\documentclass{article}
\usepackage{fontspec}
\usepackage{luacode,listings,filecontents,lipsum}
\begin{document}
\directlua{
local unicode = {}
local xstrings={}
unicode["Für"] = "for"
tex.print(unicode["Für"])
 
unicode["garçon"] = "boy"
tex.print(unicode["garçon"])
 
unicode["∆"]=1
tex.print(unicode["∆"])
tex.print("garçon")

xstrings["chapter"]="Chapteris"
tex.print(xstrings.chapter)
tex.print(string.upper("ação"))


}

The function, time and date, provide all date and time functionality in Lua. The time function when used with no argument, returns the current date  

\begin{luacode*}
function getdate (a)
  local m = "test"
  return "testing"
end
tex.sprint(os.date("%x",890000).."\\par")
tex.sprint(os.date("%d %B %Y").."\\par")
tex.sprint(os.time())
local x=os.clock()
local s=0
for i=1, 11000000 do s=s+1 end
tex.print(string.format("elapsed time: %.2f\n",os.clock()-x).."s")

tex.sprint(getdate(2))
\end{luacode*}
\directlua { tex.enableprimitives('',tex.extraprimitives()) }
\initcatcodetable1

\the\luatexversion

 \luatexdatestamp 

\formatname

regular expressions

\begin{luacode*}
test = "My name is Lua."
pattern = ".*name is (%a*)."
 
if test:match(pattern) then
    print("Name found.")
end
sub, num_matches = test:gsub(pattern, "Hello, %1!")
tex.print(sub)
\end{luacode*} 

\edef\tempstring{\string\\par..is this is a is test \string\\par and is this another?}
\begin{luacode}
local s = "\tempstring".."This is is a string and this is another"
local t = {}
local i= 0
while true do
    i = string.find(s, "\\par", i+1)
    if i == nil then break end
    t[#t + 1] = i
    tex.print(i)
end
   
\end{luacode}

\begin{luacode*}
keyval="name = Yiannis"
key, value = string.match(keyval, "(%a+)%s*=%s*(%a+)");
tex.print(key.." = "..value)
\end{luacode*}

\begin{luacode*}
local lpeg = require "lpeg"

-- matches a word followed by end-of-string
p = lpeg.R"az"^1 * -1

tex.print(p:match("hello"))        --> 6
tex.print(lpeg.match(p, "hello"))  --> 6

\end{luacode*}

\begin{filecontents*}
This is line 1
This is line 2
This is line 3
This is line 4
\end{filecontents*}
\begin{luacode*}
filename = "input.txt"
fp = io.open( filename, "r" )
 
for line in fp:lines() do
    tex.print( line )
end
 
fp:close()
\end{luacode*}

















\end{document}