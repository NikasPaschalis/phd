\cxset{style87/.style={
 chapter opening=any,
 chapter name=none,
 % positioning and float - inline is 0
 %  float right is 2
 number display=block,
 number float=right,
 number shape=starburst,
 numbering=Words,
 number spaceout=none,
 number font-size=huge,
 number font-weight=bold,
 number font-family=rmfamily,
 number font-shape=normal,
 number before=,
 number display=inline,
 number float=none,
% 
 number border-top-width=0pt,
 number border-right-width=0pt,
 number border-bottom-width=0pt,
 number border-left-width=0pt,
 number border-width=1pt,
%  
 number padding-left=0em,
 number padding-right=0.5em,
 number padding-top=0em,
 number padding-bottom=0pt,
  %number margin-top=, to do
 %number margin-left=0pt,  to create
 %
 number after=\par,
 number dot=,
 number position=rightname,
 number color=sweet,
 number background-color=white,
 %chapter name
 chapter display=block,
 chapter float=left,
 chapter shape=ellipse,
 chapter color=black,
 chapter background-color=sweet,
 chapter font-size= Huge,
 chapter font-weight=bfseries,
 chapter font-family=itshape,
 chapter before=,
 chapter spaceout=none,
 chapter after=,
 chapter margin-left=0cm,
 chapter margin-top=0pt,
 %
 chapter border-width=2pt,
 chapter border-top-width=1pt,
 chapter border-right-width=1pt,
 chapter border-bottom-width=1pt,
 chapter border-left-width=4pt,
% 
 chapter padding-left=20pt,
 chapter padding-right=20pt,
 chapter padding-top=20pt,
 chapter padding-bottom=10pt,
  %chapter title
 title font-family=rmfamily,
 title font-color=spot!80,                    %CHANGED
 title font-weight=bfseries,
 title font-size=huge,
 chapter title align=none,
 title margin-left=1cm,
 title margin-bottom=1.3cm,
 title margin-top=30pt,
 % title borders
 title border-width=0pt,
 title padding=0pt,
 title border-color=black!80,
 title border-top-color=spot!50,
 title border-top-width=2pt,
 title border-left-color=black!80,
 title border-left-width=2pt,
 title border-color=black!80,
 title padding-top=0pt,
 title padding-bottom=0pt,
 title padding-left=0pt,
 title padding-right=0pt,
 title border-right-color=spot!50,
 title border-right-width=2pt,
 title border-bottom-color=spot!50,
 title border-bottom-width=2pt,
 %
 chapter title align=left,
 chapter title text-align=left,
 chapter title width=0.8\textwidth,
 title before=,
 title after=,
 title display=block,
 title beforeskip=12pt,
 title afterskip=12pt,
 author block=false,
 section font-family=rmfamily,
 section font-size=LARGE,
 section font-weight=bfseries,
 section indent=0pt,
  section font-weight=mdseries,
 section align=left,
 subsubsection font-family=tiresias,
 subsubsection font-shape=upshape,
 subsubsection font-weight=mdseries,
 subsubsection align=flushleft,
 epigraph width=\dimexpr(\textwidth-2cm)\relax,
 epigraph align=center,
 epigraph text align=center,
 epigraph rule width=0pt,
 header style=plain}}
 
\cxset{style87}
\renewsection\renewsubsection\renewsubsubsection
\ExplSyntaxOff
\makeatother
\endinput

\makeatletter
\cxset{enumerate numberingi/.is choice,
  enumerate numberingi/.code={\renewcommand\theenumi {\csname#1\endcsname{enumi}}},
  enumerate numberingii/.code={\renewcommand\theenumii {\csname#1\endcsname{enumii}}},
  enumerate numberingiii/.code={\renewcommand\theenumiii {\csname#1\endcsname{enumiii}}},
  enumerate numberingiv/.code={\renewcommand\theenumiv {\csname#1\endcsname{enumiv}}},
  enumerate labeli punctuation/.store in=\enumeratepunctuationi@cx,
  enumerate labeli/.is choice,
  enumerate labeli/brackets/.code={\renewcommand\labelenumi{(\theenumi\enumeratepunctuationi@cx)}},
  enumerate labeli/square brackets/.code={\renewcommand\labelenumi{[\theenumi\enumeratepunctuationi@cx]}},
  enumerate labeli/right bracket/.code={\renewcommand\labelenumi{\theenumi\enumeratepunctuationi@cx)}},
  enumerate label left/.store in=\enumeratelabelleft@cx,
  enumerate label right/.code=\renewcommand\labelenumi{\enumeratelabelleft@cx\theenumi\enumeratepunctuationi@cx#1},
  enumerate leftmargini/.code={\setlength\leftmargini{#1}},
  enumerate leftmarginii/.code={\setlength\leftmarginii{#1}},
  enumerate leftmarginiii/.code={\setlength\leftmarginiii{#1}},
  enumerate leftmarginiv/.code={\setlength\leftmarginiv{#1}},
  listi topsep/.store in=\listitopsep@cx,
  listi partopsep/.store in=\listipartopsep@cx,
  listi itemsep/.store in=\listiitemsep@cx,
  listi parsep/.store in=\listiparsep@cx,
  listii topsep/.store in=\listiitopsep@cx,
  listii partopsep/.store in=\listiipartopsep@cx,
  listii itemsep/.store in=\listiiitemsep@cx,
  listii parsep/.store in=\listiiparsep@cx,
  listiii topsep/.store in=\listiiitopsep@cx,
  listiii partopsep/.store in=\listiiipartopsep@cx,
  listiii itemsep/.store in=\listiiiitemsep@cx,
  listiii parsep/.store in=\listiiiparsep@cx,
}
\cxset{compact1/.style={%
  enumerate numberingi=arabic,
  enumerate numberingii=alph,
  enumerate numberingiii=alph,
  enumerate numberingiv=roman,
  enumerate labeli punctuation=.,
  enumerate label left=,
  enumerate label right=,
  enumerate leftmargini=2.2em,
  enumerate leftmarginii=2.1em,
  enumerate leftmarginiii=1.5em,
  enumerate leftmarginiv=2em,
  listi topsep=8\p@ \@plus2\p@ \@minus\p@,
  listi itemsep=0\p@ \@plus2\p@ \@minus\p@,
  listi parsep=0\p@ \@plus2\p@ \@minus\p@,
  listii topsep=0\p@ \@plus2\p@ \@minus\p@,
  listii itemsep=0\p@ \@plus2\p@ \@minus\p@,
  listii parsep=0\p@ \@plus2\p@ \@minus\p@,
  listiii topsep=0\p@ \@plus2\p@ \@minus\p@,
  listiii itemsep=0\p@ \@plus2\p@ \@minus\p@,
  listiii parsep=0\p@ \@plus2\p@ \@minus\p@,
}}
\cxset{compact2/.style={%
  enumerate numberingi=alph,
  enumerate numberingii=roman,
  enumerate numberingiii=alph,
  enumerate numberingiv=roman,
  enumerate labeli punctuation=,
  enumerate label left=(,
  enumerate label right=),
  enumerate leftmargini=2.2em,
  enumerate leftmarginii=2.1em,
  enumerate leftmarginiii=1.5em,
  enumerate leftmarginiv=2em,
  listi topsep   = 8\p@ \@plus2\p@ \@minus\p@,
  listi itemsep = 0\p@ \@plus2\p@ \@minus\p@,
  listi parsep   = 0\p@ \@plus2\p@ \@minus\p@,
  listii topsep  = 0\p@ \@plus2\p@ \@minus\p@,
  listii itemsep= 0\p@ \@plus2\p@ \@minus\p@,
  listii parsep  = 0\p@ \@plus2\p@ \@minus\p@,
  listiii topsep = 0\p@ \@plus2\p@ \@minus\p@,
  listiii itemsep= 0\p@ \@plus2\p@ \@minus\p@,
  listiii parsep  = 0\p@ \@plus2\p@ \@minus\p@,
}}

\ExplSyntaxOn
\def\setenumerate#1{
\cxset{#1}
\def\@listi{%
           \leftmargin\leftmargini
            \parsep\listiparsep@cx
            \topsep\listitopsep@cx\relax
            \itemsep\listiitemsep@cx}
            
\def\@listii{\leftmargin\leftmarginii
            \parsep\listiiparsep@cx
            \topsep\listiitopsep@cx\relax
            \itemsep\listiiitemsep@cx}
            
\def\@listiii{\leftmargin\leftmarginiii
            \parsep\listiiiparsep@cx
            \topsep\listiiitopsep@cx\relax
            \itemsep\listiiiitemsep@cx}
}


\setenumerate{compact1}

\def\authorblockformat{\hfill\hfill}
\cxset{author block=true,
          authors names=Dr Y. Lazarides}
\MakePercentComment

\chapter[Template 87]{{\itshape Ukunxityiswa kwempimpi itayari\\
                                 njengotshaba lomzabalazo}: An\\
                                 exploratory study of insider\\
                                 accounts of necklacing in three\\
                                 Port Elizabeth townships}
\thispagestyle{plain}
\pagestyle{headings}
The original inspiration for this is an ugly duckling from a book with a horrid story. The original is shown in Figure~\ref{necklacing}.
The human eye perceives beauty in symmetry although in a human face apparently we expect some imperfections.
I am not too sure about typography and symmetry though, but there is what I call a bit of balance. 

\begin{figure}[ht]
\centering
\includegraphics[width=0.9\textwidth]{necklace}
\caption{Style 87 spread.}
\label{necklacing}
\end{figure}


When the South African townships exploded in violence in the 1980s, the most potent weapon to terrorize
police informers was \emph{necklacing}. 

The liberation struggle was also a perfect cover-up for criminal elements (\emph{tsotsis}) to persuade businessmen to provide money or goods to promote some or other liberation activity. 

\cxset{quotation left margin=\parindent,
          quotation above=-10pt,
          quotation parindent=0pt,
          quotation font-size=\normalsize}%

\begin{quotation}
Even if you saw through the bad potatoes, you had to think of your own safety. As the numbers swelled in our base,
you were scared of criticizing arrogant and aggresive members who made it clear that they were ready to kill. Hence you had to join the struggle actively or await certain necklacing. 
\end{quotation}

Now this is a serious book and a serious subject and worth reading if you pondering about the changing world and the recent ISIS cruel burning of the Jordanian pilot. Murdering humans by burning has not been the expertize of only witch-hunters.

Back to the template. I will spend a bit more time on this particular template as it demonstrates the use of some properties we have hardly touched so far  \emph{display, float} and \emph{shape}.

\begin{verbatim}
\cxset{
 chapter opnening=anywhere,
 chapter display=block,
 chapter float=right,
 chapter shape=ellipse,}
\end{verbatim}

\cxset{chapter shape=starburst,
          chapter opening=anywhere,}
          
\chapter{Why was Marilyn Monroe’s Mole Beautiful?}

Scientists spend a lot of effort in pondering questions, such as the above. The answer of course because 
people just looked at her legs and sexy body. Just kidding! Now and then I will write things that are not true, you need to take them with a punch of salt. According to Marcelo Gleiser:

\begin{quotation}
For facial (or body) moles to be considered aids to beauty they must have the right placement and be of just the right size (for example, large dark moles on the middle of the forehead are usually not found very appealing): too small they are unnoticeable and hence of no seen; too large and they are plain ugly. 
\end{quotation}

So how big must a mole be to be considered attractive? Before I give you the answer in Gleiser’s words do note the colon after the right bracket in the text above (which was copied verbatim from the book), don’t do that. It is worse than the wrong size mole on your face. Back to Gleiser he writes that a mole size must be:

\begin{quotation}
Apparently, the critical size, the boundary between the beautiful and the grotesque, is about one centimeter across. Marilyn Monroe and Cindy Crawford had their moles “just right” and are considered beautiful. 
\end{quotation}


\section{The chapter title}

The chapter title of this template (87) has a border to the left of the text block. Remember borders can be drawn individually for every side of the rectangle. In a similar fashion, padding can also be added to individual sides of the containing block. Let us review these settings once again. We also change some of the settings
in order for us to reproduce the title in a somewhat smaller scale.

\begin{verbatim}
\cxset{chapter opening=anywhere,
          title font-size=LARGE,
          chapter title width=0.6\textwidth,
          title padding=5pt,
          title border-width=2pt,
          title border-top-width=0pt,
          title border-bottom-width=0pt,
          title border-right-width=0pt,
          title margin left=30pt,
          author block=false,
          title margin top=30pt,
         }
\end{verbatim}

\cxset{chapter opening=anywhere,
          title font-size=LARGE,
          chapter title width=0.6\textwidth,
          title padding=5pt,
          title border-width=2pt,
          title border-top-width=0pt,
          title border-bottom-width=0pt,
          title border-right-width=0pt,
          author block=false,
          title margin-left=0pt,
          title margin top=30pt,
         }        
\chapter[Template 87]{{\itshape Ukunxityiswa kwempimpi itayari
                                 njengotshaba lomzabalazo}: An
                                 exploratory study of insider
                                 accounts of necklacing in three
                                 Port Elizabeth townships}

\texttt{title top-margin} when this is specified the whole block is separated from the chapter and number blocks, i.e., \tex is going to a vertical mode and skips the amount of space required. This is not a glue length. We do not allow flexibility in balancing the vertical glue of the page. How much this space should be depends on the looks of
the template, normally I would iterate  couple of times before I eye-ball an amount that looks pleasant to me.
For example trying a smaller length:
\begin{verbatim}
\cxset{title margin top=10pt}
\end{verbatim}
does not come out very well but you may disagree. This is one aspect of aesthetics that different people perceive typographical beauty differently, although most would agree on general principles fairly easily. When it comes to spacing though I have had arguments with many people that might want the titles a little up or a little down.

\vbox{%
\cxset{title margin top=10pt}
\chapter[Template 87]{{\itshape Ukunxityiswa kwempimpi itayari
                                 njengotshaba lomzabalazo}: An
                                 exploratory study of insider
                                 accounts of necklacing in three
                                 Port Elizabeth townships (style87)}}

\section{The border property}

We tried to emulate at least partially at this stage to define border properties similar to the familiar CSS model. This has also evolved at a point and was otne of the hot issues during the early dates of CSS.

\begin{verbatim}
p {
    border: 5px solid blue;
}
\end{verbatim}

There is also an extended version of the properties where individual borders can be set. The extended version is a much more powerful model and more suitable in what we are doing. Similar keys are provided for example in |tcolorbox|. 

I had to recreate the code from scratch in order to fine tune it to the requirements of the |phd| package, which is a pity.

The model currently is short of what it should be, but it is a start. A full and more powerful version should probably be fully painted using graphics rather than text commands. There are about 32 properties in all. As you can see this
is very far from the humble \cmd{\fbox} and similar commands provided by \latexe. In our case, besides the |width|
property we may also want to control the |\hsize| of the rules. Things get complicated. 

The basic properties:

\begin{verbatim}
title border = none,
border-color=black,
border-image = none,
border-style = solid,
\end{verbatim}

\begin{verbatim}
border-bottom,
border-bottom-color = blue,
border-bottom-left-radius,
border-bottom-right-radius,
border-bottom-style = solid,
border-bottom-width = 15pt,
border-collapse %not used this is only for tables
border-image-outset
border-image-repeat
border-image-slice
border-image-source
border-image-width
border-left
border-left-color
border-left-style
border-left-width
border-radius
border-right
border-right-color
border-right-style
border-right-width
border-top
border-top-color
border-top-left-radius
border-top-right-radius
border-top-style
border-top-width
border-width
\end{verbatim}

Keeping with the style we have developed so far all these properties are prefixed with |chapter title| or simply |title|
etc. I found this more intuitive and since I expect people to be cutting and pasting and only modifying selected values, it is very intuitive. 







The study published as a book by the Human Sciences Research Council in South Africa, \textit{Violence in South Africa: A variety of perspectives} by Bornman et al (1998) examines the chilling story in an article by. 

The book’s cover and layout is credited to  Suzan Smith. She appears to have designed a number of HSRC books. I found the style a bit too rough with the chapter and number boxed, although I like the left border in the title. The titles are slightly offset from their optical center. 

\begin{figure}[ht]
\centering
\includegraphics[width=1.0\textwidth]{swazi-01}
\caption{A rural primary school in Swaziland.}
\end{figure}

\section{Template 87-doc}

\begin{quotation}
In the afternoon, as I had heard from Musa that the wives of the king and the  princes
were fattened to such an extent that they could not stand upright, I paid my respects to 
Wazezeru
\end{quotation}

\drawfontbox{Ukunxityiswa.}

\drawfontframe{\HUGE Ukunxity}\drawfontframe{\HUGE Ukunxit}

 
%https://books.google.ae/books?id=aPAUAAAAIAAJ&printsec=frontcover&dq=swaziland&hl=en&sa=X&ei=ZfsKVZ_KEMX4UJ2Jg9AD&ved=0CDsQ6AEwBg#v=onepage&q=swaziland&f=false