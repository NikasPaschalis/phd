\makeatletter\@specialfalse\makeatother

\def\cows{\raisebox{1sp}{\hbox to \textwidth{\hfill\includegraphics[width=3.6cm]{kennedy-seal}\hfill}}}
\cxset{style78/.style={
chapter opening=any,
toc image =,
 chapter name=CHAPTER,
 numbering=WORDS,
 number font-size= Large,
 number font-family=sffamily,
 number font-weight= mdseries,
 number before=,
 number dot=,
 number after=\hfill\hfill\par\vspace*{15pt},
 number position=rightname,
 number before=\kern1em,
 chapter font-family= \sffamily\upshape,
 chapter font-weight=mdseries ,
  chapter font-size= Large,
 chapter color=black!90,
 chapter before=\cows\vskip10pt\hfill,
 chapter margin left=0pt,
 chapter margin top=0pt,
 chapter spaceout=none,
 chapter after=,
 number color=black!90,
 %
 title margin top=60pt,
 title margin bottom=30pt,
 title margin left=1sp,
 chapter title align=center,
 chapter title width=\textwidth,
%
 title before=\par,
 title after=\par,
 title font-family= sffamily,
 title font-color= black,
 title font-weight= bfseries,
 title font-size= LARGE,
 title font-shape=upshape,
 title spaceout=soul,
 header style= plain,
 section font-size=LARGE,
 section color=black,
 section numbering prefix=,
 section numbering=none,
 subsection numbering=none,
 subsection color=black,
 section indent=0pt,
 epigraph width=0.55\textwidth,
 epigraph align=center,
 epigraph rule width=0pt,
 epigraph text align=center,
 epigraph font-size=\large,
 epigraph source align=center,
 }}

\cxset{style78}

\chapter[Template 78]{FOREIGN POLICY}
 
 \epigraph{\itshape  
 Let every nation know, whether it wishes us well or ill, that we shall pay any price, bear any burden, meet any hardship, support any friend, oppose any foe to assure the survival and the success of liberty.
 }{\small\sffamily\mdseries---John F. Kennedy, inaugural address, \mbox{January 20, 1961}}

\large

The Black Death of 1348 and 1349 has been one of the most devastating natural disasters ever to strike Europe. There are no exact figures as to the devastation it caused, but estimates generally agree that between 70 or 80 percent of the populatio perished. Europe in about 1420 could have counted barely more than a third of the people it contained one hundred years before.

\begin{figure}[ht]
\centering
\includegraphics[width=0.9\textwidth]{kennedy-01}
\end{figure}

The European pestilence (dubbed the Black Death centuries later by northern European scholars) began in 1348 and ravaged the continent in intermittent waves for a century. In that time it killed millions; Herlihy estimates that in villages as far apart as England and Italy populations were reduced by as much as 70 or 80 percent. It is regarded as one of European history's watershed events. While not disputing that, Herlihy revisits much of the conventional wisdom about the demographic, cultural, and even medical impact of the plague. Indeed, he questions whether the Black Death even was plague: He notes that medieval chroniclers did not mention epizootics (mass deaths of rats or other rodents, which are a necessary precursor to plague) and did mention lenticulae or pustules or boils over the victims' bodies, which is not characteristic of plague. 

Herlihy observes that the illness showed some signs of bubonic plague, some of anthrax, and some of tuberculosis, and speculates that perhaps several diseases ``sometimes worked together synergistically to produce the staggering mortalities.'' Herlihy sees Europe before the Black Death as engaged in a ``Malthusian deadlock'' in which a stable population devoted most of its energy to production of food and subsistence goods. The precipitous population decline occasioned by the Black Death compelled Europe to devise labor-saving technologies that transformed the economy. In more controversial theories, Herlihy argues from the increased use of Christian given names that the Black Death caused the Christianization of what had formerly been a pagan society with a Christian veneer, and contends that in the wake of the pestilence Europeans turned to preventive measures such as birth control to check explosive population growth. A stimulating discussion of some rarely considered aspects of one of history's turning points.

\emph{The Black Death and the Transformation of the West} by David Herlihy (1997) is the inspiration of template 76. The unique feature of the template is the graphic before the chapter number and which is a recurring theme right through the book. The book is typical of the Harvard University Press typographical style. The tone is academic and the language even more so. 

Living in the age of AIDS and Ebola fever, this historical perspective helps visualize what might happen to our society if any such epidemic strikes humanity.

\section{Technical}

The template is easy to develop using the \pkgname{phd} package techniques. You can load the package using
|\cxsetstyle| and modify any of the parameters afterwards.

\begin{verbatim}
\cxset{style76}
\end{verbatim}

The graphic is inserted by first defining a macro to make it easier to insert in the |before number| key. Depending on what you want to do with such graphics you could also have it in a box or a minipage to enable it to be vertically centered exactly. I opted to manually adjust it by eye-balling the settings. Sometimes this is easier to    optically center it. 

\subsection{Lower level headings}

The book does not have any headings, as each chapter narrates the story in a chronological sequence. As we needed them for this chapter, I have made them unnumbered and position at the left margin.


\cxset{chapter opening=anywhere}



\chapter{Why Vampires don’t Like Garlic}


\makeatother




