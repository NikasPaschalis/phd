\makeatletter\@specialfalse\makeatother
\newfontfamily\vampire{Buffied.ttf}
\cxset{style73/.style={
 toc image =,
 name={},
 numbering=WORDS,
 number font-size= huge,
 number font-family= \vampire,
 number font-weight= mdseries,
 number before=\vskip0pt\hfill,
 number dot=,
 number after=\hfill\hfill\par\vspace*{10pt}%
\hspace*{1.2em}\hbox to \textwidth{\hfill\includegraphics[width=0.3\textwidth]{vampire-line}\hfill},
 number position=rightname,
 chapter font-family= vampire,
 chapter font-weight=mdseries ,
 chapter font-size= Large,
 chapter before={\vspace*{15pt}},
 chapter after=,
 chapter spaceout=none,
 number color=black!90,
 %
 title margin top=-15pt,
 title margin bottom=60pt,
 chapter title align=center,
 chapter title width=\textwidth,
%
  title before={},
 title after=,
 title font-family= rmfamily,
 title font-color= black,
 title font-weight= mdseries,
 title font-size= huge,
 title spaceout=none,
 header style= plain,
 section font-size=Large,
 section color=black,
 section numbering prefix=,
 section indent=0pt,
 }}


\cxset{style73}
\debugchapter

\chapter[Template 72]{Blood Lust---End of the One Night Bites\\ (How AIDS Killed the Vampire)}

Since the publication of Dracula the entertainment business had a never ending string of vampire productions. They have also been the subject of numerous fiction books. I must admit I am not one for fiction books and I rarely read any,  but it always fascinates me how many people are hooked by them. This template (73) is about using simple 
templates to match fonts to the book’s subject. 

\begin{figure}[ht]
\centering
\includegraphics[width=0.8\textwidth]{vampire-01}
\end{figure}

Vampire fonts can be found on the web and provided you obtain one, you can incorporate it into the template. The line below the chapter number is simply a graphic but if you are familiar with \tikzname\ you can arrange to draw it
in real time. 

The template was inspired by \emph{Vampire Nation} by Arlene Russo (2008). The book was published by Llewellyn Publications in the USA. The book was designed by Steffani Sawyer. One can only ponder that a disclaimer in the book states: “The activities in this book may be dangerous or even illegal. Please exercise responsibility. Remember that what works for some may not work for you. As you proceed, use care, caution, and common sense. Llewellyn will not be held responsible for personal actions taken in response to this book.”  Vampires and lawyers are always fascinating subjects!

\cxset{chapter opening=anywhere}
\chapter{Why Vampires don’t Like Garlic}
\begin{figure}[ht]
\centering
\includegraphics[width=0.8\textwidth]{vampire}
\end{figure}

When you feel bored and have nothing to do, run your thesis with this template and see how it looks and you will understand better the power of fonts on human perception. You thought you were unbiased did you? Would fonts affect the marks you give to a student?




%https://books.google.ae/books?id=Bz83hzRnri8C&printsec=frontcover&source=gbs_ge_summary_r&cad=0#v=onepage&q&f=false













\makeatother
