\restoregeometry
\newgeometry{left=4.5cm,right=4.5cm}
\newcommand\dashedrule{%
    \dashuline{{\color{spot!50}\hbox to \textwidth{}}}%
}

\newcommand\uwaverule{%
    \uwave{{\color{black}{\hbox to \textwidth{}}}}}
    

\makeatletter
\def\ps@curly{\leftskip\z@\let\@mkboth\@gobbletwo\vfuzz=5\p@
    \def\@oddhead{}%
    \def\@evenhead{}%
  \def\@oddfoot{\small
    \vbox{\vspace{15pt}%
      \global\hoffset=0pc%
      \noindent\hbox to\textwidth{\mbox{}\hfill\{{\rm\thepage}\}\hfill}%
      \makebox[\z@][l]{\@c@pyrightline}%
%     \noindent\hspace*{-9pc}\rule{37pc}{0.25pt}%
    }%
  }%
  \let\@evenfoot\@oddfoot
  \def\sectionmark##1{}%
  \def\subsectionmark##1{}%
 }
 
 % Renew the figure numbering
 \let\oldthefigure\thefigure
 \renewcommand \thefigure
     {\ifnum \c@chapter>\z@ \fi \@arabic\c@figure}%
 \makeatother

\cxset{style64/.style={
 name=,
 numbering=WORDS, 
 number font-weight=normalfont,
 number font-size=large,
 number font-family=rmfamily,
 number before=\hfill,
 number after=\hfill\hfill\vskip1pt%
                        \uwave{\hbox to \textwidth{}}\mbox{}\par%
                        ,
 number dot=,
 number position=rightname,
 number color=black,
 chapter color=black,
 chapter font-size=Large,
 chapter before=,
 chapter after=,
 chapter margin left=0pt,
 chapter margin top=0pt,
 title font-family=rmfamily,
 title font-color=black!80,
 title font-weight=,
 title font-size=Huge,
 title before=\vskip1pt\par,
 chapter title align=centering,
 chapter title width=0.8\textwidth,
 title after=,
 title beforeskip=,
 title margin top=0pt,
 title margin bottom=20pt,
 header style=curly,
 section numbering prefix=,
 }}
\debugchapter

\cxset{style64}

\chapter{Tulipmania}
\label{ch:tulipmania}

\pagestyle{curly}
\thispagestyle{curly}

\newthought{Something extraordinary happened in 2007}. Perfectly normal economies such as those in the Gulf changed overnight from booming to recession. This is the subject of template 64. Many economists consider the 2007--2008 crisis to be the worst financial crisis since the Great Depression. The U.S. Senate's Levin–Coburn Report concluded that the crisis was the result of ``high risk, complex financial products; undisclosed conflicts of interest; the failure of regulators, the credit rating agencies, and the market itself to rein in the excesses of Wall Street."

\begin{figure}[ht]
\centering
\includegraphics[width=0.8\textwidth]{tulipmania-01}
\caption{Style 64 Opening pages.}
\end{figure}

While regulators eager to find explanations and to appease their constituents commissioned one report after another the crisis is simpler explained by the words, predatory lending and weak and fraudulent underwriting practices. In religious texts the prophets would have preached the perils of greed.

Back to our template. The book 
\emph{Tulipmania,
Money, honor, and knowledge in the Dutch Golden Age}
by Anne Goldgar, 2007, is a historical study

%% widen the page again
\restoregeometry
\newpage

\noindent of the  Netherlands in the 1630s when it was gripped by tulipmania: a speculative fever unprecedented in scale and, as popular history would have it, folly. We all know the outline of the story—how otherwise sensible merchants, nobles, and artisans spent all they had (and much that they didn’t) on tulip bulbs. We have heard how these bulbs changed hands hundreds of times in a single day, and how some bulbs, sold and resold for thousands of guilders, never even existed. Tulipmania is seen as an example of the gullibility of crowds and the dangers of financial speculation. The book received the 
American Historical Association: AHA-Leo Gershoy Award and others. It is a long book with 446 pages 13 color plates, 69 halftones, 3 line drawings. The book is  6 x 9. 

\section{The chapter head}

The chapter heading can be constructed with standard settings, but there is one minor hiccup. The text that follows it is as narrow as the heading, while on the next page it goes to normal text width. This is almost impossible to achieve with \tex. The best way to simulate it, is to construct the first page manually and adjust the text until it fills the page.

The main feature of the chapter head is the wavy line. This is easily achieved either using \tikzname\ or \tex techniques. The \pkgname{ulem} provides a macro named \cmd{\uwave} that can be used to underline text using a wavy line. This is provided as an option to the |number after| key. 

\begin{scriptexample}{}{}
\begin{verbatim}
\cxset{
    number after=\hfill\hfill\par\vspace*{10pt}%
                           \uwave{\hbox to \textwidth{}}\par,
}
\end{verbatim}
\end{scriptexample}

There are many other possibilities to modify this line, such as making it narrower. Building the wavy line rule into
a macro is a better idea, than just inserting it into the key. Other designers might prefer a dash line---the subject of another style.
 
 Before we develop our generic macro (to also cater for other symbols) let us study how Donald Arsenau
 developed the macro. The wavy line is picked up from the font |lasy| at 6-pt. 
 The command is  \cmd{\uwave} (under-wave). Provided we have the right font, we can extend the package to
 other styles.

\begin{scriptexample}{}{}
\begin{verbatim}
\def\uwave{\bgroup \markoverwith{\lower3.5\p@\hbox{\sixly \char58}}\ULon}
\font\sixly=lasy6 % does not re-load if already loaded, so no memory problem.
\end{verbatim} 
\end{scriptexample}

You can view the contents of the |lasy| font, using |tex testfont| and providing appropriate answers to the prompts. The font is very sparse and other alternative symbol fonts are possible. The wavy line is produced by repeateatedly printing |\char58| ({\sixly\char58}) from the |lasy6| font. If you have a strong interest to expand your knowledge of
\tex Donald Arsenau’s packages are a good point to start reading and studying how complex typesetting is achieved.
\bigskip

In our case we will settle in just defining two commands \cmd{\dashedrule} and \cmd{\uwaverule} that we can use for typesetting these unusual headings.

\emphasis{dashedrule,uwaverule}
\begin{texexample}{}{}
\def\dashedrule{\dashuline{\hbox to \textwidth{}}}
\def\uwaverule{\uwave{\hbox to \textwidth{}}}
\end{texexample}

Now that we have defined the two new macros we can change our settings to:

\begin{texexample}{}{}
\cxset{chapter opening=anywhere,
           number after=\hfill\hfill\par\vspace*{10pt}%
                        \dashedrule\par,}
\end{texexample}

We get,

\bgroup
\cxset{chapter opening=anywhere,
           number after=\hfill\hfill\par\vspace*{10pt}%
                        \color{spot!50}\dashedrule\par,}
                        
 {\centering
\begin{minipage}{0.7\textwidth}
\chapter{A Dashed Chapter Head}
\lorem
\end{minipage}
\par
}
\egroup

\bigskip

\section{Headings}

The book has the folio number placed at the foot of the page in curly brackets. It never cease to amaze me what Book Designers have achieved with headings and their simple design. In this case the  curly bracket with its flowery design is a very appropriate choice for \emph{Tulipania}. 

This uses a predefined heading named |curly|. For the code see the template code.


\section{Illustrations}

Approximately 20\% of the book’s pages have illustrations. Notwithstanding that the choice of illustrations was superb the styling of the captions follows the headings, with a curly line at the bottom.  We will now see how to style them. Before we even attempt to define parameters, keys and settings let us have an attempt to manually style the block. I ususally do this in a minimal until I get the macro more or less to where it should be and then transfer the code to either the package, if it has a broader use or to the template if it is localized.

\makeatletter
\def\uwave{\bgroup \markoverwith{\lower12\p@\hbox{\sixly \char58}}\ULon}
\makeatother



\begin{texexample}{Defining the caption.}{}
\makeatletter
\def\uwave{\bgroup \markoverwith{\lower12\p@\hbox{\sixly \char58}}\ULon}
\makeatother
{
\begin{center}
{\rmfamily\scshape FIGURE 1.} The captions are underlined at the last line with a wavy line.\par
Some of the illustrations are full width and others are full margin as shown above. \\ 
\uwave{A special caption needs to be provided to achieve this.}\par
\end{center}
}
\end{texexample}





The illustration captions are troublesome. First we need to center them and then capture the last line and insert a
wavy underline. We must also control the space below the last line of the caption so that it does not look like an underline. 

\begin{figure}[ht]
\bgroup
\captionsetup[figure]{%
labelfont=sc, 
justification=centering,
labelsep=space,
singlelinecheck=false,
format=plain,
name=\textsc{FIGURE},}
\def\thechapter{}
\makeatletter
\def\thesection{\@arabic\c@section}
\makeatother
\includegraphics[width=\textwidth]{tulipmania-02}

{
\centering {\rmfamily\scshape FIGURE 1.} The captions are underlined at the last line with a wavy line.\\ \uwave{A special caption needs to be provided to achieve this.}\par

}
\egroup
\end{figure}

\begin{figure}[ht]
\bgroup
\captionsetup[figure]{%
labelfont=sc, 
justification=centering,
labelsep=space,
singlelinecheck=false,
format=plain,
name=\textsc{FIGURE},}
\def\thechapter{}
\makeatletter
\def\thesection{\@arabic\c@section}
\makeatother
\includegraphics[width=\textwidth]{tulipmania-03}
{
\centering {\rmfamily\scshape FIGURE 1.} The captions are underlined at the last line with a wavy line.\\ 
Some of the illustrations are full width and others are full margin as shown above.\\
 \uwave{A special caption needs to be provided to achieve this.}\par
}
\egroup
\end{figure}

\subsection{Full page images}

Many of the images are on float pages on their own. They follow the same style of the other figures. When the image does not extend to the full text height, I prefer to position the caption at the bottom of the page. The template comes to life with these larger images and in the book a number of images are included as plates. 

As discussed on the chapters for image placement there will always be a need to manually adjust image sizes and image placements. Visually display of figures can never be fully automated as still the best compositor is human creativity. However, with the technicques outline in this book and the \pkgname{phd} one hopes to achieve a high degree of automation with many documents requiring very little manual intervention.

\begin{figure}[ht]
\bgroup
\captionsetup[figure]{%
labelfont=sc, 
justification=centering,
labelsep=space,
singlelinecheck=false,
format=plain,
name=\textsc{FIGURE},}
\def\thechapter{}
\makeatletter
\def\thesection{\@arabic\c@section}
\makeatother
\vbox{\includegraphics[width=\textwidth]{sebastian}}\par
\vfill\vfill
\vspace*{1.5cm}

\vbox{%
\centering {\rmfamily\scshape FIGURE 1.} The captions are underlined at the last line with a wavy line.\\ 
Some of the illustrations are full width and others are full margin as shown above.\\
 \uwave{A special caption needs to be provided to achieve this.}\par
}
\egroup
\end{figure}

\begin{figure}[htb]
\vbox to \textheight\bgroup
\captionsetup[figure]{%
labelfont=sc, 
justification=centering,
labelsep=space,
singlelinecheck=false,
format=plain,
name=\textsc{FIGURE},}
\def\thechapter{}
\makeatletter
\def\thesection{\@arabic\c@section}
\makeatother
\hspace*{-.1\textwidth}\includegraphics[width=1.2\textwidth]{floral-wagon}\par
\vfill\vfill

{
\Centering {\rmfamily\scshape FIGURE \makeatletter\@arabic\c@figure\makeatother.} Flora's mallewagen.
Description	
Allegory of the Tulip Mania. Hendrik Gerritsz Pot. The goddess of flowers is riding along with three drinking and money weighing men and two women on a car.\\
Weavers from Haarlem have thrown away their equipment and are following the car. \\
\uwave{The destiny of the car is shown in the background: it will disappear in the sea.}\par
}
\egroup
\end{figure}


%% Restore changes
\let\thefigure\oldthefigure
\restoregeometry



