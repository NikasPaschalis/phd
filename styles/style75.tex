\makeatletter\@specialfalse\makeatother

\cxset{style76/.style={
chapter opening=any,
 toc image =,
 name=,
 numbering=arabic,
 number font-size= huge,
 number font-family=sffamily,
 number font-weight= mdseries,
 number before=\par\hfill,
 number dot=,
 number after=\hfill\hfill\par\vspace*{15pt},
 number position=rightname,
 chapter font-family= sffamily,
 chapter font-weight=mdseries ,
 chapter font-size= large,
 chapter color=black!90,
 chapter before=\par,
 chapter margin left=0pt,
 chapter spaceout=none,
 chapter after=,
 number color=black!90,
 %
 title margin top=30pt,
 title margin bottom=60pt,
 title margin left=0pt,
 chapter title align=center,
 chapter title width=0.85\textwidth,
%
 title before=,
 title after=\par\vskip24pt\hrule\vskip1.5pt
                  \hbox to \textwidth{\hfill\vrule width3cm height2pt \hfill},
 title font-family= sffamily,
 title font-color= black,
 title font-weight= bfseries,
 title font-size= LARGE,%no effect since we have defined the font-family as gyre with a scale
 title font-shape=itshape,
 title spaceout=none,
 header style= plain,
 section font-size=LARGE,
 section color=black,
 section numbering prefix=,
 section numbering=none,
 subsection numbering=none,
 subsection color=black,
 section indent=0pt,
 }}

\cxset{style76}

\chapter[Template 76]{Bubonic Plague:\\ Historical Epidemiology and Medical Problems}
\label{style76}
\thispagestyle{plain}

During the Cold War, the respective political leaderships of the two Germanys developed very different narratives concerning the legacy of the Third Reich and of the Holocaust. Herf (History/Ohio Univ.) describes how, in Communist East Germany (GDR), the prevailing ideology of ``antifascism'' came to be divorced from Nazism; rather, it stood for opposition to the ``bourgeois capitalists'' in Bonn, London, Washington, and, ultimately, Israel. The GDR's leaders viewed themselves as victims of the Nazis, rather than as heads of one of the Third Reich's successor states, with all the obligations that might entail. Thus, in the early '50s, when some of the GDR's leading theorists advocated reparations to Jewish Holocaust survivors, they were purged from the party. 

\begin{figure}[ht]
\centering
\fbox{\includegraphics[width=0.45\textwidth]{germany-01}%
\includegraphics[width=0.45\textwidth]{germany-02}}
\end{figure}

The history of Holocaust memory in West Germany is decidedly more ambivalent. Chancellor Konrad Adenauer accepted the policy of reparations to the Jews, but he did so grudgingly while also ``integrating'' ex-Nazis into his Christian Democratic government and proceeding sluggishly in prosecuting suspected Nazi criminals. The ``heros'' of Herf's study are a number of West German presidents, particularly Theodor Heuss (in office 1949-59), who took the initially highly unpopular stance that postwar Germans should feel collective shame, if not collective guilt, for the Nazis' war crimes, as well as such Social Democratic leaders as Kurt Schumacher, Ernst Reuter, and Willy Brandt. Herf focuses almost exclusively on policy-makers; there is unfortunately little here on the role of public opinion in West Germany, and nothing on such cultural influences as the writer G. Grass, or on the roles of the small Jewish communities in each country. Still, this illuminates much of the political cultures of the two Germanys. Herf also has provided a valuable case study of how the quest for memory and justice are largely subsumed by present- day nationalist and other political needs.



\section{Technical}

The template is easy to develop using the \pkgname{phd} package techniques. You can load the package using
|\cxsetstyle| and modify any of the parameters afterwards.

\begin{verbatim}
\cxset{style75}
\end{verbatim}

The rule is actually two rules, one with a larger depth so that the square box at its end points downwards.
This is placed in a zero box so that we can overflow it into the margin. A similar approach is implemented for the rules in the running headers.

\subsection{Lower level headings}

The book does not have any headings, as each chapter narrates the story in a chronological sequence. As we needed them for this chapter, I have made them unnumbered and position at the left margin.


\cxset{chapter opening=anywhere}

%https://books.google.ae/books?id=yahqnBEEbQEC&printsec=frontcover&dq=harvard+university+press&hl=en&sa=X&ei=seICVf2QBI2zabeWgOgP&ved=0CFAQ6AEwCDhG#v=onepage&q&f=false
%https://www.kirkusreviews.com/book-reviews/jeffrey-herf/divided-memory/
\chapter{Why Vampires don’t Like Garlic}

When you feel bored and have nothing to do, run your thesis with this template and see how it looks and you will understand better the power of fonts on human perception. You thought you were unbiased did you? Would fonts affect the marks you give to a student?




%https://books.google.ae/books?id=Bz83hzRnri8C&printsec=frontcover&source=gbs_ge_summary_r&cad=0#v=onepage&q&f=false













\makeatother
