\cxset{chapter name=,
          title margin-top=0pt}

\chapter{Short Report on Critical Interruptions, Delays and Difficulties at St Regis, Habtoor City}

\section{Introduction}

This is a brief summary of recent selected instructions for additional works that have impacted  MEP Progress. 
In addition to these additional works another critical factor that affected progress was the congestion of services and the numerous RFIs and responses we had to raise in order to resolve them.

\begin{itemize}
\item Relocation of Kitchen Extract ducting Ground Floor, Mezzanine and Podium BOH areas.
\item  Additional AV points in all public areas.
\item  Additional telephone, data and CCTV points in all Public Areas.
\item  Motorized curtains Meeting Rooms.
\item Lighting Control System.
\item Changes to Electrical DBs, SMDBs due to late receipt of DEWA approved drawings.
\end{itemize}

We have reacted as fast as possible to all instructions and as soon they were received we have added resources to mitigate delays. Where days slipped these are only by a few days and we are confident that by end of this month all physical installations will be completed with the exception of the English Pub, Banquet and Royal Suite. We have started flushing on the 7 April as planned and we expect to provide \emph{wild air} before the end of April ahead of the projected date of the 7 May 2015.

\section{Back of the House Areas}

All back of the House Areas experienced serious productivity issues, due to the lack of primary co-ordination at design stage. This caused delays until solutions were found enabling us to install the services.

\subsection{Ground Floor Kitchen and Related Areas}

At the outset please note that ceiling grid clearances for all the BOH Ground Floor areas related to the kitchen are to be released for inspections latest tomorrow afternoon. This will exclude one shaft, where the black steel ductwork for the kitchen extract is distributed to Mezzanine and Podium 1. The reason for the exclusion is that this is a new shaft that was introduced in the first week of March in order to resolve space constraint issues. This had to be cut-out by the Main Contractor and still has to be reinforced by a Specialist Fiber Contractor before we are given access.

The allowable ceiling height in this area was impossible to be achieved and the kitchen extract duct eventually was split in two sections and distributed through two different routes in order to avoid passing it through the corridors which could not accomodate it.

In addition a new roller shutter window was introduced, that made it impossible to install the fresh air ducts feeding the kitchen. After several attempts by |K&A| to find an acceptable solution the roller shutter  window was abandoned as per the instructions of the Client Representative. 

\section{Basement Kitchen and Related Areas at B1}

Please note that these areas (with the exception of the corridor) have been cleared for ceiling grid closures in most areas and the balances are as per target to close by the 15 April 2015, including additional works. The additional works were mostly for additional ELV points on walls and for which we have received drawings on the 29 March 2015. We have instituted overtime and added additional crews to complete the works as fast as possible. Most rooms in the area have been affected. 


\end{document}
\chapter{Busbar System}

As per the approved Baseline Program we expected to place the busbar order for all three hotels on 27 February 2014. However, HLS DSE-JV were unable to place any orders due to the events that are outlined below, with finality on all busbars only achieved in April 2015. 

\begin{enumerate}
\item On the 23 December 2013 we were requested to change the specification for some busbars via HLG transmittal Ref. No. HLG-626-DT-HLS-0628 dated 23 Decemeber 2013 \textit{Fire Resistance Bus Bar Specification}.

\item On the 25 February 2014 we were issued revised designs via tranmittal Ref. No. HLG-626-DT-HLS-0873 \textit{Revised Electrical Drawings}.

\end{enumerate}


\chapter{Generators}

\section{Generator Ventilation}

\subsection{Background}

The original tender drawings indicated the Generator Ventilation to be by means of Louvres. When such an approach is taken normally the ventilation openings are dictated by the size of the generators.


HLS DSE-JV have submitted as early as 2014 RFIs outlining concerns regarding the adequacy of the ventilation openings and sizing of Generator rooms in the basements.

On the 25 March 2015, we were instructed to proceed with the purchase of additional fans from Systemaire. We issued the order request on the ..... and the order placed on the ......  without formal approval of the amounts in order to speed up the purchase. This affected the commissioning of the generators.

\chapter{Transformer Room Ventilation}

\subsection{Background}

\subsection{Design Errors}










