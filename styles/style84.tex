%\makeatletter\@runinheadtrue\makeatother
\def\puttilde{%
   \HHHUGE\mdseries\raisebox{20pt}{\mbox{\HUGE\char`\~}%
   }}
\cxset{style84/.style={
 name=,
 numbering=arabic,
 number font-size=HHHUGE,
 number font-weight=mdweight,
 number font-family= rmfamily,
 number font-shape=upshape,
 number padding-left=0pt,
 number padding-right=10pt,
 number before=\puttilde\kern0pt,
 number after=\puttilde,
 number dot=,
 number position=rightname,
 number color=black!80,
 number float=center,
 number display=block,
 %chapter name
 chapter color=black!80,
 chapter font-size= Large,
 chapter font-weight=bfseries,
 chapter font-family=rmfamily,
 chapter before=,
 chapter spaceout=none,
 chapter after=\par,
 chapter margin left=0cm,
 chapter margin top=0sp,
  %chapter title
 title font-family=rmfamily,
 title font-color=black!80,
 title font-weight=bold,
 title font-size=HUGE,
 chapter title align=center,
 title display=block,
 title margin-left=0cm,
 title margin bottom=2cm,
 title margin top=10pt,
 chapter title align=centering,
 chapter title text-align=center,
 chapter title width=\textwidth,
 title before=,
 title after=,
 title beforeskip=,
 title afterskip=,
 author block=false,
 section font-size=\Large,
 section font-weight=bfseries,
 section indent=0pt,
 epigraph width=\dimexpr(\textwidth-2cm)\relax,
 epigraph align=center,
 epigraph text align=center,
 section font-weight=mdseries,
 epigraph rule width=0pt,
 header style=plain}}

\cxset{style84}

\chapter[Template 84]{MACMILLAN’S PACT\\ {\huge Eisenhower Deal Frees Makarios}}
\thispagestyle{plain}

On April 1, 1955, the crash of dynamite and the chatter of machine guns ended the colonial peace of Cyprus. Depending from which angle it was viewed this was the beginning of a heroic fight for Independence or the beginning of Terrorism. 

\begin{figure}[ht]
\centering
\includegraphics[width=0.9\textwidth]{macmillan}
\caption{Style 84 spread.}
\end{figure}



\section{Template 84}



\cxset{chapter opening=anywhere}

\example
\begin{verbatim}
\cxset{style84,
   chapter opening=anywhere,
  }
\end{verbatim}


% reset settings for other chapters
\cxset{chapter opening=anywhere,
          chapter toc=none,
          section font-weight=bold,
          section afterskip=1pt plus0.5em minus0.5em}
 
\chapter{THE CRAP OF THE DOCUMENT\\ {\LARGE How Graphic Designers Apply the Science }}    
  
Graphic designers have studied legibility factors and have used the science as
a basis on which to construct textual design principles. Graphic designers, like
attorneys, are paid to create persuasive documents that maximize
comprehension and retention of the printed material. In light of the legibility
studies discussed earlier, the methodology of large-scale design in a purely
textual document becomes relatively easy to comprehend. Graphic designers
refer to four major elements in a document’s structure, which you can
remember by Robin William’s acronym “CRAP”: 93

 \noindent Contrast\\
 Repetition\\
 Alignment\\
 Proximity
 
The acronym “CRAP” is wonderfully memorable and even parallels legal
writing’s beloved “CRAC,”94 but it makes little sense to talk about the four
elements in that order. Instead, for ease of comprehension, this section is
organized as follows:

 Contrast\\
 Proximity\\
 Alignment\\
 Repetition
 
\section{Proximity: Keep related items related in layout}
 
Because writers want their readers to be able to understand the hierarchy
when they look at the document, as well as when they read it, writers must also
consider proximity as a design principle. Proximity is important because
aligning items on the page “creates a stronger cohesive unit.”100 The science
indicates that too many fixation pauses create a more difficult document; thus,
the writer should strive for more visual uniformity. For example, a heading
that is followed by too many vertical spaces (hard returns) will create too
many fixation pauses and a less legible document.101
 Proximity, in other words, provides organization.102 Placing things closer
together denotes relationship whereas the opposite is true when items are
spaced apart. 


\section{Repetition}                

Repetition is about being consistent and uniformity throughout the document’s overall design. 
 If a heading
that is numbered I. (Roman numeral) appears in a certain size and type, then
the next Roman numeral should appear in the same size and type. Spacing
between headings should remain consistent throughout. Chunking styles
should also repeat. Readers crave consistency because it helps organize the
information and unify the hierarchy.  This is achieved automatically using \latexe and that
is one could say the reason of its existence.


Consistency is also required in the body text of the document. If the second level headings
are all lowercased then all headings should be the same. These rules belong to a house
style manual rather than the code, however the code can enforce such rules by means
of using the |capitalization| keys. For example to capitalize only the first letters of a title:

\begin{scriptexample}{}{}
\begin{verbatim}
\cxset{chapter title capitalization=ucfirst}
\end{verbatim}
\end{scriptexample}

In English many words such as |and,the| are not capitalized. In more case you might be better off
doing all these manually and just use |capitalization=none|. 

                                                             
%https://books.google.ae/books?id=4dotrHSZLGUC&printsec=frontcover&dq=indiana+university+press+freedom+cyprus&hl=en&sa=X&ei=PhgHVYXTGdWxaeKvguAJ&ved=0CCgQ6AEwATgo#v=onepage&q&f=false