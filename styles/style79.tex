\makeatletter\@specialfalse\makeatother
\def\seminole{%
  \vfill\vfill
  \vbox{%
  \fboxrule1pt\fboxsep1pt% 
  \par\leavevmode
      \fbox{\hbox to \textwidth{\hfill\includegraphics[width=\dimexpr(\textwidth-5pt)]{seminole-03}\hfill}}%
  }%   
  %
\parbox{\textwidth}{{\bfseries\large\raggedright
The Seminole leader Osceola defies his\\
white opponents with uncompromising\\
style in signing their proposed treaty.
}}
\thispagestyle{plain}
\newpage
\vspace*{38pt}
}%
 
     
\cxset{style79/.style={
chapter opening=left,
toc image =,
 name=,
 numbering=Roman,
 number font-size= HUGE,
 number font-family=rmfamily,
 number font-weight= bfseries,
 number before=,
 number dot=,
 number after=\hfill\vskip1.5pt\hrule\vskip1.5pt,
 number position=rightname,
 number before=,
 chapter font-family= \sffamily\upshape,
 chapter font-weight=mdseries ,
  chapter font-size= Large,
 chapter color=black!90,
 chapter before=,
 chapter margin left=0pt,
 chapter margin top=0pt,
 chapter spaceout=none,
 chapter after=,
 number color=black!90,
 %
 title margin top=0pt,
 title margin bottom=1sp,
 title margin left=1sp,
 chapter title align=raggedright,
 chapter title width=\textwidth,
%
 title before=,
 title after=\par\offinterlineskip
                 \seminole\par,
 title font-family= rmfamily,
 title font-color= black,
 title font-weight= bfseries,
 title font-size= HUGE,
 title font-shape=upshape,
 title spaceout=none,
 header style= plain,
 section font-size=LARGE,
 section color=black,
 section numbering prefix=,
 section numbering=none,
 subsection numbering=none,
 subsection color=black,
 section indent=0pt,
 epigraph width=0.55\textwidth,
 epigraph align=center,
 epigraph rule width=0pt,
 epigraph text align=center,
 epigraph font-size=\large,
 epigraph source align=center,
 }}

\cxset{style79}

\chapter[Template 79]{PEPSI COLA AND WESTERNS}
\thispagestyle{plain}

\large
\renewcommand{\LettrineFontHook}{\gyre}

\lettrine{O}{}nly an endless stream of spagetti cow-boy movies can imprint into a child’s mind that Indian’s
were and are the enemy. As kids our Sunday mornings were very predictable. Pepsi-Cola who at the time were
at war with Coca Cola were sponsoring each Sunday movies for us. At the time cold drinks came in bottles and the 
caps were to be collected and used as tickets. Now the movies were always Spaghetti Westerns interspersed with
the odd World War II hero fighting the Germans. 
\begin{figure}[ht]
\centering
\includegraphics[width=0.9\textwidth]{seminole}
\end{figure}
The typical Spaghetti Western team was made up of an Italian director, Italo-Spanish[4] technical staff, and a cast of Italian, Spanish, German and American actors, sometimes a fading Hollywood star and sometimes a rising one like the young Clint Eastwood in three of Sergio Leone's films.

We sort of liked the Westerns and there was also now and then a Ciccio and Franki movie. Not to content with Pespi Cola and Spaghetti Westerns, before the movie started there was always a news reel in Greek with mostly 
news about the then Junta’s achievements. One football field for every village was very prominent and there was
also always talk about the enemies. A kid’s mind is a very congested field and like we are very receptive to learning languages our tens and teens can influence our later lives, if a kid is not careful.

I don’t know how we managed but we outsmarted all these bigots and business men thanks to some unusual series of events that accidentally were all stringed together by some higher providence. At the time my mother
enrolled me to a highschool which was called the \emph{The American Academy}. Now at the Academy there
was another War for the Child’s Mind. The school was run by well intentioned missionaries from the United States
who gave us daily doses of a good Protestant teaching. Now and I think this is the important part---living in a society of Orthodox Christians, that socialized with Turks and there were also Mosques all over the place, made these young human minds \emph{query everything}. 

I never made it to America and never lived the American Dream. The Dream did take my sister away, after the Turks invaded the Island and I am happy for her as she has had a good life. My brother succumbed and to this day his is an \emph{Evangelist} and I am also happy for him, as he has made hundreds of friends all over the place via the Church. One thing I absorbed was that religion is about cultivating the good part of human nature and rejecting the stuff that hurts others. I am probably what one would label as a Christian Buddhist Atheist, if there is such a 
thing. Another thing the Academy told me was that that life is full of arse holes and you need to accept it and handle it. 

\section{Template 79}

\emph{Indian Wars} by Robert Marshall Utley, Wilcomb E. Washbur is an unimpressive history of the Indian Wars. As it has an unusual style I have included it in the selections.
Template 79 needs both the right size images at the chapter opening page, as well as care in selecting fonts


\subsection{Lower headings and images}

\begin{figure}[ht]
\centering
\includegraphics[width=0.9\textwidth]{seminole-04}
\end{figure}

\begin{verbatim}
\cxset{style79}
\end{verbatim}







