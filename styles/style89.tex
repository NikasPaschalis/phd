\cxset{style89/.style={
 chapter opening=any,
 name=,
 numbering=arabic,
 number font-size=LARGE,
 number font-weight=mdseries,
 number font-family=sffamily,
 number font-shape=upshape,
 number before=,
% number after=\vskip2pt\drawrule[solid]{\textwidth-2cm+1.5pt}{1.5pt}%
%                       \drawrule[solid]{2cm}{5pt}%
%                       \vskip0pt,
% number after=,
 number dot=,
 number position=rightname,
 number color=black!80,
 number padding=10pt,
 number shape=rectangle,
 number border-width=5pt,
 number display=block,
 number float=right,
 %chapter 
 chapter color=black!80,
 chapter font-size= LARGE,
 chapter font-weight=mdseries,
 chapter font-family=sffamily,
 chapter before=,
 chapter spaceout=soul,
 chapter after=,
% chapter margin-left=1cm,
 chapter margin top=1sp,
 chapter padding-left=0pt,
 chapter padding-right=0pt,
 chapter border-left-width=0pt,
 chapter border-top-width=0pt,
 chapter border-right-width=0pt,
 chapter border-bottom-width=0pt,
  %chapter title
 title font-family=sffamily,
 title font-color=black!80,
 title font-weight=mdseries,
 title font-size=LARGE,
 title spaceout=none,
 chapter title align=none,
 title margin-left=0cm,
 title margin bottom=30pt,
 title margin top=30pt,
 title border-left-width=0pt,
 chapter title align=left,
 chapter title text-align=raggedleft,
 chapter title width=\textwidth,
 title before=,
 title after=,
 title beforeskip=,
 title afterskip=,
 author block=false,
 author names=Rebecca Moore,
 section font-size=large,
 section numbering=none,
 section font-family=sffamily,
 section font-weight=bfseries,
 section spaceout=soul,
 section indent=0pt,
 section align=left,
 section color=black,
 epigraph width=\dimexpr(\textwidth-1cm)\relax,
 epigraph align=left,
 header style=empty}}



\cxset{style89}

\chapter[template 89]{ROBOT BELIEFS}

The book is set in Chaparral Pro. One of themost important things to say about beliefs is that they are
(or at least should be) tentative and changeable. Cognitive scientists distinguish various kinds of knowledge.
Some beliefs are termed “declarative” because beliefs are stated as declarative sentences. No one really understands how beliefs are represented in the brain. The philosopher and cognitive scientist Jerry Fodor proposes that they are represented as sentence-like forms, in a “language of thought” that he calls \emph{mentalese}.
\lorem\lorem

\begin{figure}[ht]
\centering
\includegraphics[width=0.45\textwidth]{beliefs}%
\includegraphics[width=0.45\textwidth]{beliefs-01}
\caption{Style 89 spread.}
\end{figure}

\emph{Understanding Beliefs}, Nils J. Nilsson, MIT Press (2010)

Functioning in a society is based mostly on beliefs. Our beliefs constitute a large part of knowledge of the world.


\cxset{chapter opening=anywhere}

\chapter{The Brain’s Model of Reality}

The physicist David Deutsch wrote:
\cxset{quotation left margin=1cm,
          quotation right margin=1cm,
          quotation font-size=\large}
\begin{quotation}
Reality is out there: objective, physical and independent of what we believe about it. But we never experience that relality directly. Every last scrap of our external experience is of virtual reality. And every scrap of our knowledge of the non-physical worlds of logic, mathematics and philosophy, and of imagination, fiction, art and fantasy---is encoded in the form of programs for the rendering of those worlds on our brain’s own virtual-relaity generator.\ldots So it is not just science---reasoning about the physical world---that involves virtual reality. All reasoning, all thinking and all external experiences are forms of virtual reality.
\end{quotation}

\chapter[Chikungunya Fever]{CHIKUNGUNYA\\FEVER}

\cxset{chapter opening=any}



%Reset it problematic in other places
\cxset{title margin-left=0pt}
%https://books.google.ae/books?id=Z_PvAwAAQBAJ&printsec=frontcover&dq=MIT+press&hl=en&sa=X&ei=_LQOVePpBoXjaoubgNgM&ved=0CFYQ6AEwCDg8#v=onepage&q=MIT%20press&f=false