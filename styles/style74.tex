\makeatletter\@specialfalse\makeatother
\newfontfamily\gyre{TeX Gyre Termes}[
UprightFeatures={Color = 000000,
SmallCapsFeatures = {Color=000000}},
ItalicFeatures={Color = 2244FF,
SmallCapsFeatures = {Color=112299}},
BoldFeatures={Color = 000000,
SmallCapsFeatures = {Color=000000}},
BoldItalicFeatures={Color = 888844,
SmallCapsFeatures = {Color=444422}},
Scale=1.35
]

\cxset{style74/.style={
chapter opening=any,
 toc image =,
 name=CHAPTER,
 numbering=WORDS,
 number font-size= large,
 number font-family=sffamily,
 number font-weight= mdseries,
 number before=\kern1.5em,
 number dot=,
 number after=%
\hbox to 0pt{\hskip.5\textwidth\vrule width8cm height1pt depth0pt\relax
                                          \vrule width5pt height 1pt depth4pt}%
                                          \vskip2pt,
 number position=rightname,
 number border-width=0pt,
 chapter font-family= sffamily,
 chapter font-weight=mdseries ,
 chapter font-size= large,
 chapter color=black!90,
 chapter before=\par,
 chapter margin left=0.5\textwidth,
 chapter spaceout=soul,
 chapter after=,
 number color=black!90,
 %
 title margin top=15pt,
 title margin bottom=60pt,
 title margin-left=0.5\textwidth,
 chapter title align=none,
 chapter title width=0.5\textwidth,
 chapter title text-align=left,
 chapter border-width=0pt,
 %
 title before=\par,
 title after=\par,
 title font-family= gyre,
 title font-color= black,
 title font-weight= mdseries,
 title font-size= Huge,
 title font-shape=scshape,
 title spaceout=none,
 title border-width=0pt,
 header style= plain,
 section font-size=LARGE,
 section color=black,
 section numbering prefix=,
 section numbering=none,
 subsection numbering=none,
 subsection color=black,
 section indent=0pt,
 }}



\cxset{style74}

\chapter[Template 74]{Silencing the\\ Leaders}
\label{style74}
\thispagestyle{plain}
\textbf{\huge O}n the morning of July 20, 1989, Aung San Suu Kyi (ahgn sahn soo chee) woke up to 
find hereself under house arrest.  The story of her arrest was narrated by Whitney Stewart in \emph{Aung San Suu Kyi: Fearless Voice of Burma.} 
This template (74) is inspired from this book and other than some minor variations
in fonts and spacing which I adjusted to make it suitable for a larger paper size, the template is very similar. This is
one of my favourite templates for non-fiction books.

\begin{figure}[ht]
\centering
\includegraphics[width=0.8\textwidth]{aung-01}
\end{figure}

For almost 22 years Aung San Suu Kyi was  kept a prisoner either under house arrest or in prison until 13 November 2010 when she was eventually released. 

The template was inspired by \emph{Vampire Nation} by Arlene Russo (2008). The book was published by Llewellyn Publications in the USA. The book was designed by Steffani Sawyer. One can only ponder that a disclaimer in the book states: “The activities in this book may be dangerous or even illegal. Please exercise responsibility. Remember that what works for some may not work for you. As you proceed, use care, caution, and common sense. Llewellyn will not be held responsible for personal actions taken in response to this book.”  Vampires and lawyers are always fascinating subjects!

\lorem\lorem\lorem

\section{Technical}

The challenging part of this template is to get the structure of the chapter head right, as well as to match the fonts.

\begin{verbatim}
\cxset{style74}
\end{verbatim}

The rule is actually two rules, one with a larger depth so that the square box at its end points downwards.
This is placed in a zero box so that we can overflow it into the margin. A similar approach is implemented for the rules in the running headers.

\subsection{Lower level headings}

The book does not have any headings, as each chapter narrates the story in a chronological sequence. As we needed them for this chapter, I have made them unnumbered and position at the left margin.


\cxset{chapter opening=anywhere}
\chapter{Why Vampires don’t Like Garlic}

When you feel bored and have nothing to do, run your thesis with this template and see how it looks and you will understand better the power of fonts on human perception. You thought you were unbiased did you? Would fonts affect the marks you give to a student?

%https://books.google.ae/books?id=Bz83hzRnri8C&printsec=frontcover&source=gbs_ge_summary_r&cad=0#v=onepage&q&f=false
\makeatother
