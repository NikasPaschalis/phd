%\makeatletter\@runinheadtrue\makeatother

\cxset{style95/.style={
chapter opening=any,
chapter border-bottom-width=1pt,
chapter padding=0pt,
chapter display=block,
chapter float=center,
chapter before = ,
 name= Chapter,
 numbering=arabic,
 number font-size=huge,
 number font-weight=mdseries,
 number font-family= rmfamily,
 number font-shape=scshape,
 number before=,
 number after content=\kern0.5em\raisebox{3.5pt}{\mbox{\vrule width0.7cm height1.2pt depth0pt}},
 number dot=,
 number position=rightname,
 number color=thelightgray,
 number border-top-width=0pt,
 number border-left-width=0pt,
 number border-right-width=0pt,
 number border-bottom-width=1pt,
 number border-style=solid,
 number padding=1.5pt,
 number padding-top=0pt,
 number padding-left=0pt,
 number padding-right=0pt,
 number padding-bottom=0pt,
 number float=none,
 number display=inline,
 %chapter name
 chapter float=center,
 chapter display=block,
 chapter color=thelightgray,
 chapter font-size= huge,
 chapter font-weight=mdseries,
 chapter font-family=rmfamily,
 chapter before content=\raisebox{3.5pt}{\mbox{\vrule width0.7cm height1.2pt depth0pt}}\kern0.5em,
 chapter spaceout=none,
 chapter after=,
 chapter margin left=0cm,
 chapter margin top=0sp,
 chapter padding=0pt,
 chapter padding-bottom=12pt,
 chapter border-bottom-color=black,
 chapter border-top-width=0pt,
 chapter border-right-width=0pt,
 chapter border-bottom-width=0pt,
 chapter border-left-width=0pt,
  %chapter title
 title font-family=rmfamily,
 title font-color=spot!50,
 title font-weight=mdseries,
 title font-shape=upshape,
 title font-size=Huge,
 chapter title align=none,
 title margin-left=0cm,
 title margin bottom=60pt,
 title margin top=0pt,
 title border-left-width=0pt,
 title border-top-width=1pt,
 title border-top-color=thelightgray,
 chapter title align=left,
 chapter title text-align=center,
 chapter title width=\textwidth,
 title before=,
 title after=,
 title beforeskip=,
 title afterskip=,
 title padding-top=30pt,
 author  names=Dr Yiannis Lazarides,
 author block=false,
 section font-size=\Large,
 section font-weight=bfseries,
 section indent=0pt,
 section beforeskip=20pt,
 section afterskip=20pt,
 epigraph width=\dimexpr(\textwidth-2cm)\relax,
 epigraph align=center,
 epigraph text align=center,
 section font-weight=mdseries,
 section align=center,
 epigraph rule width=0pt,
 header style=plain}}

\cxset{style95}
\parindent2em

\chapter[Template 95]{The Evolution of Co-operation}
\label{style95}
\thispagestyle{plain}
\pagestyle{headings}

\dropcap{A}{} fundamental question about all forms of  collective animal behavior is how they evolved through natural selection.  The theory of natural selection is grounded in the idea than those individuals exhibiting a behavior that provides them with a higher than average fitness pass their genes, and thus their particular behaviour, on to future generations.  David J.T. Sumpter’s book \emph{Collective Animal Behavior} is a good introduction to the field.  


{%
\centering
\includegraphics[width=0.45\textwidth]{evolution-of-cooperation}
\captionof{figure}{Style 95 spread.}
}

Animal groups vary from two magpies sitting on a branch to plagues of millions of locusts crossing the desert. 
Human settlements also show similar variety in size, from tiny villages to massive cities, with differences in size arising without large differences in the environment in which they were originally founded (Reed 2001).

Animal behavior provides a wealth of interesting and accessible examples
of collective phenomena and complex systems. Most people are
familiar with ant trails; cockroach aggregations; fish schools; bird migrations;
honeybee swarms; web construction by spiders; and locust marching,
even if they have not observed them personally. In these systems
there are two clearly defined levels of organization that we aim to link
together: the animal and the group. This clarity stands in contrast to
many other collective phenomena, such as protein interactions or ecological
webs, where it can be difficult to establish exactly on which level
to observe a system. Thus animal behavior provides much needed case
studies of how complex systems theory can be put into practice. 

\section{Technical}

You may need to set the fonts and spacing to your liking here. The main point was to introduce the property
|border-style=double|. This can be applied either selectively or globally to all elements. The border styles can be extended by the use of the property |number border-bottom-style-custom|, where the value is 
any valid CSS name. 

\cxset{chapter opening=anywhere}

\example
\begin{verbatim}
\cxset{style95,
   chapter opening=anywhere,
   number border-bottom-style=solid
  }
\end{verbatim}


% reset settings for other chapters
\cxset{chapter opening=anywhere,
          chapter toc=none,
          section font-weight=bold,
          section afterskip=1pt plus0.5em minus0.5em,
          number border-style=double,
          number border-bottom-width=3pt}
          
 \chapter[Template 95]{Information Transfer}         
          
 \dropcap{O}{}ne reason why animal groups are such a popular subject for scientific
study is the importance of social interactions in our own everyday experience.
Humans are inherently social animals, whose activities exhibit
many of the elements of co-operation and conflict found in other animal
societies. these social activities are extremely important to us: they determine
our economic welfare; they produce a great deal of emotional
turmoil, often providing the main reasons for whether we are happy
or not; they determine how we are governed and how we structure our
workplaces; and they even determine simple every day activities, such as
how long we have to wait in queues.

Can some of the techniques used to study collective animal behavior
be applied to understanding human societies? the answer is a qualified
“yes.” in narrowly defined social situations, such as in pedestrian movement
and spectator crowds, some of the techniques used to understand
collective animal behavior can be applied to humans. in wider situations,
such as consumer decision-making and the “evolution” of fads and fashions,
there could also be applications. recent studies have looked at how
our tendencies to buy particular items, find employment, and even commit
crime change with the behaviors of those around us. Many of the underlying
dynamics of these processes are similar to those seen in animal
groups and this book seeks to highlight how these similarities arise. 

\cxset{chapter before content=,
          number after content=,
          chapter opening=any}
 % https://books.google.ae/books?id=JwdOrSMmdkUC&pg=PA243&dq=evolution+of+co-operation+tit+for+tat&hl=en&sa=X&ei=fLAaVZS7A4LgaKjNguAF&ved=0CEgQ6AEwCA#v=onepage&q&f=false    
