\cxset{
 name={},
 numbering=none,
 number font-size=\HUGE,
 number font-family=\sffamily,
 number font-weight=\bfseries,
 number before={},
 number dot=,
 number position=rightname,
 chapter font-family=neutra,
 chapter font-weight=\normalfont,
 chapter font-size=Large,
 number after={\vskip1pt},
 chapter before={\vspace*{10pt}},
 chapter after={\par\vskip12pt},
 chapter color= black!90,
 number color=black!90,
 title beforeskip={\vskip10pt\par},
 title afterskip=\par,
 title before=\par,
 title after={\par},
 chapter title text-align=center,
 title margin bottom=30pt,
 title font-family=neutra,
 title font-color= black!90,
 title font-weight=bfseries,
 title font-size=huge,
 title font-color=black!90,
 chapter title width=\textwidth,
 chapter title align=centering,
% section numbering = arabic,
% section numbering prefix = \thechapter.,
 section indent=0pt,
 section font-size=Large,
 chapter afterindent=false}
 
 \cxset{section font-family=neutra,
          subsection font-family=neutra,
          subsection font-shape=upshape,
          subsubsection font-size=large,
          subsection align=flushleft}
\renewsubsection          
 \cxset{
  author block=true,
 author block format=\par\Large\aegean\begin{center},
 author block afterskip=\vspace*{30pt},
 author names=Karin Wahl-Jorgensen and Thomas Hanitzch\end{center}}
 
\chapter{INTRODUCTION TO STYLE 56 }

The looting of Iraq’s past during the Gulf War will be a permanent stain on humanity’s history. The story is described in a publication by the Oriental Intstitute of the University of Chicago \cite{looting}. 
\lipsum[3]

\cxset{chapter opening=anywhere}

 \cxset{
 author block=true,
 author block format=\par\medskip\hfill\minipage{0.7\textwidth}\Large\aegean\begin{center},
 author names={Donny George, Stony Brook University, with McGuire Gibson, Oriental Institute University of Chicago}\par\end{center}\endminipage\hfill\hfill}

\chapter{THE LOOTING OF THE IRAQ MUSEUM COMPLEX}

This is a large size book $9\times11.50$ inches, so it is difficult to reproduce here the actual feel of the book. The book also depends on specific fonts to achieve the typographical rhythm with the main head set in |neutra|. The neutra typeface was designed by Christian Swartz and is widely used, with some\footnote{\protect\url{http://blog.threestepsahead.com/rants/neutraface-is-the-new-helvetica/}} even claiming that it is becoming the new helvetica.

The book has a good style for images and this was discussed briefly at \autoref{looting}. There is also a good interplay between images and text.



\begin{figure}[p]
\centering
\includegraphics[height=.5\textheight]{oriental-01}
\caption{Wrapped Illustrations are set on a background.}
\end{figure}

\cxset{author block=false,
          chapter opening=any}



          
\section{LOWER LEVEL HEADINGS}

The lower level headings are set centered and in bold. I have set the subsections centered, but they can also be set flush left without distracting from the good looks of the template.

\subsection{OTHER SETTINGS}
\lorem \lorem

\section{IMAGES}




