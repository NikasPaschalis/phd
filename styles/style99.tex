%\makeatletter\@runinheadtrue\makeatother
\makeatletter
\@specialtrue
\definecolor{pythagoras}{rgb}{0.9,0.9,0.9}
\newcommand\doublechaptertemplate[2][]{%
\def\doublenumbers##1{%
\bgroup             
\parindent0pt
\noindent
\begin{tcolorbox}[arc=0pt, outer arc=0pt,colback=pythagoras,colframe=pythagoras, 
boxsep=0pt, bottom=0pt, top=2pt, left=3pt, borderline=0pt]
   \mbox{{\color{white}\bfseries\sffamily\HHHUGE\@arabic\c@chapter}}% 
   \hspace*{-2em}\mbox{\color{thelightgray}\bfseries\sffamily\LARGE\so##1}}
 \def\tempa{\expandafter\words@cx{\expandafter\@arabic\c@chapter}}
\doublenumbers{\tempa}%
\end{tcolorbox}
\vspace*{12pt}
\setchaptertitlefont\language-1\raggedright #2\par%
\vspace*{48pt}
\egroup
}

\cxset{style99/.style={
custom=doublechaptertemplate,
chapter opening=any,
chapter border-width=1pt,
chapter padding=0pt,
chapter shape=star,
chapter display=block,
chapter float=left,
 name=CHAPTER,
 numbering=arabic,
 number font-size=LARGE,
 number font-weight=mdweight,
 number font-family= rmfamily,
 number font-shape=,
 number before=,
 number after=,
 number dot=,
 number position=rightname,
 number color=black!80,
 number border-left-width=1pt,
 number border-right-width=1pt,
 number padding=1em,
 number display=inline,
 chapter display=inline,
  %chapter name
 chapter color=black!80,
 chapter font-size= LARGE,
 chapter font-weight=bfseries,
 chapter font-family=rmfamily,
 chapter before=,
 chapter spaceout=soul,
 chapter after=\par,
 chapter margin left=0cm,
 chapter margin top=0sp,
  %chapter title
 title font-family=rmfamily,
 title font-color=black!80,
 title font-weight=mdseries,
 title font-size=Huge,
 chapter title align=none,
 title margin-left=0cm,
 title margin bottom=2cm,
 title margin top=30pt,
 chapter title align=left,
 chapter title text-align=left,
 chapter title width=\textwidth,
 title before=,
 title after=,
 title beforeskip=,
 title afterskip=,
 author block=false,
 section font-size=\Large,
 section font-weight=bfseries,
 section indent=0pt,
 epigraph width=\dimexpr(\textwidth-2cm)\relax,
 epigraph align=center,
 epigraph text align=center,
 section font-weight=mdseries,
 section align=center,
 epigraph rule width=0pt,
 header style=plain}}
\cxset{style99}
\cxset{custom=doublechaptertemplate,chapter margin top=0pt, title margin-top=-50pt,
          title font-family=sffamily, title font-weight=bfseries}

\chapter[Template 99]{Ancient Greek Number Theory}


\thispagestyle{plain}
\pagestyle{headings}

The Spiritual-Industrial Complex: America's Religious Battle against Communism in the Early Cold War
 By Jonathan P. Herzog. \lipsum[1]

\begin{figure}[ht]
\centering
\includegraphics[width=0.9\textwidth]{pythagoras}
\caption{Style 99 spread.}
\end{figure}


\section{Technical}

So how are we to develop such a heading from our standard CSS like properties? Unfortunately at this point once
again we are to delve deeply into a custom made template. However, the nut and bolts are available to minimize the amount of code we need to write. 

\begin{texexample}{Developing the Heading}{}
\begin{tcolorbox}[arc=0pt, outer arc=0pt]
\thechapter  
\cxset{numbering=WORDS}
\thechapter
\end{tcolorbox}
\end{texexample}

We observe that we can display the number, using our standard techniques of setting properties. We just need to set the keys twice. Font sizing and positing can come in later.

\begin{texexample}{Developing the Heading}{}
\begin{tcolorbox}[arc=0pt, outer arc=0pt]
\cxset{number font-size=HHUGE,
          numbering=arabic}
\thechapter\par
  
\cxset{chapter numbering=words,
          number font-size=LARGE}
\thechapter\par
\end{tcolorbox}
\end{texexample}



\begin{tcolorbox}[arc=0pt, outer arc=0pt,colback=pythagoras,colframe=pythagoras]
\makeatletter
\def\doublenumbers#1{%
   \mbox{{\color{white}\bfseries\sffamily\HHHUGE\@arabic\c@chapter}}% 
   \hspace*{-2.5em}\mbox{\color{thelightgray}\bfseries\sffamily\LARGE\so#1}}
\doublenumbers{\thechapter}
\makeatother
\end{tcolorbox}




 
