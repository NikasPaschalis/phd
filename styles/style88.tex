\cxset{style88/.style={
 name=,
 numbering=arabic,
 number font-size=HHHUGE,
 number font-weight=bfseries,
 number font-family=rmfamily,
 number font-shape=upshape,
 number before=,
 number after=,
 number dot=,
 number position=rightname,
 number float=right,
 number display=block,
 number color=thelightgray,
 number shape=star,
 %chapter 
 chapter shape=circle,
 chapter color=black!80,
 chapter font-size= Huge,
 chapter font-weight=bfseries,
 chapter font-family=sffamily,
 chapter before=,
 chapter spaceout=soul,
 chapter after=,
% chapter margin-left=1cm,
 chapter margin top=1sp,
 chapter padding-left=0pt,
 chapter padding-right=0pt,
 chapter border-left-width=0pt,
 chapter border-top-width=0pt,
 chapter border-right-width=0pt,
 chapter border-bottom-width=0pt,
  %chapter title
 title font-family=rmfamily,
 title font-color=black!80,
 title font-weight=bold,
 title font-size=huge,
 title spaceout=none,
 chapter title align=none,
 title margin-left=5cm,
 title margin bottom=30pt,
 title margin top=10pt,
 chapter title align=left,
 chapter title text-align=raggedright,
 chapter title width=0.6\textwidth,
 title before=,
 title after=,
 title beforeskip=,
 title afterskip=,
 author block=false,
 author names=Rebecca Moore,
 section font-size=large,
 section numbering=none,
 section font-family=sffamily,
 section font-weight=bfseries,
 section spaceout=soul,
 section indent=0pt,
 section align=left,
 section color=black,
 epigraph width=\dimexpr(\textwidth-1cm)\relax,
 epigraph align=left,
 header style=empty}}



\cxset{style88}

\chapter[template 83]{An Overview of\\ Newly Emerging\\ Viral Plagues: The\\ Hemorrhagic Fevers}
\large
The Red Queen hypothesis, which now offers one of the most promising
explanations of sex, was first suggested by Leigh and Van Valen in 1973. Van Valen discovered that
the probability of marine organisms becoming extinct at any one time bears no relation to how long it has already survived.

\begin{figure}[ht]
\centering
\includegraphics[width=0.9\textwidth]{viruses-01}
\caption{Style 88 spread.}
\end{figure}

It is a humbling thought that the struggle for existence never gets any easier: however well adapted an animal may become, it still has the same chance of extinction as a newly formed species. Van Valen was reminded of the Red Queen in Alice in Wonderland, who ran fast with Alice only to stand still. 

The application of this theory to the problem of the maintenance of sex is captured by the phrase “genetics arm race”. A typical animal must constantly run the genetic gauntlet   of being able to chase its prey, run away from predators and resist infection from parasites. Parasite infection in particular means that the parasite and its host are locked in an “evolutionary embrace” (Ridley, 1993). Each reproduces sexually in the desperate hope that some combination will gain a tactical advantage in an attack or defence. 

Further support for the parasite exclusion theory comes from the fact that genes that code for the immune response---the majorhistocompatibilty complex (MHC)---are incredibly variable. This is consistent with the idea that variability is needed to keep an advantage over parasites.

Lassa fever virus, Hantavirus, and Ebola virus---all equally lethal infectious agents but members of different viral families---share the ability to cause hemorrhagic fever. Once the person is infected with the viruses, the victim soons suffers profuse breaks in small bloods vessels, causing blood to ooze from the skin, mouth, and rectum. Internally, blood flows into the leural cavity where the lungs are lkocated, into the pericardial cavity surrounding the heart, into
the abdomen, and into organs like the liver, kidney, heart spleen and death. Once it strikes hemorrhagic fever is relentless and devastating.

Another alarming factor is that the number of individuals that are susceptible to these viruses has swelled markedly due to the ever growing population that are on immunosuppressive drugs or are infected by such pathogens susch as HIV, measles virus, malaria and tuburculosis, all of which suppress the immune system. Since 1969, thirty-nine new pathogens have emerged including SARS, HIV, and Ebola. 

\cxset{chapter opening=anywhere}

\chapter{Smallpox}

Smallpox killed nearly 300 million people in the twentieth century alone. The story of this most universally feared disease is told by Oldstone in \emph{Viruses, Plagues, and History: Past, Present and Future} and from whichthe design of this template (88) was inspired. There are many books on the subject of plagues but Scipps 
Research Institute virologist Oldstone, is less passionate than Laurie Garrett's The Coming Plague, being more or less a prosaic, factual account of viral plagues in recorded history. Oldstone provides background chapters on the nature of viruses and the ways the body's immune system combats them, then launches into a detailed description of the plagues themselves.

He devotes a fair amount of space to smallpox, following its depredations from ancient Greece all the way up to the work of the Centers for Disease Control in Atlanta. Modern strategies have led to the total eradication of smallpox-- a major success, given that the disease killed 300 million people in the 20th century. Other success stories cited by Oldstone include the treatment of yellow fever, measles, and polio, although the lack of immunization programs still racks up enormous tolls. The World Health Organization estimates that in the 1980s and early '90s as many as 2.5 million children died of measles annually. The second half of the book deals with such unconquered viral diseases as Lassa fever, Ebola, Hantavirus, and AIDS. The role of urbanization and air travel in spreading viruses to large pools of susceptible people, the unpredictable nature of viral genetics and evolution, and the impact of politics on medicine are among the variables Oldstone cites to remind us that as a species we are always vulnerable. Interestingly, while the author loudly condemns governments and corporations for suppressing information, he is silent on the rivalries and contentions among scientists themselves: nary a word on Salk vs. Sabin, for example, nor Gallo vs. Montagnier. A bit of the old-boy network? In sum, a somewhat sanitized, professorial account of the ever-fascinating legacy of viral disease on human history.

\chapter[Chikungunya Fever]{CHIKUNGUNYA\\FEVER}

The illness was named by local Tanzania villagers and it means “bending up or contorting” similar to the way a leaf’s edges curl as it dries and becomes brittle. Because of the severe pain induced by the reaction of the body to the virus the patirnts assume a stiff, bent posture to limit movement of painful joints. The virus is caused by a mosquito-borne virus in the Semliki Forest complex of the \emph{Alphavirus} genus of the family Togaviridae. 

\begin{figure}[htbp]
\includegraphics[width=\textwidth]{reunion-pest-control}
\caption{Insecticide spraying of mosquitoes on Reunion Island}
\end{figure}

The virus is actively studied by epidemiologists, virologists, entomologists, pathologists, parasitologists and many other scientists. Details can be found in Zoonoses and medical  handbooks and many countries carry out health campaigns to make people aware of the symptoms in an attempt to limitits spread.

Since 2005, there have been 1.9 million cases in India, Indonesia, Thailand, the Maldives and Burma also known as Myanmar, according to the World Health Organization.

There is currently no cure or vaccine and experts advise that the best way to stop it is to avoid being bitten in the first place. The good news? Once you've been infected you are immune and you don't have to suffer again.\footnote{\protect\url{http://www.bbc.com/news/health-26724645}}


\chapter{Lymphatic Filariosis}

\begin{figure}[htbp]
\includegraphics[width=0.5\textwidth]{elephantiasis}
\caption{Dramatic photo taken by A.Edgar of a Tahitian man, with extreme case of elphantiasis. The disease involved the left arm, left leg, and scrotum; the photograph was taken during World War II. From \emph{Medical and Veterinary Entomology}, \emph{editor} editor Gary R. Mullen, Lance A. Durden (2009), Academic Press. }
\end{figure}




%Reset it problematic in other places
\cxset{title margin-left=0pt}
%https://www.kirkusreviews.com/book-reviews/michael-ba-oldstone/viruses-plagues-and-history/

%https://books.google.ae/books?id=2XbHXUVY65gC&pg=PA219&dq=history+of+ebola&hl=en&sa=X&ei=oosOVa32F4vgaL26gMAI&ved=0CDcQ6AEwBQ#v=onepage&q&f=false

