\cxset{style69/.style={
 name=CHAPTER,
 numbering=arabic,
 number font-size=Large,
 number font-weight=mdweight,
 number font-family=rmfamily,
 number before=\kern0.5em,
 number after=\kern.5cm\raisebox{1pt}{\hbox{\textbullet}}\hfill\hfill\par,
 number dot=,
 number position=rightname,
 number color=black!80,
 %chapter name
 chapter color=black!80,
 chapter font-size= Large,
 chapter font-weight=mdseries,
 chapter font-family=sffamily,
 chapter before=\hfill\raisebox{1pt}{\hbox{\textbullet}}\kern.5cm,
 chapter spaceout=none,
 chapter after=,
 chapter margin left=0cm,
 chapter margin top=1sp,
  %chapter title
 title font-family=rmfamily,
 title font-color=black!80,
 title font-weight=bold,
 title font-size=LARGE,
 chapter title align=none,
 title margin-left=0cm,
 title margin bottom=2cm,
 title margin top=1sp,
 chapter title align=center,
 chapter title width=\textwidth,
 title before=,
 title after=,
 title beforeskip=,
 title afterskip=,
 author block=false,
 section font-size=\Large,
 section font-weight=bfseries,
 section indent=0pt,
 epigraph width=\dimexpr(\textwidth-1cm)\relax,
 epigraph align=left,
 section font-weight=\normalfont,
 header style=empty}}

\cxset{style69}

\chapter[Template 69]{PUBLISHING BEGINS IN BERKLEY}

Academic printing in the United States does not have a long history compared to European institutions such as the University of Oxford Press or the University of Cambridge Press. The book \emph{The University of California Press: The Early Years, 1893-1953,} by Albert Muto narrates the story of the publishing and printing ventures of the University of California. The book besides the many facts that it provides was also a springboard for my research into how books are designed. Although none mentioned the tracing of its history led me to search for titles from every decade of its printing history. It can be seen that the designs eveolved over the years in conjuction with changing technologies and fashion trends. 

\begin{figure}[ht]
\centering
\includegraphics[width=0.9\textwidth]{ashkenazi}
\caption{Style 68 spread.}
\end{figure}

There are a number of ways to get the two bullets that frame the chapter