\documentclass{article}
\usepackage{fontspec}
\usepackage{underscore,amsmath,stackengine,luacode,xcolor,listings,
            filecontents,lipsum,graphicx,listings,caption}
\begin{document}
\newfontfamily\hiero{NotoSansEgyptianHieroglyphs-Regular.ttf}
%\newfontfamily\aegyptusi{NewGardiner.ttf}
\newfontfamily\aegyptus{JSeshFont.ttf}
\newfontfamily\arial{Arial Unicode MS}
\parindent0pt
\raggedright


\begin{luacode*}
    local h = {}
          h = dofile("hiero.lua")
    local options = {style="block",
                     echo=true,
                     direction="RL",
                     size = "\\Huge",
                     color = "green",
                     headings = "captionof{figure}"  -- section/tablecaption/figurecaption
                     }
   -- prints full symbol list
   h.printgardiner(t,options)

   tex.print("\\par")
   local options = {style="block",
                     echo=true,
                     heading="\\par",
                     direction="RL",
                     color = "teal",
                     scale = 8}

   h.printhierochar("aegyptus","1317D",options)
   h.printhierochar("aegyptus","13000",{direction="RL",
                                        color = "teal",
                                        scale = 8})
   h.printhierochar("aegyptus","13003",{direction="LR",
                                        color = "teal",
                                        scale = 1})
   h.parseMdC([[M23-X1-R4-X8-Q2-D4-W17-R14-G4-R8-O29-
               V30-U23-N26-D58-O49-Z1-F13-N31-V30-N16-
               N21-Z1-D45-N25!]])

   tex.print("\\par")
   h.printgardinercat("B")

   tex.sprint()

\end{luacode*}

\newcommand\hierochar[2][direction = "LR",
                         color     = "teal",
                         scale     = 1]{% 
               \luaexec{
                h = h or {}
                h = require("hiero.lua")  
                h.parseMdC(#2,{#1})}}
               
\newcommand\printhierochar[3][direction = "LR",
                              color     = "teal",
                              scale     = 4]{% 
               \luaexec{
                h = h or {}
                h = require("hiero.lua")  
                h.printhierochar(#2,#3,{#1})}}

This file just tests the various commands available for manipulating hieroglyphics. We tried to 
generalize the commands, so they can be re-used for other type of hieroglyphics.

\hierochar{"A1-A2-A3!"}

\centering 

\def\options{direction = "LR",
             color     = "teal",
             scale     = 7}

\def\fontname{"aegyptus"}

\def\hierochar#1{\printhierochar[\options]{\fontname}{#1}}

\hierochar{"13051"}
\kern-4mm
\hierochar{"13003"}


\begin{luacode*}
local fontpath = kpse.show_path("opentype fonts")

tex.print("\\begin{verbatim}",fontpath,"\\end{verbatim}")


database = {}
local storeinfo

function explorefonts (dir)
   if not lfs.isdir(dir) then
      return
   end
   for name in lfs.dir(dir) do
   if name~="." and name ~=".." then
      local file = dir .."/"..name
      local attr = lfs.attributes(file)
      if attr then
      if attr.mode == "file" and
        name:lower():match("/.[to]tf$") then 
          storeinfo(file, name)
          tex.print(file)
      elseif attr.mode == directory then
          tex.print(file)
      end
    end
   end
end
end

fontpath = fontpath:match(":")
           and fontpath:explode(":")
           or fontpath:explode(";")

for _,dir in ipairs(fontpath) do
     dir = dir:gsub("^!!","")
     explorefonts(dir)
end


local f = fontloader.open("c:/windows/fonts/NotoSansEgyptianHieroglyphs-Regular.ttf")
    fonttable = fontloader.to_table(f)

tex.print(fonttable.design_size*6553.6)

tex.print(fonttable)
local size=1
size = size * tex.sp("10pt") / -1000
tex.print(size)
\end{luacode*}

\raggedright

We first load the table and convert the info to a Lua table. We use a font provide with
an article by Paul called TestLibertine.otf. 

\includegraphics[width=0.6\textwidth]{./images/fontforge.jpg}

\begin{luacode*}
local z={}
tf=fontloader.to_table(fontloader.open("c:/windows/fonts/andlso.ttf"))

-- we sort the keys to create a table
-- important keys to us are tf.glyphs

for k,v in pairs (tf) do
   tex.print(k,type(v),"\\par")
   table.insert(z, k)
end
table.sort(z)
for k,v in pairs (z) do
   local s = tf[v]
   tex.print(z[k],type(s),"\\par")
end

for k,v in pairs (tf.map) do
   tex.print(k,tf.map.enc_name,type(tf.map.enc),"\\par ")
end

-- the encoding
for k,v in pairs (tf.map.enc) do
   tex.print(k,type(k),v, "\\par ")
end

tex.print("Units per em",tf.units_per_em,"\\par",
          "version", tf.version,"\\par",
          "glyph count", tf.glyphcnt,"\\par",
          "design size", tf.design_size,"\\par",
           tf.fontstyle_id,tf.copyright,"\\par",
           tf.glyphs[5].unicode,"\\par", 
          "uwidth",tf.uwidth,"\\par")

-- print the glyph
for k,v in pairs (tf.glyphs[6]) do
    tex.print(v,"zz ")
    if (type(tf.glyphs[6].v)) then tex.print("\\par",k,tf.glyphs[6].name,tf.glyphs[6].boundingbox[1],"!!!\\par")  
      for i,j in ipairs (tf.glyphs[6].boundingbox) do
        tex.print("bounding box["..i.."]".." = ", j,"\\par")
      end 
    end
end




-- presents nicely a table 

local rep, write = string.rep, tex.print
function ExploreTable (tab, offset)
    offset = offset or ""
    for k, v in pairs (tab) do
        local newoffset = offset .. "\\mbox{.}"
        if type(v) == "table" then
           -- if k == "boundingbox" then write("BB") end
           write(offset..k .. " = \\{\\par ")
           ExploreTable(v, newoffset)
           write(offset..newoffset .. "\\}\\par")
         else
           write(offset..k .. " = "..tostring(v),"\\par")
         end
      end
end

write("\\fboxrule0pt\\par{\\ttfamily ")
 ExploreTable(tf.glyphs[38],"\\mbox{.}")
write("}")

\end{luacode*}
\end{document}


-- Sort the Hieroglyphics database per unicode symbol
-- we will use this later to insert the metrics

H  = H or {}
Hs = Hs or {}
local j=0

for k,v in pairs (t) do
    local index = tostring(t[k].unicode)                -- sanity check if some unicode are numbers
    local info  = info or {}                            -- table will hold general info
    metrics     = metrics or {}                         -- will add the metrics later for specific font
    info        = {unicode     = tostring(t[k].unicode),
                   gardiner    = t[k].gardiner,
                   mnemonic    = t[k].mnemonic,
                   phonetic    = '',
                   description = '',}
    H[index]    = {info        = info, 
                   metrics     = metrics}
    tex.print("\\arial ", H[index].info.gardiner)
end

-- get metrics

function toHex(num)
if  num>-1 then  
         num  = string.format("%x", num)
         num  = string.upper(num) 
         return num 
  else 
      num = "ERROR"
  end
  return nil
end

tex.print("FETCH AND INSERT METRICS FROM FONT ")

local i=0
while (i <tf.glyphcnt ) do -- use f.glyphmax 
		local g = tf.glyphs[i]
        local bb = {}
              bb[1] = tf.glyphs[i].boundingbox[1]
			  bb[2] = tf.glyphs[i].boundingbox[2]
              bb[3] = tf.glyphs[i].boundingbox[3]
              bb[4] = tf.glyphs[i].boundingbox[4]
        local idx = toHex(g.unicode)
        if H[idx] ~=nil then
           local boundingbox = boundingbox or {}
           H[idx].metrics.boundingbox = {bb[1],bb[2],bb[3],bb[4]}
           tex.print("\\arial", H[idx].info.unicode, 
                                H[idx].metrics.boundingbox[1],
                                H[idx].metrics.boundingbox[2],
								H[idx].metrics.boundingbox[3],
								H[idx].metrics.boundingbox[4],
                                "\\hiero\\char\""..H[idx].info.unicode.."\\par\\arial")
        end
  		i = i + 1
end


table.sort(H)

for k,v in pairs (H) do
   table.insert(Hs,k)
   -- tex.print("\\arial ",H[k], k, "\\par")
end

table.sort(Hs)

for k,v in pairs (Hs) do
    tex.print("\\arial ",Hs[k],"\\hiero","\\char\"" .. Hs[k],"\\arial")
end

for n in pairs(_G) do tex.print(n) end


s = string.gsub(package.cpath,"\\", "\\textbackslash ")
s = string.gsub(s,";", ";\\par ")

tex.print(s)
\end{luacode*}

\end{document}

\raggedright

While letters like é exist "on their own", they can also be written as the sequence [e´]. Intelligent fonts can actually either reposition the accent, so that the combination looks like an é, or they can perform a "ligature" substittion and replace the two characters [e]+[´] with the single character [é].


\end{document}
\begin{luacode*}
function toHex(num)
  local  num  = string.format("%x", num)
         num  = string.upper(num)
  return num
end

function prettyprint ()
    f = {}
    tex.print("\\footnotesize")
    f = fontloader.to_table(fontloader.open("c:/windows/fonts/Cambriai.ttf"))
    

    local i = 3
    local glyphname
    local ts = {}
    
   
	while (i+41 <f.glyphmax ) do -- use f.glyphmax 
		local g = f.glyphs[i]
		local uhex
    	if g then
      		glyphname = g.name     --string.sub(g.unicode,2) 
 	   		--- uhex = toHex(g.unicode)
 			---tex.print(""..glyphname.."=U+"..toHex(g.unicode)..", ")
 			---- printglyphRL("aegyptus", uhex) -- texprints the glyph
            tex.print("\\par",i, glyphname , g.unicode,  "\\par")
		end
  		i = i + 1
	end
	fontloader.close(f)
end

prettyprint()

local n = font.max()

tex.print("highest font number",n)

number, metrics=font.each()
--for key,value in pairs(metrics) do tex.print(key,value) end

tex.print(metrics)
metrics = font.getfont(21)
map = fonttable.map.map
local glyph = fonttable.glyphs[map[78200]]

tex.print("Glyph width",glyph.width)
tex.print("Glyph name",glyph.name,"\\par")
tex.print("Glyph unicode",glyph.unicode,"\\par")
tex.print("Glyph boundingbox[1]",glyph.boundingbox[1],"\\par")
tex.print("Glyph boundingbox[2]",glyph.boundingbox[2],"\\par")
tex.print("Glyph boundingbox[3]",glyph.boundingbox[3],"\\par")
tex.print("Glyph boundingbox[4]",glyph.boundingbox[4],"\\par")
--tex.print("Glyph boundingbox[5]",glyph.boundingbox[5],"\\par")
--tex.print("Glyph boundingbox[6]",glyph.boundingbox[6],"\\par")

tex.print(glyph.class,"\\par")
tex.print(font.current())

local g = node.new("glyph")
g.font = font.current()
g.lang = tex.language
g.char = 7824 -- U+13410

local hbox = node.hpack(g)
local vbox = node.vpack(hbox)

node.write(vbox)
\end{luacode*}
\arial

\aegyptus
\par
\begin{luacode*}
local g = node.new("glyph")

g.lang = tex.language
g.char = 80 -- U+13410

local hbox = node.hpack(g)
local vbox = node.vpack(hbox)

node.write(vbox)
\end{luacode*}

\aegyptus

\char8890
\end{document}
% We provide some author commands to assist
% with typesetting a hieroglyphics string 


% Use as |\hieroprint{"D50-D50a-D50b-D50i!"}| to
% print \hieroprint{"D50-D50a-D50b-D50i!"}

% The string must be quoted, otherwise Lua will be
% looking for a string variable. Some variables are build
% for example \hieroprint{women} will print Gardiner's list
% of `women and their occupations.'

\resizebox{!}{2cm}{\hiero\char"13000}\\
\resizebox{.6cm}{!}{\hiero\char"13000}\resizebox{.6cm}{!}{\hiero\char"13000}\resizebox{.6cm}{!}{\hiero\char"13000}

\cxset{geometry units=mm}
\drawfontbox{\resizebox{!}{2cm}{\hiero\char"130B3}}
\drawfontbox{\resizebox{!}{1cm}{\hbox{\hiero\char"13000\char"13000}}}


\drawfontbox{\resizebox{51.05pt}{!}{\hiero\char"13000\char"1306D\char"13000}}


This is using \verb+\shortunderstack+

abcde f \Shortunderstack[c]{{\hiero\char"13000} {\hiero\char"13001} {\hiero\char"13002\char"13006}} gyu\par

abcde f \Shortstack[c]{{\hiero\char"13000} {\hiero\char"13001} {\hiero\char"13070\char"1308A\char"1308B}} gyu\par

This is \resizebox{2.5em}{!}{\fbox{\Huge\hsize=1em\vbox{%
\hiero\char"1308F\par
\char"130BB
}}}
\resizebox{2.5em}{!}{\fbox{\Huge\hsize=1em\vbox{%
\hiero\char"1309F\par
\char"130BB
}}}

\Huge
$$\stackrel{\mbox{\hiero\char"1309F}}{\mbox{\hiero\char"130BB}}    $$

$$\mbox{\fontsize{3em}{1em}\selectfont\hiero \char"13005}\overset{\mbox{\hiero\char"13000\char"13000}}{\overset{\scalebox{-1}[1]{\hiero\char"1309F}}{\scalebox{-1}[1]{\hiero\char"13000}}}$$

$$\mbox{\fontsize{3em}{1em}\selectfont\hiero \char"13005}\overset{\mbox{\hiero\char"13000\char"13000}}{\overset{\scalebox{-1}[1]{\hiero\char"1309F}}{\scalebox{-1}[1]{\hiero\char"13000}}}$$

$$\mbox{\fontsize{3em}{1em}\selectfont\hiero \char"13005}\overset{\mbox{\hiero\char"13000\char"13000}}{\overset{\scalebox{-1}[1]{\hiero\char"1309F}}{\mbox{}}}$$

\normalsize

\arial
\begin{luacode*}
local g = node.new("glyph")
g.font = font.current()
g.lang = tex.language
g.char = 7824 -- U+13410

local hbox = node.hpack(g)
local vbox = node.vpack(hbox)

node.write(vbox)
\end{luacode*}
\end{document}


{\Huge
$$\text{\hiero\char"13000}\frac{\text{\hiero \char"13000\char"13001}}{\text{\hiero \char"130FE\char"130FF}}$$
}


$\stackrel{\hbox to 3em{\hiero\hfill\char"13001\hfill}}{\hbox to 3em{\hiero \hfill\char"130FE\char"130FF\hfill}}$

\begin{lstlisting}[language={[5.2]Lua},
basicstyle=\ttfamily\footnotesize,
keywordstyle=\bfseries\color{blue},
 stringstyle=\color{orange},
 commentstyle=\color{green},
]
-- test
local words = io.open('hyphens-' .. tex.jobname .. '.txt', 'w');
local outchar = unicode.utf8.char
local function dumphyphens (head)
   local data = {}
   for v in node.traverse(head) do
       if v.id == node.id('glyph') then
         data[#data+1] = outchar(v.char);
       elseif v.id == node.id('disc') then
          data[#data+1] = '-'
       elseif v.id == node.id('glue') then
         data[#data+1] = outchar(32)
       elseif v.id == node.id('hlist') then
         data[#data+1] = dumphyphens(v.list)
       end
   end
   return table.concat(data)
end
callback.register ('hyphenate', function (head,tail)
   lang.hyphenate(head, tail) 
   words:write (dumphyphens(head) .. outchar(10))
   end)
\end{lstlisting}




\end{document}
\Large
\scalebox{-2}[2]{\hiero\char"13000\char"13011\char"13020}

\scalebox{2}[2]{\hiero\char"13000\char"13011\char"13020}
\end{document}

removing a specific number of lines

\begin{luacode*}
function remove( filename, starting_line, num_lines )
    local fp = io.open( filename, "r" )
    if fp == nil then return nil end
 
    content = {}
    i = 1;
    for line in fp:lines() do
        if i < starting_line or i >= starting_line + num_lines then
	    content[#content+1] = line
	end
	i = i + 1
    end
 
    if i > starting_line and i < starting_line + num_lines then
	print( "Warning: Tried to remove lines after EOF." )
    end
 
    fp:close()
    fp = io.open( filename, "w+" )
 
    for i = 1, #content do
	 fp:write( string.format( "%s\n", content[i] ) )
    end
 
    fp:close()
end
remove("input.txt",3,2)

filename = "./languages/hieroglyphics.txt"
fp = io.open( filename, "r" )
for line in fp:lines() do
    tex.print(line.."\\par")
end
fp:close()

--If the file opens with no problems, io.open will return a
--handle to the file with methods attached.
--If the file does not exist, io.open will return nil and
--an error message.
--assert will return the handle to the file if present, or
--it will throw an error with the message returned second
--by io.open.
local file = assert(io.open(filename))
--Without wrapping io.open in an assert, local file would be nil,
--which would cause an 'attempt to index a nil value' error when
--calling file:read.
 
--file:read takes the number of bytes to read, or a string for
--special cases, such as "*a" to read the entire file.
local contents = file:read'*a'
 
--If the file handle was local to the expression
--(ie. "assert(io.open(filename)):read'a'"),
--the file would remain open until its handle was
--garbage collected.
tex.sprint(contents)
file:close()
\end{luacode*}
















\end{document}