%%%%%%%%%%%%%%%%%%%%%%%%%%%%%%%%%%%
% TEST FOR MATH faces and symbols
%%%%%%%%%%%%%%%%%%%%%%%%%%%%%%%%%%%%
\magnification=\magstep1
\input testmac.tex
%\hsize=17truecm\hoffset=-0.54truecm
%\vsize=25.7truecm\voffset=-0.54truecm
\input amssym.def
\input amssym
\long\def\comments#1\endcomments{}
%%%%%%%%%%%%%%%%%%%%%%%%%%%%%%%%%%%%%
% conservation des anciennes macros
%%%%%%%%%%%%%%%%%%%%%%%%%%%%%%%%%%%%%
\let\Surd=\surd
\let\infinity=\infty
\let\voidset=\emptyset
\let\In=\in
\let\Subset=\subset
\let\Forall=\forall
\let\Exists=\exists
\let\Dagger=\dagger
\let\DDagger=\ddagger
\let\oldpartial=\partial
\let\oldflat=\flat
\let\oldnatural=\natural
\let\oldsharp=\sharp
\let\oldnorm=\|
\let\oldbigotimes=\bigotimes
\let\oldsum=\sum
\let\oldprod=\prod
\let\oldcoprod=\coprod
\let\oldint=\intop
\let\oldoint=\ointop
\let\oldsqrt=\sqrt
\let\oldwidetilde=\widetilde
\let\oldwidehat=\widehat
%
% in math (display) mode
%
\let\normall=\mathopen
\let\normalm=\mathrel
\let\normalr=\mathclose
\catcode`\@=11
\def\biggg#1{{\hbox{$\left#1\vbox to20.5\p@{}\right.\n@space$}}}
\catcode`\@=12
\def\bigggl{\mathopen\biggg}
\def\bigggr{\mathclose\biggg}
\def\bigggm{\mathrel\biggg}
\def\testdelimiter#1{
	\csname#1l\endcsname({\rm H}
	\csname#1m\endcsname|{\rm O}
	\csname#1r\endcsname)
	=\csname#1l\endcsname[{\rm H}
	\csname#1m\endcsname\thickvert{\rm O}
	\csname#1r\endcsname]
	=\csname#1l\endcsname\{{\rm H}
	\csname#1m\endcsname|{\rm O}
	\csname#1r\endcsname\}
	=\csname#1l\endcsname\langle {\rm H}
	\csname#1m\endcsname|{\rm O}
	\csname#1r\endcsname\rangle
	=\csname#1l\endcsname\ldbrack{\rm H}
	\csname#1m\endcsname|{\rm O}
	\csname#1r\endcsname\rdbrack
	=\csname#1l\endcsname\lceil{\rm H}
	\csname#1m\endcsname\thickvert{\rm O}
	\csname#1r\endcsname\rceil
\cr
	=\csname#1l\endcsname\lfloor{\rm H}
	\csname#1m\endcsname\thickvert{\rm O}
	\csname#1r\endcsname\rfloor
	=\csname#1l\endcsname\lgroup{\rm H}
	\csname#1m\endcsname\thickvert{\rm O}
	\csname#1r\endcsname\rgroup
%=\csname#1l\endcsname\lmoustache {\rm H}\csname#1r\endcsname\rmoustache
	=\csname#1l\endcsname|{\rm H}\csname#1r\endcsname|
	=\csname#1l\endcsname\Vert {\rm H}\csname#1r\endcsname\Vert
	=\csname#1l\endcsname\vvvert {\rm H}\csname#1r\endcsname\vvvert
	=\csname#1l\endcsname\uparrow {\rm H}
	\csname#1m\endcsname\updownarrow{\rm H}
	\csname#1r\endcsname\downarrow
	=\csname#1l\endcsname\Uparrow {\rm H}
	\csname#1m\endcsname\Updownarrow{\rm H}
	\csname#1r\endcsname\Downarrow
	=\sqrt{\csname#1m\endcsname|}}
\def\testwideaccent#1#2{\hbox{\tt\string#2}\hfill
	#1#2{I}=#1#2{H}=#1#2{II}=#1#2{III}=
	#1#2{IIII}=#1#2{IIIII}=#1#2{IIIIII}=#1#2{IIIIIIII}}
\def\testcs#1#2{\allowbreak\noindent\leavevmode
	\hbox to 0.1\hsize{$#1{#2}$\hss}\nobreak
	\hbox to 0.4001\hsize{\tt\string#1\hss}\ignorespaces}
%%%%%%%%%%%%%%%%%%%%%%%%%%%%%%%%%%%%%

\title{\shadedtext{Mathabx series}}
\subtitle{\shadedtext{Informations and tests}}
\author{\shadedtext{Anthony Phan}}
\maketitle

The mathabx series are a large mathematical symbols set designed and
defined in MetaFont language. Many common and uncommon symbols can be
found in these series.  These programs are intended to produce bitmap
fonts and a lot of care is given about rasterization.  Encoding,
metrics, designs are not supposed to be frozen since improvements,
changes of mind can always happen.  Thus, {\it this distribution is
still (\today) at a ``merely for evaluation'' level}.  Please check my
home page to get the last updates:
$$
\hbox{\catcode`\~=12\tt http://www-math.univ-poitiers.fr/~phan/metafont.html}
$$

By now, there are three main series of fonts: {\it matha},
{\it mathb} and {\it mathx}. The {\it matha}\/
series consist in quite usual mathematical symbols, more precisely they
contain the, say, 64 mathematical symbols one can use
and suppose other ones know their meaning. The {\it  mathb}\/
series is a kind of twin of {\it  matha}, but it is the one
people should not show outside of the house: these symbols
do not have a very well known meaning and, thus, should not
be used. The {\it  mathx}\/ series is the set of extensible
delimiters and large operators fitting {\it  matha}\/ and
{\it  mathb}\/; its encoding doesn't match at all the {\it  cmex}\/
one, but it may change in the future.

Since a metafont designer doesn't always know when to stop
creating or coding stuff, many pieces of code remained once
the three former series where filled. The remaining stuff
has been put in {\it  mathu}\/ ({\it u}\/ stands for
unsupported) and in {\it  mathux}\/ ({\it ux}\/ stands for
unsupported extensible). There is also
``work in progress''-material: some full OT1 implementation
of calligraphic characters. If this last task find an end,
there would be a {\it mathc}\/ series---where {\it c}\/ would
stand for calligraphic.

The translation of this document into english is not yet finished.
We do apologize for this.
 
\section*{Progress}

\subsection*{April 29, 2002}
First posting on CTAN.

\subsection*{June 19, 2002}
A few changes have been done:

\item{$\bullet$}
Some large symbols are now thicker (\cs{\bigcup}, etc.);

\item{$\bullet$}
``Flat cups'' have been introduced (\cs{\bigcup}, etc.),
that means that the roundish parts of those symbols are
now ``flatter''.

\subsection*{November 16, 2002}

\item{$\bullet$}
Some work has been done on the calligraphic set. The uppercase
subset should be exactly the {\it Computer Modern}\/'s one. So
kerning has to be done in order to get a nice stuff.

\item{$\bullet$}
The series\/ {\it mathux}\/ are in progress. It will contain
more extensible symbols (unnecessary I think, so still unsupported).

\item{$\bullet$}
Binary operators {\tt\string\sprod} and {\tt\string\scoprod}
have been added to {\it matha}\/ at locations {\tt"3C} and {\tt "3D}.
I love those symbols and I believe that they must lie on the
main symbols series.

\subsection*{July 29, 2003}

\item{$\bullet$}
Some \LaTeX\ bugs have been fixed in {\tt mathabx.dcl}
with the help of some very kind users (Patrick Cousot, Hung N. Duong,
Kohsaku Hotta).

\item{$\bullet$}
The file {\tt mathgrey.mf} is no longer used. This means
that {\tt mathu10} has no more grey characters. It seems that
those characters are not compatible with some postscript
or such translation (with {\tt TeXtrace} for instance).

\item{$\bullet$}
The \LaTeX\ style file {\tt mathabx.sty} allows options
which are {\it matha}, {\it mathb}\/ and {\it mathx}.
These options define which series will be used.
If no option is given, the three series will be used.
(The plain\TeX\ file {\tt mathabx.tex} does not allow
anything of this kind.) 

\subsection*{October 23, 2003}

\item{$\bullet$} The {\tt\string\bar} and {\tt\string\widebar}
accents have changed of height (which is now smaller).

\item{$\bullet$} Arrows have a wider breadth and a lighter
head.

\item{$\bullet$} A little work on calligraphic digits has been done.

\item{$\bullet$} Double brackets formerly named
{\tt\string\lbbbrack} and {\tt\string\rbbbrack} are now
named {\tt\string\ldbrack} and {\tt\string\rdbrack}, and
they still have their curious aliases {\tt\string\lsemantic}
and {\tt\string\rsemantic}.

\subsection*{May 18, 2005}

\item{$\bullet$} I didn't pay attention until recently to
the fact that square roots may appear also in
{\tt\string\scriptstyle} and {\tt\string\scriptscriptstyle}
modes. This makes necessary to make the corresponding symbols
go across some usual symbols set and the extensible characters set.
This is repaired: a basic radical symbol now lies in {\it matha}\/
at location {\tt"37}.
In fact, the former basic radical symbol in {\it mathx}\/ still exists
at place {\tt"60} and there was no problem if {\it mathx}\/ is used
in 3 different sizes.

\item{$\bullet$} The {\tt\string\bar} and {\tt\string\widebar}
accents have been restored to their former heights.

\item{$\bullet$} Large greek like symbols (sums, products, coproducts)
have been revised. Text\-style product and coproduct widths have been
increased by $u\#$. Also, in {\it matha}, small sum, product, coproduct
have changed of encoding since the small sum has been introduced.

\item{$\bullet$} {\it mathc}\/ has finally been introduced.
It is still an uncomplete calligraphic set. Also existing glyphs
may be revised.

\item{$\bullet$} Integrals have been revised. Comments are welcome.

\item{$\bullet$} Astronomical/logical symbols are in progress.

\item{$\bullet$} Plain\TeX/\LaTeX\ files have been revised.
 
\section{List of every mathabx' symbols}

Thereafter will be loaded
plain\TeX\ definitions files related to these fonts families.
Assigning any value to the control sequence\/ {\tt\string\proofmode}
like\/ {\tt\string\let\string\proofmode=!}, for instance,
the definition of every symbol will be made together with the
printing of the related informations. The control sequence
{\tt\string\proofmode} will be reset to\/ {\tt\string\undefined}
at the end of the loading. This can be helpful for instantaneous
documentation.

About the names of the different control sequences, we mostly tried to
conform to the usual names. If in the following there is some apparent
mismatch with\/ {\it AMS}\/ denomination, it is normally supported.
Below is printed first the symbol, then if it already exists the {\it
Computer Modern}\/ or\/ {\it AMS}\/ one, the name of the control
sequence, in exponent the spacing value of the symbol (``other''
means that it is an accent or such, it does not really matter),
at last---if any---its aliases.

\let\proofmode=!
\fuzzytext
\input mathabx.tex
\normaltext

\section{Special constructions}

The special constructions described here are defined in the file
{\tt mathabx.dcl}. Usual plain\TeX/\LaTeX\ constructions should
be supported if not replaced by some {\it mathabx}\/ symbol.

\item{$\bullet$}
The control sequences
{\tt\string\not}, {\tt\string\varnot}, {\tt\string\changenotsign}
are described in some next section.

\item{$\bullet$} The control sequences {\tt\string\overbrace}, etc.,
have been defined as suggested by Matthias Clasen and Ulrik Vieth
in {\it newmath}.
\medbreak
\testcs\overbrace{abcde}
\testcs\underbrace{abcde}
\testcs\overgroup{abcde}
\testcs\undergroup{abcde}
\medbreak
\item{$\bullet$} The control sequences {\tt\string\overrightarrow},
etc., have been defined anew ({\it see}\/~{\tt mathabx.dcl}). They
will certainly be improved since they work only in textstyle and
displaystyle modes (as for standard plain\TeX/\LaTeX).
$$
\overrightarrow{abcde}^{\overrightarrow{abcde}^{\overrightarrow{abcde}}}
$$
\medbreak
\testcs\overrightarrow{abcde}
\testcs\overleftarrow{abcde}
\testcs\overleftrightarrow{abcde}
\testcs\underrightarrow{abcde}
\testcs\underleftarrow{abcde}
\testcs\underleftrightarrow{abcde}
\testcs\overRightarrow{abcde}
\testcs\overLeftarrow{abcde}
\testcs\overLeftRightarrow{abcde}
\testcs\underRightarrow{abcde}
\testcs\underLeftarrow{abcde}
\testcs\underLeftRightarrow{abcde}
\medbreak

\item{$\bullet$} The build-in accent {\tt\string\wideparen}
allows to build other accent-like control sequences.
\medbreak
\testcs\widering{abcde}
\testcs\widedot{abcde}
\testcs\wideddot{abcde}
\testcs\widedddot{abcde}
\testcs\wideddddot{abcde}
\medbreak

\item{$\bullet$}
At last, some symbols like {\tt\string\int},
{\tt\string\bigcomplement},
{\tt\string\surd}, must be defined by some {\tt\string\def}
because of limits or such.

\noindent
But it seems clear when viewing what is done elsewhere that the already too large mathabx set should (really?) be extended. This may be done by combining already existing symbols. If not, I would once again take my pen, some paper and my keyboard if necessary.

\section{Global installation}

In the {\it mathabx.me}\/ distribution, MetaFont source files
({\tt xxx.mf}) are all located in the {\tt source/} directory.
They may be moved to a (new) subdirectory named {\tt mathabx}
of {\tt\$TEXMF/fonts/source/public/}, thus in
$$
\hbox{\tt\$TEXMF/fonts/source/public/mathabx/}
$$
where {\tt\$TEXMF}
stands for the root directory of the \TeX MF distribution of
the computer. Plain\TeX/\LaTeX input files are all located
in the {\tt texinputs} directory of this distribution.
The three files {\tt mathabx.tex}, {\tt mathabx.sty}
and {\tt mathabx.dcl} may be moved to the directory
$$
\hbox{\tt\$TEXMF/tex/generic/misc/}
$$
(other location may be also fine). Other \TeX\ files are
there for documentation. They can be removed.

At last, the \TeX MF system needs to know that new files
have been added. This can be done by executing from a console
{\tt texhash} or such, some command that refreshes the \TeX MF
database. We don't provide more informations on
this last subject since it may depend on every particular \TeX MF
distribution and computer system.

If a previous {\it mathabx}\/ distribution has been already
installed, please remove every bitmap fonts ({\tt xxx.yyypk})
and metrics ({\tt xxx.tfm}) related to {\it mathabx} (only!)
since all of them may change from a ditribution to another.

We won't write anything about local installation (on Unices
systems for instance), nor about Type 1 conversions and installation
of the {\it mathabx}\/ fonts. One can find informations about these
two last topics on the World Wide Web (but maybe in japanese).

\section{Use with plain\TeX}

The basic input file is {\tt mathabx.tex}. It requires {\tt mathabx.dcl}
which is common to plain\TeX\ and \LaTeX. So one should type
$$
	\cs{\input\ mathabx.tex}
$$
at the beginning of his (her) plain\TeX\ document. This sets up
all the symbols previously described and defines $3$ new families
of mathematical symbols whose numbers are \cs\mathafam,
\cs\mathbfam\ and \cs\mathxfam. Pointsizes are $10\,\rm pt$, {\it i.e.},
textstyle is $10\,\rm pt$, scriptstyle is $7\,\rm pt$
and scriptscriptstyle is $5\,\rm pt$ for \cs\mathafam\ and \cs\mathbfam\
families. For \cs\mathxfam\ the three styles correspond to a pointsize
equal to $10\,\rm pt$ as for {\tt cmex} in plain\TeX. Changing pointsizes
is easy since any plain\TeX user know how to do so (one can also look
into {\tt mathabx.tex} to make sure). Remember that setting the
control sequence {\tt\string\proofmode} to a known value
before inputing {\tt mathabx.tex} would lead to the verbose
mode as illustrated in Section~1 of this document.

\section{Use with \LaTeX}

The basic package is {\tt mathabx.sty}. It requires {\tt mathabx.dcl}
which is common to plain\TeX\ and \LaTeX. So one should type
$$
	{\tt\cs\usepackage\{mathabx\}}
$$
in the preamble of his (her) \LaTeX\ document. This sets up
all the symbols previously described and defines $3$ new families
of mathematical symbols whose \LaTeX\ names are {\tt matha},
{\tt mathb} and {\tt mathx} (according to \LaTeX\ font selection scheme).
These families behave as expected with pointsize changes.
The possible options of the {\tt mathabx} package are
{\tt matha}, {\tt mathb} and {\tt mathx}. They allow to select
which families would be actually defines. For instance
$$
	{\tt\cs\usepackage[matha,mathx]\{mathabx\}}
$$
ignores the {\it mathb}\/ family and load only {\it matha}\/
and {\it mathx}\/ families. Remember that no option means
that the three families would be loaded. No individual
symbol selection has been setted. If one wants to use only,
say, a couple of symbols in the {\it mathabx}\/ series,
he (she) would have to it by him(her)self. 

\section{The control sequence \tt\string\not}

With plain\TeX\ or \LaTeX, the control sequence\/ {\tt\string\not}
only invocates a particular mathematical character (slanted line)
which is of relation-type. This character, which width is zero, \dots

Ce caract\`ere, qui est de longueur
nulle, recouvre le caract\`ere suivant d'autant mieux que  son
mode d'espacement est du type relation et que sa largeur a une
certaine valeur (celle des signes\/ $+$ ou\/ $=$). Autrement, le
recouvrement peut \^etre assez mauvais (en fait, inadapt\'e).

\par

Dans les s\'eries pr\'esent\'ees ici, certaines n\'egations
ont \'et\'e d\'efinies. Il semblait alors souhaitable que
la commande\/ {\tt\string\not} suivie par l'appel d'un caract\`ere
poss\'edant sa n\'egation propre ait pour r\'esultat cette derni\`ere.
Il suffisait pour cela de d\'efinir\/ {\tt\string\not} comme une
commande \`a un argument qui teste si cet argument est une commande
dont la n\'egation est d\'efinie
(si celle-ci est\/ {\tt\string\xxx}, le test porte sur l'existence
de\/ {\tt\string\notxxx} ou de\/ {\tt\string\nxxx}),
auquel cas ce sera elle qui sera appliqu\'ee, sinon 
(si l'argument n'est pas une commande, par exemple si c'est un caract\`ere,
ou si la n\'egation n'est pas d\'efinie) la m\'ethode
de superposition sera utilis\'ee.

\par

Le caract\`ere de n\'egation est appel\'e par\/ {\tt\string\notsign},
il appartient \`a la s\'erie\/ {\it matha}\/ et est droit. On peut y
pr\'ef\'erer une ligne inclin\'ee pr\'esente dans la s\'erie
{\it mathb}\/ et nomm\'ee\/ {\tt\string\varnotsign}. Il suffit alors
d'\'echanger les noms. C'est ce que fait la commande
{\tt\string\changenotsign} et ce de mani\`ere
\'eventuellement locale.
Ainsi on aura~:
$$
A\not=B
\qquad\hbox{\tt\string\changenotsign}\qquad
\changenotsign
A\not=B
$$
bien qu'on doive pr\'ef\'erer sur cet exemple l'emploi
de la commande\/ {\tt\string\neq}~:\/ $A\neq B$.
Ci-dessous nous avons recours \`a la commande\/ {\tt\string\not}
(sauf pour\/ {\tt\string\neq}) pour des relations dont la n\'egation
est d\'efinie~:
$$
a=b\neq c\equiv d\not\equiv e\sim f\not\sim g \approx h\not\approx i
\simeq j\not\simeq k\cong l\not\cong m,
$$
puis pour des relations (entre autres) dont la n\'egation
n'est pas d\'efinie~:
$$
\changenotsign
a\topdoteq b\not\topdoteq c\botdoteq d\not\botdoteq
e\dotseq f\not\dotseq g
\changenotsign
\risingdotseq h\not\risingdotseq i
\fallingdotseq j\not\fallingdotseq
\changenotsign
k S l\not S m,
$$
o\`u on se sera servi de\/ {\tt\string\changenotsign}
\`a certains endroits. Il est \`a remarquer que
l'espacement est perturb\'e de mani\`ere coh\'erente,
c'est-\`a-dire de la m\^eme fa\c con qu'il l'aurait \'et\'e
par le\/ {\tt\string\not} classique.
\par
Nous avons aussi d\'efini la commande\/ {\tt\string\varnot}
de fonctionnement semblable \`a celui de l'instruction
{\tt\string\not}~:\/ {\tt\string\varnot\string\xxx} teste si
{\tt\string\varnotxxx} est d\'efini et, si oui ex\'ecute
cette derni\`ere, sinon applique\/ {\tt\string\varnotsign\string\xxx}.
Nous reprenons la formule pr\'ec\'edente avec cette commande~:
$$
a\topdoteq b\varnot\topdoteq c\botdoteq d\varnot\botdoteq
e\dotseq f\varnot\dotseq g
\risingdotseq h\not\risingdotseq i
\fallingdotseq j\not\fallingdotseq
k S l\varnot S m,
$$
o\`u on ne voit aucune diff\'erence. En revanche,
$$
a=b\varnot= c\equiv d\varnot\equiv e\sim f\varnot\sim g
\approx h\varnot\approx i
\simeq j\varnot\simeq k\cong l\varnot\cong m
$$
emploie dans ce cas uniquement la m\'ethode de juxtaposition.

\section{Mayan numerals}

The presence of Mayan numerals in these series is related only to the history
of the development of them. At the beginning, we wanted to keep some\dots

La pr\'esence de chiffres mayas dans ces s\'eries
n'est li\'ee qu'\`a l'histoire du d\'evelop\-pement
de celles-ci. Nous d\'esirions au d\'epart conserver
quelques ressemblances avec les s\'eries destin\'ees
au texte (caract\`eres alphab\'etiques et num\'eraux).
L'absence ou la raret\'e de possibilit\'e de composer
selon des num\'erations anciennes nous aura pouss\'e
\`a nous y int\'eresser un peu. Le r\'esultat est
illustr\'e ci-dessous~: 
$$
\mayadelimiters([,])
\maya{1251}+\maya{2135}=\maya{3386}\neq\mayadigit{0}.
$$
This line has been typesetted with
\smallbreak
{\tt\string\mayadelimiters([,])\par
\string\maya$\{$1251$\}$%
+%
\string\maya$\{$2135$\}$%
=%
\string\maya$\{$3386$\}$%
\string\neq
\string\mayadigit$\{$0$\}$}.
\smallbreak
\noindent
Keeping these characters and the corresponding control sequences
is always an open question.

\section{La commande\/ {\tt\string\prime} et ses amies}

Nous avons r\'eintroduit les signes\/ {\tt\string\prime}
multiples associ\'es aux commandes\/ {\tt\string\prime},
{\tt\string\second},\/ {\tt\string\third} et\/ {\tt\string\fourth}.
Une commande naturellement associ\'ee est\/ {\tt\string\degree}
correspondant \`a un symbole semblable \`a celui appel\'e
par\/ {\tt\string\circ}. Il est n\'eanmoins diff\'erent car
il doit \^etre homog\`ene \`a l'ensemble des symboles
pr\'ec\'edents (comparer\/ $44^\circ$ et\/ $44^\degree$).
$$
44^\degree+36^\prime+89^\second+46^\third+99^\fourth
$$
La construction habituelle ({\it i.e.}\/ {\tt 99''''})
supporte une propri\'et\'e de ligaturage
sur les caract\`eres correspondants.
C'est ce que l'on voit ci-dessous~:
$$
44^\degree+36'+89''+46'''+99'''',
\qquad
\hbox{mais}
\quad
99'''''
\quad
\hbox{ou}
\quad
99''''''''.
$$
(Il faut y regarder de tr\`es pr\`es pour y voir
ce qu'il faut voir\dots)

\section{Various trials}

We begin by some meaningless expressions:
$$
\displaylines{
G\triangleleft H\trianglelefteq A \trianglerighteq B \triangleright C
\cr
|G|\not\triangleleft|H|\not\trianglelefteq|A|
\not\trianglerighteq|B|\not\triangleright|C|.
}
$$
Then we look at variations with mathematical style:
$$
\infty^{\infty^\infty}
\qquad
\infinity^{\infinity^\infinity}
\qquad
\in^{\in^\in}
\qquad
\subset^{\subset^\subset}
\qquad
\subseteq^{\subseteq^\subseteq}
\qquad
<^{<^<}
\qquad
\sqrt x^{\sqrt x^{\sqrt x}}
$$
The first {\tt\string\infty} symbol is from {\it matha}, the next one is
the {\it Computer Modern}\/ one.

Other trials with sometimes {\it Computer Modern}\/ symbols for comparison:
$$
\displaylines{
\Gamma\ssum \Sigma\sprod \Pi\scoprod D,\qquad f:X\righttoleftarrow\cr
{\partial f\over\partial x}(x)
\simeq
{\oldpartial f\over\oldpartial x}(x)
\qquad
\oldflat\flat\oldnatural\natural\oldsharp\sharp\hbox{bof bof}
\cr
\oldnorm T(h)f-f\oldnorm=\|T(h)f-f\|
\cr
A\cap B\cup C\uplus D \vee E\wedge F
\cr
A\sqcap B\sqcup C\squplus D \veebar E\barwedge F\veedoublebar G
\doublebarwedge F \curlyvee H \curlywedge I
\cr
\dagger\Dagger\ddagger\DDagger
\qquad
A^{\dagger^\dagger}
\cdot A^{\Dagger^\Dagger}
\qquad
A\Asterisk A\coAsterisk\{y\}\star[x]\ast(z)
\qquad
\surd\Surd
\cr
\forall x\in y^\perp,\ \exists S\subseteq R,
\ \exists x,\ (\Exists x),\quad
z\cap y,\ x\in y
\cr
f\mapsto g\longmapsto h,\qquad
f\hookrightarrow g \hookleftarrow h,\qquad
A\Mapstochar\Relbar\joinrel\Rightarrow B
\Leftarrow\joinrel\Relbar\Mapsfromchar C,\qquad
a\leftsquigarrow b\leftrightsquigarrow
c\rightsquigarrow d.
}
$$
Integrals in displaystyle then in textstyle (I don't want to forget that these signs are derived from the letter ``S'' even if I draw them my way):
$$
\displaylines{
\int\int\int_0^t f\circ g(x)\,{\rm d}x=
\int\iint_0^t f\circ g(x)\,{\rm d}x=
\iiint_0^t f\circ g(x)\,{\rm d}x
\cr
\oldint\nolimits_0^t\oldoint_C f\circ g(x)\,{\rm d}x=
\int_0^t\oint_C f\circ g(x)\,{\rm d}x=
\oiint_S f\circ g(x)\,{\rm d}x
\cr
\textstyle\int\int\int_0^t f\circ g(x)\,{\rm d}x=
\int\iint_0^t f\circ g(x)\,{\rm d}x=
\iiint_0^t f\circ g(x)\,{\rm d}x
\cr
\textstyle\oldint_0^t\oldoint_C f\circ g(x)\,{\rm d}x=
\textstyle\int_0^t\oint_C f\circ g(x)\,{\rm d}x=
\oiint_S f\circ g(x)\,{\rm d}x
}
$$
Sums and products, etc., in displaystyle:
$$
\displaylines{
\prod_{i=0}^{i=n}\prod_{j=0}^{j=n}
\prod_{k=0}^{k=n}\Gamma^{ij}_k
=
\prod_{i=0}^{i=n}\coprod_{j=0}^{j=n}
\prod_{k=0}^{k=n}\Gamma^{ij}_k
=
\coprod_{i=0}^{i=n}\coprod_{j=0}^{j=n}
\coprod_{k=0}^{k=n}\Gamma^{ij}_k
=
\oldprod_{i=0}^{i=n}\oldcoprod_{j=0}^{j=n}
\oldprod_{k=0}^{k=n}\Gamma^{ij}_k
\cr
\prod_{i=0}^{i=n}\prod_{j=0}^{j=n}
\prod_{k=0}^{k=n}\Gamma^{ij}_k
=
\biggl(\prod_{i=0}^{i=n}\sum_{j=0}^{j=n}
\prod_{k=0}^{k=n}\Gamma^{ij}_k\biggr)
=
\biggl[
\sum_{i=0}^{i=n}\sum_{j=0}^{j=n}
\sum_{k=0}^{k=n}\Gamma^{ij}_k\biggr]
=
\biggl[\oldsum_{i=0}^{i=n}\oldsum_{j=0}^{j=n}
\oldsum_{k=0}^{k=n}\Gamma^{ij}_k\biggr]
\cr
\bigcup_{i=0}^{i=n}\bigcap_{j=0}^{j=n}
\bigcup_{k=0}^{k=n}\Gamma^{ij}_k
=
\bigcup_{i=0}^{i=n}\bigvee_{j=0}^{j=n}
\bigcup_{k=0}^{k=n}\Gamma^{ij}_k
=
\bigcup_{i=0}^{i=n}\bigwedge_{j=0}^{j=n}
\bigcup_{k=0}^{k=n}\Gamma^{ij}_k
=
\bigvee_{i=0}^{i=n}\bigwedge_{j=0}^{j=n}
\bigvee_{k=0}^{k=n}\Gamma^{ij}_k
}
$$
Sums and products, etc., in textstyle:
$$
\displaylines{
\textstyle
\prod\prod
\prod\Gamma^{ij}_k
=
\prod\coprod
\prod\Gamma^{ij}_k
=
\coprod\coprod
\coprod\Gamma^{ij}_k
=
\oldprod\oldcoprod
\oldprod\Gamma^{ij}_k
\cr
\textstyle
\prod\prod
\prod\Gamma^{ij}_k
=
\bigl(\prod\sum
\prod\Gamma^{ij}_k\bigr)
=
\bigl[
\sum\sum
\sum\Gamma^{ij}_k\bigr]
=
\bigl[\oldsum\oldsum
\oldsum\Gamma^{ij}_k\bigr]
\cr
\textstyle
\bigcup\bigcap
\bigcup\Gamma^{ij}_k
=
\bigcup\bigvee
\bigcup\Gamma^{ij}_k
=
\bigcup\bigwedge
\bigcup\Gamma^{ij}_k
=
\bigvee\bigwedge
\bigvee\Gamma^{ij}_k
\cr}
$$

\section{Delimiters}

The whole set of extensible delimiters is presented below.
Some of those delimiters are quite close to\/ {\it Computer
Modern}\/'s ones, for instance parentheses are almost the same.
By now there are a few differences. Left and right groups
are fully supported, i.e. every sizes exist. Moustaches
are built in the font but not in a satisfactory way, so
that the corresponding control sequences have not been
written down. Some vertical lines are not supported, these
are the ones that could be built with various extension moduli
as in\/ {\it Computer Modern}.
$$
\displaylines{
(X,X)=[X,X]=\{X,X\}=
\langle X^c,X^c\rangle
\equiv
[X^c,X^c]
=
\ulcorner X\urcorner
=
\llcorner X\lrcorner
=
\ulcorner X\lrcorner
=
\llcorner X\urcorner
\cr
\testdelimiter{normal}
\cr
\testdelimiter{big}
\cr
\testdelimiter{Big}
\cr
\testdelimiter{bigg}
\cr
\testdelimiter{Bigg}
\cr
\testdelimiter{biggg}
\cr
}
$$

\section{Accents and wide accents}
Here are some basic accents.
$$
\def\test#1{\hbox{\tt\string#1\ o}\quad#1o}
\displaylines{
\test\ring,\quad
\test\dot,\quad\test\ddot,\quad\test\dddot,\quad\test\ddddot,\cr
}
$$
Thus,
$$
\ddddot y -4\dddot y +5x\ddot y +f(x)\dot y=g(x)\in\ring C
$$
Here we have extensible accents. Control sequences {\tt\string\oldxxx}
just invoke former symbols when they exist, these control sequence
are defined only for this test file.
$$
\displaylines{
\testwideaccent{\skew3}{\widehat}
\cr
\testwideaccent{\skew3}{\oldwidehat}
\cr
\testwideaccent{\skew3}{\widecheck}
\cr
\testwideaccent{\skew3}{\widetilde}
\cr
\testwideaccent{\skew3}{\oldwidetilde}
\cr
\testwideaccent{\skew3}{\widearrow}
\cr
\testwideaccent{\skew3}{\wideparen}
\cr
\testwideaccent{\skew3}{\widering}
\cr
\testwideaccent{\skew3}{\widedot}
\cr
\testwideaccent{\skew3}{\wideddot}
\cr
\testwideaccent{\skew3}{\widedddot}
\cr
\testwideaccent{\skew3}{\wideddddot}
\cr
\testwideaccent{\skew3}{\widebar}
\cr
\hbox{{\tt\string\overleftarrow} is not of accent type}\hfill
\overleftarrow{IIIIIIIIIIIIIIIIIIIIIIIIIII}
}
$$
According to Matthias Clasen's construction: {\tt\string\overbrace},
{\tt\string\underbrace}, {\tt\string\overgroup}, {\tt\string\undergroup}
$$
\overbrace{\underbrace{HHHHHHHHHH}}^{AAAAAAAAAAAA}_{HHHHHHHHHHH}
\qquad
\overgroup{\undergroup{HHHHHHHHHH}}^{AAAAAAAAAAAA}_{HHHHHHHHHHH}
$$

\section{Astronomical symbols}

Astronomical/logical symbols are in progress (coding, design, etc.).
There is not enough room yet in the {\it mathb}\/ series to provide a
complete set of such symbols. If Mayan numerals are supressed,
maybe\dots
$$
\left\lfilet\enspace\vcenter{\advance\hsize by -4\parindent
The Earth~$\Earth$ (or~$\varEarth$)
is in rotation around the Sun~$\Sun$
like Mercury~$\Mercury$, Venus~$\Venus$,
Mars~$\Mars$, Saturn~$\Saturn$, Jupiter~$\Jupiter$,
Uranus~$\Uranus$, Neptune~$\Neptune$ and Pluto~$\Pluto$.
But the Moon~$\Moon$ is not.\par}\enspace\right\rfilet
$$
Also, there are Aries $\Aries$, Taurus $\Taurus$, Gemini $\Gemini$,
Leo $\Leo$, Libra $\Libra$, Scorpio $\Scorpio$, etc. (Notice the use of
{\tt\string\lfilet} and {\tt\string\rfilet} in the previous
paragraph---which names may be changed.)
 
Of course, some symbols have an {\it alias}\/
such as\/ {\tt\string\girl} and\/ {\tt\string\boy}:
$$
	\{(\boy,\boy),(\boy,\girl),(\girl,\boy),(\girl,\girl)\}.
$$
These symbols are nice in some usual exercices of
elementary Probability Theory.

The {\tt\string\rip} sign is mostly for fun. It is not an
astronomical/logical symbol but is located among them in {\it
mathb}. In the Theory of Markov Processes, a cemetery sign is often
needed. We have designed the following ugly and not so necessary one:
$\rip{}\rip\rip\rip$. Many successive {\tt\string\rip} signs glue to
each others.

\section{Unsupported}

{\font\unsupported=mathc10
\unsupported
As one can see further on, many things are unsupported. But this does not mean that few things extracted from unsupported stuff are not interesting.  Below, one can see some shape that comes from my favorite pen and another I saw once on the web and thought it was astonishingly beautiful.
{\font\unsupported=mathu10
\def\greekfill{\hss\cleaders\hbox{\unsupported\char"9D}\hfill}%
\def\ghaneanfill{\hss\cleaders\hbox{\unsupported\char"9E}\hfill}%
\medbreak
\line{\greekfill}
\line{\ghaneanfill}}
\medbreak\noindent
See {\it mathc10}, {\it mathu10}, {\it mathux10} in the next pages.}
\newpage

\section{Mathabx font tables}

\subsection*{Matha, major symbols series}

{\def\fontname{matha10 }
\startfont\table}
%{\def\fontname{matha5 }
%\startfont\table}

\newpage

\subsection*{Mathb, minor symbols series}

{\def\fontname{mathb10 }
\startfont\table}
%{\def\fontname{mathb5 }
%\startfont\table}

\newpage

\subsection*{Mathx, major extensible symbols series}
{\centerlargechars
\def\fontname{mathx10 }
\startfont\table}
%{\centerlargechars
%\def\fontname{mathx5 }
%\startfont\table}

\newpage

\subsection*{Mathc, unsupported calligraphic series}
{\font\unsupported=mathc10
\unsupported
The series\/ {\it mathc}\/ have some features that may interest
people: it contains the whole set of calligraphic characters
of\/ {\tt cmsy} and also extends it. The first part of this extension
is the latin lowercase letters, and also the punctuation which
make it a quite complete OT1 font. This part of the extension is due
to me. Hebrew characters have been converted to MetaFont,
adapted and extended (dagesh sign) also by me, but the source
is some ``professional'' or commercial font.

Things are in progress: <\,punctuation\,>, <<\,ligatures\,>>, greek letters\dots\
Designs are deeply based on {\it Computer Modern}.
Thus these series should be named
cmchXX\dots }

\docomparison{mathc10 }{cmsy10}from 65 to 90.
\docomparison{cmsy10 }{cmmi10}from 65 to 90.
\docomparison{mathc10 }{cmmi10}from 65 to 90.
\docomparison{mathc10 }{cmmi10}from 97 to 122.

\UsualTest{mathc10 }


\subsection*{Random test of gray}

{\font\currentfont=mathc10\currentfont
\mixfrom 97 to 122.
\mixfrom 160 to 191.
\digits
}
%\mixfrom 192 to 255.}

\newpage

\subsection*{Mathu, unsupported symbols series}

{\centerlargechars
\def\fontname{mathu10 }
\startfont\table}

\subsection*{Mathux, unsupported extensible symbols series}

{\centerlargechars
\def\fontname{mathux10 }
\startfont\table}

\subsection*{Mathastrotest10, about the metaness
of astronomical/logical symbols}\break
Who cares about astronomical/logical symbols? So why trying to do
something great with them? General shapes are even unstable: they are
never the same from a reference to another. I think that I've been
convinced by the presence of some such symbols in the fonts tables of
the famous book ``The Printing of Mathematics''. By the way it
remembers me that if I want to extend this subset of {\it mathb}, I
would have to take into account that I have already put some metaness
in these designs.

\medbreak\noindent
{\font\currentfont=mathastrotest10
\currentfont
\count0=0
\loop\ifnum\count0<32\relax\count1=\count0
{\loop\ifnum\count1<256\char\the\count1\advance\count1 by 32\repeat}
\hfil\allowbreak\advance\count0 by 1\repeat}

{\def\fontname{mathastrotest10 }
\startfont\table}


\bye




