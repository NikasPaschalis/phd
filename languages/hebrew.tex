\subsection{Hebrew}
\newfontfamily\hebrew[Script=Hebrew, Scale=1]{Miriam}
\fontspec{Arial Unicode MS}
To properly typeset Hebrew texts you first need to choose an appropriate font and also set the directionality of the text. This
is done using the etex commands:

\CMDI{\beginL} and \CMDI{\beginR} 

For \XeTeX\ you also need to add near the top of your document |\TeXXeTstate=1|. The package \pkgname{bidi} can be used to set all parameters. Be warned that it redefines almost all of \latexe's commands, so for short mixed texts, I wouldn't recommend its usage. 



The Hebrew alphabet (Hebrew: אָלֶף־בֵּית עִבְרִי[a], alefbet ʿIvri ), known variously by scholars as the Jewish script, square script, block script, is used in the writing of the Hebrew language, as well as other Jewish languages, most notably Yiddish, Ladino, and Judeo-Arabic. There have been two script forms in use; the original old Hebrew script is known as the paleo-Hebrew script (which has been largely preserved, in an altered form, in the Samaritan script), while the present "square" form of the Hebrew alphabet is a stylized form of the Assyrian script. Various "styles" (in current terms, "fonts") of representation of the letters exist. There is also a cursive Hebrew script, which has also varied over time and place. On Windows you can use the \texttt{Miriam} font or \texttt{Arial Unicode MS} or \texttt{Miriam Fixed}.
\medskip

\topline

\bgroup\TeXXeTstate=1
\raggedleft\hebrew{}\beginR

הכתב הכנעני הקדום הלך והתפשט וסימניו היו מוכרים כל כך, עד כי המשתמשים בו התחילו "להתעצל" בהשלמת הציורים, והניחו כי הקורא יבין גם מתוך שרטוטים סכמתיים באיזו אות מדובר. כך, למשל, הפך הראש למשולש עם צוואר; כף היד מלאת האצבעות הפכה לשרטוט דל, ומהדג נותר רק הזנב. כשהעברים אמצו את הכתב הכנעני הם התקשו לזהות חלק מהציורים המקוריים והניחו למשל כי הסימן המתאר את המילה "זהה" הוא כלי נשק; שזנב הדג המשולש הוא דלת, ושדווקא הנחש הוא דג. כך נולדו שמותיהם העבריים של האותיות זי"ן, דל"ת ונו"ן (נון הוא דג, כמו אמנון, שפמנון וכו'). הציורים שהפכו לסימנים התגלגלו לכתבים נוספים, ואפילו ליוונית וללטינית. גם בכתב העברי המודרני ניתן לזהות המשך התפתחותי ברור מן הכתב הכנעני הקדום, והשתמרות שמות האותיות מקלה מאוד על פענוח המקור.


בתקופת בית שני, אומץ האלפבית הארמי לשימוש השפה העברית במקום האלפבית העברי העתיק, כאשר בזה האחרון נעשה שימוש מועט כגון כתיבת השמות הקדושים והטבעת מטבעות. עם הזמן, נעלם גם שימוש זה של הכתב העתיק. האלפבית העברי של ימינו הוא אפוא פיתוח של האלפבית הארמי ולא של הכתב העברי העתיק.	
{}

\HHUGE לֹ֥א תִשָּׂ֛א

\endR


\egroup
\bottomline
\medskip

To make all paragraphs  RL use the \cmd{\everypar}\footnote{See discussions at \url{http://tex.stackexchange.com/questions/141867/minimal-bidi-for-typesetting-rl-text} and \url{http://www.tug.org/pipermail/xetex/2004-August/000697.html}}. 

\begin{verbatim}
\newbox\mybox \everypar{\setbox\mybox\lastbox\beginR\box\mybox}
\everypar={% at the start of each paragraph, do....
    \setbox0=\lastbox % save the paragraph indent, if any
    \beginR % set R-L direction
    \box0 % then re-insert the indent
	}
\end{verbatim}

The Hebrew alphabet has 22 letters, of which five have different forms when used at the end of a word. Hebrew is written from right to left. Originally, the alphabet was an abjad consisting only of consonants. Like other \textit{abjads}, such as the Arabic alphabet, means were later devised to indicate vowels by separate vowel points, known in Hebrew as niqqud. In rabbinic Hebrew, the letters א ה ו י are also used as matres lectionis to represent vowels. When used to write Yiddish, the writing system is a true alphabet (except for borrowed Hebrew words). In modern usage of the alphabet, as in the case of Yiddish (except that ע replaces ה) and to some extent modern Israeli Hebrew, vowels may be indicated. Today, the trend is toward full spelling with these letters acting as true vowels.
