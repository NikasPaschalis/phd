
\begin{description}
\item[Abkhazia] (Abkhaz: Аҧсны́ Apsny [apʰsˈnɨ]; Georgian: აფხაზეთი Apkhazeti; Russian: Абхазия Abkhaziya) is a disputed territory and partially recognised state controlled by a separatist government on the eastern coast of the Black Sea and the south-western flank of the Caucasus.

\item[Achinese] Acehnese language (Achinese) is a Malayo-Polynesian language spoken by Acehnese people natively in Aceh, Sumatra, Indonesia. This language is also spoken in some parts in Malaysia by Acehnese descendents there, such as in Yan, Kedah.

Formerly, Acehnese language was written in Arabic script called Jawoë or Jawi in Malay language. The script is less common nowadays.[citation needed] Now, Acehnese language is written in Latin script since colonization by the Dutch; with the addition of supplementary letters. The additional letters are é, è, ë, ö and ô.[8] The sound ɨ is represented by 'eu' and the sound ʌ is represented by 'ö' respectively. The letter 'ë' is used to represent the schwa sound which forms the second part in the diphthongs.

\item[Adyghe] Adyghe (/ˈædɨɡeɪ/ or /ˌɑːdɨˈɡeɪ/;[3] Adyghe: Адыгэбзэ adyghabze), also known as West Circassian (КӀахыбзэ), is one of the two official languages of the Republic of Adygea in the Russian Federation, the other being Russian. It is spoken by various tribes of the Adyghe people: Abzekh,[4] Adamey, Bzhedug;[5] Hatuqwai, Temirgoy, Mamkhegh; Natekuay, Shapsug;[6] Zhaney, Yegerikuay, each with its own dialect. The language is referred to by its speakers as Adygebze or Adəgăbză, and alternatively spelled in English as Adygean, Adygeyan or Adygei. The literary language is based on the Temirgoy dialect.
There are apparently around 128,000 speakers of the language on the native territory in Russia, almost all of them native speakers. In the whole world, some 300,000 speak the language. The largest Adyghe-speaking community is in Turkey, spoken by the post Russian–Circassian War (circa 1763–1864) diaspora; in addition to that, the Adyghe language is spoken by the Cherkesogai in Krasnodar Krai.

Ублапӏэм ыдэжь Гущыӏэр щыӏагъ. Ар Тхьэм ыдэжь щыӏагъ, а Гущыӏэри Тхьэу арыгъэ. Ублапӏэм щегъэжьагъэу а Гущыӏэр Тхьэм ыдэжь щыӏагъ. Тхьэм а Гущыӏэм зэкӏэри къыригъэгъэхъугъ. Тхьэм къыгъэхъугъэ пстэуми ащыщэу а Гущыӏэм къыримыгъгъэхъугъэ зи щыӏэп. Мыкӏодыжьын щыӏэныгъэ а Гущыӏэм хэлъыгъ, а щыӏэныгъэри цӏыфхэм нэфынэ афэхъугъ. Нэфынэр шӏункӏыгъэм щэнэфы, шӏункӏыгъэри нэфынэм текӏуагъэп.

Translation: In the beginning was the Word, and the Word was with God, and the Word was God. The same was in the beginning with God. All things were made by him, and without him was not any thing made that was made. In him was life, and the life was the light of men. And the light shineth in darkness, and the darkness comprehended it not.

\item[Albanian]Albanian (shqip [ʃcip] or gjuha shqipe [ˈɟuha ˈʃcipɛ], meaning Albanian language) is an Indo-European language spoken by approximately 7.6 million people,[3] primarily in Albania, Kosovo, the Republic of Macedonia and Greece, but also in other areas of Southeastern Europe in which there is an Albanian population, including Montenegro and Serbia (Presevo Valley). Centuries-old communities speaking Albanian-based dialects can be found scattered in Greece, southern Italy,[4] Sicily, and Ukraine.[5] As a result of a modern diaspora, there are also Albanian speakers elsewhere in those countries and in other parts of the world, including Scandinavia, Switzerland, Germany, Austria and Hungary, United Kingdom, Turkey, Australia, New Zealand, Netherlands, Singapore, Brazil, Canada, and the United States.

Letter:	A	B	C	Ç	D	Dh	E	Ë	F	G	Gj	H	I	J	K	L	Ll	M	N	Nj	O	P	Q	R	Rr	S	Sh	T	Th	U	V	X	Xh	Y	Z	Zh\\
IPA value:	a	b	t͡s	t͡ʃ	d	ð	e	ə	f	ɡ	ɟ	h	i	j	k	l	ɫ	m	n	ɲ	o	p	c	ɾ	r	s	ʃ	t	θ	u	v	d͡z	d͡ʒ	y	z	ʒ\\

\end{description}

\begin{multicols}{5}
\raggedright
Abkhazian\\
Abron\\
Achinese\\
Acoli\\
Adyghe\\
Afar\\
Afrikaans\\
Aghem\\
Akan\\
Akoose\\
Albanian\\
Albay\\
Bikol\\
Amo\\
Asturian\\
Asu\\
Atikamekw
Atsam
Avaric
Aymara
Azerbaijani (Cyrillic script)\\
Azerbaijani (Latin script)\\
Bafia\\
Bafut\\
Balinese\\
Balkan Gagauz Turkish
Bambara (Latin script)
Banjar
Baoulé
Basaa
Bashkir
Basque
Batak
Batak Toba
Belarusian
Bemba
Bena
Betawi
Bikol
Bini
Bislama
Bomu
Bosnian (Cyrillic script)
Bosnian (Latin script)
Breton
Bube
Buginese
Buhid
Bulgarian
Bulu
Buriat
Bushi
Catalan
Cebaara Senoufo
Cebuano
Central Atlas Tamazight (Latin script)
Central-Eastern Niger Fulfulde
Central Huasteca Nahuatl
Central Mazahua
Chamorro
Chechen
Chiga
Chipewyan
Church Slavic
Chuukese
Chuvash
Colognian
Congo Swahili
Cornish
Corsican
Croatian
Czech
Dan
Danish
Dargwa
Dogrib
Duala
Dutch
Dyula
Eastern Huasteca Nahuatl
East Futuna
Efik
Embu
English
Erzya
Esperanto
Estonian
Ewe
Ewondo
Fang
Faroese
Fijian
Filipino
Finnish
Fon
French
Friulian
Fulah
Ga
Gagauz
Galician
Ganda
German
Ghomala
Gilbertese
Gorontalo
Greek
Gronings
Guajajára
Guarani
Guianese Creole French
Gusii
Gwichʼin
Haitian
Hanunoo
Hausa (Latin script)
Hawaiian
Hiligaynon
Hiri Motu
Hungarian
Ibibio
Icelandic
Igbo
Iloko
Inari Sami
Indonesian
Ingush
Interlingua
Inuinnaqtun
Inuktitut (Latin script)
Inupiaq
Irish
Italian
Javanese
Jenaama Bozo
Jju
Jola-Fonyi
Kabardian
Kabuverdianu
Kabyle
Kaingang
Kako
Kalaallisut
Kalanga
Kalenjin
Kalo Finnish Romani
Kamba
Karachay-Balkar
Kara-Kalpak
Karelian
Kashubian

Kazakh (Cyrillic script)

Kerinci
Khasi
Kʼicheʼ
Kikuyu
Kimbundu
Kinyarwanda
Kita Maninkakan
Kom
Komering
Komi
Komi-Permyak
Kongo
Koro
Koro Wachi
Kosraean
Koyraboro Senni
Koyra Chiini
Kpelle
Krio
Kuanyama
Kumyk
Kurdish (Latin script)

Kwasio

Kyrgyz (Cyrillic script)

Kyrgyz (Latin script)

Lak\\
Lakota\\
Lampung Api\\
Langi\\
Lango\\
Latin\\
Latvian\\
Lezghian\\
Limburgish\\
Lingala\\
Lithuanian\\
Lombard
Lomwe
Lower Sorbian
Low German
Lozi
Luba-Katanga
Luba-Lulua
Lule Sami
Luo
Luxembourgish
Luyia
Maasina Fulfulde
Macedonian
Machame
Madurese
Mafa
Maguindanaon
Makasar
Makhu
Makhuwa-Meetto
Makonde
Malagasy
Malay (Latin script)
Maltese
Mandar
Mandingo (Latin script)
Manx
Manyika
Maori
Mapuche
Mari
Marshallese
Masaaba
Masai
Mbunga
Medumba
Mende
Meru
Meta’
Minangkabau
Mohawk
Moksha
Mongo
Mongolian (Cyrillic script)
Montagnais
Morisyen
Mossi
Mundang
Nama
Nauru
Navajo
Naxi
Ndau
Ndonga
Neapolitan
Negeri Sembilan Malay
Ngaju
Ngiemboon
Ngomba
Nigerian Fulfulde
Nigerian Pidgin
Niuean
Northern Sami
Northern Sotho
North Ndebele
North Slavey
Norwegian Bokmål
Norwegian Nynorsk
Nuer
Nyamwezi
Nyanja
Nyankole
Occitan
Oromo
Ossetic
Palauan
Pampanga
Pangasinan
Papiamento
Pohnpeian
Pökoot
Polish
Portuguese
Punu
Quechua
Rajasthani
Rejang
Réunion Creole French
Riang
Rinconada Bikol
Romanian
Romansh
Rombo
Ronga
Rundi
Russian
Rusyn
Rwa
Safaliba
Saho
Sakha
Samburu
Samoan
Sangir
Sango
Sangu
Santali
Sasak
Scots
Scottish Gaelic
Sena
Serbian (Cyrillic script)
Serbian (Latin script)
Serer
Seselwa Creole French
Shambala
Shona
Sicilian
Sidamo
Sinte Romani
Skolt Sami
Slave
Slovak
Slovenian
Soga
Somali
Soninke
Southern Altai
Southern Sami
Southern Sotho
South Ndebele
Spanish
Sranan Tongo
Sukuma
Sundanese
Susu
Swahili
Swati
Swedish
Swiss German
Tachelhit (Latin script)
Tae’
Tagbanwa
Tahitian
Taita
Tajik (Cyrillic script)
Tamashek
Taroko
Tasawaq
Tatar
Tausug
Tavringer Romani
Teso
Tetum
Timne
Tiv
Tokelau
Tok Pisin
Tolaki
Tomo Kan Dogon
Tongan
Tooro
Tornedalen Finnish
Tsonga
Tswana
Tumbuka
Turkish
Turkmen (Latin script)
Tuvalu
Tuvinian
Tyap
Uab Meto
Udmurt
Ukrainian
Ulithian
Umbundu
Unknown Language
Uyghur (Cyrillic script)
Uzbek (Cyrillic script)
Uzbek (Latin script)
Vai (Latin script)
Venda
Vietnamese
Virgin Islands Creole English
Vunjo
Wallisian
Walloon
Walser
Waray
Welsh
Western Frisian
Western Huasteca Nahuatl
Western Mari
Wolof
Xaasongaxango
Xavánte
Xhosa
Yangben
Yao
Yapese
Yemba
Yoruba
Yucatec Maya
Zarma
Zaza
Zeelandic
Zhuang
Zulu
\end{multicols}

