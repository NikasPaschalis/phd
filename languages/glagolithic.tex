\let\glagolithic\pan

\section{Glagolitic}
\label{s:glagolitic}
\fboxrule0pt\fboxsep0pt

\noindent
The Glagolitic alphabet /{\glagolithic ˌɡlæɡɵˈlɪtɨk/}, also known as Glagolitsa, is the oldest known Slavic alphabet, from the 9th century.

The Glagolitic alphabet was added to the Unicode Standard in March 2005 with the release of version 4.1.
The Unicode block for Glagolitic is U+2C00–U+2C5F.



\begin{scriptexample}[]{Glacolitic}

\unicodetable{glagolithic}{%
"2C00,"2C10,"2C20,"2C30,"2C40,"2C50}

\end{scriptexample}
\arial

The name was not coined until many centuries after its creation, and comes from the Old Church Slavonic glagolъ "utterance" (also the origin of the Slavic name for the letter G). The verb glagoliti means "to speak". It has been conjectured that the name glagolitsa developed in Croatia around the 14th century and was derived from the word glagolity, applied to adherents of the liturgy in Slavonic.[1]

In Old Church Slavonic the name is {\glagolithic ⰍⰫⰓⰊⰎⰎⰑⰂⰋⰜⰀ}, Кѷрїлловица.
The name Glagolitic in Bulgarian, Russian, Macedonian глаголица (glagolica), Belarusian is глаголіца (hłaholica), Croatian glagoljica, Serbian глагољица / glagoljica, Czech hlaholice, Polish głagolica, Slovene glagolica, Slovak hlaholika, and Ukrainian глаголиця (hlaholyća).