\section{Deseret script}
\newfontfamily\deseret{code2001.ttf}

The Deseret alphabet (dɛz.əˈrɛt.) (Deseret: {\deseret 𐐔𐐯𐑅𐐨𐑉𐐯𐐻 or 𐐔𐐯𐑆𐐲𐑉𐐯𐐻}) is a phonemic English spelling reform developed in the mid-19th century by the board of regents of the University of Deseret (later the University of Utah) under the direction of Brigham Young, second president of The Church of Jesus Christ of Latter-day Saints.

In public statements, Young claimed the alphabet was intended to replace the traditional Latin alphabet with an alternative, more phonetically accurate alphabet for the English language. This would offer immigrants an opportunity to learn to read and write English, he said, the orthography of which is often less phonetically consistent than those of many other languages. Similar experiments were not uncommon during the period, the most well-known of which is the Shavian alphabet.

Young also prescribed the learning of Deseret to the school system, stating "It will be the means of introducing uniformity in our orthography, and the years that are now required to learn to read and spell can be devoted to other studies".[2]


Deseret script {\deseret 𐐔𐐯𐑅𐐨𐑉𐐯𐐻}  [U+10400-U+1044F]
\medskip

\bgroup
\par
\noindent
\colorbox{graphicbackground}{\color{black}^^A
\begin{minipage}{\textwidth}^^A
\parindent1pt
\vskip10pt
\leftskip10pt \rightskip\leftskip
\deseret
\large

𐐂 𐑌𐐲𐑉𐑅𐐨𐑉𐐮 𐐮𐑆 𐐪 𐐹𐐨𐑅 𐐱𐑂 𐑊𐐰𐑌𐐼 𐐱𐑌 𐐸𐐶𐐮𐐽 𐑁𐑉𐐭𐐻𐐻𐑉𐐨𐑆 𐐪𐑉 𐑅𐐻𐐪𐑉𐐻𐐯𐐼,


\par
\vspace*{10pt}
\end{minipage}
}

Text: Deseret alphabet http://www.omniglot.com/writing/deseret.htm
\medskip
\egroup

\PrintUnicodeBlock{./languages/deseret.txt}{\deseret}
