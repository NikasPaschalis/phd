\subsection{Sundanese}
\newfontfamily\sundanese{SundaneseUnicode-1.0.5.ttf}
\newfontfamily\arial{Arial Unicode MS}
\def\ublock#1{\texttt{{\arial #1}}}

Sundanese script (Aksara Sunda, {\sundanese ᮃᮊ᮪ᮞᮛ ᮞᮥᮔ᮪ᮓ}) is a writing system which is used by the Sundanese people. It is built based on Old Sundanese script (Aksara Sunda Kuno) which was used by the ancient Sundanese between the 14th and 18th centuries.


\sundanese
\obeylines

{\arial 0	1	2	3	4	5	6	7	8	9	A	B	C	D	E	F}
\ublock{U+1B8x}	ᮀ	ᮁ	ᮂ	ᮃ	ᮄ	ᮅ	ᮆ	ᮇ	ᮈ	ᮉ	ᮊ	ᮋ	ᮌ	ᮍ	ᮎ	ᮏ
\ublock{U+1B9x}	ᮐ	ᮑ	ᮒ	ᮓ	ᮔ	ᮕ	ᮖ	ᮗ	ᮘ	ᮙ	ᮚ	ᮛ	ᮜ	ᮝ	ᮞ	ᮟ
\ublock{U+1BAx}	ᮠ	ᮡ	ᮢ	ᮣ	ᮤ	ᮥ	ᮦ	ᮧ	ᮨ	ᮩ	᮪	 ᮫ 	ᮬ	ᮭ	ᮮ	ᮯ
\ublock{U+1BBx}	᮰	᮱	᮲	᮳	᮴	᮵	᮶	᮷	᮸	᮹	ᮺ	ᮻ	ᮼ	ᮽ	ᮾ	ᮿ

᮱ {\arial= 1}	᮲ {\arial= 2}	᮳{\arial = 3}
᮴ {\arial= 4}	᮵ {\arial = 5} 	᮶ {\arial= 6}
᮷ {\arial= 7}	᮸ {\arial= 8}	᮹ {\arial= 9}
᮰ {\arial= 0}

\begin{scriptexample}[]{Sundanese}
\bgroup
\sundanese
\centering

◌ᮃᮄᮅᮆᮇᮈᮉᮊᮋᮌᮍᮎᮏᮐᮕᮔᮓᮑᮖᮗᮚᮛᮜᮝᮞᮟᮠᮠ
\egroup
\end{scriptexample}