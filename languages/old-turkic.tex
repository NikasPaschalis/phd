\subsection{Old Turkic}

\newfontfamily\oldturkic{Segoe UI Symbol}
\begin{scriptexample}[]{Old Turkish}
\oldturkic
\obeylines
Orkhon	Yenisei
variants	Transliteration / transcription
Old Turkic letter  𐰀	𐰁 𐰂	a, ä
Old Turkic letter  𐰃	𐰄 𐰅	y, i (e)
Old Turkic letter  𐰆		o, u
Old Turkic letter  𐰇	𐰈	ö, ü

	0	1	2	3	4	5	6	7	8	9	A	B	C	D	E	F
U+10C0x	𐰀	𐰁	𐰂	𐰃	𐰄	𐰅	𐰆	𐰇	𐰈	𐰉	𐰊	𐰋	𐰌	𐰍	𐰎	𐰏
U+10C1x	𐰐	𐰑	𐰒	𐰓	𐰔	𐰕	𐰖	𐰗	𐰘	𐰙	𐰚	𐰛	𐰜	𐰝	𐰞	𐰟
U+10C2x	𐰠	𐰡	𐰢	𐰣	𐰤	𐰥	𐰦	𐰧	𐰨	𐰩	𐰪	𐰫	𐰬	𐰭	𐰮	𐰯
U+10C3x	𐰰	𐰱	𐰲	𐰳	𐰴	𐰵	𐰶	𐰷	𐰸	𐰹	𐰺	𐰻	𐰼	𐰽	𐰾	𐰿
U+10C4x	𐱀	𐱁	𐱂	𐱃	𐱄	𐱅	𐱆	𐱇	𐱈	

\hfill  Typeset with \texttt{Segoe UI Symbol} \cmd{\oldturkic} 
\end{scriptexample}

Irk Bitig or Irq Bitig (Old Turkic: {\bfseries\Large\oldturkic 𐰃𐰺𐰴 𐰋𐰃𐱅𐰃𐰏}), known as the Book of Omens or Book of Divination in English, is a 9th-century manuscript book on divination that was discovered in the "Library Cave" of the Mogao Caves in Dunhuang, China, by Aurel Stein in 1907, and is now in the collection of the British Library in London, England. The book is written in Old Turkic using the Old Turkic script (also known as "Orkhon" or "Turkic runes"); it is the only known complete manuscript text written in the Old Turkic script.[1] It is also an important source for early Turkic mythology.

The Old Turkic text does not have any sentence punctuation, but uses two black lines in a red circle as a word separation mark in order to indicate word boundaries as shown in Figure~{\ref{omen}}

\begin{figure}[htb]
\includegraphics[width=0.7\textwidth]{./images/omen.jpg}
\caption{Omen 11 (4-4-3 dice) of the Irk Bitig (folio 13a): "There comes a messenger on a yellow horse (and) an envoy on a dark brown horse, bringing good tidings, it says. Know thus: (The omen) is extremely good."}
\label{omen}
\end{figure}