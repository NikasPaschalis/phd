\section{Myanmar}
\index{Myanmar}\index{Burmese}\index{Mon}\index{Unicode!Myanmar}\index{Fonts!Padauk}

\newfontfamily\myanmar{Padauk}

The Burmese script (Burmese:{\myanmar မြန်မာအက္ခရာ}; MLCTS: mranma akkha.ra; pronounced: [mjəmà ʔɛʔkʰəjà]) is an abugida in the Brahmic family, used for writing Burmese. It is an adaptation of the Old Mon script[2] or the Pyu script. In recent decades, other alphabets using the Mon script, including Shan and Mon itself, have been restructured according to the standard of the now-dominant Burmese alphabet. Besides the Burmese language, the Burmese alphabet is also used for the liturgical languages of Pali and Sanskrit.

The characters are rounded in appearance because the traditional palm leaves used for writing on with a stylus would have been ripped by straight lines.[3] It is written from left to right and requires no spaces between words, although modern writing usually contains spaces after each clause to enhance readability.

The earliest evidence of the Burmese alphabet is dated to 1035, while a casting made in the 18th century of an old stone inscription points to 984.[1] Burmese calligraphy originally followed a square format but the cursive format took hold from the 17th century when popular writing led to the wider use of palm leaves and folded paper known as parabaiks.[3] The alphabet has undergone considerable modification to suit the evolving phonology of the Burmese language.

Mon/Burmese script was added to the Unicode Standard in September, 1999 with the release of version 3.0. It was extended in October, 2009 with the release of version 5.2 and again in June, 2014 with the release of version 7.0.

\begin{scriptexample}[]{Myanmar}
\unicodetable{myanmar}{"1000,"1010,"1020,"1030,"1040,"1050,"1060,"1070,"1080,"1090}
\end{scriptexample}






