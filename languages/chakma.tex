
\section{Chakma}
\label{s:chakma}

\newfontfamily\chakma{RibengUni.ttf}

The Chakma alphabet (Ajhā pāṭh), also called Ojhapath, Ojhopath, Aaojhapath, is an abugida used for the Chakma language and which is being adapted for the Tanchangya language.[1] The forms of the letters are quite similar to those of the Burmese script.

\bgroup
\obeylines
\chakma
𑄇𑄳𑄇 Kkā = 𑄇 Kā + 𑄳 VIRAMA + 𑄇 Kā
𑄇𑄳𑄑 Ktā = 𑄇 Kā + 𑄳 VIRAMA + 𑄑 Tā
𑄇𑄳𑄖 Ktā = 𑄇 Kā + 𑄳 VIRAMA + 𑄖 Tā
𑄇𑄳𑄟 Kmā = 𑄇 Kā + 𑄳 VIRAMA + 𑄟 Mā
𑄇𑄳𑄌 Kcā = 𑄇 Kā + 𑄳 VIRAMA + 𑄌 Cā
𑄋𑄳𑄇 ńkā = 𑄋 ńā + 𑄳 VIRAMA + 𑄇 Kā
𑄋𑄳𑄉 ńkā = 𑄋 ńā + 𑄳 VIRAMA + 𑄉 Gā
𑄌𑄳𑄌 ccā = 𑄌 cā + 𑄳 VIRAMA + 𑄌 Cā

\egroup

Fonts for the script are not available in general but the
the script can be typeset using \texttt{RibengUni.ttf} which is available at \url{http://uni.hilledu.com/}. 

\begin{scriptexample}[]{Chakma}
\unicodetable{chakma}{"11100,"11110,"11120,"11130,"11140}
\end{scriptexample}


\printunicodeblock{./languages/chakma.txt}{\chakma}
