\subsection{Tifnagh}

\newfontfamily\tifinagh{code2000.ttf}

Tifinagh (Berber pronunciation: [tifinaɣ]; also written Tifinaɣ in the Berber Latin alphabet, {\tifinagh  ⵜⵉⴼⵉⵏⴰⵖ} in Neo-Tifinagh, and تيفيناغ in the Berber Arabic alphabet) is a series of abjad and alphabetic scripts used by Berber peoples to write Berber languages.[1]
A modern derivate of the traditional script, known as Neo-Tifinagh, was introduced in the 20th century. A slightly modified version of the traditional script, called Tifinagh Ircam, is used in a number of Moroccan elementary schools in teaching the Berber language to children as well as a number of publications.[2][3]

The word tifinagh is thought to be a Berberized feminine plural cognate of Punic, through the Berber feminine prefix ti- and Latin Punicus; thus tifinagh could possibly mean "the Phoenician (letters)"[4][5] or "the Punic letters".

\bgroup

\noindent\tifinagh
\colorbox{graphicbackground}{\color{black}^^A
\begin{minipage}{\textwidth}
\parindent1pt
\vskip10pt
\leftskip10pt \rightskip\leftskip
Tifnagh     ⵜⵉⴼⵉⵏⴰⵖ [U+2D30-U+2D7F]

ⴰⴳⵍⴷⵓⵏ ⴰⵎⵥⵥⴰ

ⵙ ⵡⴰⵡⴰⵍ ⴳⵔⵉ ⵉⴷⵙ, ⵙⵙⵏⵖ ⵢⴰⵜ ⵜⵖⴰⵡⵙⴰ ⵜⵉⵙⵙ ⵙⵏⴰⵜ  ⵉⵅⴰⵜⵔⵏ: ⵉⵜⵔⵉ ⵙⴳ ⴷⴷ ⵉⴷⴷⴰ ⵓⵔ ⵉⵎⵇⵇⵓⵔ, ⵉⵍⵍⴰ ⵖⴰⵙ ⴰⵏⵛⵜ ⵏ ⵢⴰⵜ ⵜⴰⴷⴷⴰⵔⵜ !

ⴰⵢⴰ ⵓⴽⵣⵖ ⵜ. ⵙⵙⵏⵖ ⵉⵙ ⴱⵕⵕⴰ ⵏ ⵉⵜⵔⴰⵏ ⵣⵓⵏⴷ ⴰⴽⴰⵍ, ⵊⵓⴱⵉⵜⵔ, ⵎⴰⵔⵙ, ⴱⵉⵏⵓⵙ – ⵉⵜⵔⴰⵏ ⵎⵉ ⵏⴽⴼⴰ ⵉⵙⵎⴰⵡⵏ – ⵍⵍⴰⵏ ⴷⵉⵖ ⵉⵜⵔⴰⵏ ⵢⴰⴹⵏ ⵎⵥⵥⵉⵢⵏⵉⵏ, ⵡⵉⵏⵏⴰ ⵓⵔ ⵏⵣⵎⵉⵔ ⴰⴷ ⵏⵥⵔ ⵙ ⵓⵜⵉⵍⵉⵙⴽⵓⴱ. ⴰⴷⴷⴰⵢ ⵢⵓⴼⴰ ⵓⴰⵙⵜⵕⵓⵏⵓⵎ ⵢⴰⵏ ⴷⵉⴳⵙⵏ, ⴷⴰ ⵢⴰⵙ ⵉⵜⵜⴳⴰ ⵙ ⵢⵉⵙⵎ ⵢⴰⵏ ⵡⵓⵜⵜⵓⵏ. ⴷⴰ ⵢⴰⵙ ⵉⵇⵇⴰⵔ ⵙ ⵓⵎⴷⵢⴰⵜ : « ⴰⵙⵜⵔⵓⵉⴷ 3251 ».

ⵓⴽⵣⵖ ⵉⵙ ⴷⴷ ⵉⴷⴷⴰ ⵓⴳⵍⴷⵓⵏ ⵎⵥⵥⵉⵢⵏ ⵙⴳ ⵉⵜⵔⵉ ⵎⵉ ⵇⵇⴰⵔⵏ ⴰⵙⵜⵔⵓⵉⴷ ⴱ612. ⴰⵙⵜⵔⵓⵉⴷ ⴰ, ⵓⵔ ⵉⵜⵓⵥⵔⴰ ⴰⵔ 1909 ⵙ ⵓⵜⵉⵍⵉⵙⴽⵓⴱ. ⵉⵥⵔⴰ ⵜ ⵢⴰⵏ ⵓⴰⵙⵜⵕⵓⵏⵓⵎ ⴰⵜⵓⵔⴽⵉⵢ. ⵉⵙⵙⴽⵏ ⵜⵓⴼⴰⵢⵜ ⵏⵏⵙ ⴳ ⵢⴰⵏ ⵓⴳⵔⴰⵡ ⴰⴳⵔⴰⵖⵍⴰⵏ ⵏ ⵍⴰⵙⵜⵕⵓⵏⵓⵎⵢ. ⵎⴰⵛⴰ, ⴰⴽⴷ ⵢⵉⵡⵏ ⵓⵔ ⵜ ⵢⵓⵎⵏ ⴰⵛⴽⵓ ⵉⵍⵍⴰ ⵉⵍⵙⴰ ⵢⴰⵜ ⵎⵍⵙⵉⵡⵜ ⵓⵔ ⵉⴳⵉⵏ ⴰⵎⵎ ⵜⵉⵏ ⵎⴷⴷⵏ. ⵎⴷⴷⵏ ⵉⵎⵇⵔⴰⵏⴻⵏ, ⴰⵎⴽⴰ ⴰⴽⴽ ⴰⵢ ⴳⴰⵏ.

ⵎⴰⵛⴰ ⵙ ⵓⵎⴷⴰⵣ ⵏ ⵜⵓⵙⵙⵏⴰ ⵏ ⴰⵙⵜⵔⵓⵉⴷ ⴱ612, ⵉⴽⴽⵔ ⵢⴰⵏ ⵓⴷⵉⴽⵜⴰⵜⵓⵔ ⴰⵜⵓⵔⴽⵢ, ⵉⴳⴳ ⴰⵙⵏ ⵛⵛⵉⵍ ⵉ ⵎⴷⴷⵏ ⴰⴷ ⵍⵙⵙⴰⵏ ⵎⵍⵙⵉⵡⵜ ⵏ ⵓⵔⵓⴱⵉⵢⵏ, ⵡⴰⵏⵏⴰ ⵢⴰⴳⵉⵏ ⵉⵏⵖ ⵜ. ⴰⵙⵜⵔⵓⵏⵓⵎ ⵏⵏⴰⵖ, ⵢⵓⵍⵙ ⴷⵉⵖ ⵉ ⵜⵎⵙⴽⴰⵏⵜ ⵏⵏⵙ ⴰⵙⴳⴳⴰⵙ ⵏ 1920, ⵜⵉⴽⴽⵍⵜ ⵏⵏⴰⵖ ⵉⵍⵍⴰ ⵉⵍⵙⴰ ⵢⴰⵜ ⵎⵍⵙⵉⵡⵜ ⵢⵖⵓⴷⴰⵏ ⵛⵉⴳⴰⵏ. ⵜⵉⴽⴽⵍⵜ ⵏⵏⴰⵖ, ⵎⴷⴷⵏ ⴰⴽⴽ ⵓⵎⴻⵏ ⴰⵡⴰⵍ ⵏⵏⵙ.
\par
\vspace*{10pt}
\end{minipage}
}
