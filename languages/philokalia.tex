%\documentclass{book}
%\usepackage{phd}
%\usepackage{philokalia}
%\begin{document}

\subsection{Philokalia}

The \pkgname{philokalia} package by Apostolos Syropoulos provides a Greek font in the style of the Philokalia manuscripts. The package modifies the lettrine package, which we cater for in the \pkgname{phd} and hence we adjusted it slightly for this. Also the package needed some modifications to work with LuaTeX.

The Philokalia (Ancient Greek: φιλοκαλία "love of the beautiful, the good", from φιλία philia "love" and κάλλος kallos "beauty") is "a collection of texts written between the 4th and 15th centuries by spiritual masters"[1] of the Eastern Orthodox hesychast tradition. They were originally written for the guidance and instruction of monks in "the practise of the contemplative life".[2] The collection was compiled in the eighteenth-century by St. Nikodemos of the Holy Mountain and St. Makarios of Corinth.

Although these works were individually known in the monastic culture of Greek Orthodox Christianity before their inclusion in The Philokalia, their presence in this collection resulted in a much wider readership due to its translation into several languages. The earliest translations included a Church Slavonic translation of selected texts by Paisius Velichkovsky (Dobrotolublye) in 1793, a Russian translation by Ignatius Bryanchaninov in 1857, and a five-volume translation into Russian (Dobrotolyubie) by St. Theophan the Recluse in 1877.

There were subsequent Romanian, Italian and French translations.[3][4]
The book is a "principal spiritual text" for all the Eastern Orthodox Churches;[5] the publishers of the current English translation state that "The Philokalia has exercised an influence far greater than that of any book other than the Bible in the recent history of the Orthodox Church."[6]
Philokalia (sometimes Philocalia) is also the name given to an anthology of the writings of Origen compiled by Saint Basil the Great and Saint Gregory Nazianzus. Other works on monastic spirituality have also used the same title over the years.[5][7]

The Philokalia fonts consist of three fonts: one that contains
the normal typeface, one that contains the ligatures and one that contains the special ornament characters that decorate the beginning of each chapter. The glyphs were generated from scanned images of the book pages and Apostolos Syropoulos described the process in detail in \cite{syropoulos}. 


{
%\newfontfamily\plk{Philokalia-Regular}
\plk
%\newfontfamily\PHtitl[Script=Greek,RawFeature=+titl;grek]{Philokalia-Regular}
 %\font\PHtitl="[Philokalia-Regular]/ICU:script=grek,+titl"

 
 \lettrine[lines=3]{\usebox{\philobox}}{ερὶ} ποιητικῆς αὐτῆς τε καὶ τῶν εἰδῶν αὐτῆς, ἥν τινα δύναμιν ἕκαστον ἔχει, 
καὶ πῶς δεῖ συνίστασθαι τοὺς μύθους  εἰ μέλλει καλῶς ἕξειν ἡ ποίησις, ἔτι δὲ ἐκ πόσων καὶ ποίων 
ἐστὶ μορίων, ὁμοίως δὲ καὶ περὶ τῶν ἄλλων ὅσα τῆς αὐτῆς ἐστι μεθόδου, λέγωμεν ἀρξάμενοι κατὰ φύσιν 
πρῶτον ἀπὸ τῶν πρώτων.
 
Ἐποποιία δὴ καὶ ἡ τῆς τραγῳδίας ποίησις ἔτι δὲ κωμῳδία καὶ ἡ διθυραμβοποιητικὴ καὶ τῆς αὐλητικῆς 
ἡ πλείστη καὶ κιθαριστικῆς πᾶσαι τυγχάνουσιν οὖσαι μιμήσεις τὸ σύνολον· διαφέρουσι δὲ ἀλλήλων τρισίν, 
ἢ γὰρ τῷ ἐν ἑτέροις μιμεῖσθαι ἢ τῷ ἕτερα ἢ τῷ ἑτέρως καὶ μὴ τὸν αὐτὸν τρόπον. 

Ὥσπερ γὰρ καὶ χρώμασι καὶ σχήμασι πολλὰ μιμοῦνταί τινες ἀπεικάζοντες (οἱ μὲν [20] διὰ τέχνης οἱ δὲ διὰ συνηθείας),
ἕτεροι δὲ διὰ τῆς φωνῆς, οὕτω κἀν ταῖς εἰρημέναις τέχναις ἅπασαι μὲν ποιοῦνται τὴν μίμησιν ἐν ῥυθμῷ καὶ λόγῳ καὶ
ἁρμονίᾳ, τούτοις δ᾽ ἢ χωρὶς ἢ μεμιγμένοις· οἷον ἁρμονίᾳ μὲν καὶ ῥυθμῷ χρώμεναι μόνον ἥ τε αὐλητικὴ καὶ ἡ κιθαριστικὴ
κἂν εἴ τινες [25] ἕτεραι τυγχάνωσιν οὖσαι τοιαῦται τὴν δύναμιν, οἷον ἡ τῶν συρίγγων, αὐτῷ δὲ τῷ ῥυθμῷ [μιμοῦνται]
χωρὶς ἁρμονίας ἡ τῶν ὀρχηστῶν (καὶ γὰρ οὗτοι διὰ τῶν σχηματιζομένων ῥυθμῶν μιμοῦνται καὶ ἤθη καὶ πάθη καὶ πράξεις)· 
 }

The package also modifies the \pkgname{lettrine} package and hence we have modified the \cmd{\lettrine} command to be called \cmd{\lettrinephilokalia} when used with the |philokalia| package. It is a bit long as a command, but easier to remember. 

