\section{Arabic}

\newfontfamily\arabian
    [Script=Arabic,        % to get correct arabic shaping
     Scale=1.2]            % make the arabic font bigger, a matter of taste
    {Scheherazade}         % whatever Arabic font you like
\newcommand{\textarabic}[1] % Arabic inside LTR
           {\bgroup\luatextextdir TRT\arabian #1\egroup}
\newcommand{\narabic}         [1] % for digits inside Arabic text
           {\bgroup\luatextextdir TLT #1\egroup}
\newcommand{\afootnote} [1] % Arabic footnotes
           {\footnote{\textarabic{#1}}}
\newenvironment{Arabic}     % Arabic paragraph
           {\luatextextdir TRT\luatexpardir TRT\arabicfont}{}
The Arabic script is a writing system used for writing several languages of Asia and Africa, such as Arabic, Sorani and Luri Dialects of Kurdish language, Persian, Pashto and Urdu.[1] Even until the 16th century, it was used to write some texts in Spanish.[2] After the Latin script, Chinese characters, and Devanagari, it is the fourth-most widely used writing system in the world.[3]
The Arabic script is written from right to left in a cursive style. In most cases the letters transcribe consonants, or consonants and a few vowels, so most Arabic alphabets are abjads.

The script was first used to write texts in Arabic, most notably the Qurʼān, the holy book of Islam. With the spread of Islam, it came to be used to write languages of many language families, leading to the addition of new letters and other symbols, with some versions, such as Kurdish, Uyghur, and old Bosnian being abugidas or true alphabets. It is also the basis for a rich tradition of Arabic calligraphy.


\begin{Arabic}
\begin{verbatim}
ّ هو إذ الغاية؛ شريف الفوائد، جم المذهب، عزيز فنّ التاريخ فنّ أنّ اعلم
والملوك سيرهم، في والأنبياء أخلاقهم، في الأمم من الماضين أحوال على يوقفنا
ّ أحوال في يرومه لمن ذلك في الإقتداء فائدة تتم حتّى وسياستهم؛ دولهم في
والدنيا. الدين

\end{verbatim}
\end{Arabic}





As of Unicode 7.0, the Arabic script is contained in the following blocks:
Arabic (0600—06FF, 255 characters)
Arabic Supplement (0750—077F, 48 characters)
Arabic Extended-A (08A0—08FF, 39 characters)
Arabic Presentation Forms-A (FB50—FDFF, 608 characters)
Arabic Presentation Forms-B (FE70—FEFF, 140 characters)
Rumi Numeral Symbols (10E60—10E7F, 31 characters)
Arabic Mathematical Alphabetic Symbols (1EE00—1EEFF, 143 characters)[1][2]

The basic Arabic range encodes the standard letters and diacritics, but does not encode contextual forms (U+0621–U+0652 being directly based on ISO 8859-6); and also includes the most common diacritics and Arabic-Indic digits. The Arabic Supplement range encodes letter variants mostly used for writing African (non-Arabic) languages. The Arabic Extended-A range encodes additional Qur'anic annotations and letter variants used for various non-Arabic languages. The Arabic Presentation Forms-A range encodes contextual forms and ligatures of letter variants needed for Persian, Urdu, Sindhi and Central Asian languages. The Arabic Presentation Forms-B range encodes spacing forms of Arabic diacritics, and more contextual letter forms. The presentation forms are present only for compatibility with older standards, and are not currently needed for coding text.[3] 

The Arabic Mathematical Alphabetical Symbols block encodes characters used in Arabic mathematical expressions.


Position in word:	Isolated	Final	Medial	Initial
Glyph form:\scalebox{3}[3]{ب}{ـب}‎	ـبـ‎	 \scalebox{3}{بـ}


\printunicodeblock[2]{./languages/arabic.txt}{\arabian}



