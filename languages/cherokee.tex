
\subsection{Cherokee}

Windows comes with |Plantagenet Cherokee| font. The |code2000| also has good support for the alphabet. The \texttt{SIL font Charis SIL} also has good support and can be downloaded at \href{http://scripts.sil.org/cms/scripts/page.php?item_id=CharisSIL_download}{scripts.sel.org}, the latest version gave me problems when used with Windows. 



\def\cherokeedefaultfont#1{%
  \newfontfamily\cherokee[Language=Cherokee,Scale=MatchLowercase]{#1}%
  \ifcsname textcherokee\endcsname\relax
  \else
    \csname textcherokee\endcsname
    \maketextfontcommand{\textcherokee}{\cherokee}
  \fi
}

^^A only preamble error - which is disabled
\def\maketextfontcommand#1#2{
   \DeclareTextFontCommand{#1}{#2}
}


\cxset{cherokee font/.code=\cherokeedefaultfont{#1}}

\cxset{language=cherokee, 
       cherokee font = FreeSerif}

\cxset{language=cherokee, 
       cherokee font = Plantagenet Cherokee}


\begin{key}{/chapter/cherokee font = \meta{font name}} Loads the font
command \cmd{\cherokee}. When the command is used it typesets text in
cherokee unicode. There is no need to load the language, unless it is the main document language. For windows the default font is  |Plantagenet Cherokee|. Another font is FreeSerif, which we are using here.
\end{key}


{\cherokee
\begin{tabular}{lp{10cm}}
Translation	  &John (ᏣᏂ) 3:16\\
American Bible Society 1860	&ᎾᏍᎩᏰᏃ ᏂᎦᎥᎩ ᎤᏁᎳᏅᎯ ᎤᎨᏳᏒᎩ ᎡᎶᎯ, ᏕᏅᏲᏒᎩ ᎤᏤᎵᎦ ᎤᏪᏥ ᎤᏩᏒᎯᏳ ᎤᏕᏁᎸᎯ, ᎩᎶ ᎾᏍᎩ ᏱᎪᎯᏳᎲᏍᎦ ᎤᏲᎱᎯᏍᏗᏱ ᏂᎨᏒᎾ, ᎬᏂᏛᏉᏍᎩᏂ ᎤᏩᏛᏗ.\\

(Transliteration)	&\titus nasgiyeno nigavgi unelanvhi ugeyusvgi elohi, denvyosvgi utseliga uwetsi uwasvhiyu udenelvhi, gilo nasgi yigohiyuhvsga uyohuhisdiyi nigesvna, gvnidvquosgini uwadvdi.\\
\end{tabular}}

\bgroup
\cherokee \Large\textbf{ᎾᏍᎩᏰᏃ}

\textcherokee{ᎾᏍᎩᏰᏃ}
\egroup

If you have trouble getting them to work\footnote{\url{http://tex.stackexchange.com/questions/132087/displaying-cherokee-text}}
