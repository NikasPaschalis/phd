\subsection{Ugaritic}

\newfontfamily\ugaritic{NotoSansUgaritic-Regular.ttf}

The Ugaritic script is a cuneiform (wedge-shaped) abjad used from around either the fifteenth century BCE[1] or 1300 BCE[2] for Ugaritic, an extinct Northwest Semitic language, and discovered in Ugarit (modern Ras Shamra), Syria, in 1928. It has 30 letters. Other languages (particularly Hurrian) were occasionally written in the Ugaritic script in the area around Ugarit, although not elsewhere.
Clay tablets written in Ugaritic provide the earliest evidence of both the North Semitic and South Semitic orders of the alphabet, which gave rise to the alphabetic orders of Arabic (starting with the earliest order of its abjad), the reduced Hebrew, and more distantly the Greek and Latin alphabets on the one hand, and of the Ge'ez alphabet on the other. Arabic and Old South Arabian are the only other Semitic alphabets which have letters for all or almost all of the 29 commonly reconstructed proto-Semitic consonant phonemes. 

According to Dietrich and Loretz in Handbook of Ugaritic Studies (ed. Watson and Wyatt, 1999): "The language they [the 30 signs] represented could be described as an idiom which in terms of content seemed to be comparable to Canaanite texts, but from a phonological perspective, however, was more like Arabic."
The script was written from left to right. Although cuneiform and pressed into clay, its symbols were unrelated to those of the Akkadian cuneiform.

\begin{scriptexample}[]{Ugaritic}
\unicodetable{ugaritic}{"10380,"10390}
\end{scriptexample}