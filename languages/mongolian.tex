\subsection{Mongolian Script}

\newfontfamily\mongolian[Language=Mongolian, Scale=1.3]{code2000.ttf}

The classical Mongolian script (in Mongolian script: {\mongolian  ᠮᠣᠩᠭᠣᠯ ᠪᠢᠴᠢᠭ᠌} Mongγol bičig; in Mongolian Cyrillic: Монгол бичиг Mongol bichig), also known as Uyghurjin Mongol bichig, was the first writing system created specifically for the Mongolian language, and was the most successful until the introduction of Cyrillic in 1946. Derived from Uighur, Mongolian is a true alphabet, with separate letters for consonants and vowels. The Mongolian script has been adapted to write languages such as Oirat and Manchu. Alphabets based on this classical vertical script are used in Inner Mongolia and other parts of China to this day to write Mongolian, Sibe and, experimentally, Evenki.
\medskip

\bgroup\par
\noindent
\colorbox{graphicbackground}{\color{black}^^A
\begin{minipage}{\textwidth}^^A
\parindent1pt
\vskip10pt
\leftskip10pt \rightskip\leftskip
\mongolian
\large
ᠬᠦᠮᠦᠨ ᠪᠦᠷ ᠲᠥᠷᠥᠵᠦ ᠮᠡᠨᠳᠡᠯᠡᠬᠦ ᠡᠷᠬᠡ ᠴᠢᠯᠥᠭᠡ ᠲᠡᠢ᠂ ᠠᠳᠠᠯᠢᠬᠠᠨ ᠨᠡᠷ᠎ᠡ ᠲᠥᠷᠥ ᠲᠡᠢ᠂ ᠢᠵᠢᠯ ᠡᠷᠬᠡ ᠲᠡᠢ ᠪᠠᠢᠠᠭ᠃ ᠣᠶᠤᠨ ᠤᠬᠠᠭᠠᠨ᠂ ᠨᠠᠨᠳᠢᠨ ᠴᠢᠨᠠᠷ ᠵᠠᠶᠠᠭᠠᠰᠠᠨ ᠬᠦᠮᠦᠨ ᠬᠡᠭᠴᠢ ᠥᠭᠡᠷ᠎ᠡ ᠬᠣᠭᠣᠷᠣᠨᠳᠣ᠎ᠨ ᠠᠬᠠᠨ ᠳᠡᠭᠦᠦ ᠢᠨ ᠦᠵᠢᠯ ᠰᠠᠨᠠᠭᠠ ᠥᠠᠷ ᠬᠠᠷᠢᠴᠠᠬᠥ ᠤᠴᠢᠷ ᠲᠠᠢ᠃
\par
\vspace*{10pt}
\end{minipage}
}
\medskip