\chapter{South East Asian Scripts}

\section{Introduction}

This section documents the facilities offered to typeset Southeast Asian Scripts. These scripts are used in most of Southeast Asia, Indonesia and the Philippines.


\begin{table}[htb]
\centering
\begin{tabular}{lll}
Thai. & Tai Tham &Balinese.\\
\nameref{s:lao}  &Tai Viet  &Javanese.\\
Myanmar &Kayah Li &Rejang\\
Khmer. &Cham &Batak\\
Tai Le &Philippine Scripts &\nameref{s:undanese}\\
\nameref{s:newtailue} & Buginese\\
\end{tabular}
\end{table}
\parindent1em
\section{Lao}
\label{s:lao}
\def\laotext#1{{\lao#1}}

The Lao alphabet, Akson Lao (Lao: \laotext{ອັກສອນລາວ} [ʔáksɔ̌ːn láːw]), is the main script used to write the Lao language and other minority languages in Laos. It is ultimately of Indic origin, the alphabet includes 27 consonants (\laotext{ພະຍັນຊະນະ} [pʰāɲánsānā]), 7 consonantal ligatures (\laotext{ພະຍັນຊະນະປະສົມ} [pʰāɲánsānā pá sǒm]), 33 vowels (\laotext{ສະຫລະ} [sálā]) (some based on combinations of symbols), and 4 tone marks (\laotext{ວັນນະຍຸດ} [ván nā ɲūt]). 



According to Article 89 of Amended Constitution of 2003 of the Lao People's Democratic Republic, the Lao alphabet is the official script to the official language, but is also used to transcribe minority languages in the country, but some minority language speakers continue to use their traditional writing systems while the Hmong have adopted the Roman Alphabet.[1] An older version of the script was also used by the ethnic Lao of Thailand's Isan region, who make up a third of Thailand's population, before Isan was incorporated into Siam, until its use was banned and supplemented with the very similar Thai alphabet in 1871, although the region remained distant culturally and politically until further government campaigns and integration into the Thai state (Thaification) were imposed in the 20th century.[2] The letters of the Lao Alphabet are very similar to the Thai alphabet, which has the same roots. They differ in the fact, that in Thai there are still more letters to write one sound and the more circular style of writing in Lao.

Lao, like most indic scripts, is traditionally written from left to right. Traditionally considered an \emph{abugida} script, where certain 'implied' vowels are unwritten, recent spelling reforms make this definition somewhat problematic, as all vowel sounds today are marked with diacritics when written according the Lao PDR's propagated and promoted spelling standard. However most Lao outside of Laos, and many inside Laos, continue to write according to former spelling standards, which continues the use of the implied vowel maintaining the Lao script's status as an \emph{abugida}. Vowels can be written above, below, in front of, or behind consonants, with some vowel combinations written before, over and after. Spaces for separating words and punctuations were traditionally not used, but a space is used and functions in place of a comma or period. The letters have no \emph{majuscule} or \emph{minuscule} (upper and lower case) differentiations

The Unicode block for the Lao script is U+0E80–U+0EFF, added in Unicode version 1.0. The first 10 characters of the row U+0EDx are the Lao numerals 0 through 9. Throughout the chart grey (unassigned) code points are shown, because the assigned Lao characters intentionally match the relative positions of the corresponding Thai characters. This has created the anomaly that the Lao letter \laotext{ສ} is not in alphabetical order, since it occupies the same codepoint as the Thai letter \laotext{ส}.

\begin{scriptexample}[]{}
\unicodetable{lao}{"0E80,"0E90,"0EA0,"0EB0,"0EC0,"0ED0,"0EE0,"0EF0}
\end{scriptexample}

\subsubsection{Numerals}
\bgroup
\lao
\begin{tabular}{rllllllllllll}
Hindu-Arabic numerals	&0	&1	&2	&3	&4	&5	&6	&7	&8	&9	&10 &	20\\
Lao numerals	&໐	&໑	&໒	&໓	&໔	&໕	&໖	&໗	&໘	&໙	&໑໐	&໒໐\\
Lao names	&ສູນ	&ນຶ່ງ	&ສອງ	&ສາມ	&ສີ່	&ຫ້າ 	&ຫົກ	&ເຈັດ	&ແປດ	&ເກົ້າ	&ສິບ	&ຊາວ\\
\end{tabular}
\egroup




\subsection{Thai}

\newfontfamily\thai[Scale=1.0,Script=Thai]{IrisUPC}

\def\thaitext#1{{\thai#1}}

\begin{scriptexample}[]{Thai}
\centerline{\LARGE\thaitext{◌ะ; ◌ัวะ; เ◌ะ; เ◌อะ; เ◌าะ; เ◌ียะ; เ◌ือะ; แ◌ะ; โ◌ะ}}


\hfill Typeset with \idxfont{IrisUPC} and the command \cmd{\thai}
\end{scriptexample}
\subsection{Balinese}

The Balinese script, natively known as Aksara Bali and Hanacaraka, is an abugida used in the island of Bali, Indonesia, commonly for writing the Austronesian Balinese language, Old Javanese, and the liturgical language Sanskrit. With some modifications, the script is also used to write the Sasak language, used in the neighboring island of Lombok.[1] The script is a descendant of the Brahmi script, and so has many similarities with the modern scripts of South and Southeast Asia. The Balinese script, along with the Javanese script, is considered the most elaborate and ornate among Brahmic scripts of Southeast Asia.[2]

Though everyday use of the script has largely been supplanted by the Latin alphabet, the Balinese script has significant prevalence in many of the island's traditional ceremonies and is strongly associated with the Hindu religion. The script is mainly used today for copying lontar or palm leaf manuscripts containing religious texts.[2][3]

\newfontfamily\balinese{AksaraBali.ttf}
\newfontfamily\indicative{code2000.ttf}

{\indicative ◌ }

\newcounter{under}
\setcounter{under}{"1B00}

\def\cb#1 {
\hspace*{2.5pt}
 \large
 $\text{◌#1}_{\pgfmathparse{Hex(\theunder)}\pgfmathresult}$
\stepcounter{under}
\vskip5pt\par
}
\begin{scriptexample}[]{Balinese}


\balinese
	 
᭐	᭑	᭒	᭓	᭔	᭕	᭖	᭗	᭘	᭙	᭚	᭛	᭜	᭝	᭞	᭟\\\
 
\def\columnseprulecolor{\color{thegray}}
\columnseprule.4pt
\begin{multicols}{8}

\texttt{U+1B0x}	

\cb{ᬀ }  \cb{ ᬁ } 	\cb{ ᬂ } 	\cb ᬃ	\cb ᬄ 	\cb ᬅ	\cb ᬆ	\cb ᬇ	\cb ᬈ	\cb ᬉ	\cb ᬊ	\cb ᬋ	\cb ᬌ	\cb ᬍ	\cb ᬎ	\cb ᬏ

\columnbreak

\texttt{U+1B1x}	 

\cb ᬐ	 \cb ᬑ 	\cb ᬒ 	\cb ᬓ	\cb ᬔ	\cb ᬕ	\cb ᬖ \cb ᬗ 	\cb ᬘ 	\cb ᬙ 	\cb ᬚ	\cb ᬛ 	\cb ᬜ 	\cb ᬝ 	\cb ᬞ	\cb ᬟ 

\columnbreak

U+1B2x	 

\cb ᬠ◌ 	\cb ᬡ	\cb ᬢ	\cb ᬣ	\cb ᬤ	\cb ᬥ	\cb ᬦ	\cb ᬧ	\cb ᬨ	\cb ᬩ	\cb ᬪ	\cb ᬫ	\cb ᬬ	\cb ᬭ	\cb ᬮ	\cb ᬯ

\columnbreak
U+1B3x 

\cb ᬰ	\cb ᬱ	\cb ᬲ	\cb ᬳ	\cb ᬴	\cb ᬵ	\cb ᬶ	\cb ᬷ	\cb ᬸ	\cb ᬹ	\cb ᬺ	\cb ᬻ	\cb ᬼ	\cb ᬽ	\cb ᬾ	\cb ᬿ


\columnbreak
U+1B4x	 

\cb ᭀ	 \cb ᭁ	\cb ᭂ	\cb ᭃ	\cb ᭄	\cb ᭅ	\cb ᭆ	\cb ᭇ	\cb ᭈ	\cb ᭉ	\cb ᭊ	\cb ᭋ

\columnbreak				
U+1B5x	 

\cb ᭐	\cb ᭑	\cb ᭒	\cb ᭓	\cb ᭔	\cb ᭕	\cb ᭖	\cb ᭗	\cb ᭘	\cb ᭙	\cb ᭚	\cb ᭛	\cb ᭜	\cb ᭝	\cb ᭞	\cb ᭟\\

\columnbreak

U+1B6x 

\cb ᭠	\cb ᭡	\cb ᭢	\cb ᭣	\cb ᭤	\cb ᭥	\cb ᭦	\cb ᭧	\cb ᭨◌ 	\cb ᭩◌ 	\cb ᭪◌ 	\cb ᭫	\cb ᭬	\cb ᭭	\cb ᭮	\cb ᭯

\columnbreak
U+1B7x	 

\cb ᭰	 \cb ᭱  \cb ᭲  \cb ᭳	 \cb ᭴	\cb ᭵	\cb ᭶	\cb ᭷	\cb ᭸	\cb ᭹	\cb ᭺	\cb ᭻	\cb ᭼


\end{multicols}

\end{scriptexample}
\defaulttext

One of the most comprehensive fonts is Aksara Bali\footnote{\url{http://www.alanwood.net/downloads/index.html}}. This is obtainable at Alan Wood's website.
\clearpage

%\subsection{Javanese}
\index{scripts>Javanese}

The Javanese script (Hanacaraka/Carakan) is a script for writing the Javanese language, the native language of one of the peoples of the Island of Java. It is a descendent of the ancient Brahmi script of India, and so has many similarities with modern scripts of South Asia and Southeast Asia. The Javanese script is also used for writing Sanskrit, Old Javanese, and transcriptions of Kawi, as well as the Sundanese language, and the Sasak language.

Javanese script was added to Unicode Standard in version 5.2 on the code points \texttt{A980 - A9DF}.


\newfontfamily\javanesei{NotoSansJavanese-Regular.ttf}

\newfontfamily\javanese{TuladhaJejeg_gr.ttf}

\begin{scriptexample}
\bgroup
\javanese

꧋ꦱꦧꦼꦤ꧀ꦮꦺꦴꦁꦏꦭꦲꦶꦂꦲꦏꦺꦏꦤ꧀ꦛꦶꦩꦂꦢꦶꦏꦭꦤ꧀ꦢꦂꦧꦺꦩꦂꦠꦧꦠ꧀ꦭꦤ꧀ꦲꦏ꧀ꦲꦏ꧀ꦏꦁꦥꦝ꧉

꧋ ꦲꦮꦶꦠ꧀ꦲꦶꦏꦁꦄꦱ꧀ꦩꦄꦭ꧀ꦭꦃ꧈ ꦏꦁꦩꦲꦩꦸꦫꦃꦠꦸꦂ ꦩꦲꦲꦱꦶꦃ꧉ 	 
 ۝꧋ ꦄꦭꦶꦥꦃ꧀ ꦭ ꦩ꧀ ꦫ ꧌ ꦏꦁ — — ꦥꦿꦶꦏ꧀ꦱ ꦏꦉꦪꦥ꧀ꦥꦩꦸꦁꦄꦭ꧀ꦭꦃꦥꦶꦪꦺꦩ꧀ꦧꦏ꧀ ꧌꧉ ꦩꦁꦪꦏꦴꦪꦤꦴ ꦲꦶꦏꦸꦄꦪꦺꦪꦠꦴꦏꦶꦠꦧ꧀ꦑꦸꦂꦄꦤ꧀ꦏꦁꦥꦿꦪꦠꦭ꧉ 	 
᭐	᭑	᭒	᭓	᭔	᭕	᭖	᭗	᭘	᭙	᭚	᭛	᭜	᭝	᭞	᭟

 
\egroup
\end{scriptexample}

As of the writing of this document (2014), there are several widely published fonts able to support Javanese, ANSI-based Hanacaraka/Pallawa by Teguh Budi Sayoga,[21] Adjisaka by Sudarto HS/Ki Demang Sokowanten,[22] JG Aksara Jawa by Jason Glavy,[23] Carakan Anyar by Pavkar Dukunov,[24] and Tuladha Jejeg by R.S. Wihananto,[25] which is based on Graphite (SIL) smart font technology. Other fonts with limited publishing includes Surakarta made by Matthew Arciniega in 1992 for Mac's screen font,[26] and Tjarakan developed by AGFA Monotype around 2000.[27] There is also a symbol-based font called Aturra developed by Aditya Bayu in 2012–2013.[28]

Due to the script's complexity, many Javanese fonts have different input method compared to other Indic scripts and may exhibit several flaws. JG Aksara Jawa, in particular, may cause conflicts with other writing system, as the font use code points from other writing systems to complement Javanese's extensive repertoire. This is to be expected, as the font was made before Javanese implementation in Unicode.[29]

Arguably, the most "complete" font, in terms of technicality and glyph count, is \texttt{Tuladha Jejeg}\footnote{\url{https://sites.google.com/site/jawaunicode/main-page}}. It comes with keyboard facilities, displaying complex syllable structure, and support extensive glyph repertoire including non-standard forms which may not be found in regular Javanese texts, by utilizing Graphite (SIL) smart font technology. |Tuladha Jejeg| uses variable stroke widths on its glyphs with serifs on some glyphs.

However, as not many writing systems require such complex feature, use is limited to programs with Graphite technology, such as Firefox browser, Thunderbird email client, and several OpenType word processor and of course XeLaTeX. The font was chosen for displaying Javanese script in the Javanese Wikipedia.[16]


\section{Khmer}
\newfontfamily\normaltext{Arial}
\normaltext

\newfontfamily\khmer[Scale=1.05]{NotoSansKhmer-Regular.ttf}
\def\khmertext#1{{\khmer#1}}

\begin{key}{/chapter/khmer font=\meta{font name} (Khmer  UI)} Loads the font
command \cmd{\khmer}. When the command is used it typesets text in
khmer unicode. There is no need to load the language, unless it is the main document language. For windows the default font is \texttt{DaunPenh} this font is in general too small to read; a better font to use is Khmer UI.
\end{key}




The Khmer script (Khmer: {\Large\khmertext{អក្សរខ្មែរ}}; IPA: [ʔaʔsɑː kʰmaːe]) [2] is an \textit{abugida} (alphasyllabary) script used to write the Khmer language (the official language of Cambodia). It is also used to write Pali among the Buddhist liturgy of Cambodia and Thailand.

It was adapted from the Pallava script, a variant of Grantha alphabet descended from the Brahmi script of India, which was used in southern India and South East Asia during the 5th and 6th Centuries AD.[3] The oldest dated inscription in Khmer was found at Angkor Borei District in Takéo Province south of Phnom Penh and dates from 611.[4] The modern Khmer script differs somewhat from precedent forms seen on the inscriptions of the ruins of Angkor.

Not all Khmer consonants can appear in syllable-final position. The most common syllable-final consonants include {\khmer កងញតនបមល}. The pronunciation of the consonant in final position may differ from it's normal pronunciation.


\begin{tabular}{l l p{9cm}}
\khmertext{ំ}	&nĭkkôhĕt (\khmertext{និគ្គហិត})	&niggahita; nasalizes the inherent vowels and some of the dependent vowels, see anusvara, sometimes used to represent [aɲ] in Sanskrit loanwords\\

\khmertext{ះ}	&reăhmŭkh (\khmertext{រះមុខ})	&"shining face"; adds final aspiration to dependent or inherent vowels, usually omitted, corresponds to the visarga diacritic, it maybe included as dependent vowel symbol\\

\khmertext{ៈ}	&yŭkôleăkpĭntŭ (\khmertext{យុគលពិន្ទុ})	&yugalabindu ("pair of dots"); adds final glottalness to dependent or inherent vowels, usually omitted\\

\khmertext{៉}	 &musĕkâtônd (\khmertext{មូសិកទន្ត})	&mūsikadanta ("mouse teeth"); used to convert some o-series consonants (\khmertext{ង ញ ម យ រ វ}) to the a-series\\

\khmertext{៊}	&treisâpt (\khmertext{ត្រីសព្ទ})	&trīsabda; used to convert some a-series consonants (\khmertext{ស ហ ប អ}) to the o-series\\
\end{tabular}




ុ	kbiĕh kraôm (ក្បៀសក្រោម)	also known as bŏkcheung (បុកជើង); used in place of the diacritics treisâpt and musĕkâtônd when they would be impeded by superscript vowels
់	bântăk (បន្តក់)	used to shorten some vowels; the diacritic is placed on the last consonant of the syllable
៌	rôbat (របាទ)
répheăk (រេផៈ)	rapāda, repha; behave similarly to the tôndâkhéat, corresponds to the Devanagari diacritic repha, however it lost its original function which was to represent a vocalic r
 ៍	tôndâkhéat (ទណ្ឌឃាដ)	daṇḍaghāta; used to render some letters as unpronounced
៎	kakâbat (កាកបាទ)	kākapāda ("crow's foot"); more a punctuation mark than a diacritic; used in writing to indicate the rising intonation of an exclamation or interjection; often placed on particles such as /na/, /nɑː/, /nɛː/, /vəːj/, and the feminine response /cah/
៏	âsda (អស្តា)	denotes stressed intonation in some single-consonant words[5]
័	sanhyoŭk sannha (សំយោគសញ្ញា)	represents a short inherent vowel in Sanskrit and Pali words; usually omitted
៑	vĭréam (វិរាម)	a mostly obsolete diacritic, corresponds to the virāma
្	cheung (ជើង)	a.w. coeng; a sign developed for Unicode to input subscript consonants, appearance of this sign varies among fonts
\section{Sundanese}
\label{s:sundanese}



The Sundanese script (Aksara Sunda, {\sundanese ᮃᮊ᮪ᮞᮛ ᮞᮥᮔ᮪ᮓ}) is a writing system which is used by the Sundanese people. It is built based on Old Sundanese script (Aksara Sunda Kuno) which was used by the ancient Sundanese between the 14th and 18th centuries.

\begin{scriptexample}[]{Sundanese}
\unicodetable{sundanese}{"1B80,"1B90,"1BA0,"1BB0}

\sundanese
\obeylines
\bgroup
᮱ {\arial= 1}	᮲ {\arial= 2}	᮳{\arial = 3}
᮴ {\arial= 4}	᮵ {\arial = 5} 	᮶ {\arial= 6}
᮷ {\arial= 7}	᮸ {\arial= 8}	᮹ {\arial= 9}
᮰ {\arial= 0}

\egroup
\end{scriptexample}

\begin{scriptexample}[]{Sundanese}
\bgroup
\sundanese
\centering

◌ᮃᮄᮅᮆᮇᮈᮉᮊᮋᮌᮍᮎᮏᮐᮕᮔᮓᮑᮖᮗᮚᮛᮜᮝᮞᮟᮠᮠ


\egroup
\end{scriptexample}

\bgroup
\def\1{\sundanese ᮱}
\TextOrMath\1\1

$\1$
\egroup

In text In texts, numbers are written surrounded with dual pipe sign \textbar \ldots \textbar. Example: {\textbar \sundanese ᮲᮰᮱᮰\textbar} = 2010












%\section{Hanunó’o}

Hanunó’o is one of the indigenous scripts of the Philippines and is used by the Mangyan peoples of southern Mindoro to write the Hanunó'o language.[1] 
It is an \emphasis{abugida} descended from the Brahmic scripts, closely related to Baybayin, and is famous for being written vertical but written upward, rather than downward as nearly all other scripts (however, it's read horizontally left to right). It is usually written on bamboo by incising characters with a knife.[2][3] Most known Hanunó'o inscriptions are relatively recent because of the perishable nature of bamboo. It is therefore difficult to trace the history of the script

\newfontfamily\hanunoo[RawFeature={script=hano,vertical:}]{NotoSansHanunoo-Regular.ttf}
^^A%\fontspec[Scale=MatchLowercase]{Quivira}
\begin{scriptexample}[width=2cm]{Hanunoo}
\hanunoo

{\Large\bfseries
\obeylines
ᜠ 
ᜫ
ᜨᜲ
ᜫᜲ
ᜰ
ᜮ
ᜥ
ᜦ᜴}


\font\f="NotoSansHanunoo:script=hano,vertical"

^^A \f\rotatebox{-90}{᜵ ᜩᜤᜮᜥᜳᜨ᜴ ᜣᜨ᜴ᜫᜳᜥᜭᜨ᜴ ᜶ ᜦᜲᜦᜨᜨ᜴ᜦᜯᜲ}


ᜠ 
ᜫ
ᜨᜲ
ᜫᜲ
ᜰ
ᜮ
ᜥ


Typeset with \texttt{NotoSansHanunoo-Regular.ttf} and the command \cmd{\hanunoo}
\end{scriptexample}

Vertically positionning the text is not currently supported by \pkgname{fontspec} and the manual says \textsc{Todo!}. You are your own here, or you can just put the characters in a box and give it a try.

\begin{minipage}[t]{2cm}
\begin{tcolorbox}[width=2cm,colback=graphicbackground,
boxrule=0pt,toprule=0pt,colframe=white]
\LARGE\hanunoo
ᜩ\\
ᜤ\\
ᜮ\\
ᜥᜳ\\
ᜨ᜴ \\
ᜨ᜴\\
ᜫᜳ\\
ᜥ\\
\end{tcolorbox}
\end{minipage}
\begin{minipage}[t]{2cm}
\begin{tcolorbox}[width=2cm,colback=graphicbackground,
boxrule=0pt,toprule=0pt,colframe=white]
\LARGE\hanunoo
ᜩ\\
ᜤ\\
ᜮ\\
ᜥᜳ\\
ᜨ᜴ \\
ᜨ᜴\\
ᜫᜳ\\
ᜥ\\
\end{tcolorbox}
\end{minipage}
\begin{minipage}[t]{\textwidth-6cm}
The script is written from \so{bottom} to \so{top}. Typesetting this type of script automatically is not without its problems. One way is to use the build-in features of the font if they are available, but currently this gives problems---at least with the fonts that I have tried. Entering the text is also problematic as you will more than likely see little boxes rather than the actual glyph with most text editors common to \latexe. If you only need a couple of characters or a short sentence, an easy solution is to use |\rotatebox|. Another solution is to use a macro that can add the letters onto a stack, then place them in a box with a limited width. We can use |\@tfor| for this.  
\end{minipage}
\section{New Tai Lue Script}
\label{s:newtailue}


New Tai Lue script, also known as Simplified Tai Lue, is an alphabet used to write the Tai Lü language. Developed in China in the 1950s, New Tai Lue is based on the traditional Tai Le alphabet developed ca. 1200 AD. The government of China promoted the alphabet for use as a replacement for the older script; teaching the script was not mandatory, however, and as a result many are illiterate in New Thai Lue. 

In addition, communities in Burma, Laos, Thailand and Vietnam still use the Tai Le alphabet. There are probably less than one million native speakers of the language who can be found in China, Burma, Laos, Thailand and Vietnam.

We use NotoSansNewTaiLue-Regular.ttf.

\begin{scriptexample}[]{Tai Lue}
{\centering\tailue \LARGE

ᦒ	ᦓ	ᦔ	ᦕ	ᦖ	ᦗ	ᦘ	ᦙ	ᦚ	ᦛ	ᦜ	ᦝ	ᦞ	

}
\end{scriptexample}

The New Tai Lue script was added to the Unicode Standard in March, 2005 with the release of version 4.1.

The Unicode block for New Tai Lue is |U+1980|–|U+19DF|:

\begin{scriptexample}[]{New Tai Lue}
\unicodetable{tailue}{"1980,"1990,"19A0,"19B0,"19C0,"19D0}
\end{scriptexample}
\subsection{Myanmar}
\newfontfamily\myanmar{Padauk}

\begin{scriptexample}[]{Myanmar}
\bgroup
Myanmar [U+1000-U+109F]
\myanmar
လူတိုင်းသည် တူညီ လွတ်လပ်သော {\bfseries ဂုဏ်သိက္ခါဖြင့်
လည်းကောင်း၊}
\egroup
\end{scriptexample}

%\subsection{Oriya alphabet}
\newfontfamily\oriya[Scale=1.1,Script=Oriya]{code2000.ttf}

\def\oriyatext#1{{\oriya#1}}
The Oriya script or Utkala Lipi (Oriya: \oriyatext{ଉତ୍କଳ ଲିପି}) or Utkalakshara (Oriya: \oriyatext{ଉତ୍କଳାକ୍ଷର}) is used to write the Oriya language, and can be used for several other Indian languages, for example, Sanskrit.

\centerline{\Huge\oriyatext{ଉତ୍କଳ ଲିପି}}

\bgroup
\oriya
୦୧୨୩୪୫୬୭୮୯
ଅ ଆ ଇ ଈ ଉ ଊ ଋ ୠ ଌ ୡ ଏ ଐ ଓ ଔ କ ଖ ଗ ଘ ଙ ଚ ଛ ଜ ଝ ଞ ଟ ଠ ଡ ଢ ଣ ତ ଥ ଦ ଧ ନ ପ ଫ ବ ଵ ଭ ମ ଯ ର ଳ ୱ ଶ ଷ ସ ହ ୟ ଲ
\egroup

\begin{quotation}
Oṛiyā is encumbered with the drawback of an excessively awkward and cumbrous written character. ... At first glance, an Oṛiyā book seems to be all curves, and it takes a second look to notice that there is something inside each.(G. A. Grierson, Linguistic Survey of India, 1903)
\end{quotation}

Comparison of Oṛiyā script with its neighbours[edit]
At a first look the great number of signs with round shapes suggests a closer relation to the southern neighbour Telugu than to the other neighbours Bengali in the north and Devanāgarī in the west. The reason for the round shapes in Oriya and Telugu (and also in Kannaḍa and Malayāḷam) is the former method of writing using a stylus to scratch the signs into a palm leaf. These tools do not allow for horizontal strokes because that would damage the leaf.

Oriya letters are mostly round shaped whereas in Devanāgarī and Bengali have horizontal lines. So in most cases the reader of Oṛiyā will find the distinctive parts of a letter only below the hoop. Considering this the  closer relation to Devanāgarī and Bengali exists than to any southern script, though both northern and southern scripts have the same origin, Brāhmī.

Oriya (\oriyatext{ଓଡ଼ିଆ} oṛiā), officially spelled Odia,[3][4] is an Indian language belonging to the Indo-Aryan branch of the Indo-European language family. It is the predominant language of the Indian states of Odisha, where native speakers comprise 80\% of the population,[5] and it is spoken in parts of West Bengal, Jharkhand, Chhattisgarh and Andhra Pradesh. Oriya is one of the many official languages in India; it is the official language of Odisha and the second official language of Jharkhand. [6][7][8] Oriya is the sixth Indian language to be designated a Classical Language in India, on the basis of having a long literary history and not having borrowed extensively from other languages.


\subsection{Mongolian Script}



The classical Mongolian script (in Mongolian script: {\mongolian  ᠮᠣᠩᠭᠣᠯ ᠪᠢᠴᠢᠭ᠌} Mongγol bičig; in Mongolian Cyrillic: Монгол бичиг Mongol bichig), also known as Uyghurjin Mongol bichig, was the first writing system created specifically for the Mongolian language, and was the most successful until the introduction of Cyrillic in 1946. Derived from Uighur, Mongolian is a true alphabet, with separate letters for consonants and vowels. The Mongolian script has been adapted to write languages such as Oirat and Manchu. Alphabets based on this classical vertical script are used in Inner Mongolia and other parts of China to this day to write Mongolian, Sibe and, experimentally, Evenki.
\medskip

\bgroup\par
\noindent
\colorbox{graphicbackground}{\color{black}^^A
\begin{minipage}{\textwidth}^^A
\parindent1pt
\vskip10pt
\leftskip10pt \rightskip\leftskip
\mongolian
\large
ᠬᠦᠮᠦᠨ ᠪᠦᠷ ᠲᠥᠷᠥᠵᠦ ᠮᠡᠨᠳᠡᠯᠡᠬᠦ ᠡᠷᠬᠡ ᠴᠢᠯᠥᠭᠡ ᠲᠡᠢ᠂ ᠠᠳᠠᠯᠢᠬᠠᠨ ᠨᠡᠷ᠎ᠡ ᠲᠥᠷᠥ ᠲᠡᠢ᠂ ᠢᠵᠢᠯ ᠡᠷᠬᠡ ᠲᠡᠢ ᠪᠠᠢᠠᠭ᠃ ᠣᠶᠤᠨ ᠤᠬᠠᠭᠠᠨ᠂ ᠨᠠᠨᠳᠢᠨ ᠴᠢᠨᠠᠷ ᠵᠠᠶᠠᠭᠠᠰᠠᠨ ᠬᠦᠮᠦᠨ ᠬᠡᠭᠴᠢ ᠥᠭᠡᠷ᠎ᠡ ᠬᠣᠭᠣᠷᠣᠨᠳᠣ᠎ᠨ ᠠᠬᠠᠨ ᠳᠡᠭᠦᠦ ᠢᠨ ᠦᠵᠢᠯ ᠰᠠᠨᠠᠭᠠ ᠥᠠᠷ ᠬᠠᠷᠢᠴᠠᠬᠥ ᠤᠴᠢᠷ ᠲᠠᠢ᠃
\par
\vspace*{10pt}
\end{minipage}
}
\medskip
^^A\section{Tibetan}

Another important Northern Indian member perhaps
derived directly from Gupta---and thus a sister script to Nagari,
Sarada and Pali---is Tibetan. However, the Tibetan
language wears this foreign Indo-Aryan script most uncomfortably.\cite{writing}
The script retains the Indic consonantal alphabet with diacritic
attachments to indicate vowels – but with only one vowel
letter, the /a/, which is the same as the system’s own ‘default’ /a/.
This /a/ letter is then used to attach other diacritics in order to
indicate further vowels. Because the Tibetan language has
changed greatly since c. AD 700 (when the script was first elaborated
from Gupta) while the script has remained almost
unchanged, Tibetan is extremely difficult to read today. Its
greatest problem is that it marks none of the tones of its tonal
language. Though Tibetans have long tried to adapt written
Tibetan to spoken Tibetan, high illiteracy has been the price of
failing to achieve this. Tibetan schools in Tibet, by governmental
decree, now teach only the Chinese script and in the Chinese
language.

^^A\newfontfamily\tibetan{TibMachUni.ttf}

^^A\newfontfamily\tibetan{Qomolangma-Chuyig.ttf}

^^A should pick it up automatically \tibetan

Fonts described in this section can be obtained from The Tibetan \& Himalayan Library
\footnote{\url{http://www.thlib.org/tools/scripts/wiki/tibetan\%20machine\%20uni.html}  }

I have tried a few \texttt{Tibetan Machine Uni (TMU)} seems to be used by a number of scholars. 

A tip when you are trying to locate fonts is to find a related article in Wikipedia, such as Tibetan alphabet and inspect the element using your browser to see what fonts are being used.


|style="font-family:'Jomolhari','Tibetan Machine Uni','DDC Uchen', 'Kailash';| 


If you cannot see the script and rather than boxes or question marks then you can search and download one of the fonts in |font-family|.

\def\tibetandefaultfont#1{\newfontfamily\tibetan[Language=Tibetan]{#1}}

\cxset{language/.store in=\tibetan}
\cxset{language=tibetan} 
\cxset{tibetan font/.code=\tibetandefaultfont{#1}}


\cxset{tibetan font = TibMachUni.ttf}




\begin{key}{/chapter/language = tibetan} The key |language=tibetan| sets the default language as Tibetan, using the main font given by the key |tibetan font=TibMachUni.ttf|.
\end{key}

\begin{key}{/chapter/tibetan font = TibMachUni.ttf} The key |tibetan font=font-name| sets the default font for the Tibetan language. It will also create the switch \cmd{\tibetan} for typesetting text in Tibetan.
\end{key}

\begin{texexample}{Tibetan language setttings}{ex:tibetan}
\cxset{language=tibetan, tibetan font = TibMachUni.ttf}
\tibetan

\tibetan Tibetan: དབུ་ཅན
\end{texexample}


The Tibetan alphabet is an \emph{abugida} of Indic origin used to write the Tibetan language as well as Dzongkha\footnote{Spoken in Bhutan.}, the Sikkimese language, Ladakhi, and sometimes Balti. 

The printed form of the alphabet is called \textit{uchen} script (Tibetan: དབུ་ཅན་, Wylie: dbu-can; "with a head") while the hand-written cursive form used in everyday writing is called umê script (Tibetan: དབུ་མེད་, Wylie: dbu-med; "headless").

The alphabet is very closely linked to a broad ethnic Tibetan identity. Besides Tibet, it has also been used for Tibetan languages in Bhutan, India, Nepal, and Pakistan.[1] The Tibetan alphabet is ancestral to the Limbu alphabet, the Lepcha alphabet,[2] and the multilingual 'Phags-pa script.[2]


The Tibetan alphabet is romanized in a variety of ways.[3] This article employs the Wylie transliteration system.

The Tibetan alphabet has thirty basic letters, sometimes known as "radicals", for consonants.[2]

ཀ ka /ká/	ཁ kha /kʰá/	ག ga /kà, kʰà/	ང nga /ŋà/
ཅ ca /tʃá/	ཆ cha /tʃʰá/	ཇ ja /tʃà/	ཉ nya /ɲà/
ཏ ta /tá/	ཐ tha /tʰá/	ད da /tà, tʰà/	ན na /nà/
པ pa /pá/	ཕ pha /pʰá/	བ ba /pà, pʰà/	མ ma /mà/
ཙ tsa /tsá/	ཚ tsha /tsʰá/	ཛ dza /tsà/	ཝ wa /wà/ (not originally part of the alphabet)[5]
ཞ zha /ʃà/[6]	ཟ za /sà/	འ 'a /hà/[7]
ཡ ya /jà/	ར ra /rà/	ལ la /là/
ཤ sha /ʃá/[6]	ས sa /sá/	ཧ ha /há/[8]
ཨ a /á/

Tibetan is not a difficult script to read or write, but it is a very complex script to deal with in terms of computer processing (as far as complexity goes I would rate it second only to the Mongolian script). The problem is that written Tibetan comprises complex syllable units (known in Tibetan as a tsheg bar {\tibetan ཚེག་བར}) which although written horizontally may include \emph{vertical} clusters of consonants and vowel signs agglutinating around a base consonant (a vertical cluster is known as a "stack"). 

Thus most words have a horizontal and a vertical dimension, with the result that text is not laid out in a straight line as in most scripts. For example, the word bsGrogs བསྒྲོགས་ (pronounced drok ... obviously!) may be analysed as follows :

\definecolor{lavenderblush}{HTML}{FFF0F5}%
\definecolor{beige}{HTML}{F5F5DC}%


{\tibetan 
\HUGE བསྒྲོགས

{\color{beige}%
\symbol{"0F56}\color{blue!40}\color{red}\symbol{"0F66}\symbol{"0F92}\color{blue!80}\symbol{"0FB2}\color{beige}\symbol{"0F7C}\color{blue!25}\symbol{"0F42}\symbol{"0F66}\symbol{"0F0B}}



\begin{tabular}{|l|}
\symbol{"0F56}\symbol{"0F7C}\\
\symbol{"0F42}\symbol{"0F7C}\\
\symbol{"0F66}\symbol{"0F7C}\\
\symbol{"0F40}\symbol{"0F7C}\\
\end{tabular}
}

\subsection{Unicode Block Tibetan}


\bgroup\large\tibetan
\begin{tabular}{llllllllllllllll l}
\toprule
	           &|0|	&|1|	&|2|	&|3|	&|4|	&|5|	&|6|	&|7|	&|8|	&|9|	&|A|	&|B|	&|C|	&|D|	&|E|	&|F|\\
\midrule
\texttt{U+0F0x}	&ༀ	&༁	&༂	&༃	&༄	&༅	&༆	&༇	&༈	&༉	&༊	&་	&༌  &	།	&༎	&༏\\
\midrule
\texttt{U+0F1x} &༐	&༑	&༒	&༓	&༔	&༕	&༖	&༗	&༘&	༙	&༚	&༛	&༜	&༝	&༞	&༟\\
\midrule
\texttt{U+0F2x} &༠	&༡	&༢	&༣	&༤	&༥	&༦	&༧	&༨	&༩	&༪	&༫	&༬	&༭	&༮	&༯\\
\midrule
\texttt{U+0F3x}	&༰ &༱	 &༲ &༳	&༴ &༵	&༶ & ༷	&༸&	༹	&༺&	༻	&༼&	༽	&༾	&༿\\
\midrule
\texttt{U+0F4x} &ཀ	&ཁ	&ག	&གྷ	&ང	&ཅ	&ཆ	&ཇ	&	&ཉ	&ཊ	&ཋ	&ཌ	&ཌྷ	&ཎ	&ཏ\\
\midrule
\texttt{U+0F5x}	 &ཐ	&ད	&དྷ	&ན	&པ	&ཕ	&བ	&བྷ	&མ	&ཙ	&ཚ	&ཛ	&ཛྷ	&ཝ	&ཞ	&ཟ\\
\midrule
\texttt{U+0F6x} &འ	&ཡ	&ར	&ལ	&ཤ	&ཥ	&ས	&ཧ	&ཨ	&ཀྵ	&ཪ	&ཫ	&ཬ	&&&\\
^^A\texttt{U+0F7x}&&	ཱ &	& &ི	ཱི&	ུ&	ཱུ&	ྲྀ&	ཷ&	ླྀ&	ཹ&	ེ&	ཻ&	ོ&	ཽ&	&ཾ	&ཿ\\
\midrule
\texttt{U+0F8x}&    ྀ   & 	ཱྀ&	ྂ&	&ྃ &	྄	&྅&	྆	&྇	ྈ&	ྉ&	ྊ&	ྋ&	ྌ&	ྍ&	ྎ&	ྏ\\
\midrule
\texttt{U+0F9x} &	ྐ&	ྑ   & 	ྒ &	ྒྷ &	ྔ &	ྕ &	ྖ &	ྗ &		ྙ &	ྚ &	ྛ &	ྜ &	ྜྷ &	ྞ &	ྟ\\
\texttt{U+0FAx} &	ྠ &	ྡ &	ྡྷ &	ྣ &	ྤ &	ྥ &		&ྦ	&ྦྷ	ྨ&	ྩ&	ྪ&	ྫ&	ྫྷ&	ྭ&	ྮ&	ྯ\\
\midrule
\texttt{U+0FBx} 
&	  ྰ 
&	
& ྱ  	 
&ྲ	
&ླ	
&ྴ
&	ྵ
&	ྶ
&	ྷ
&ྸ
&
&
&
&	
&྾	
&྿\\
\midrule
\texttt{U+0FCx}	 &࿀&	࿁&	࿂&	࿃&	࿄&	࿅&	&࿇	&࿈	&࿉	&࿊	&࿋	&࿌	&&	࿎	&࿏\\
\midrule
\texttt{U+0FDx}	&࿐	&࿑	&࿒	&࿓	&࿔	&࿕	&࿖	&࿗	&࿘	&࿙	&࿚	&&&&&\\
\midrule
\texttt{U+0FEx} &&&&&&&&&&&&&&&&\\
\midrule
\texttt{U+0FFx}  &&&&&&&&&&&&&&&&\\
\bottomrule
\end{tabular}
\egroup




\subsection{Fonts for Tibetan}

Fonts for Tibetan need to be downloaded one set of fonts are the \texttt{Qomolangma}. They come in different flavours, but they appear
to offer advantages as compared to the Tibetan Machine Uni.
\medskip


\newfontfamily\betsu{Qomolangma-Betsu.ttf}
\newfontfamily\drutsa{Qomolangma-Drutsa.ttf}
\newfontfamily\chuyig{Qomolangma-Chuyig.ttf}
\newfontfamily\tsumachu{Qomolangma-Tsumachu.ttf}
\newfontfamily\uchensutung{Qomolangma-UchenSutung.ttf}
\newfontfamily\uchensuring{Qomolangma-UchenSuring.ttf}
\newfontfamily\uchensarchen{Qomolangma-UchenSarchen.ttf}
\newfontfamily\uchensarchung{Qomolangma-UchenSarchung.ttf}
\newfontfamily\tsuring{Qomolangma-Tsuring.ttf}
\newfontfamily\TMU{TibMachUni.ttf}
\newfontfamily\himalaya{Microsoft Himalaya}


{
\centering

\renewcommand{\arraystretch}{1.5}

\begin{tabular}{lr}
\toprule
|Qomolangma-Betsu.ttf| & {\betsu  དབུ་མེད }\\
\midrule
|Qomolangma-Chuyig.ttf| &{\chuyig  དབུ་མེད}\\
\midrule
|Qomolangma-Drutsa.ttf| &{\drutsa  དབུ་མེད}\\
\midrule
|Qomolangma-Tsumachu.ttf|&{\tsumachu  དབུ་མེད}\\
\midrule
|Qomolangma-Tsuring.ttf| &{\tsuring  དབུ་མེད}\\
\midrule
|Qomolangma-UchenSarchen.ttf| &{\uchensarchen དབུ་མེད}\\
\midrule
|Qomolangma-UchenSarchung.ttf|&{\uchensarchung དབུ་མེད }\\
\midrule
|Qomolangma-UchenSuring.ttf|&{\uchensuring དབུ་མེད}\\
\midrule
|Qomolangma-UchenSutung.ttf|&{\uchensutung དབུ་མེད }\\
\midrule
|TibMachUni.ttf| &{\TMU དབུ་མེད }\\
\midrule
|Microsoft Himalaya| &{\himalaya དབུ་མེད ཽ}\\
\bottomrule
\end{tabular}

}
\bigskip

\bgroup
\LARGE\tsuring
\noindent༆ །ཨ་ཡིག་དཀར་མཛེས་ལས་འཁྲུངས་ཤེས་བློ  འི་\par
གཏེར༑ །ཕས་རྒོལ་ཝ་སྐྱེས་ཟིལ་གནོན་གདོང་ལྔ་བཞིན།།\par
ཆགས་ཐོགས་ཀུན་བྲལ་མཚུངས་མེད་འཇམ་དབྱངསམཐུས།།\par
མཧཱ་མཁས་པའི་གཙོ་བོ་ཉིད་འགྱུར་ཅིག། །མངྒལཾ༎\par
བསྒྲོགས
\egroup

\subsubsection{Tibetan numbers}
\cxset{language=tibetan, tibetan font = TibMachUni.ttf}

{
\obeylines
\small
TIBETAN DIGIT ZERO	༠
TIBETAN DIGIT ONE	༡	
TIBETAN DIGIT TWO	༢	
TIBETAN DIGIT THREE	༣	
TIBETAN DIGIT FOUR	༤	
TIBETAN DIGIT FIVE	༥	
TIBETAN DIGIT SIX	༦	
TIBETAN DIGIT SEVEN	༧	
TIBETAN DIGIT EIGHT	༨	
TIBETAN DIGIT NINE	༩	
TIBETAN DIGIT HALF ONE	\tibetan༪	
TIBETAN DIGIT HALF TWO	༫	
TIBETAN DIGIT HALF THREE	༬
TIBETAN DIGIT HALF FOUR ༭	
TIBETAN DIGIT HALF FIVE ༯	
TIBETAN DIGIT HALF SIX	 ༯	
TIBETAN DIGIT HALF SEVEN	༰	
TIBETAN DIGIT HALF EIGHT	༱	
TIBETAN DIGIT HALF NINE	༲	
TIBETAN DIGIT HALF ZERO	༳	
}


Tibetan numbers

The usage is not certain. By some interpretations, this has the value of 9.5. Used only in some traditional contexts, these appear as the last digit of a multidigit number, eg. ༤༬ represents 42.5. These are very rarely used, however, and other uses have been postulated.


\PrintUnicodeBlock{./languages/tibetan.txt}{\himalaya}


^^A
^^A
^^A

^^A\section{Tamil}
\label{s:tamil}
\newfontfamily\tamil[Scale=1.1,Script=Tamil]{code2000.ttf}

\def\tamiltext#1{{\tamil#1}}

The Tamil script (\tamiltext{தமிழ் அரிச்சுவடி} tamiḻ ariccuvaṭi) is an abugida script that is used by the Tamil people in India, Sri Lanka, Malaysia and elsewhere, to write the Tamil language, as well as to write the liturgical language Sanskrit, using consonants and diacritics not represented in the Tamil alphabet.[1] Certain minority languages such as Saurashtra, Badaga, Irula, and Paniya are also written in the Tamil script

The Tamil script has 12 vowels (\tamiltext{உயிரெழுத்து} uyireḻuttu "soul-letters"), 18 consonants (\tamiltext{மெய்யெழுத்து} meyyeḻuttu "body-letters") and one character, the āytam \tamiltext{ஃ (ஆய்தம்)}, which is classified in Tamil grammar as being neither a consonant nor a vowel (\tamiltext{அலியெழுத்து} aliyeḻuttu "the hermaphrodite letter"), though often considered as part of the vowel set (\tamiltext{உயிரெழுத்துக்கள்} uyireḻuttukkaḷ "vowel class"). The script, however, is syllabic and not alphabetic.[3] The complete script, therefore, consists of the thirty-one letters in their independent form, and an additional 216 combinant letters representing a total 247 combinations (\tamiltext{உயிர்மெய்யெழுத்து} uyirmeyyeḻuttu) of a consonant and a vowel, a mute consonant, or a vowel alone. These combinant letters are formed by adding a vowel marker to the consonant. Some vowels require the basic shape of the consonant to be altered in a way that is specific to that vowel. Others are written by adding a vowel-specific suffix to the consonant, yet others a prefix, and finally some vowels require adding both a prefix and a suffix to the consonant. In every case the vowel marker is different from the standalone character for the vowel.
The Tamil script is written from left to right.

Tamil is a Unicode block containing characters for the Tamil, Badaga, and Saurashtra languages of Tamil Nadu India, Sri Lanka, Singapore, and Malaysia. In its original incarnation, the code points U+0B02..U+0BCD were a direct copy of the Tamil characters A2-ED from the 1988 ISCII standard. The Devanagari, Bengali, Gurmukhi, Gujarati, Oriya, Telugu, Kannada, and Malayalam blocks were similarly all based on their ISCII encodings.

\begin{scriptexample}[]{Tamil}
\unicodetable{tamil}{"0B80,"0B90,"0BA0,"0BB0,"0BC0,"0BE0,"0BF0}

\hfill  Typeset with \cmd{\tamil} and \texttt{code2000.ttf}
\end{scriptexample}

\subsection{Tamil Numbers and Numerals}

Originally, Tamils did not use zero, nor did they use positional digits (having separate 
symbols for the numbers 10, 100 and 1000). Symbols for the numbers are similar to 
other Tamil letters, with some minor changes. 

For example, the number 3782 is not written as \tamiltext{௩௭௮௨} as in modern usage. Instead it 
is written as \tamiltext{௩ ௲ ௭ ௱ ௮ ௰ ௨}. This would be read as they are written as 
Three Thousands, Seven Hundreds, Eight Tens, Two; or in Tamil as 
\tamiltext{௩௲௭௱௮௰௨ž}.\footnote{https://cloud.github.com/downloads/raaman/Tamil-Numeral/tamilnumbers.html}

\subsection{Dates}

Once the script is loaded the day, month and year can be loaded using the command  \cmd{\tamildate}, which returns the |\today| formatted as per custom Tamil. 

\begin{center}
\bgroup
\tamil
\begin{tabular}{lll}
day	 &month	&year	\\

௳	&௴	      &௵	\\

u	&mee	      &wa	\\
\end{tabular}
\egroup
\end{center}


\section{Grantha}
Grantha is a Unicode block containing the ancient Grantha script characters of 6th to 19th century Tamil Nadu and Kerala for writing Sanskrit and Manipravalam. Battled to get it working, as I could not find an appropriate unicode font. The font would need remapping.

\newfontfamily\grantha{gr2.ttf}%e-Grantamil 7

\begin{scriptexample}[]{Tamil}
\unicodetable{grantha}{"0D0,"0D1,"0D2,"1133,"1134,"1135,"1136,"1137}

\hfill  Typeset with \cmd{\grantha} and \texttt{e-Granthamil 7.ttf}
\end{scriptexample}

{
\grantha \char"11311

}





^^A
^^A\subsection{Kannada alphabet}

\newfontfamily\kannada[Scale=1.0,Script=Kannada]{Lohit-Kannada.ttf}

\def\kannadatext#1{{\kannada#1}}

The Kannada alphabet (\kannadatext{ಕನ್ನಡ ಲಿಪಿ}) is an abugida of the Brahmic family,[2] used primarily to write the Kannada language, one of the Dravidian languages of southern India. Several minor languages, such as Tulu, Konkani, Kodava, and Beary, also use alphabets based on the Kannada script.[3] The Kannada and Telugu scripts share high mutual intellegibility with each other, and are often considered to be regional variants of single script. Similarly, Goykanadi, a variant of Old Kannada, has been historically used to write Konkani in the state of Goa.[4]

\begin{scriptexample}[]{Kannada}
\centerline{\LARGE\kannadatext{ಙ	ಙ್ಕ	ಙ್ಖ	ಙ್ಗ	ಙ್ಘ	ಙ್ಙ	ಙ್ಚ	ಙ್ಛ	ಙ್ಜ	ಙ್ಝ	ಙ್ಞ	ಙ್ಟ	ಙ್ಠ	ಙ್ಡ	ಙ್ಢ}}
\end{scriptexample}

\medskip

The Kannada script (aksharamale or varnamale) is a phonemic abugida of forty-nine letters, and is written from left to right. The character set is almost identical to that of other Brahmic scripts. Consonantal letters imply an inherent vowel. Letters representing consonants are combined to form digraphs (ottaksharas) when there is no intervening vowel. Otherwise, each letter corresponds to a syllable.
The letters are classified into three categories: swara (vowels), vyanjana (consonants), and yogavaahaka (part vowel, part consonant).
The Kannada words for a letter of the script are akshara, akkara, and varna. Each letter has its own form (ākāra) and sound (shabda), providing the visible and audible representations, respectively. Kannada is written from left to right.[7]

^^A
^^A\section{Osmanian Alphabet}

\bgroup
\newfontfamily\osmanian{code2001.ttf}
\osmanian
𐒚𐒁𐒖𐒄 𐒚𐒐 𐒚 𐒎𐒚𐒍𐒚𐒐 𐒑𐒚𐒒𐒠𐒚𐒐 𐒎𐒚𐒑𐒁𐒗 𐒚𐒁𐒖𐒄 𐒚𐒌𐒖𐒄 𐒚𐒁𐒖𐒄𐒖 𐒚
𐒌𐒜
\egroup



^^A

\cxset{steward,
  numbering=arabic,
  custom=stewart,
  offsety=0cm,
  image={ethiopianbride.jpg},
  texti={An introduction to the use of font related commands. The chapter also gives a historical background to font selection using \tex and \latex. },
  textii={In this chapter we discuss keys that are available through the \texttt{phd} package and give a background as to how fonts are used
in \latex.
 },
 pagestyle = empty,
}




\cxset{steward,
  numbering=arabic,
  custom=stewart,
  offsety=0cm,
  image={fellah-woman.jpg},
  texti={An introduction to the use of font related commands. The chapter also gives a historical background to font selection using \tex and \latex. },
  textii={In this chapter we discuss keys that are available through the \texttt{phd} package and give a background as to how fonts are used
in \latex.
 },
 pagestyle = empty
}
