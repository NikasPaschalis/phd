\cxset{section numbering=arabic}

\newfontfamily\miao{MiaoUnicode-Regular.ttf}


\section{Miao}
\label{s:miao}
\parindent1em

The Pollard script, also known as Pollard Miao (Chinese:{\pan 柏格理苗文 Bó Gélǐ} Miao-wen) or Miao, is an abugida loosely based on the Latin alphabet and invented by Methodist missionary Sam Pollard. 

The history of the Miao writing system is very much the story of the myth about the loss of the old Miao writing
system and how this was later recovered. wrote that in all parts of the areas inhabited by the Miao there are
legends of a lost writing system, Due to the expansion ofthe Han people, the Miao
had to migrate southwards, and in connection with that, the writing was
lost during a river crossing or eaten accidentally.
This kind ofmyth is not unique; a similar legend exists among the Karen
in Burma. In the beginning, when the creator was dispensing books to the
various peoples ofthe earth, the Karen overslept and missed out on the gift
of literacy. In some versions of this myth, they were given a book, but it
was consumed in the fires with which they bum their swidden fields. The
Kachin also have a myth that they devoured their own writing out of
hunger,1 as do the Akha,1 while Graham mentions that the ‘legend of a
lost book’ was also found among the Qiang of west Sichuan.2 This myth
about lost books radically influenced the readiness of the Miao to accept
writing.

Peter Mühlhäusler states that it is almost never the case that writing is
created to meet the needs of an aboriginal society, and that writing systems
introduced from the outside are often met with suspicion as the potentialities
of writing are unknown to the people.3 It may seem quite useless, or at
best as some kind of magic. However, counter-examples do exist, like the
Maori in New Zealand and the Cree in Canada. The Miao have, of course,
had contacts with their neighbours, and, with the Chinese as their main
neighbour, the power of writing must always have been well known to
them. The Chinese have probably attached more importance to writing
than any other people in history and this may have strengthened the need
for explaining the absence of writing in Miao society. 

One version of the myth is that the ancient Miao script survived in their embroidery.

The Miao are
famous for their embroidery and usually attach very strong importance to
the amount and quality of embroidery, especially on wedding dresses and
even on the ordinary dresses still worn in most Miao areas even today. The
myth is also partly an explanation of the intricate patterns found in those
embroideries. One such myth is presented by WangJianguang:

\begin{latexquote}
The Miao people originally had writing, but unfortunately it has not been preserved.
As Chiyou was beaten at the battle of Zhuolu by the Yellow Emperor, the Miao
were driven towards the south. When they had to cross various waters, they did not
have time to build boats, so when they forded the riven they were afraid that their
books should become wet. In order to avoid such a disaster they carried the books
on their heads. In this way the people wandered. When they came to the Yangtze
river they all wanted to cross as quickly as possible, but unfortunately the current was
very strong as they came to the middle, and most ofthem were drowned and only a
few managed to get over. The books were also lost and they could not be retrieved.
As [the migration] continued somebody invented a method of embroidering these
characten onto the clothes as a memorial. Therefore traces of the Miao history are
preserved in their clothes and skirts.6
\end{latexquote}

\begin{figure}[htp]
\includegraphics[width=\textwidth]{miao-01}
\caption{According to lengend the ancient Miao script, survived in the Miao embroidery.
source:\protect\href{http://themiaoculture.tumblr.com/}{themiaoculture}}
\end{figure}
Pollard invented the script for use with A-Hmao, one of several Miao languages. The script underwent a series of revisions until 1936, when a translation of the New Testament was published using it. The introduction of Christian materials in the script that Pollard invented caused a great impact among the Miao. Part of the reason was that they had a legend about how their ancestors had possessed a script but lost it. According to the legend, the script would be brought back some day. When the script was introduced, many Miao came from far away to see and learn it.[1][2]

\subsection{Eating Books and Getting a Good Memory}

Another version of the stories claims that the writing was for some reason
eaten by the Miao, resulting in inner qualities, like a good memory for
traditional songs and stories and general cleverness. The first is a legend from
the ‘short skirt Miao’ in Leishan County, Guizhou Province, recorded by Li
Tinggui during the Spring Festival of 1980:

\begin{latexquotation}
In the past the Miao and the Han were brothers who studied under the same
teacher. Both invented a script. Once they had to cross a river and big brother Miao
carried his younger brother Han on his back and therefore he put his script in his
mouth. As he came to the middle ofthe river he slipped and happened to swallow
the script. Therefore the Miao script is in the stomach and is recorded in the heart,
whereas little brother, who sat on his back, held the script in his hand and preserved
it. Thus the Han have a script which they write with their hands and see with their
eyes.23
\end{latexquotation}

Pollard credited the basic idea of the script to the Cree syllabics designed by James Evans in 1838–1841, “While working out the problem, we remembered the case of the syllabics used by a Methodist missionary among the Indians of North America, and resolved to do as he had done” (1919:174). He also gave credit to a Chinese pastor, “Stephen Lee assisted me very ably in this matter, and at last we arrived at a system” (1919:174). In listing the phrases he used to describe devising the script, there is clear indication of intellectual work, not revelation: “we looked about”, “resolved to attempt”, “adapting the system”, “solved our problem” (Pollard 1919:174,175).

Changing politics in China led to the use of several competing scripts, most of which were romanizations. The Pollard script remains popular among Hmong in China, although Hmong outside China tend to use one of the alternative scripts. A revision of the script was completed in 1988, which remains in use.

As with most other abugidas, the Pollard letters represent consonants, whereas vowels are indicated by diacritics. Uniquely, however, the position of this diacritic is varied to represent tone. For example, in Western Hmong, placing the vowel diacritic above the consonant letter indicates that the syllable has a high tone, whereas placing it at the bottom right indicates a low tone.

A still experimental font, that supports Graphite technology is \idxfont{Mia Unicode}\footnote{\url{http://phjamr.github.io/miao.html\#intro}}. The font is licenced under the SIL terms and we are using it in the |phd| package as the default font for the Miao script.



\begin{scriptexample}[]{Miao}
\unicodetable{miao}{"16F00,"16F10,"16F20,"16F30,"16F40,"16F70,"16F80,"16F90}
\end{scriptexample}

{\miao 𖼴	𖼵	𖼶	𖼷	𖼸	𖼹	𖼺	}

Features for Miao
There are three features currently available for the Miao script:
\bgroup
\miao
Chuxiong ‘wart’ variant
Stylistic alternates for 𖼳 and 𖼴
Aspiration marker always on right
The ‘wart’ (a translated technical term!) is the small circle in characters like 𖼁, 𖼅, and 𖼾. In the Chuxiong orthography, it is rendered not as a circle but as a dot on the right of the letter, as shown in point 5 here (pdf).

Miao Unicode has a feature called “chux” for handling this. In LibreOffice you can use this style by typing “Miao Unicode:chux=1” into the font field.


Samuel Pollard was born in Cornwall in 1864.66 After finishing school he
started to work at a bank in London, but in 1886 he decided to become a
missionary. He arrived in China in the year 1887 in order to work for the
Bible Christian China Mission in north Yunnan. After studying Chinese at
the language school in Anqing he and another young missionary, Frank
Dymond, came to the city of Zhaotong in 1888, where missionary work
had been started just a few months before. Premises had been rented in
Jixian Street near the east gate.

In 1890 Pollard married Emma Hainge, who was a missionary of the
CIM at Kunming. Progress was slow in the missionary work among the
Chinese, and the first two Chinese were baptized in 1893. In 1895-6
Pollard and his wife went to England on their first furlough. On his return
two Chinese students of good family took interest in Christianity and were
baptized. Their names were Li Sitifan ‘Stephen Lee’ and Li Yuehan
‘John Lee’.67 They were to play an important role in the work
among the Miao.









