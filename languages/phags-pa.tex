\section{Phags-pa}
\label{s:phagspa}
\newfontfamily\phagspa{code2000.ttf}
\arial 
The 'Phags-pa script,[1], (Mongolian: дөрвөлжин үсэг "Square script") was an alphabet designed by the Tibetan monk and vice-king Drogön Chögyal Phagpa for the Mongol Yuan emperor Kublai Khan as a unified script for the literary languages of the Yuan.

Widespread use was limited to about a hundred years during the Yuan Dynasty, and it fell out of use with the advent of the Ming dynasty. The documentation of its use provides clues about the changes in the varieties of Chinese, the Tibetic languages, Mongolian and other neighboring languages during the Yuan era.
\medskip


\includegraphics[width=1\linewidth]{./images/phags-pa.jpg}

credit \protect\url{http://turfan.bbaw.de/dta/monght/images/monght009_seite2.jpg}



\begin{scriptexample}[]{Phags-pa}
\bgroup
\unicodetable{phagspa}{"A840,"A850,"A860,"A870}

\arial
\hfill Typeset with \texttt{code2000.ttf} and \cmd{\phagspa}

\egroup
\end{scriptexample}
\medskip

Phags-pa is a historical script related to Tibetan that was created as the national script of
the Mongol empire. Even though Phags-pa was used mostly in Eastern and Central Asia for
writing text in the Mongolian and Chinese languages, it is discussed in this chapter because
of its close historical connection to the Tibetan script. The script has very limited modern use. It bears similarity to Tibetan and has no case distinctions. It is written vertically in columns running for left to right, like Mongolian. Units are often composed of several syllables and sometimes are separated by whitespace.


\printunicodeblock{./languages/phags-pa.txt}{\phagspa}

\cxset{script/.code={}}
\cxset{script=phags-pa}

\begin{key}{/chapter/script = \meta{phags-pa}} The key |script| will activate the commands available for typesetting the phags-pa script.
\end{key}



