\subsection{Osmanya}

\newfontfamily\osmanya{NotoSansOsmanya-Regular.ttf}

\begin{scriptexample}[]{Osmanya}
\unicodetable{osmanya}{"10480,"10490,"104A0}
\end{scriptexample}

The Osmanya alphabet (Somali: Cismaanya; Osmanya: {\osmanya 𐒋𐒘𐒈𐒑𐒛𐒒𐒕𐒀}), also known as Far Soomaali ("Somali writing"), is a writing script created to transcribe the Somali language. It was invented between 1920 and 1922 by Osman Yusuf Kenadid of the Majeerteen Darod clan, the nephew of Sultan Yusuf Ali Kenadid of the Sultanate of Hobyo.

While Osmanya gained reasonably wide acceptance in Somalia and quickly produced a considerable body of literature, it proved difficult to spread among the population mainly due to stiff competition from the long-established Arabic script as well as the emerging Somali alphabet developed by the Somali linguist, Shire Jama Ahmed, which was based on the Latin script.

As nationalist sentiments grew and since the Somali language had long lost its ancient script,[1] the adoption of a universally recognized writing script for the Somali language became an important point of discussion. After independence, little progress was made on the issue, as opinion was divided over whether the Arabic or Latin scripts should be used instead.

In October 1972, due to its simplicity, the fact that it lent itself well to writing Somali since it could cope with all of the sounds in the language, and the already widespread existence of machines and typewriters designed for its use,[2][3] the government of Somali president Mohamed Siad Barre unilaterally elected to use only the Latin script for writing Somali instead of the Arabic or Osmanya scripts.[4] Barre's administration subsequently launched a massive literacy campaign designed to ensure its sole adoption. This led to a sharp decline in use of Osmanya.