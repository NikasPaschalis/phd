\section{Saurashtra}
\label{sec:saurashtra}
\idxlanguage{Saurashtra}\idxlanguage{Sourashtra}

\index{Saurashtra fonts>code2000}
\newfontfamily\saurashtra{code2000.ttf}
\def\test{}
\cxset{saurashtra font/.code=\test}
\cxset{saurashtra font=code2000.ttf}

\begin{key}{/chapter/saurashtra font = \meta{fontname}}

\end{key}
Saurashtra or Sourashtra or {\saurashtra ꢱꣃꢬꢵꢰ꣄ꢜ꣄ꢬꢵ} or Palkar or Patkar (Sanskrit: सौराष्ट्र, Tamil: சௌராட்டிரம்) is an Indo-Aryan language[3] spoken by the Saurashtrian community native to Gujarat, who migrated and settled in Southern India. Madurai in Tamil Nadu has the highest number of people belonging to this community and also remains as their cultural center.

The language is largely only in spoken form even though the language has its own script. The lack of schools teaching Saurashtra script and the language is often cited as a reason for the very few number of people who actually know to read and write in Saurashtra script. Latin, Devanagari or Tamil script is used as alternative for Saurashtra Script by many Saurashtrians.

Census of India places the language under Gujarati. Official figures show the number of speakers as 185,420 (2001 census).[4]


\begin{scriptexample}[]{Saurashtra}
\unicodetable{saurashtra}{"A880,"A890,"A8A0,"A8B0,"A8C0,"A8D0}
\end{scriptexample}


\begin{scriptexample}[]{Saurashtra}
\bgroup
\saurashtra

ꢮꢶꢯ꣄ꢮ ꢱꣃꢬꢵꢰ꣄ꢜ꣄ꢬꢪ꣄ ꢦꢡ꣄ꢬꢶꢒꢾ ꢱꢵꢡ꣄ꢡꢒꢸ ꢂꢮꢬꢾ
ꢮꣁꢭꢱ꣄ꢢꢵꢥꢪꢸꢒ꣄(ꣀꢵꢮꢾꢔꢹ ꢂꢮ꣄ꢬꢶꢫꣁ


\arial

Text: Vishwa Sourashtram \url{http://www.sourashtra.info/ghEr.htm}
\egroup
\end{scriptexample}


\printunicodeblock{./languages/saurashtra.txt}{\saurashtra}
