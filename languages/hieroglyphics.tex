\newfontfamily\aegyptus{AegyptusR.ttf}

\bgtitle{Aegyptian\\ Hieroglyphics}{Aegyptian Hieroglyphics}
\chapter{Aegyptian Hieroglyphics}


\index{fonts>Aegyptus}\index{Aegyptus (font)}
\index{fonts>Hieroglyphics}\index{languages>hieroglyphics}

\newfontfamily\hiero{NotoSansEgyptianHieroglyphs-Regular.ttf}

Hieroglyphic writing appeared in Egypt at the end of the fourth millennium bce. The writing
system is pictographic: the glyphs represent tangible objects, most of which modern
scholars have been able to identify. A great many of the pictographs are easily recognizable
even by nonspecialists. Egyptian hieroglyphs represent people and animals, parts of the
bodies of people and animals, clothing, tools, vessels, and so on.

Hieroglyphs were used to write Egyptian for more than 3,000 years, retaining characteristic
features such as use of color and detail in the more elaborated expositions. Throughout the
Old Kingdom, the Middle Kingdom, and the New Kingdom, between 700 and 1,000 hieroglyphs
were in regular use. During the Greco-Roman period, the number of variants, as
distinguished by some modern scholars, grew to somewhere between 6,000 and 8,000.

Hieroglyphs were carved in stone, painted on frescoes, and could also be written with a reed
stylus, though this cursive writing eventually became standardized in what is called \emph{hieratic}
writing. Unicode does not encode the hieratic forms separately, but ust considers them as cursive forms of the hieroglyphs encoded block.

The Demotic script and then later the Coptic script replaced the earlier hieroglyphic and
hieratic forms for much practical writing of Egyptian, but hieroglyphs and hieratic continued
in use until the fourth century ce. An inscription dated August 24, 394 ce has been
found on the Gateway of Hadrian in the temple complex at Philae; this is thought to be
among the latest examples of Ancient Egyptian writing in hieroglyphs

\begin{figure}[htb]
\includegraphics[width=\textwidth]{./images/bookofthedead.jpg}
\end{figure}

In hieroglyphic texts, these drawings are not only simply arranged in sequential order, but also grouped on top of and next to each other. This rather complicates matters trying to register and reproduce hieroglyphic texts using a computer.

\section{Computer Typesetting}

Typesetting hieroglyphics with computers presents a number of problems. First is the method of inputting the characters and second the various methods required to stack hieroglyphics, the direction of writing which can be one of four different directions.

When the first computers were introduced in Egyptology in the late 1970s and the beginning of the 1980s, the graphical capacity of the machines was still in its infancy. Early attempts to register the hieroglyphic pictorial writing on computer therefore chose an encoding system to do this, using alphanumeric codes to represent or replace the graphics. To prevent many people from reinventing the wheel, during the first "Table Ronde Informatique et Egyptologie" in 1984 a committee was charged with the task to develop a uniform system for the encoding of hieroglyphic texts on computer. The resulting Manual for the Encoding of Hieroglyphic Texts for Computer-input (Jan Buurman, Nicolas Grimal, Jochen Hallof, Michael Hainsworth and Dirk van der Plas, Informatique et Egyptologie 2, Paris 1988), simply called Manuel de Codage, presents an easy to use and intuitive way of encoding hieroglyphic writing as well as the abbreviated hieroglyphic transcription (transliteration). The system proposed by the Manuel de Codage has since been adopted by international Egyptology as the official common standard for registering hieroglyphic texts on computer. Mark-Jan Nederhof proposed an enhanced encoding scheme to remove many of the limitations in the Manuel de Codage.

\pkgname{HieroTeX} is a \latexe package developed by to typeset hieroglyphic texts and still works well. The advantages of using \tex is of course its excellent typesetting capabilities and the usage of macros. Although inputting the texts as MdC codes is not that difficult, repeating the same codes over and over can be avoided with easily constructed simple substitution macros. 

\subsection{fonts}

One of the best fonts I came across is \idxfont{Aegyptus} from \url{http://users.teilar.gr/~g1951d/}\footnote{The site also has fonts for Aegean Numbers, Ancient Greek Musical Notation, Ancient Greek Numbers, Ancient Roman Symbols, Arkalochori Axe, Carian, Cypriot Syllabary, Dispilio tablet, Linear A, Linear B Ideograms, Linear B Syllabary, Lycian, Lydian, Old Italic, Old Persian, Phaistos Disc, Phoenician, Phrygian, Sidetic, Troy vessels’ signs and Ugaritic. Cretan Hieroglyphs and Cypro-Minoan script(s) are offered in separate files.}. The font provides all the unicode characters and also offers an additional number of glyphs that are not in the Unicode standard. The font uses the Unicode Private Use Areas to encode the glyphs. 

Another font is the Noto Egyptian Hieroglyphics from Google. This is a lightweight font with the symbols in their proper unicode slots. Mark-Jan Nederhof's \idxfont{NewGardiner} font is another one with support only for the Gardiner set. The codepoint mappings are incorrect, as the font has been  
encoded to EGPZ. The font is similar to the Aegyptus font, however it is just transposed and not recommended unless it is transposed. 

The editor software JSesh\footnote{\protect\url{http://jsesh.qenherkhopeshef.org/}} also provides a free font |JSeshFont.ttf|. This offers a correctly mapped unicode and is another good alternative. The symbols are drawn somewhat simpler and is just a matter of taste what you want to use.

My recommendation is for short demonstration purposes, the Noto font is to be preferred while for more serious work the Aegyptus font will be more useful. Using Lua the font can be transposed automatically to allow the use of commands that refer to unicode numbers. Another advantage of the Aegyptus font is that the glyphs are named with their Gardiner numbers, so it is somewhat easier to programmatically access them by name.\footnote{Unicode does not name the glyphs, but simply calls the Egyptian Hieroglyph $n$. } 

\medskip

\ifxetex
\bgroup
\centering 
\font\myfont = "Aegyptus"
\scalebox{7}{\myfont\XeTeXglyph 201}
\scalebox{7}{\myfont\XeTeXglyph 203}
\scalebox{7}{\myfont\XeTeXglyph 163}
\scalebox{7}{\myfont\XeTeXglyph 164}
\scalebox{7}{\myfont\XeTeXglyph 165}
\scalebox{7}{\myfont\XeTeXglyph 168}
\captionof{table}{Example of Egyptian Hieroglyphics typeset with the \textit{Aegyptus} font.} 
\egroup
\fi

\ifluatex
\bgroup
\centering 
\aegyptus
\scalebox{7}{\char"F300C}
\scalebox{7}{\char"F3001}
\scalebox{7}{\char"F3010}
\scalebox{7}{\char"F308B}
\scalebox{7}{\char"F3097}
\scalebox{7}{\char"F3091}
\captionof{table}{Example of Egyptian Hieroglyphics typeset with the \textit{Aegyptus} font.} 
\egroup

\fi


\subsection{Unicode Block}

Egyptian hieroglyphs is a Unicode block containing the Gardiner's sign list of Egyptian hieroglyphics.
The code points, in the range |0x13000| to |0x1342E|, are available starting from
\href{http://unicode.org/charts/PDF/U13000.pdf}{Unicode 5.2}

\begin{scriptexample}[]{Hieroglyphic}
\bgroup
\unicodetable{hiero}{"13000,"13010,"13020,"13030,"13040,"13050,"13060,"13070,%
"13080,%
"13090,"130A0,"130B0,"130C0,"130D0,"130E0,"130F0,%
"13100,"13110,"13120,"13130,"13140,"13150,"13060,"13070,"13080,"13090}
\egroup
\end{scriptexample}

\subsection{Gardiner's classification}

The standard reference on Egyptian hieroglyphics is Gartiner's Sign List, which lists common Egyptian hieroglyphs. These are grouped in categories from A-Aa. Each category represents a theme for example category A, is "man and his occupations". Based on this list ``Queen with flower" is denoted as \texttt{B7}. 

\subsection{Character Names} 

Egyptian hieroglyphic characters have traditionally been designated in
several ways:

\begin{enumerate}
\item  By complex description of the pictographs: \texttt{GOD WITH HEAD OF IBIS}, and so forth.
\item By standardized sign number: C3, E34, G16, G17, G24.
\item For a minority of characters, by transliterated sound.
\end{enumerate}

The characters in the Unicode Standard make use of the standard Egyptological catalog
numbers for the signs. Thus, the name for {\hiero\char"130F9} |U+13049| egyptian hieroglyph e034 refers
uniquely and unambiguously to the Gardiner list sign E34, described as a “{\aegean DESERT HARE}” ({\hiero \char"130FA}) and used for the sound “wn”. The Unicode catalog values are padded to three places with
zeros, so where the Gardiner classification is shown as \texttt{E34}, the unicode value is \texttt{E034}. 

Names for hieroglyphic characters identified explicitly in Gardiner 1953 or other sources as
variants for other hieroglyphic characters are given names by appending “A”, “B”, ... to the sign number. In the sources these are often identified using asterisks. Thus Gardiner’s G7,
G7*, and G7** correspond to U+13146 egyptian sign g007 {\hiero \char"13147}, U+13147 egyptian sign g007a, and U+13148 egyptian sign g007b, respectively.

\def\texthiero#1{{\color{black!95}\hiero #1}}

\begin{longtable}{>{\Large}lll>{\ttfamily}l}
{\hiero \char"13000}&A1-A70 & Man and his occupations &U+13000-1304F\\
{\hiero \char"13050}&B1-B9  &Woman and her occupations &U+13050-13059\\
{\hiero \char"1305A} &C1-C24 &Anthropomorphic Deities &U+1305A-13075\\
{\hiero \char"13076} &D1-D67 &Parts of the Human Body &U+13076-130D1\\
{\hiero \char"130D2} &E1-E38 &Mammals &U+13076-130D1\\
{\hiero \char"130FE}  &F1-F53	&Parts of Mammals &U+130FE-1313E\\
{\hiero\char"1313F}	&G1-G54	&Birds &U+1313F-1317E\\
{\hiero \char"1317F}	&H1-H8	&Parts of Birds &U+1317F-13187\\
\texthiero{\char"13188}	&I1-I15	&Amphibious Animals, Reptiles, etc. &U+13188-1319A\\
\texthiero{\char"1319B}	&K1-K8	&Fishes and Parts of Fishes &U+1319B-131A2\\
\texthiero{\char"131A3}	&L1-L8	&Invertebrata and Lesser Animals &U+131A3-131AC\\
\texthiero{\char"131AD}	&M1-M44	&Trees and Plants &U+13AD-131EE\\
\texthiero{\char"131EF}	&N1-N42	&Sky, Earth, Water &U+131EF-1321F\\
\texthiero{\char"13250}	&O1-O51	&Buildings and Parts of Buildings &U+13250-1329A\\
\texthiero{\char"1329B}	&P1-P11	&Ships and Parts of Ships &U+1329B-132A7\\
\texthiero{\char"132A8}	&Q1-Q7	& Domestic and Funerary Furniture &U+132A8-132AE\\
\texthiero{\char"132AF}	&R1-R29	&Temple Furniture and Sacret Emblems &U+132AF-132D0\\
\texthiero{\char"132D1}	&S1-S46	&Crowns, Dress, Staves, etc. &U+132D1-13306\\
\texthiero{\char"13307}	&T1-T36	&Warfare, Hunting, Butchery &U+13307-13332\\
\texthiero{\char"13333}	&U1-42	&Agriculture, Crafts and Professions &U+13333-13361\\
\texthiero{\char"13362}	&V1-V40a	&Rope, Fibre, Baskets, Bags, etc. &U+13362-133AE\\
\texthiero{\char"133AF}	&W1-W25	&Vessels of Stone and Earthenware &U+133AF-133CE\\
\texthiero{\char"133CF}	&X1-X8a	&Loaves and Cakes &U+133CF-133DA\\
\texthiero{\char"133DB}	&Y1-Y8	&Writing, Games, Music &U+133DB-133E3\\
\texthiero{\char"133E4}	&Z1-Z16H	&Strokes, Geometrical Figures, etc. &U+133E4-1340C\\
\texthiero{\char"1340D}	&Aa1-Aa32	&Unclassified &U+1340D-1342E\\
\end{longtable}

I particularly like the crocodile sign \def\crocodile{\color{teal}{\Huge\texthiero{\char"13188}}} {\crocodile}, as it is applicable to describe people in my field of work. 

\begin{scriptexample}[]{Woman and her occupations}
\unicodetable{hiero}{"13050}
\end{scriptexample}

\section{Positioning}

One of the core assumptions of any hieroglyphic encoding or mark-up scheme following the MdC is that signs and groups of signs maybe positioned next to each other or above each other. The former is indicated by the operator * and the latter by :. One may also use -, which functions as * for horizontal texts and as : for vertical text. 

In some dialects of the MdC relative positioning has been extended by the use of the |&| operator. This is used to form a kind of ligature, such as |D&t| can be defined to represent the \textit{Cobra at rest} sign I10 with sign X1 underneath, as follows:

\begin{center}
{\hiero\HUGE
       \mbox{\rlap{\char"133CF}\char"13193\hfill\hfill}\\
       {\large|insert[bs](I10,X1)|}

\mbox{\rlap{\scalebox{0.5}{\char"133E3}}\char"13193\hfill\hfill}\\
 	
}
\end{center}

This is only a partial solution and to automate it via kerning tables, will require hundreds of entries in the kerning tables. It will also need constant modifications as researchers discover new combinations. A better approach and which is easily applied to \tex based systems would be to adopt Nederhof's method of creating a new command |insert[bs](I10,X1)|. 

In \tex one could simply define a command \cmd{\insert} with one optional argument to handle the positioning. The positioning uses the letters [b,t,s,e] to position the glyph. the letters s and e stand for start and end, whereas b,t for bottom and top respectively. When there are only two symbols involved, this is not such a difficult operation, but when three or more symbols are to be grouped and kerned together, inserting with some form of scaling is necessary.

\subsection{Enclosures}

Enclosures. The two principal names of the king, the \emph{nomen} and \emph{prenomen}, were normally
written inside a \emph{cartouche}: a pictographic representation of a coil of rope.

In the Unicode representation of hieroglyphic text, the beginning and end of the cartouche
are represented by separate paired characters, somewhat like parentheses. The Unicode manual states that `rendering of a full cartouche surrounding a name requires specialized layout software', which is of course an easy task for \tex.

\begin{macro}{\cartouche}
The commands \cmd{\cartouche} and \cmd{\cartouche}, from Peter Wilson's \pkgname{hierglyph} package have been used for many years to demonstrate the use of hieroglyphics with \latexe. 
\end{macro}

There are a several characters for these start and end cartouche characters, reflecting various styles for the enclosures.

\cartouche{{\hiero \char"13147}$sin^{2} x + cos^{2} x = 1$}
\Cartouche{{\hiero \char"13147}$sin^{2} x + cos^{2} x = 1$}

Unicode:{\hiero 𓇓𓏏𓊵𓏙𓊩𓁹𓏃𓋀𓅂𓊹𓉻𓎟𓍋𓈋𓃀𓊖𓏤𓄋𓈐𓎟𓇾𓈅𓏤𓂦𓈉 }

\textpmhg{\HQ} 

\cartouche{\pmglyph{K:l-i-o-p-a-d:r-a}}
%\translitpmhg{\HK\Hl\Hi\Ho\Hp\Ha\Hd\Hr\Ha}

\printunicodeblock{./languages/hieroglyphics.txt}{\hiero}
\printunicodeblock{./languages/hieroglyphics-13100.txt}{\hiero}
\printunicodeblock{./languages/hieroglyphics-13200.txt}{\hiero}
\printunicodeblock{./languages/hieroglyphics-13300.txt}{\hiero}
\printunicodeblock{./languages/hieroglyphics-13400.txt}{\hiero}
\section{Numerals}

Egyptian numbers are encoded following the same principles used for the
encoding of Aegean and Cuneiform numbers. Gardiner does not supply a full set of
numerals with catalog numbers in his Egyptian Grammar, but does describe the system of
numerals in detail, so that it is possible to deduce the required set of numeric characters.

Two conventions of representing Egyptian numerals are supported in the Unicode Standard.
The first relates to the way in which hieratic numerals are represented. Individual
signs for each of the 1s, the 10s, the 100s, the 1000s, and the 10,000s are encoded, because in
hieratic these are written as units, often quite distinct from the hieroglyphic shapes into
which they are transliterated. The other convention is based on the practice of the \emph{Manual
de Codage}, and is comprised of five basic text elements used to build up Egyptian numerals.
There is some overlap between these two systems.

%% Needs some work to get it into LuaLaTeX
%% omitted for the time being
%\ifxetex
%\begin{texexample}{TeXeXglyph}{ex:xetexglyph}
%\raggedright
%\font\myfont = "Aegyptus"
%\setcounter{glyphcount}{136}
%
%\whiledo
%{\value{glyphcount}<\XeTeXcountglyphs\myfont}
%{\arabic{glyphcount}:~
%{\myfont\XeTeXglyph\arabic{glyphcount}}\quad
%\stepcounter{glyphcount}}
%\end{texexample}
%\fi

\section{Input Methods}

If you writing a document with a lot of hieroglyphics inputting of hieroglyphics can be problematic. Most researchers in the field will use special keyboards or editors. They also use MS/Word or OpenOffice. They can both be coerced to produce reasonable documents, but with \tex obviously better results can be achieved. One such editor is \href{http://jsesh.qenherkhopeshef.org/}{jsesh}. 


\begin{luacode*}
    local h = {}
          h = dofile("hiero.lua")
    local options = {style="block",
                     echo=true,
                     direction="RL",
                     size = "\\Huge",
                     color = "green",
                     headings = "captionof{figure}"  -- section/tablecaption/figurecaption
                     }
   -- prints full symbol list
   h.printgardiner(t,options)

   tex.print("\\par")
   local options = {style="block",
                     echo=true,
                     heading="\\par",
                     direction="RL",
                     color = "teal",
                     scale = 8}

   h.printhierochar("hiero","1317D",options)
   h.printhierochar("hiero","13000",{direction="RL",
                                        color = "teal",
                                        scale = 8})
   h.printhierochar("hiero","13003",{direction="LR",
                                        color = "teal",
                                        scale = 1})
   h.parseMdC([[M23-X1-R4-X8-Q2-D4-W17-R14-G4-R8-O29-
               V30-U23-N26-D58-O49-Z1-F13-N31-V30-N16-
               N21-Z1-D45-N25!]])

   tex.print("\\par")
   h.printgardinercat("B")

\end{luacode*}

\newcommand\hierochar[2][direction = "LR",
                         color     = "teal",
                         scale     = 1]{% 
               \luaexec{
                h = h or {}
                h = require("hiero.lua")  
                h.parseMdC(#2,{#1})}}
               
\newcommand\printhierochar[3][direction = "LR",
                              color     = "teal",
                              scale     = 4]{% 
               \luaexec{
                h = h or {}
                h = require("hiero.lua")  
                h.printhierochar(#2,#3,{#1})}}

This file just tests the various commands available for manipulating hieroglyphics. We tried to 
generalize the commands, so they can be re-used for other type of hieroglyphics.

{
\hierochar{"A1-A2-A3!"}

\centering 

\def\options{direction = "LR",
             color     = "teal",
             scale     = 7}

\def\fontname{"hiero"}

\def\hierochar#1{\printhierochar[\options]{\fontname}{#1}}
}


\begin{scriptexample}[]{Some Example}
Sometimes kerning might be required, especially if the
glyphs are scaled.This is easily achieved with a \cmd{\kern}
command and a suitable skip dimension.

\medskip

\bgroup
\fboxsep=0pt\fboxsep.4pt
\def\options{direction = "RL",
             color     = "black!95",
             scale     = 5}
\centering

\color{teal}
\fbox{\hierochar{"13051"}}
\kern-4mm
\hierochar{"13003"}
\def\options{direction = "LR",
             color     = "black!95",
             scale     = 5}
\fbox{\hierochar{"13003"}}\color{red}
\kern-4mm
\hierochar{"13051"}
\color{black!95}
\egroup
\begin{verbatim}
\centering
\hierochar{"13051"}
\kern-4mm
\hierochar{"13003"}
\def\options{direction = "RL",
             color     = "black!95",
             scale     = 5}
\hierochar{"13003"}
\kern-4mm
\hierochar{"13051"}
\end{verbatim}
\end{scriptexample}

A bit of a diversion is appropriate at this point. Our attempt after the historical overview, is to provide some routines for the capturing and display of hieroglyphic texts using LuaTeX. This involves getting low level information from the system regarding fonts. 

\begin{figure}[ht]
\begin{minipage}{0.45\textwidth}
\centering
\includegraphics[width=0.6\textwidth]{./images/fontforge.jpg}
\end{minipage}
\begin{minipage}[t]{0.45\textwidth}
\caption{Viewing font information with fontforge.}
\end{minipage}
\end{figure}

For each glyph, we are interested to get its unicode number, the position in the font table, its name and most importantly the font metrics. The font metrics are a set of parameters that are used to measure the bounding box, any ascenders or descenders and similar information. Using fontforge, these parameters can easily be viewed. However, we are not interested to make any modifications manually; what we are interested is to programmatically obtain this information using Lua. Lua's philosophy and a mantra repeated often by the developers, is that it provides the tools and not the solutions. What this means to the LuaTeX programmer, is that we need to reach very low level  to get this information, which is a road with many bumps. Luckily the tools have been provided by the LuaTeX developers. This comes with a lot of benefits as we can also do our own on the fly mapping, such as creating an index table holding all the Gardiner numbers. 

The |fontloader.open| function loads a font, but it's not usable by itself; the result should be turned into a table with
\textbf{fontloader.to\_table}, as follows:

\begin{verbatim}
  local f = fontloader.open
     ("c:/windows/fonts/NotSansEgyptianHieroglyphics-
       Regulat.ttf")
  fonttable = fontloader.to_table(f)
  fontloader.close(f)
\end{verbatim}

We will use the Google No Tofu Egyptian Hieroglyphic font to experiment with our hieroglyphics. I have used a full path to load the font, which resides on my windows machine in the fonts folder. Once we load all the information in the |fonttable| we use |fontloader.close| to discard the userdata from which the table is extracted. 

What makes OpenType fonts special is that they describe every aspect that you might be able to think of when you think of putting letters together to form words. In addition to the obvious "this is what letters look like" information, OpenType fonts also specify things like the name of each letter that is available in the font, how much of the Unicode standard the font implements, which horizontal and vertical metrics apply to which letters, exactly how the letters are arranged inside the font so that they can quickly be read out, what kind of font classifications apply (is it a fantasy font? is it bold face? is it fixed width? etc), what kind of memory allocation a printer needs to perform in order to be able to even load the font, etc. etc. etc. All these are stored in tables upon tables, similat to a collection of Russian dolls.

To view the values in the fonttable, we will first iterate over the \textbf{fonttable} and extract all the first level keys.

\begin{texexample}{Iterating through a font table}{}
\begin{luacode*}
local z={}
tf=fontloader.to_table(fontloader.open("c:/windows/fonts/NotoSansEgyptianHieroglyphs-Regular.ttf"))

-- we sort the keys to create a table
-- important keys to us are tf.glyphs

for k,v in pairs (tf) do
   --tex.print(k.."\\par")
   table.insert(z, k)
end

table.sort(z)
tex.print("\\begin{multicols}{3}\\raggedright")
for k,v in pairs (z) do
   z[k] = string.gsub(z[k],"%_","\\textunderscore ")
   local s = tf[v]
   tex.print("\\textbullet\\hskip3pt\\hangindent2em " .. z[k].." [\\textit{"..type(s).."}] ","\\par")
end
tex.print("\\end{multicols}")
\end{luacode*}
\end{texexample}

We iterate through the \textbf{fonttable} using the Lua  "pair" iterator and we simply print all the keys and the type of the values in a human readable form as shown in the example. Note the use of |\textunderscore| that replaces all underscores in the fields with its text equivalent to sanitize the output. This is a quick and dirty way to avoid the use of catcodes. Many of the keys, bear intuitive names and are not difficult to discern: \textit{version}, \textit{copyright} and the like. Getting the type of Lua variables is important in order to use them for error trapping. When you attempt for example to print a nil value an error will occur.

Now that we have peeked under the font we will iterate and capture the information of interest, which we will put into another table with two keys \textbf{info}  and \textbf{metrics}. In the metrics file we will get the bounding box related metrics of each and every glyph in the font and save it, into our own table. 

\begin{texexample}{More Metrics}{}
  \begin{luacode*}
   tex.print("units per em = ", tf.units_per_em,"\\par")
   for i,j in ipairs (tf.glyphs[6].boundingbox) do
      tex.print("bounding box["..i.."]".." = ", j,"\\par")
   end 
   local w = (tf.glyphs[6].boundingbox[3]-tf.glyphs[6].boundingbox[1])/tf.units_per_em
   local h = tf.glyphs[6].boundingbox[4]/tf.units_per_em
   tex.print("glyph width = ", w,"em\\par")
   tex.print("glyph height = ", h,"em\\par")

-- presents a nicely typeset table 

local rep, write = string.rep, tex.print
function ExploreTable (tab, offset)
    offset = offset or ""
    for k, v in pairs (tab) do
        local newoffset = offset .. "\\mbox{.}"
        if type(v) == "table" then
           -- if k == "boundingbox" then write("BB") end
           write(offset..k .. " = \\{\\par ")
           ExploreTable(v, newoffset)
           write(offset..newoffset .. "\\}\\par")
         else
           write(offset..k .. " = "..tostring(v),"\\par")
         end
      end
end

write("\\par{\\ttfamily ")
ExploreTable(tf.glyphs[38],"\\mbox{.}")
write("}")
  \end{luacode*}
\end{texexample}

The OpenType fonts standard, provides for so much information that we will ignore most of the items and focus on only a few tables and fields. A small utility after Paul Isambert's article is necessary to enable us to view tables easily within this book,


\begin{texexample}{ExploreTable utility}{}
\begin{luacode*}
-- presents a nicely typeset table 

local rep, write = string.rep, tex.print
function ExploreTable (tab, offset)
    offset = offset or ""
    for k, v in pairs (tab) do
        local newoffset = offset .. "\\mbox{.}"
        if type(v) == "table" then
           -- if k == "boundingbox" then write("BB") end
           write(offset..k .. " = \\{\\par ")
           ExploreTable(v, newoffset)
           write(offset..newoffset .. "\\}\\par")
         else
           write(offset..k .. " = "..tostring(v),"\\par")
         end
      end
end

write("\\par{\\ttfamily ")
ExploreTable(tf.glyphs[38],"\\mbox{.}")
write("}")
  \end{luacode*}
\end{texexample}

A good utility also is |TTX| that will convert an OTF font to XML and back. This requires that you have python installed.\footnote{See some good guidelines as to how to install it at \url{http://www.glyphrstudio.com/ttx/}.} The utility uses python to do the conversion. The archive can be downloaded from \url{http://sourceforge.net/projects/fonttools/files/latest/download}. This is a three prong attack. You need to have python install, the numpy library and then the TTX package. The |TTX| program was written by the font designer Just van Rossum, brother of the creator of the Python language, Guido van Rossum. The tool converts TrueType into human-readable |XML| format. The most attractive feature of this tool is that it also perform the opposite operation that is create a TruType font from an |XML| file. The |XML| format makes the hierarchy of the format clearer. Since SVG fonts are also described in |XML| it becomes an easier task to convert an |SVG| font to a TrueType font. To convert |bar.ttf| into |bar.ttx| you simply write:

\begin{verbatim}
ttx bar.ttf
\end{verbatim}

Similarly for the opposite conversion, from |.ttx| to |.ttf|

\begin{verbatim}
ttx bar.ttx
\end{verbatim}

The generated ttx file is approximately ten times larger than the original |.ttf| file. The files generated are huge affairs and difficult to manage.The command line option |-l| prints a list of the tables in the font. |TTX| is indispensable in the ``humanization'' of TrueType fonts. The details of the tables and what each field represents are eloquently described in that indispensable book by Yannis Haralambous \textit{Fonts \& Encodings.} Although the book is now somewhat dated, it is still the best source of information on many esoteric topics related to fonts. 





