\chapter{East Asian Scripts}
\newfontfamily\yi{Microsoft Yi Baiti}
\index{Yi fonts>Microsoft Yi Baiti}
\parindent1em

\begin{tabular}{lll}
Han &Hiragana &Hangul\\
Bopomofo &Katakana &Yi\\
\end{tabular}
\index{Katakana}\index{Hiragana}
\index{Bopomofo}\index{Hangul}\index{Yi}
\index{East Asian Scripts>Katakana}
\index{East Asian Scripts>Hiragana}
\index{East Asian Scripts>Hangul}
\index{East Asian Scripts>Bopomofo}
\index{East Asian Scripts>Yi}


\section{Han CJK Unified Ideographs}
\index{CJK}
The Chinese, Japanese and Korean (CJK) scripts share a common background. In the process called Han unification the common (shared) characters were identified, and named "CJK Unified Ideographs". Unicode defines a total of 74,617 CJK Unified Ideographs.[1]

The terms ideographs or ideograms may be misleading, since the Chinese script is not strictly a picture writing system.
Historically, Vietnam used Chinese ideographs too, so sometimes the abbreviation "CJKV" is used. This system was replaced by the Latin-based Vietnamese alphabet in the 1920s.


\newfontfamily\cjk{Arial Unicode MS}
\unicodetable{cjk}{"4E00,"4E10,"4E20,"4E30,"4E40,"4}




\section{Bopomofo}
\label{s:bopomofo}
Bopomofo is the colloquial name of the \textit{zhuyin fuhao} or \textit{zhuyin} system of phonetic notation for the transcription of spoken Chinese, particularly the Mandarin dialect. Consisting of 37 characters and four tone marks, it transcribes all possible sounds in Mandarin. 

Bopomofo was introduced in China by the Republican Government, in the 1910s and used alongside the Wade-Giles system, which used a modified Latin alphabet. The Wade system was replaced by \textit{Hanyu Pinyin} in 1958 by the Government of the People's Republic of China,[1] at the International Organization for Standardization (ISO) in 1982 (ISO 7098:1982). Bopomofo remains widely used as an educational tool and electronic input method in Taiwan. On Windows the font Microsoft JhengHei can be used. 

Windows fonts that can be used \texttt{Microsoft JhengHei} and \texttt{SimSun}.

U+3100–U+312F
\newfontfamily\bopomofo{Microsoft JhengHei}

\begin{scriptexample}[]{Bopomofo}
{\centering\bopomofo 

伯帛勃脖舶博渤霸壩灞

}

\hfill \texttt{Typeset with \cmd{\bopomofo} and Microsoft JhengHei font }
\end{scriptexample}

\begin{scriptexample}[]{Bopomofo}
\newfontfamily\simsun{SimSun}
{\centering\simsun 

伯帛勃脖舶博渤霸壩灞

}
\hfill \texttt{Typeset with \cmd{\bopomofo} and SimSun font }
\end{scriptexample}

\arial

The Bopomofo Extended block, running from \unicodenumber{U+31A0-U31BF}, contains less universally recognized Bopomofo characters used to write various non-Mandarin Chinese languages. A few additional tone marks are unified with characters in the Spacing Modifier Letters block. 












\section{Yi}


The Yi script (Yi: {\yi ꆈꌠꁱꂷ} nuosu bburma [nɔ̄sū bū̠mā]; Chinese: {\cjk 彝文}; pinyin: Yí wén) is an umbrella term for two scripts used to write the Yi language; Classical Yi, an ideogram script, the later Yi Syllabary. The script is also historically known in Chinese as Cuan Wen (Chinese: 爨文; pinyin: Cuàn wén) or Wei Shu (simplified Chinese: 韪书; traditional Chinese: 違書; pinyin: Wéi shū) and various other names (夷字、倮語、倮倮文、毕摩文), among them "tadpole writing" (蝌蚪文).[1]

This is to be distinguished from romanized Yi (彝文罗马拼音 Yiwen Luoma pinyin) which was a system (or systems) invented by missionaries and intermittently used afterwards by some government institutions.[2][3] There was also a Yi abugida or alphasyllabary devised by Sam Pollard, the Pollard script for the Miao language, which he adapted into "Nasu" as well.[4][5] Present day traditional Yi writing can be sub-divided into five main varieties (Huáng Jiànmíng 1993); Nuosu (the prestige form of the Yi language centred on the Liangshan area), Nasu (including the Wusa), Nisu (Southern Yi), Sani (撒尼) and Azhe (阿哲).[6][7]

The Unicode block for Modern Yi is Yi syllables (U+A000 to U+A48C), and comprises 1,164 syllables (syllables with a diacritic mark are encoded separately, and are not decomposable into syllable plus combining diacritical mark) and one syllable iteration mark (U+A015, incorrectly named YI SYLLABLE WU). In addition, a set of 55 radicals for use in dictionary classification are encoded at U+A490 to U+A4C6 (Yi Radicals).[11] Yi syllables and Yi radicals were added as new blocks to Unicode Standard Version 3.0.[12]

Classical Yi - which is an ideographic script like the Chinese characters - has not yet been encoded in Unicode, but a proposal to encode 88,613 Classical Yi characters was made in 2007.[13]

\bgroup
\yi \char"A000: Yi Syllable It\\

\yi \char"A001: Yi Syllable Ix\\

\yi \char"A002: Yi Syllable I\\
\egroup

\begin{scriptexample}[]{Yi}
\unicodetable{yi}{"A000,"A010,"A020,"A030,"A040,"A050,"A060,"A070,"A080,"A090,"A0A0,"A0B0,"A0C0}
\end{scriptexample}


