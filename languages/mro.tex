\section{Mro (Mru language)}

\newfontfamily\mro{MroUnicode-Regular.ttf}
\def\textmro#1{{\mro #1\xspace}}

 Mro (or Mru) is a Tibeto-Burman language spoken primarily in Bangladesh with a few
speakers in India. 

Mru is a Tibeto-Burman language and one of the recognized languages of Bangladesh. It is spoken by a community of Mros (Mru) inhabiting the Chittagong Hill Tracts of Bangladesh and also in Burma with a population of 22,000 in Bangladesh according to the 1991 census. The Mros are the second-largest tribal group in Bandarban District of the Chittagong Hill Tracts. A small group of Mros also live in Rangamati Hill District.

The Mru language is considered "definitely endangered" by UNESCO in June 2010.[4]

The script was invented in the 1980s and is of the class of “messianic”
scripts with no genetic relationship with existing scripts. In the last 10 years there has been an acceptance
among all the Mro to use this script and literacy levels among the 100,000 Mro exceed 80\%.

Some of the characters of the Mro alphabet have a visual similarity to those from other alphabets, but this
relationship is purely coincidental, and the Mro alphabet stands alone as a unity.


The Mro script has no technical complexity: it is a simple left to right alphabet with no
combining characters or characters with special function. There are no tone marks. Some sounds are
represented by more than one letter. The sound [k] is usually represented by \textmro{𖩌} KEAAE kəɘ, as in \textmro{𖩌𖩑𖩗} kow
‘village’, \textmro{𖩄𖩑𖩁𖩌𖩑} boŋko ‘owl’, but in a few words the letter 𖩙 KOO ko is used, as in \textmro{𖩙𖩑} ko ‘gold’. The sound
[m] is usually represented by \textmro{𖩎} MAEM mɘm, as in \textmro{𖩎𖩆𖩁} maŋ ‘go’, \textmro{𖩔𖩎𖩑} śmo ‘fool’, but in a few words the
letter \textmro{𖩃} MIM mim is used, as in \textmro{𖩃𖩊𖩏} min ‘cat’, \textmro{𖩋𖩃𖩊} cmi ‘rice’. The sound [l] is usually represented by \textmro{𖩍} OL
\textmro{ɔl}, as in \textmro{𖩍𖩝𖩁} lɔŋ ‘boat’, \textmro{𖩈𖩍𖩆} khla ‘spoon’, but in a few words the letter \textmro{𖩛} LA la is used, as in \textmro{𖩛𖩆𖩎𖩖} lamɘ
‘moon’, and in a few words \textmro{𖩚} LAN lan is used (we have no example). The vowels \textmro{𖩑𖩖} oɘ are used as a
digraph to describe the vowel [ø].

We are using Philip Reimer's font which is freely available under SIL OFL licence at \href{http://phjamr.github.io/mro.html}{github}. Philip has also produced fonts for two other scripts: Lisu (Fraser) and Miao (Pollard). All three scripts were added to Unicode 7.0 in 2014.



\begin{scriptexample}[]{Mro}
\unicodetable{mro}{"16A40,"16A50,"16A60}
\end{scriptexample}


\printunicodeblock{./languages/mro.txt}{\mro}





