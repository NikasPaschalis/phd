\section{Ancient Anatolian Alphabets}
\label{s:anatolian}
The Anatolian scripts described in this section all date from the first millenium BCE, and were used to write various ancient Indo-European languages of western and southwestern Anatolia (now Turkey). All are related to the Greek script and are probably adaptations of it. 

\newfontfamily\lycian{Aegean.ttf}
\let\lydian\lycian
\let\carian\lydian

\section{Lycian}
\label{sec:lycian}
The Lycian alphabet was used to write the Lycian language. It was an extension of the Greek alphabet, with half a dozen additional letters for sounds not found in Greek. It was largely similar to the Lydian and the Phrygian alphabets.

 
\begin{scriptexample}[]{Lydian}
\unicodetable{lydian}{"10280,"10290}

Typeset with the \idxfont{Aegean.ttf} and the command \cmd{\lycian}
\end{scriptexample}

\begin{figure}[htb]

\begin{minipage}[b]{0.5\textwidth}
\includegraphics[width=1\linewidth]{./images/xanthian-obelisk.jpg}
\end{minipage}\hspace*{1em}
\begin{minipage}[b]{0.45\textwidth}
\captionof{figure}{Part of the Xanthian obelisk inscription. The Xanthian Obelisk, also known as the Xanthos or Xanthus Stele, the Xanthos or Xanthus Bilingual, the Inscribed Pillar of Xanthos or Xanthus, the Harpagus Stele, and the Columna Xanthiaca, is a stele bearing an inscription currently believed to be trilingual, found on the acropolis of the ancient Lycian city of Xanthos, or Xanthus, near the modern town of Kınık in southern Turkey. The three languages are Ancient Greek, Lycian and Milyan (the last two are Anatolian languages and were previously known as Lycian A and Lycian B respectively).}
\end{minipage}
\end{figure}


\printunicodeblock{./languages/lycian.txt}{\lycian}


\section{Lydian}
\label{sec:lydian}
 Lydian script was used to write the Lydian language. That the language preceded the script is indicated by names in Lydian, which must have existed before they were written. Like other scripts of Anatolia in the Iron Age, the Lydian alphabet is a modification of the East Greek alphabet, but it has unique features. The same Greek letters may not represent the same sounds in both languages or in any other Anatolian language (in some cases it may). Moreover, the Lydian script is alphabetic.



Early Lydian texts are written both from left to right and from right to left. Later texts are exclusively written from right to left. One text is boustrophedon. Spaces separate words except that one text uses dots. Lydian uniquely features a quotation mark in the shape of a right triangle.

The first codification was made by Roberto Gusmani in 1964 in a combined lexicon (vocabulary), grammar, and text collection.

\begin{scriptexample}[]{Lydian}
\unicodetable{lydian}{"10920,"10930}

\medskip

Typeset with the \idxfont{Aegean.ttf} and the command \cmd{\lydian}
\end{scriptexample}

Examples of words

\bgroup\lydian
𐤬𐤭𐤠  - Ora - "Month"

𐤬𐤳𐤦𐤭𐤲𐤬𐤩  - Laqrisa - "Wall"

𐤬𐤭𐤦𐤡  - "House, Home"

\egroup

Herodotus Hdt. 1.94 
Chapter on the Lydians is well known, but in order to evaluate it properly it will be
helpful to recall exactly what it says54:

\begin{latexquotation}
The Lydians have about the same customs as the Greeks, except that the
Lydians prostitute their female children. The Lydians are the first people
we know to have coined money of silver and gold, and they were the first to
be shopkeepers. The Lydians themselves also claim the invention of the
games that both they and the Greeks now play. They say that the invention
occurred at the same time that they colonized Tyrsenia. What they say
about these things goes like this (the following is in indirect discourse):
In the reign of Atys, son of Manes, there was a terrible famine
throughout Lydia. Although in hard straits, the Lydians persevered for
some time. But finally, when there was no let-up, they sought respite,
some trying one thing and others another. It was then that they invented
dice, and astragals, and ball, and all the other kinds of games, except for
draughts. For the Lydians don't claim to have invented draughts. After
their inventions, this is what they did about the famine. Every second
day they would play, all day, so as not to want food, and on the day
between they would eat, and not play. In this way they persevered for
eighteen years. Since the evil did not abate, but pressed them even
worse, their king divided them up into two parts, by lot: the one group
for staying on, the other group to emigrate from the country. And the
king himself was to be in charge of the group that remained, while in
charge of the departing group was the king's son, whose name was
Tyrsenos. The group whose lot it was to depart from the land went down
to Smyrna and built boats. They put everything they needed into the
boats and sailed away in search of life and land; passing by many
nations, they sailed until they reached the Ombrikians, where they built

cities for themselves and they still live there today. Instead of "Lydians",
they adopted a new name from the king's son, the man who led them.
Taking their eponym from him, they were called Tyrsenoi.

Well, then, the Lydians were enslaved by the Persians.
\end{latexquotation}



\section{Carian}
\label{sec:carian}
The Carian alphabets are a number of regional scripts used to write the Carian language of western Anatolia. They consisted of some 30 alphabetic letters, with several geographic variants in Caria and a homogeneous variant attested from the Nile delta, where Carian mercenaries fought for the Egyptian pharaohs. They were written left-to-right in Caria (apart from the Carian–Lydian city of Tralleis) and right-to-left in Egypt. Carian was deciphered primarily through Egyptian–Carian bilingual tomb inscriptions, starting with John Ray in 1981; previously only a few sound values and the alphabetic nature of the script had been demonstrated. The readings of Ray and subsequent scholars were largely confirmed with a Carian–Greek bilingual inscription discovered in Kaunos in 1996, which for the first time verified personal names, but the identification of many letters remains provisional and debated, and a few are wholly unknown.

Carian was added to the Unicode Standard in April, 2008 with the release of version 5.1. It is encoded in Plane 1 (Supplementary Multilingual Plane).
The Unicode block for Carian is \unicodenumber{U+102A0–U+102DF}:

\begin{scriptexample}[]{Carian}
\unicodetable{carian}{"102A0,"102B0,"102C0,"102D0}
\end{scriptexample}


\PrintUnicodeBlock{./languages/carian.txt}{\carian}



