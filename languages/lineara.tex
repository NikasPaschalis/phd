\section{Linear A}
\label{s:lineara}
\newfontfamily\lineara{Aegean.ttf}

Linear A is one of two currently undeciphered writing systems used in ancient Greece. Cretan hieroglyphic is the other. Linear A was the primary script used in palace and religious writings of the Minoan civilization. It was discovered by archaeologist Arthur Evans. It is the origin of the Linear B script, which was later used by the Mycenaean civilization.

In the 1950s, Linear B was largely deciphered and found to encode an early form of Greek. Although the two systems share many symbols, this did not lead to a subsequent decipherment of Linear A. Using the values associated with Linear B in Linear A mainly produces unintelligible words. If it uses the same or similar syllabic values as Linear B, then its underlying language appears unrelated to any known language. This has been dubbed the Minoan language.\footnote{\url{http://www.people.ku.edu/~jyounger/LinearA/LinAIdeograms/}}

\begin{scriptexample}[]{Linear A}
\unicodetable{lineara}{^^A  
\number"10600,"10610,"10620,"10630,"10640,"10650,"10660,"10670,
"10680,"10690,"106A0,"106B0,"106C0,"106D0,"106E0,"106F0,"10710,"10720,"10730,"10740,"10750,"10760,"10770}
\end{scriptexample}

Many of the characters form group and specialists name them such as vases in transliterations.

\begin{scriptexample}[]{Vases}
\begin{center}
\scalebox{3}{{\lineara \char"106A6}}
\scalebox{3}{{\lineara \char"106A5}}
\scalebox{3}{{\lineara \char"106A7}}
\scalebox{3}{{\lineara \char"106A9}}
\end{center}
\end{scriptexample}

Linear A contains more than 90 signs (open vowels and consonants+vowels) in regular use and a host of
logograms, many of which are ligatured with syllabograms and/or fractions; about 80\% of these
logograms do not appear in Linear B. While many of Linear A’s signs are also found in Linear B, some
signs are unique to A (e.g., A *301 and following), while some signs found in Linear B are not yet found
in Linear A (e.g., B 12, 14-15, 18-19, 25, 32-33, 36, 42-43, 52, 62-64, 68, 71-72, 75, 83-84, 89-91).

The Unicode Linear A encoding is broadly based on the GORILA ([{\arial ɡɔɹɪˈlɑː}]) catalogue
(Godart and Olivier 1976–1985), which is the basic set of characters used in decipherment efforts.However, “ligatures” which consist of simple horizontal juxtapositions are not uniquely encoded here, as
these may be composed of their constituent parts. On the other hand, “ligatures” which consist of stacked
or touching elements have been encoded. 





