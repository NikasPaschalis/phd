\section{Meroitic}
\label{s:meroitic}

The Meroitic script is an alphabetic script, used to write the Meroitic language of the Kingdom of Meroë in Sudan. It was developed in the Napatan Period (about 700–300 BCE), and first appears in the 2nd century BCE. For a time, it was also possibly used to write the Nubian language of the successor Nubian kingdoms. Its use was described by the Greek historian Diodorus Siculus (c. 50 BCE).

Although the Meroitic alphabet did continue in use by the Nubian kingdoms that succeeded the Kingdom of Meroë, it was replaced by the Coptic alphabet with the Christianization of Nubia in the sixth century CE. The Nubian form of the Coptic alphabet retained three Meroitic letters.

The script was deciphered in 1909 by Francis Llewellyn Griffith, a British Egyptologist, based on the Meroitic spellings of Egyptian names. However, the Meroitic language itself has yet to be translated. In late 2008 the first complete royal dedication was found,[1] which may help confirm or refute some of the current hypotheses.

The longest inscription found is in the Museum of Fine Arts, Boston.

\newfontfamily\meroitic{Nilus.ttf}^^A

\unicodetable{meroitic}{"109A0,"109B0,"109C0,"109E0,"109F0}%


The examples here use the \idxfont{Nilus.ttf} font of George Douros\footnote{\url{http://users.teilar.gr/~g1951d/}}.

The name of the queen of Amenhotp III is rendered Teie, i.e. Teye, in the Armana 
tablets. The name of the city dedicated to her in Nubia was therefore pronounced 
Ha-Teye and appears in Meroitic as eyita (1916:119)








