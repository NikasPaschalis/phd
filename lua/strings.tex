\chapter{Lua Strings Library}

\section{Introduction}


\tex never provided any mechanisms for the manipulation of strings and neither did \latex. This is a serious limitation for libraries that have to deal mostly with the manipulation and typesetting of textual data and one of the reasons to use \LUA. 

The \LUA string library in \LUA\tex has been extended to provide a number of additional functions: a very useful function is \luacmd{string.explode} which returns a table with the exploded string. 


\luacmd{string.len(string)} can be used to find the length of a string.
The \luafunction{string.len} function is not unicode aware and the slunicode library which be used for cases where utf characters above unicode codepoint 127 is required.

\begin{texexample}{Exploding strings}{}
\begin{luacode}
local z = string.explode("a b c d ", " ")
local   writeln = function (v) 
   return   tex.print("key = ", v, "\\par")
end

for _,v in ipairs (z) do
  writeln(v)
end
\end{luacode}
\end{texexample}

If you notice in the example above the space at the end of "d" was captured and inserted into the return table. The \luafunction{string.trim} trims a string and removes and spaces from the beginning and the end of a string.

\begin{texexample}{Trimming strings}{}
\begin{luacode*}


string.trim = function (s)
   return (string.gsub(s, "^%s*(.-)%s*$", "%1"))
end

-- string.trim = trim

local  writeln = function (v) 
   return   tex.print("key = ", v, "\\par")
end

local str = " a b c d e "

tex.print(str:trim(str).."zzzz\\par")

local z = string.explode(str, " ")
  
for _,v in ipairs (z) do
  writeln(v)
end
\end{luacode*}
\end{texexample}

One of the advantages of Lua, is that any library is just a table and hence can be extended easily. In the example above we extended the string library by adding the \textbf{line string.trim = trim}.  




\begin{texexample}{Finding the length  of a string}{}
\begin{luacode*}
local z = "ąΒβ"
tex.print(string.len(z),"\\par")
tex.print(unicode.utf8.len(z))
str="äöABCäö"
 i = font.id("pan")
 tex.print(i, font.current())
\end{luacode*}
\end{texexample}

For more details on the rest of the available methods, see the Lua and LuaLaTeX manuals.
