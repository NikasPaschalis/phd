\chapter{Variadic Functions}

Variadic functions are present in most modern languages and Lua is no exception. For example, the following function returns the summation of all its arguments\footnote{There is a difference from the way the |arg| was accessed in Lua 5.1 to Lua 5.2}:

\begin{texexample}{Variadic Functions}{ex:variadic}
\begin{luacode*}
function add (...)
	local s = 0
	for i, v in ipairs{...} do
   		s = s + v
	end
	return s
end
tex.print(add(3, 4 ,7,9,10))
\end{luacode*}
\end{texexample}

The interesting part of the function is the three dots notation |(...)| in the parameter list.

The three dots (...) in the parameter list indicate that the function is variadic.
When we call this function, Lua collects all its arguments internally; we call
these collected arguments the extra arguments of the function. A function can
access its extra arguments using again the three dots, now as an expression.
In our example, the expression {...} results in an array with all collected
arguments. The function then traverses the array to add its elements.

This type of  expression |...| is called  a \emph{vararg} expression. It behaves like a multiple
return function, returning all extra arguments of the current function. For
instance, the command \texttt{print(...)} prints all extra arguments of the function. The three dots is just a s

Lua allows to have any number of fixed arguments before the dots. Lua assigns the fixed arguments first and the rest as extra arguments.

\begin{texexample}{Named Arguments}{ex:namedargs}
\begin{luacode*}
function Call(...)
   local arg = {n=select('#',...),...}
  -- arg.n is the real size
  for i = 1,arg.n do
    tex.print(arg[i])
  end
 end

Call("sth", 1, 2, 6, "yo", "test" )	

\end{luacode*}
\end{texexample}


\section{Named Arguments}

The Lua mentality is that it provides tools and not solutions. In languages like |python| or |php| one can create functions with named parameters for example: \texttt{myfunction(a=1,b=10,c=20)}, where the values of the arguments are default values, in case the function is called without them. This is very useful also when functions can have many optional parameters.








