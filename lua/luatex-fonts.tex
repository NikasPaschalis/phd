
\let \handlegroupnormalbefore \relax
\let \handlegroupnormalafter  \relax

\protected \def \handlegroupnormal #1#2{^^A
  \bgroup
  \def \handlegroupbefore {#1}^^A
  \def \handlegroupafter  {#2}^^A
  \afterassignment \handlegroupnormalbefore
  \let \next =^^A
}

\def \handlegroupnormalbefore {^^A
  \bgroup 
  \handlegroupbefore
  \bgroup 
  \aftergroup \handlegroupnormalafter%
}

\def \handlegroupnormalafter {%
  \handlegroupafter
  \egroup 
  \egroup 
}

\let \groupedcommand \handlegroupnormal 

\def \definehighlight [#1][#2]{%
  \ifcsname #1\endcsname\else
    \expandafter\def\csname #1\endcsname{%
      \leavevmode
      \groupedcommand {#2}\empty%
    }
  \fi%
}



\def\restoreunderscore{\catcode`\_=12\relax}

\definehighlight         [fileent][\ttfamily\restoreunderscore]         %% files, dirs
\definehighlight        [texmacro][\sffamily\itshape\textbackslash]     %% cs
\definehighlight     [luafunction][\sffamily\itshape\restoreunderscore] %% lua identifiers
\definehighlight      [identifier][\sffamily]                           %% names
\definehighlight          [abbrev][\rmfamily\scshape]                   %% acronyms
\definehighlight        [emphasis][\rmfamily\slshape]                   %% level 1 emph

\definehighlight       [Largefont][\Large]                              %% font size
\definehighlight       [smallcaps][\sc]                                 %% font feature
\definehighlight [nonproportional][\tt]                                 %% font switch
%\let \inlinecode \lstinline
\protected \def \inlinecode {\lstinline}

\chapter{LuaTeX and Fonts}
\label{c:luatexfonts}


\section{Introduction}
With \LuaTeX/\LuaLaTeX a completely different mechanism to that of \XeLaTeX or \pdfLaTeX is used to load fonts.

Font management and installation has always been painful with \tex.  A
lot of files are needed for one font ({tfm},{pfb},
{map}, {fd}, {vf}), and due to the 8-Bit encoding
each font is limited to 256 characters.

But the font world has evolved since the original \tex, and new
typographic systems have appeared, most notably the so called
\emph{smart font} technologies like \OpenType fonts ({otf}).

These fonts can contain many more characters than \tex fonts, as well
as additional functionality like ligatures, old-style numbers, small
capitals, etc., and support more complex writing systems like Arabic
and Indic
  Unfortunately, \pkgname{luaotfload} doesn‘t support many Indic
  scripts right now.
scripts.

In \identifier{luaotfload}, the canonical syntax for font requests
requires a \emphasis{prefix}:
%
\begin{quote}
  \nonproportional{\string\font\string\fontname\space= }%
  \meta{prefix}%
  \nonproportional{:}%
  \meta{fontname}%
  \dots
\end{quote}
%
where \meta{prefix} is either \inlinecode{file:} or \inlinecode {name:}.
  The development version also knows two further prefixes,
  \inlinecode {kpse:} and \inlinecode {my:}.
  %
  A \inlinecode {kpse} lookup is restricted to files that can be found by
  \identifier{kpathsea} and
  will not attempt to locate system fonts.
  %
  This behavior can be of value when an extra degree of encapsulation is
  needed, for instance when supplying a customized tex distribution.

  The \inlinecode {my} lookup takes this a step further: it lets you define
  a custom resolver function and hook it into the 

   \verb+\luafunction{resolve_font}+

  callback.
  %
  This ensures full control over how a file is located.
  %
  For a working example see the
  \url {https://bitbucket.org/phg/lua-la-tex-tests/src/5f6a535d/pln-lookup-callback-1.tex}.

%
It determines whether the font loader should interpret the request as
a \emphasis{file name} or
  \emphasis{font name}, respectively,
which again influences how it will attempt to locate the font.
%
Examples for font names are
            “Latin Modern Italic”,
            “GFS Bodoni Rg”, and
            “PT Serif Caption”
-- they are the human readable identifiers
usually listed in drop-down menus and the like.\footnote{%
  Font names may appear like a great choice at first because they
  offer seemingly more intuitive identifiers in comparison to arguably
  cryptic file names:
  %
  “PT Sans Bold” is a lot more descriptive than \fileent{PTS75F.ttf}.
  On the other hand, font names are quite arbitrary and there is no
  universal method to determine their meaning.
  %
  While \identifier{luaotfload} provides fairly sophisticated heuristic
  to figure out a matching font style, weight, and optical size, it
  cannot be relied upon to work satisfactorily for all font files.
  %
  For an in-depth analysis of the situation and how broken font names
  are, please refer to
  \hyperlink [this post]{http://www.ntg.nl/pipermail/ntg-context/2013/073889.html}
  by Hans Hagen, the author of the font loader.
  %
  If in doubt, use filenames.
  %
  \fileent{luaotfload-tool} can perform the matching for you with the
  option \inlinecode {--find=<name>}, and you can use the file name it returns
  in your font definition.}

%

\subsection {Compatibility Layer}

In addition to the regular prefixed requests, \identifier{luaotfload}
accepts loading fonts the \XETEX way.
%
There are again two modes: bracketed and unbracketed.
A bracketed request looks as follows.

\begin{quote}
  \nonproportional{\string\font\string\fontname\space = [}%
  \meta{/path/to/file}%
  \nonproportional{]}
\end{quote}

\noindent
Inside the square brackets, every character except for a closing
bracket is permitted, allowing for specifying paths to a font file.
%
Naturally, path-less file names are equally valid and processed the
same way as an ordinary \inlinecode {file:} lookup.

\begin{quote}
  \nonproportional{\string\font\string\fontname\space= }%
  \meta{font name}
  \ldots
\end{quote}

Unbracketed (or, for lack of a better word: \emphasis{anonymous})
font requests resemble the conventional \TEX syntax.
%
However, they have a broader spectrum of possible interpretations:
before anything else, \identifier{luaotfload} attempts to load a
traditional \TEX Font Metric (\abbrev{tfm} or \abbrev{ofm}).
%
If this fails, it performs a \inlinecode {name:} lookup, which itself will
fall back to a \inlinecode {file:} lookup if no database entry matches
\meta{font name}.

Furthermore, \identifier{luaotfload} supports the slashed (shorthand)
font style notation from \XETEX.

\begin{quote}
  \nonproportional{\string\font\string\fontname\space= }%
  \meta{font name}%
  \nonproportional{/}%
  \meta{modifier}
  \dots
\end{quote}





\texttt{OpenType} fonts are widely deployed and available for all modern
operating systems.

As of 2014 they have become the de facto standard for advanced text
layout.

However, until recently the only way to use them directly in the \tex
world was with the \XeTeX engine.

Unlike \XeTeX, \LUATEX has no built-in support for \OpenType or
technologies other than the original \tex fonts.

Instead, it provides hooks for executing LUA code during the TEX run
that allow implementing extensions for loading fonts and manipulating
how input text is processed without modifying the underlying engine.

This is where \pkgname{luaotfload} comes into play:
Based on code from \CONTEXT, it extends \LUATEX with functionality necessary
for handling \OpenType fonts.

Additionally, it provides means for accessing fonts known to the operating system conveniently by indexing the metadata. The |luaotfload| package is an adaptation of the \CONTEXT font loading system, allowing for loading \OpenType fonts with an extended syntax and adds support for a variety of font features. With current developments of moving \xetex to \luatex it is the expected way of \latex to evolve.

\section{Loading Fonts}

|luaotfload| supports an extended syntax for font loading, similar to that used by \xetex.



\CMDI{\font}=\meta{prefix}:\meta{font name}:\meta{font features}\meta{\tex features}

The curly brackets are optional and escape the spaces in the enclosed
font name. Alternatively, double quotes serve the same purpose.

\section{Below the Hood}

Most \tex users like to have full control and are curious to understand how things work. Using |luaotfload| we can have a look at the |otf| tables, and although a bit error prone can 
see how the fonts store information. The first example we will examine is a short programme to print 50 of the glyph names. In Example \ref{ex:symbola}, we use the built-in  \luacmd{fontloader} function. Many font symbolic names have underscores or other problematic characters. We change the underscore using \luacmd{string.gsub} from the |strings| module, which is loaded automatically with LuaTeX.

\newfontfamily{\symbola}{Symbola.ttf}

\begin{texexample}{Getting the name of Glyphs}{ex:symbola}
\bgroup
\begin{luacode}
tex.print("\\footnotesize")
f=fontloader.open("c:/windows/fonts/Symbola.ttf")
tex.print (f.fontname,'\\par\\symbola')
local i = 65
while (i < 200) do  
local g = f.glyphs[i]
if g then
  local s = string.gsub(g.name, '%_', '\\textunderscore  ')
  tex.print(s .. "(\\symbola\\char" .. g.unicode ..")" )
end
  i = i + 1
end
fontloader.close(f)
\end{luacode}
\egroup
\end{texexample}

If you are compiling this document in Windows, the above example would probably find your fonts. However, this is not a very good way to search for fonts. The TeX world has settled over many years on two principles when distributing files. The TeX Directory Structure (TDS) which is a directory hierarchy for macros, fonts, and the other implementation-independent TeX system files and the kpathsea utility for locating these files and for running the many scripts that are necessary while typesetting.

\begin{texexample}{Getting the name of Glyphs}{}
\begin{luacode}
local filename = "Symbola.ttf"
local fullname = kpse.find_file(filename, 'truetype fonts') 
if fullname then tex.print(fullname) else tex.print("font ".. filename .. " not found") end
\end{luacode}
\end{texexample}

The various fontloader programs go at great lengths to ensure that all possible directory paths are searched and the main reason, why when using a font for the first time it takes so long to process. With Lua once the
 file is located, we can read it and manipulate the data in any way we want.





