 %%% TODO HOW TO HANDLE BOTH LOCALLY
 %%%
 \subsection{Integrals}
 \label{integrals}
 Integrals come in two styles, the slanted versions shown below (|$\intsl$|,
 etc.)\ and right versions such as~|$\int$|. By default, the symbol names
 listed below will give you the slanted style, but if you specify the |int|
 package option, they will give you the corresponding right symbols.

 It is highly recommended that authors stick to the names below and use the
 |int| package option to choose a style globally for their document.
 However, in recognition of the fact that it might occasionally be necessary
 to mix the two styles, alternative names have been provided for all
 integrals. Append |sl| or || to the names below to request either the
 \emph{sl}anted or the \emph{}right variant. Thus, |$\intsl$| will always
 yield~|$\intsl$| and |$\int$| will always yield~|$\int$|, and similarly for
 the other integrals.
%
 \begin{multicols}{2}
 \integralsymbol\int{222B}{}
 \integralsymbol\iint{222C}{}
 \integralsymbol\iiint{222D}{}
 \integralsymbol\oint{222E}{}
 \integralsymbol\oiint{222F}{}
 \integralsymbol\oiiint{2230}{}
 \integralsymbol\intclockwise{2231}{}
 \integralsymbol\varointclockwise{2232}{}
 \integralsymbol\ointctrclockwise{2233}{}
 \integralsymbol\sumint{2A0B}{}
 \integralsymbol\iiiint{2A0C}{}
 \integralsymbol\intbar{2A0D}{}
 \integralsymbol\intBar{2A0E}{}
 \integralsymbol\fint{2A0F}{}
 \integralsymbol\cirfnint{2A10}{}
 \integralsymbol\awint{2A11}{}
 \integralsymbol\rppolint{2A12}{}
 \integralsymbol\scpolint{2A13}{}
 \integralsymbol\npolint{2A14}{}
 \integralsymbol\pointint{2A15}{}
 \integralsymbol\sqint{2A16}{}
 \integralsymbol\intlarhk{2A17}{}
 \integralsymbol\intx{2A18}{}
% \integralsymbol\intcapup{2A19}{}
 %\integralsymbol\intc{2A1A}{}
% \integralsymbol\intcup{2A1B}{}
 \integralsymbol\lowint{2A1C}{}
 \end{multicols}

\endinput
