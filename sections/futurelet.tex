\chapter{Futurelet}
\precis{A discussion on one of the most esoteric commands of \protect\tex, with examples as to how to write macros with optional arguments.}
\addtocimage{-12pt}{-20pt}{../images/tocblock-futurelet.jpg}
\epigraph{Life can only be understood backwards; but it must be lived forwards.}{
---S Kierkegaard}

The \cmd{\futurelet} primitive deserves its own chapter, as most people have difficulty in understanding the command. The instruction allows the user to \textit{look ahead}. By look ahead we mean that \tex will look at a future token\footnote{remember that a token is either a single character or a macro command} without absorbing it, i.e, without removing that token from the token list. This operation allows the programmer to perform a test to check what token is 'coming'. You can read a couple of articles about it for example \citep{Eijkhout2001}, but generally they are difficult to follow. The information about the command is also very sparse in the TeXBook.  Another TUGboat article is \citep{bechto88}, which gives pretty much the same example as we describe below. 

The token looked at through
|\futurelet| will be removed later, typically as part
of an argument of a later macro call as we will see
shortly. It is not removed by the action of the
|\futurelet| primitive.

Let us be more precise now; the |\futurelet|
instruction has the following format:


\begin{teX}
\futurelet (tokenl) (token2) (token3)
\end{teX}


\begin{enumerate}
\item  \tex will execute a \cmd{\let}\meta{tokenl}=\meta{token3}.
We therefore have generated a copy of (token3)
stored under the name of (tokenl).\label{lettoken}


\item  removes (tokenl) from the main token list.

\item \tex expands (token2). This token is for all
practical purposes a macro with the following
properties:

(a) The macro will use (tokenl), which is a
copy of (token3), to find out what (token3)
is, in other words what token is to be
expected later.
(b) It will cause another macro to be expanded
which will ultimately absorb (token3).

This other macro ordinarily depends on
what $<token_l>$ is.

\end{enumerate}

The description above, is a bit of a mouthful and it is better to describe it with an example. In Example~\ref{futurelet} we will try and find if the next token is the opening square bracket `['. We then according to the definition in \ref{lettoken} this should be stored in \cs{tokenone}. We verify this by peeking at its meaning.

\begin{texexample}{futurelet}{futurelet}
\def\tokentwo#1{}
\futurelet\tokenone\tokentwo[
\meaning\tokenone
\end{texexample}

The second token \cs{tokentwo} we have defined it, so that it justs absorbs its next argument and does nothing for the time being. As you can see its meaning is \texttt{the character [}. Now what happens if there was a space between the \cs{tokentwo} and the `['?

\begin{texexample}{futurelet second}{futurelet2}
\def\tokentwo#1{}
\futurelet\tokenone\tokentwo     [
\meaning\tokenone
\end{texexample}

As you can see so far the spaces have been absorbed, but let us now change the definition of \cs{tokentwo}.

\begin{texexample}{futurelet second}{futurelet2}
\def\tokentwo#1{}
\futurelet\tokenone\tokentwo     
\meaning\tokenone
\end{texexample}



\begin{texexample}{futurelet}{futurelet}
\def\tokentwo#1{%
   \ifx\tokenone[ true [\else false\fi
}
\futurelet\tokenone\tokentwo[
\meaning\tokenone
\end{texexample}

We try again with spaces,

\emphasis{tokentwo,[}
\begin{texexample}{futurelet}{futurelet}
\def\tokentwo#1{%
   \ifx\tokenone[ true [\else false\fi
}
\futurelet\tokenone\tokentwo     [
\meaning\tokenone
\end{texexample}

As you can see from the examples we cannot capture the spaces. This might present a problem, if we enclose everything in other macros as \tex might leave extra spaces in the stream. Better to absorb them. We will see how later, using LaTeX. 


\section{Applications}

There are many applications of |\futurelet|.
will here present only one example, although
we will present it in quite some detail so the user
will know how to apply |\futurelet| in different
circumstances.

\subsection{Using \textbackslash futurelet in Macros with Optional
Arguments}

A typical application of |\futurelet| is the handling
of macros with optional arguments\cite{Becht1988} as they are used,
for instance, in \latex. By "optional argument" we
mean an argument which in most cases is omitted,
and is provided only occasionally in macro calls.\footnote{See also the discussion at \url{http://tex.stackexchange.com/questions/4557/how-to-use-futurelet-to-define-optional-parameters}}

\textbf{Defining the Problem}

Let us give a specific example: we would like to
define a macro \cmd{xx}, which can be called in two
different ways:

\begin{enumerate}
\item With optional argument as in |\xx [opt]{arg}|
where opt is the optional argument enclosed
in square brackets and \meta{arg} is the mandatory argument
argument.

\item Without optional argument as in |\xx{arg}|
where \meta{arg} is again the regular argument.

\end{enumerate}


Before we discuss how this can be done in \tex,
observe that we do not really have to use an
optional argument. We could simply define two
different macros \cmd{xxwithoptions} for the case where an
optional argument is given, and \cmd{xxnooptions} for the
case where no optional argument is given:


\begin{texexample}{two macros}{ex:twomacros}
\def\xxWithOpt [#1]#2{...}
\def\xxNoOpt #1{...}
\def\xxWithOpt (#1)#2{\fbox{#2}}
\xxWithOpt (box){Testing}
\end{texexample}

How we can use |\futurelet| to find out
whether an optional argument was given or not?

We will define a macro |\xx| whose only function is
to check whether there is an opening square bracket
(optional argument is present) or not (no optional
argument). The |\xx| macro will, after this has been
determined, cause the |\xxWithOpt| macro to be invoked
when there is an optional argument, and the
|\xxNoOpt| macro to be called if there is no opening
bracket. In other words the macros |\xxWithOpt|
and |\xxNoOpt| do the "real work while the only
purpose of the |\xx| macro is to decide which of the
two macros should be invoked.


Here is the completely worked out example.


\begin{teX}
\def \xxWithOpt [#1] #2{...}
\def\xxNoOpt #2{...}

\def\xx {%
\futurelet\xxLookedAtToken
    \xxDecide
}

% (3) The \xxDecide macro, based on
% the lookahead of \xx, calls
% either \xxWithOpt or \xxNoOpt .
\def\xxDecide {%
 \ifx\xxLookedAtToken [%
\let\next = \xxWithOpt
\else
 \let\next = \xxNoOpt
 \fi
\next
}
\end{teX}

\section{Other Applications in the LaTeX kernel}

\begin{teX}
\def\elidebefore[#1]#2{[$\ldots$] #2}
\def\elideafter#1{#1$\ldots$}

\def\elide {%
\futurelet\ifoptions
    \choosemacro
}

\elide{Lorem Ipsum}

\elide[b]{Lorem ipsum}
\end{teX}

\begin{comment}
% The \choosemacro, based on
% the lookahead of \elide, calls
% either \elidebefore or \elideafter 
\end{comment}

\begin{teX}
\def\choosemacro{%
 \ifx\ifoptions [%
     \let\choice = \elidebefore 
 \else
    \let\choice = \elideafter
 \fi
\choice
}
\end{teX}



\begin{teX}
\elide{Lorem Ipsum}

\elide[b]{Lorem ipsum}

\end{teX}

\begin{teX}
\def \xxWithOpt [#1] #2{...}
\def\xxNoOpt #2{...}

\def\xx {%
\futurelet\xxLookedAtToken
    \xxDecide
}

% (3) The \xxDecide macro, based on
% the lookahead of \xx, calls
% either \xxWithOpt or \xxNoOpt .
\def\xxDecide {%
 \ifx\xxLookedAtToken [%
\let\next = \xxWithOpt
\else
 \let\next = \xxNoOpt
 \fi
\next
}
\end{teX}



To build a command with any optional parameter, as you find in many of LaTeX's commands, you will need two things:

\begin{itemize}
\item a macro with delimited parameters

\item a way to grab the first non-space token that follows the command
\end{itemize}


The first part is fairly easy using delimited argument macros, for example we can say

\begin{verbatim}
\def\test(#1)#2#3{#1, #2, #3}
\end{verbatim}

We can then call this macro as:

\begin{verbatim}
\test(a){b}{c}
\end{verbatim}


resulting in a,b,c

To define the |()| as an optional parameter, we effectively need to define the macro as a conditional a sort of a "yes-no" switch. If \tex finds the "(" bracket the "yes-code" will be called and if it finds only the normal arguments the "no-code" will be executed.

For this we can use the |\@ifnextchar| macro from the LaTeX kernel.
You can say |@ifnextchar{char}{yes-code}{no-code}| to test for |(|. The result then will depend on the token that follows. If this token is the same as the first argument, then the "yes-code" is executed, otherwise the "no-code" is executed. The first argument should be a single token (for instance a character). Spaces are ignored. 

As for example we can redefine the LaTeX code for `rule` to accept an optional parameter in round brackets, rather than the traditional square brackets.

\begin{texexample}{Using ifnextchar}{}
\makeatletter
\def\Rule{\@ifnextchar(\@Rule%
        {\@Rule(\z@)}}
\def\@Rule(#1)#2#3{%
 \leavevmode
 \hbox{%
 \setlength\@tempdima{#1}%
 \setlength\@tempdimb{#2}%
 \setlength\@tempdimc{#3}%
 \advance\@tempdimc\@tempdima
 \vrule\@width\@tempdimb\@height\@tempdimc\@depth-\@tempdima}}
\makeatother

A test \Rule(6.5pt){100pt}{1pt}

Another test \Rule{100pt}{3pt}

Not that difficult but you will need to , but why on earth do you need this?
\end{texexample}




\begin{comment}
\def\elidebefore[#1]#2{[$\ldots$] #2}
\def\elideafter#1{#1$\ldots$}

\def\elide {%
\futurelet\ifoptions
    \choosemacro
}

% The \choosemacro, based on
% the lookahead of \elide, calls
% either \elidebefore or \elideafter 


\def\choosemacro{%
 \ifx\ifoptions [%
     \let\choice = \elidebefore 
 \else
    \let\choice = \elideafter
 \fi
\choice
}

Testing \elide[b]{Lorem ipsum}

\elide{Lorem Ipsum}

\elide[b]{Lorem ipsum}

\end{comment}


\section{Using LaTeX \protect\textbackslash @ifnextchar}

\latex defines the |\@ifnextchar| kernel command that is used effectively to
determine the token that follows the command. It is used in the definitions
of macros with optional arguments amongst other things.

\begin{teXXX}
\@ifnextchar]{true}{false}] 
\@ifnextchar[{true}{false}[
\end{teXXX}
The result would both be true,

\begin{texexample}{Example ifnextchar}{ifnextchar}
\makeatletter
\@ifnextchar]{true ]}{false} ] %notice ]
\@ifnextchar[{true [}{false} [ %notice [
\makeatother
\end{texexample}





















