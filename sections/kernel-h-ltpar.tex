\chapter{Paragraphs \texttt{File h: ltpar.dtx}}

\index{LaTeX kernel classes>File h ltpar.dtx}

This section of the kernel declares the commands used to set |\par| and |\everypar|
when ever their function needs to be changed for a long time. It is a very small class
and the commands that are defined are used extensively by other sections of the kernel.

As the kernel authors note, There are two situations in which |\par| may be changed:

\begin{enumerate}
\item  Long-term changes, in which the new value is to remain in effect until the
current environment is left. The environments that change |\par| in this way
are the following:

 All list environments (itemize, quote, etc.)
 
 Environments that turn |\par| into a noop: tabbing, array and tabular.

\item Temporary changes, in which |\par| is restored to its previous value the next
time it is executed. The following are all such uses.
|\end| when preceded by |\@endparenv|, which is called by |\endtrivlist|

 The mechanism for avoiding page breaks and getting the spacing right
after section heads.
\end{enumerate}

\Paragraph{\textbackslash @setpar} This initializes a long-term change to |\par|. The default definition of |\@par| will ensure that if |\@restorepar| defines |\par|
to execute |\@par| it will redefine itself to the primitive |\@@par| after one iteration.

\startlineat{3}
\begin{teX}
3 \def\@setpar#1{\def\par{#1}\def\@par{#1}}
4 \def\@par{\let\par\@@par\par}
\end{teX}

\Paragraph{\textbackslash @restorepar} Restore from a short-term change to |\par|.

\begin{teXX}
6 \def\@restorepar{\def\par{\@par}}
\end{teXX}

