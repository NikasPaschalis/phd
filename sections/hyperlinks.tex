\chapter{Electronic Documents}
\cxset{section align=left}
\label{ch:hyperlinks}

One of the reasons this package was developed was to handle the many conflicts of packages with the
\pkgname{hyperref} package. This package developed by Sebastian Rahtz and Heiko Oberdiek \citeyearpar{hyperref}
has brought the capabilities of the |PDF| format to the \tex world. For a good introductory article
see \cite{garcia}.

\section{Special Effects}

When the |hyperred| package is loaded a number of things happen without further intervention.  
The items in the table of contents, the list of figures, and the list of tables, will be links: when
the reader clicks on them, the cursor will jump to the corresponding target. 

\begin{enumerate}
\item  The superscript that calls for a footnote will be
a link to the footnote itself.
\item Bibliographical references through \cmd{\cite} will
create links to the entries in the final bibliography
list.
\item All pairs of  |\label-\ref| will also produce links
(the result of a |\ref| leading by mouse click to
the corresponding |\label|).
\end{enumerate}


%\begin{macro}{\pageref}
\begin{docCommand}{nameref}{ \marg{label}}
\label{hyperref01}
The macro \CMDI{\pageref} prints the number of the \emph{page} where the referenced elements appears (rather
than the element itself). The command |\pageref{label}|  \pageref{hyperref01} shows the pagenumber of this page.
\end{docCommand}


A third command \CMDI{\nameref} will typeset the name of the chapter or section. This is a much better way than
just printing the number of the chapter.




In Example~\ref{ex:hyperlinks} we demonstrate the use of the hyperlinking capabilities of the \cmd{\nameref}
command.  We
also demonstrate the built-in command \CMDI{\pkgname} that we use to link package names to their |ctan| repository.

\begin{texexample}{Hyperlinks with hyperref}{ex:hyperlinks}
In Chapter \ref{ch:hyperlinks} we added the |\label{ch:hyperlinks}|. In the Chapter \nameref{ch:hyperlinks}, we discuss
the use of the \pkgname{hyperref} package.
\end{texexample}




\section{Arbitrary cross references}

\begin{docCommand}{hypertarget}{\marg{key}\marg{text}}
 will link the target with the key and text.
\end{docCommand}

The macro \CMDI{\hyperlink}\marg{key}\marg{text}


\hypertarget{linktest}{This is a link}.   The \hyperlink{linktest}{test} when clicked will take you to the text this is a
link.

\def\sectionautorefname{Section}
\def\exampleautorefname{Example}
\def\refexample#1{Example~\ref{#1}\xspace}
\let\exampleref\refexample
\section{Automating referencing}



\label{sec:references}

In \autoref{sec:references} we typed the \cmd{\label}\marg{sec:references}. When we use the \CMDI{\autoref}\marg{sec:references} we get a  reference back to the label. 
The command automatically adds the refname of the sectioning command, appropriately.


\begin{texexample}{Auto referencing}{ex:autoref}
\begin{teX}
The \autoref{ex:autoref} creates a link to this example. 
The \refexample{ex:autoref} also uses a built-in command (*@ \label{refex} @*)
provided by the \pkgname{phd}
\end{teX}
\end{texexample}

As you can observe the Example (see line \ref{refex}), could not be automatically detected by the \pkgname{hyperref} package, so we have build another command \CMDI{refexample} to achieve the effect.

\section{Options}

\begin{docCommand}{hypersetup}{\marg{options}}
The |hyperref| package comes with over 60 options. It is not possible to cover them all here, but we have included
them in the phd package. The options can be set using the \docAuxCommand{hypersetup}, which provides a convenient API to hook into options after the loading of the package. 
\end{docCommand}

\section{How to Link to the Web}

\begin{docCommand}{url}{}
\begin{docCommand}{href}{}
The command \CMDI{\url} takes one argument---the destinations \textsc{URL} address---and creates a link to it. For example, |\url{www.tug.org}}| will open the default system browser at the url address. There is also a related
command \CMDI{\href} that creates a hypelink to the address. This can also be used to indicate an email
address using |\href{mailto:}|\meta{email address}\meta{link text}.
\end{docCommand}
\end{docCommand}

\section{Including PDFs in Documents}

The \pkgname{pdfpages}  developed by Andreas Mattias \citeyearpar{pdfpages} can be used to insert external
multi-page PDFs or PS documents. It supports pdfTeX, VTeX and XeTeX. One of the limitations of the current
engines is that any included pdfs will be inserted without any links. 

\begin{docCommand}{includepdf}{}

The main command of the package is \CMDI{\includepdf}\oarg{key=val}\marg{filename}. The key-value list
is used to specify a list of options. The filename is any name that can be found on a TeXpath. (See the \nameref{ch:graphics} for details on paths). 
\end{docCommand}

On of the pitfalls of the package is if you use a \CMDI{\pagecolor} to set the background of pages. The first
\cmd{\pagecolor}  must precede the loading of the package. 
With VTeX---and since LuaLaTeX is using VTEX---this is not a limitation and hence we have avoided any routines to by-pass the limitation.







