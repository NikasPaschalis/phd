% !TeX root = tcolorbox.tex
% include file of tcolorbox.tex (manual of the LaTeX package tcolorbox)
\cxset{section align=left, chapter border-left-width=1pt, 
         chapter border-left-color=sweet,
         chapter border-top-style=solid,
         chapter border-bottom-style=solid,
         chapter border-left-style=solid,
         chapter border-right-style=solid,
         chapter border-style=solid,
         subsubsection font-shape=upshape}

\chapter{How to Design and Adjust Chapter Heading Parameters}

\section{Quick Reference}\label{sec:quickref}

Each element of a heading such as \textit{chapter} or \textit{title} is drawn as a box, with a large number of parameters that can adjust spacing, borders, fonts and other typographic parameters.

\bigskip\bigskip\bigskip\bigskip
\let\oldrefkey\refKey
\let\refKey\texttt
\makeatletter
\long\def\demobox#1#2{%
\par\bigskip\bigskip\bigskip
\begin{tcolorbox}[enhanced,left=0pt, top=0pt, bottom=0pt,width=\textwidth,
  enlarge top initially by=1cm,enlarge bottom finally by=1cm,left skip=1cm,right skip=1cm,
  colframe=white,colback=white,
  colbacktitle=red!30!white,colupper=black!7!white,
  code={\appto\kvtcb@shadow{%
    \path[fill=white,draw=yellow!50!black,dashed,line width=0.4pt]
      ([xshift=-1cm,yshift=-1cm]frame.south west) rectangle
      ([xshift=1cm,yshift=1cm]frame.north east);
     \path[fill=blue!20!white, 
              opacity=0.3, draw=yellow!50!black,solid,line width=1pt]
      ([xshift=-2cm,yshift=-2cm]frame.south west) rectangle
      ([xshift=2cm,yshift=2cm]frame.north east);  
    }},
  finish={
  \draw[thick,<->] ([yshift=-1.3cm]frame.north west)-- node[below]{\texttt{#1 width}}
    ([yshift=-1.3cm]frame.north east);
  \draw[thick,<->] ([xshift=-15mm]frame.north east)-- node[above]{\refKey{#1 height}}
    ([xshift=-15mm]frame.south east);
  \draw[thick,<->] (frame.north)-- node[right]{\refKey{#1 padding-top}} +(0,1);
  \draw[thick,<->] ([yshift=1cm]frame.north)-- node[right]{\refKey{#1 margin-top}} +(0,1);
  \draw[thick,<->] (frame.south)-- node[right, align=left]{\refKey{#1 padding-bottom}}+(0,-1);
  %left padding
  \draw[thick,<->] (frame.west)-- node[below right,align=center]{\refKey{#1 padding-left }}+(-1,0);
  %left margin
  \draw[thick,<->] ([xshift=-1cm,yshift=-0.9cm]frame.west)-- node[below right,xshift=-1,align=left]{\refKey{#1 margin-left }\\\refKey{#1 grow to left by}}+(-1,0);
  %right padding
  \draw[thick,<->] (frame.east)-- node[below left,align=center]{\refKey{#1 padding-right}}+(1,0);
 %right margin
  \draw[thick,<->] ([xshift=1cm,yshift=-0.9cm]frame.east)-- node[below left,xshift=1, align=right]{\refKey{#1 margin-right}\\\refKey{#1 grow to right by}}+(1,0);
 \draw[thick,<->] ([yshift=-2cm]frame.south)-- node[right, align=left]{\refKey{#1 margin-bottom},\\ \refKey{#1 after skip}}+(0,1);
  }
    ]
#2%
%\hrule width0pt height4.5cm depth0pt\relax% \vspace*{4.5cm}% \lipsum[1]
\end{tcolorbox}\par
\bigskip\bigskip\bigskip}
\makeatother

\demobox{chapter}{\scalebox{1.17}{\HHHUGE Chapter}}

The number box is again drawn in a box similar to a chapter with all properties generalized.

\demobox{number}{\scalebox{1.15}{\HHHUGE Thirteen}}



All parameters shown in the diagram can be set using the command \cs{cxset}. The property names follow conventions similar to those of |css|, rather than typical conventions of \tikzname that are more widely known to the programming community. The prefix to these properties (in the example \textit{chapter}) can be thought of
as similar to a |class| or |id| name in |css|.  

\begin{docCommand}{cxset}{\marg{options}}
  Sets options for every following \refEnv{tcolorbox} inside the current \TeX\ group.
  By default, this does not apply to nested boxes, see \Vref{subsec:everybox}.\par
  For example, the colors of the boxes may be defined for the whole document by this:
\begin{dispListing}
\tcbset{colback=red!5!white,colframe=red!75!black}
\end{dispListing}
\end{docCommand}

\begin{docKey}[]{chapter padding-top}{=\meta{dimension}}{no default, initial value 0pt}
All padding keys take one argument, which is a dimension. The length is also stored in a register
\cmd{\chapterpaddingtop}. In this chapter it was set at \the\chapterpaddingtop.
\end{docKey}

\begin{docKey}[]{chapter padding-right}{=\meta{dimension}}{no default, initial value 0pt}
All padding keys take one argument, which is a dimension. The length is also stored in a register
\cmd{\chapterpaddingright}.  In this chapter it was set at \the\chapterpaddingright.
\end{docKey}

\begin{docKey}[]{chapter padding-bottom}{=\meta{dimension}}{no default, initial value 0pt}
All padding keys take one argument, which is a dimension. The length is also stored in a register
\cmd{\chapterpaddingbottom}.  In this chapter it was set at \the\chapterpaddingbottom.
\end{docKey}

\begin{docKey}[]{chapter padding-left}{=\meta{dimension}}{no default, initial value 0pt}
All padding keys take one argument, which is a dimension. The length is also stored in a register
\cmd{\chapterpaddingleft}.  In this chapter it was set at \the\chapterpaddingleft.
\end{docKey}

%% margin

\begin{docKey}[]{chapter margin-top}{=\meta{dimension}}{no default, initial value 0pt}
All padding keys take one argument, which is a dimension. The length is also stored in a register
\cmd{\chaptermargintop}. In this chapter it was set at .
\end{docKey}

\begin{docKey}[]{chapter margin-right}{=\meta{dimension}}{no default, initial value 0pt}
All padding keys take one argument, which is a dimension. The length is also stored in a register
\cmd{\chapterpaddingright}.  In this chapter it was set at \the\chapterpaddingright.
\end{docKey}

\begin{docKey}[]{chapter margin-bottom}{=\meta{dimension}}{no default, initial value 0pt}
All padding keys take one argument, which is a dimension. The length is also stored in a register
\cmd{\chapterpaddingbottom}.  In this chapter it was set at \the\chapterpaddingbottom.
\end{docKey}

\begin{docKey}[]{chapter margin-left}{=\meta{dimension}}{no default, initial value 0pt}
All padding keys take one argument, which is a dimension. The length is also stored in a register
\cmd{\chaptermarginleft}.  In this chapter it was set at \the\chaptermarginleft.
\end{docKey}

\subsection{Borders}

Border have three properties \emph{width, color} and \emph{style}. They can set individually for
each side of the box or using the shorter key .

\begin{docKey}[]{chapter border-top-width}{=\meta{dimension}}{no default, initial value 0pt}
All border keys take one argument, which is a dimension.
\end{docKey}

\begin{docKey}[]{chapter border-right-width}{=\meta{dimension}}{no default, initial value 0pt}
All border keys take one argument, which is a dimension.
\end{docKey}

\begin{docKey}[]{chapter border-bottom-width}{=\meta{dimension}}{no default, initial value 0pt}
All border keys take one argument, which is a dimension.
\end{docKey}

\begin{docKey}[]{chapter border-left-width}{=\meta{dimension}}{no default, initial value 0pt}
All border keys take one argument, which is a dimension.
\end{docKey}

\subsubsection{Border Colors}

The colors follow the same pattern for |border-width| and again they can be set individually or using
a shorter key to set all of them in one color. 

\begin{docKey}[]{chapter border-top-color}{=\meta{color name}}{no default, initial value black}
All border keys take one argument, which is a dimension.
\end{docKey}

\begin{docKey}[]{chapter border-right-color}{=\meta{color name}}{no default, initial value black}
All border keys take one argument, which is a dimension.
\end{docKey}

\begin{docKey}[]{chapter border-bottom-color}{=\meta{color name}}{no default, initial value black}
All border keys take one argument, which is a dimension.
\end{docKey}

\begin{docKey}[]{chapter border-left-color}{=\meta{color name}}{no default, initial value black}
This key is stored in \cmd{\chapterborderrightcolor} and the value in this chapter is \texttt{\chapterborderrightcolor}.
\end{docKey}

\subsubsection{Border Styles}

Standard |css|  offers four styles \emph{dotted, solid, double, dashed}. We offer almost an unlimited set of styles.

\begin{docKey}[]{chapter border-top-style}{=\meta{style name}}{no default, initial value \texttt{none}}
The |border-style| properties take a value, which can be |solid, double, dotted, dashed, asterisk|.
\end{docKey}

\begin{docKey}[]{chapter border-right-style}{=\meta{style name}}{no default, initial value \texttt{none}}
The |border-style| properties take a value, which can be |solid, double, dotted, dashed, asterisk|.
\end{docKey}

\begin{docKey}[]{chapter border-bottom-style}{=\meta{style name}}{no default, initial value \texttt{none}}
The |border-style| properties take a value, which can be |solid, double, dotted, dashed, asterisk|.
\end{docKey}

\begin{docKey}[]{chapter border-left-style}{=\meta{style name}}{no default, initial value \texttt{none}}
The |border-style| properties take a value, which can be |solid, double, dotted, dashed, asterisk|.
\end{docKey}

\begin{docKey}[phd]{chapter border-style}{=\meta{style name}}{no default, initial value \texttt{none}}
This key sets all chapter-border-\meta{top,right,bottom,left}-style to a single value.
\end{docKey}
%\bigskip
%\bigskip
%
%\begin{tcolorbox}[enhanced,title={tcolorbox},before skip=5mm,after skip=5mm,
%  colframe=red!50!black!30!white,colback=red!10!white!40!white,
%  colbacktitle=red!30!white,coltext=black!20!white,
%  toptitle=1mm,bottomtitle=1mm,
%  overlay={\begin{tcbclipinterior}%
%    \path[fill=red!10!white!40!yellow!20!white,draw=yellow!50!black,dotted]n
%      ([xshift=1mm,yshift=1mm]interior.south west)
%      rectangle ([xshift=-1mm,yshift=-1mm]interior.north east);
%    \path[fill=red!10!white!40!white,draw=yellow!50!black,dotted] (
%      [xshift=5mm,yshift=3mm]interior.south west)
%      rectangle ([xshift=-5mm,yshift=-3mm]interior.north east);
%    \path[fill=red!10!white!40!yellow!20!white,draw=yellow!50!black,dotted]
%      ([xshift=5mm,yshift=-1mm]segmentation.south west)
%      rectangle ([xshift=-5mm,yshift=1mm]segmentation.north east);
%    \path[fill=red!10!white!40!white,draw=yellow!50!black,dotted]
%      ([xshift=5mm,yshift=1mm]segmentation.south west)
%      rectangle ([xshift=-5mm,yshift=-1mm]segmentation.north east);
%    \path[dashed,draw=red!50!black!30!white] (segmentation.west) -- (segmentation.east);
%    \end{tcbclipinterior}%
%    \begin{tcbcliptitle}
%    \path[fill=red!30!white!70!yellow,draw=yellow!50!black,dotted]
%      ([xshift=1mm,yshift=1mm]title.south west)
%      rectangle ([xshift=-1mm,yshift=-1mm]title.north east);
%    \path[fill=red!30!white,draw=yellow!50!black,dotted]
%      ([xshift=5mm,yshift=2mm]title.south west)
%      rectangle ([xshift=-5mm,yshift=-2mm]title.north east);
%    \end{tcbcliptitle}},
%  finish={
%  \coordinate (A) at ([yshift=-0.25mm]frame.north);
%  \draw[thick,<-] (A) -- +(-1,0.3) node[left]{\refKey{/tcb/toprule}};
%  \coordinate (A) at ([yshift=-0.75mm]A);
%  \draw[thick,<-] (A) -- +(1,0) node[right]{\refKey{/tcb/boxsep}};
%  \coordinate (A) at ([yshift=-1mm]A);
%  \draw[thick,<-] (A) -- +(-1,0) node[left]{\refKey{/tcb/toptitle}};
%  %
%  \coordinate (A) at ([yshift=1.00mm]interior.north);
%  \draw[thick,<-] (A) -- +(1,0) node[right]{\refKey{/tcb/boxsep}};
%  \coordinate (A) at ([yshift=1mm]A);
%  \draw[thick,<-] (A) -- +(-1,0) node[left]{\refKey{/tcb/bottomtitle}};
%  \coordinate (A) at ([yshift=0.25mm]interior.north);
%  \draw[thick,<-] (A) -- +(-1,-0.4) node[left]{\refKey{/tcb/titlerule}};
%  \coordinate (A) at ([yshift=-0.5mm]interior.north);
%  \draw[thick,<-] (A) -- +(1,-0.2) node[right]{\refKey{/tcb/boxsep}};
%  \coordinate (A) at ([yshift=-1.5mm]A);
%  \draw[thick,<-] (A) -- +(-1,-0.6) node[left]{\refKey{/tcb/top}};
%  %
%  \coordinate (A) at ([yshift=2.0mm]segmentation);
%  \draw[thick,<-] (A) -- +(-1,0) node[left]{\refKey{/tcb/middle}};
%  \coordinate (A) at ([yshift=0.5mm]segmentation);
%  \draw[thick,<-] (A) -- +(1,0.2) node[right]{\refKey{/tcb/boxsep}};
%  \coordinate (A) at ([yshift=-0.5mm]segmentation);
%  \draw[thick,<-] (A) -- +(1,-0.2) node[right]{\refKey{/tcb/boxsep}};
%  \coordinate (A) at ([yshift=-2.0mm]segmentation);
%  \draw[thick,<-] (A) -- +(-1,0) node[left]{\refKey{/tcb/middle}};
%  %
%  \coordinate (A) at ([yshift=0.25mm]frame.south);
%  \draw[thick,<-] (A) -- +(-1,-0.3) node[left]{\refKey{/tcb/bottomrule}};
%  \coordinate (A) at ([yshift=0.75mm]A);
%  \draw[thick,<-] (A) -- +(1,0) node[right]{\refKey{/tcb/boxsep}};
%  \coordinate (A) at ([yshift=1.5mm]A);
%  \draw[thick,<-] (A) -- +(-1,0) node[left]{\refKey{/tcb/bottom}};
%  %
%  \coordinate (A) at ([xshift=0.25mm]frame.west);
%  \draw[thick,<-] (A) -- +(-0.3,-1) node[below]{\refKey{/tcb/leftrule}};
%  \coordinate (A) at ([xshift=0.75mm]A);
%  \draw[thick,<-] (A) -- +(0,1) node[above]{\refKey{/tcb/boxsep}};
%  \coordinate (A) at ([xshift=2.5mm]A);
%  \draw[thick,<-] (A) -- +(0.7,0.5) node[above right]{\refKey{/tcb/left}};
%  %
%  \coordinate (A) at ([xshift=-0.25mm]frame.east);
%  \draw[thick,<-] (A) -- +(0.3,-1) node[below]{\refKey{/tcb/rightrule}};
%  \coordinate (A) at ([xshift=-0.75mm]A);
%  \draw[thick,<-] (A) -- +(0,1) node[above]{\refKey{/tcb/boxsep}};
%  \coordinate (A) at ([xshift=-2.5mm]A);
%  \draw[thick,<-] (A) -- +(-0.7,0.5) node[above left]{\refKey{/tcb/right}};
%  }
%    ]
%  \lipsum[1]
%  \tcblower
%  \lipsum[2]
%\end{tcolorbox}
%
%\let\refKey\oldrefkey
%
%\end{document}