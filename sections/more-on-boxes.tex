\cxset{
         section font-family=tiresias,
         title font-family=tiresias,
         chapter numbering=arabic,
         chapter title align=none,
         title border-top-width=5pt,
         title border-left-width=5pt,
         title border-bottom-width=5pt,
         title border-right-width=5pt
         title padding-bottom=0pt,
         title margin-bottom=30pt,
         title border-width=0pt,
         chapter float=center,
         chapter font-family= ttfamily}
\chapter{More on Boxes: Using packages to automate boxing calculations and the drawing of borders.}
\thispagestyle{plain}
\pagestyle{headings}

\section{Testing title values}
\makeatletter
\ExplSyntaxOn
%title~display = \int_use:c {chapter_title_display} \\
title~border~left~width = \dim_use:N \title_border_left_width\\
title~border~top~width = \dim_use:N \title_border_top_width\\
title~border~right~width =\dim_use:N \title_border_right_width\\
title~border~bottom~width =\dim_use:N \title_border_bottom_width\\
%chapter~title-display= \int_use:c {chapter_title_display} \\
title~display =\meaning\chapter_title_display_tl\\
chapter~title~text-align = \meaning\chaptertitletextalign@cx\\
%chapter~title~align = \tl_use:c {tl_chapter_title_align}
\meaning\chaptertitletextalign@cx
\makeatother
\ExplSyntaxOff

The boxing of contents is such an important concept in \tex and also in typography that it is worth examining some of the available packages that can be used.

Another elaborate package is Martin Scharrer’s package \pkgname{adjustbox}. The package uses numerous keys
to draw borders, adjust spacing and margins but also for the clipping of images. At the background the package uses and extends the \pkgname{graphicx} key value system.

This package allows to adjust general \latexe material in several ways using a key=value interface.
It got inspired by the interface of \cmd{\includegraphics} from the \pkg{graphicx} package.
 This package also loads the \pkg{trimclip} package which code was once included in this package.


 \subsection{Trimming and Clipping}
 
 Trimming and clipping is achieved by loading the \pkgname{trimclip}. This package forms part of the suit of packages developed by Martin Scharrer and or related to adjusting boxes and their sizes. The package allows for
 verbatim material as well. 
 
 \let\Macro\cmd
 
 The following keys allow content to be trimmed (i.e.\ the official size is made smaller, so the remaining material
 laps over the official boundaries) or clipped (overlapping material is not displayed).
 These keys come in different variants, where the lower-case keys represent the behavior of
 the corresponding \Macro\includegraphics keys. The corresponding macros (\Macro\trimbox, \Macro\clipbox, etc.)
 and environments (\env{trimbox}, \env{clipbox}, etc.) are included in the
 accompanying \pkg{trimclip} package and are explained in its manual.

 
 This key represents the original \option{trim} key of \Macro\includegraphics but accepts its value in different forms.
 Unlike most other keys it always acts on the original content independent in which order it is used with other keys.
 The key trims the given amounts from the lower left (ll) and the upper right (ur) corner of the box. This means that
 the amount \meta{llx} is trimmed from the left side, \meta{lly} from the bottom and \meta{urx} and \meta{ury} from the
 right and top of the box, respectively.
 If only one value is given it will be used for all four sites.
 If only two values are given the first one will be used for the left and right side (llx, urx) and the second for the
 bottom and top side (lly, ury).
 
\begin{texexample}{Example}{clipping}
\adjustbox{Clip=1, min width=8cm, center,}{This is some test}
\medskip

The untrimmed version is shown below
\medskip

\adjustbox{Clip=.1, min width=8cm, center,}{This is some test}%
\end{texexample}
 
\section{tcolorbox} 

\subsection{Breakable Boxes}

 \begin{verbatim}
 \begin{tcolorbox}[enhanced, breakable,
  colback=blue!5!white,colframe=blue!75!black,title=Breakable box,
  watermark color=white, watermark text=\Roman{tcbbreakpart}]
  \lipsum[1-18]
\end{tcolorbox}

	See \ref{test} for details and an example.
 \end{verbatim}
 
 
 

 