\chapter{Coding Styles}

\epigraph{A survey conducted in 2010 shows that \latex is mostly a world of dwarfs.}{--Didier Verna, in \textit{Towards LaTeX coding standards} TUGboat, Volume 32 (2011), No.3}

As Didier Verna \citep{verna} wrote in a seminal article at TUGboat, most \latex macro authors are loners and there is little co-operative effort when producing packages. Once you start writing your own macros, that grow into packages and classes the matter of coding style will arise.

\section{Learning by example}

Like most programming, learning by example is the best approach. Many \latex programmers including myself started by reading code, and what other people did. In doing so as vera says: `they implicitly (and unconsiously) inherit the coding style or lack of it. This behavious actually encourages legacy (the good \textit{and} the bad and leads to a very heterogeneous code base.

\section{Some advice for readability}

\subsection{Allow spaces}

Generally programmers prefer to leave spaces before and after |=,+,-|. Consider writing:

\begin{verbatim}
$ f(x) = f(x-1) + f(x-2) $
\end{verbatim}

This is more readable than:

\begin{verbatim}
$f(x)=f(x-1)+f(x-2)$
\end{verbatim}

\section{Documentation}













