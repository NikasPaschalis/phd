
\cxset{chapter numbering=arabic,
          chapter name = Chapter,
          title margin-top=2cm}
\chapter{The xtemplate package of LaTeX3 and how to use it effectively}

Back in 1999 Frank Mittelbach together with David Carlisle and Chris Rowley published a paper in TUGboat describing their ideas of  \enquote{New Interfaces for \latex Class Design.} 
 
 \begin{latexquote}
 Traditional \latex class files typically implement one
fixed design via ad hoc, and often low-level, \latex
code. This style of implementation makes it much
harder than is either desirable or necessary to produce
classes that implement a specific visual design.
Moreover, the construction of such classes typically
involves a lot of work that is essentially programming
and thus does not live easily with the declarative
kind of design specification for a document (or
range of documents) that would be produced by a
professional typographic designer.
\end{latexquote}

The \emph{declarative kind} of design specification for a document, mentioned by the authors has been the holy grail of \latex for sometime. With the proliferation of key value packages it came closer to fruition and my own work in the |phd| package had this goal as one of its primary objectives. The \pkgname{xtemplate} is at a much lower level than the phd package and I have struggled in my head as to how to integrate the two, so far unsuccessfully. There are very few articles on |xtemplate| but a good introductory one is \emph{Some notes on templates} by  Lars Hellström’s and which was published in TUGboat. The \pkgname{xgalley} still under develpment makes use of templates extensively and is worth to have a good look at the code.

\section{Objects, templates and instances}

\subsection{Object types}

An \emph{object type} sometimes termed \enquote{object} is an abstract idea of a document element that has a fixed number of arguments corresponding to the information from the document author that it is representing.  A sectioning object, for example, might take three inputs: \enquote{title}, \enquote{short title}, and \enquote{label}.

\begin{docCommand} {DeclareObjectType} { \meta{object type} \meta{no of args}}
This function defines an \meta{object type} taking \meta{number of arguments}, where the \meta{object type} is an abstraction as discussed above. For example:
   \begin{verbatim}
     \DeclareObjectType{chapter}{3}
   \end{verbatim}
This would create an object type \enquote{sectioning}, where each use of that object type will need three arguments.   
\end{docCommand}

The object type doesn’t do much when it is declared. It just records the name and the number of arguments in a property store, as can be seen in the code below, extracted from the |xtemplate| package:

\begin{teXXX}
\cs_new_protected:Npn \@@_declare_object_type:nn #1#2
  {
    \int_set:Nn \l_@@_tmp_int {#2}
    \bool_if:nTF
      {
        \int_compare_p:nNn {#2} > \c_nine ||
        \int_compare_p:nNn {#2} < \c_zero
      }
      {
        \msg_error:nnxx { xtemplate } { bad-number-of-arguments }
          {#1} { \exp_not:V \l_@@_tmp_int }
      }
      {
        \msg_info:nnxx { xtemplate } { declare-object-type }
          {#1} {#2}
        \prop_gput:NnV \g_@@_object_type_prop {#1}
          \l_@@_tmp_int
      }
  }
\end{teXXX}

    
\subsection{Templates}

Once an object is created a \emph{template} can be used to generalize a design solution for representing the information of a specific object type. A template has a name and a parent object. There are two important parts to a template:

\begin{enumerate}
\item The parameters it takes to vary the design it is producing.
\item The implementation of the design.
\end{enumerate}

The template definition is split into two parts using \cs{DeclareTemplateInterface} and \cs{DeclareTemplateCode}.
We will first examine \docAuxCommand*{DeclareTemplateInterface}.

\begin{docCommand} {DeclareTemplateInterface} {\meta{object} \marg{key value list}}
The key value list is of the form:
\begin{verbatim}

    key1 : key type1,
    key2 : key type2,
    key3 : key type3  = default3,
    key4 : key type4  = default4,
\end{verbatim}

An important item to note is that spaces in key names are ignored so writing |my key| and |mykey| is one and the same. 

Essentially the |DeclareTemplateInterface| is a command that initializes the list of key values applicable to the object type. The key list must be the same as declared for |object|. I didn’t know the \latex guys were fans of Java. The code pattern here is very similar. You declare an object and then its interface. Once this is done then the code can be developed. The key types available are shown in Table~\ref{tab:key-types}, which has been extracted from the documentation.

\end{docCommand}

\begin{docKey}{boolean} { boolean type for template interface}{\meta{true or false}}
a true or false value
\end{docKey}

   \begin{table}
     \centering
     \begin{tabular}{>{\ttfamily}ll}
       \toprule
       \multicolumn{1}{l}{Key-type} & Description of input \\
       \midrule
       boolean    & \texttt{true} or \texttt{false}            \\
       choice\marg{choices}
         & A list of pre-defined \meta{choices} \\
       code
         & Generalised key type: use |#1| as the input to the key \\
       commalist  & A comma-separated list                        \\
       function\marg{$N$}
         & A function definition with $N$ arguments
          ($N$ from $0$ to $9$) \\
       instance\marg{name}
                      & An instance of type \meta{name} \\
       integer    & An integer or integer expression            \\
       length     & A fixed length                              \\
       muskip    & A math length with shrink and stretch components \\
       real         & A real (floating point) value               \\
       skip         & A length with shrink and stretch components \\
       tokenlist  & A token list: any text or commands          \\
       \bottomrule
     \end{tabular}
     \caption{Key-types for defining template interfaces with
       \cs{DeclareTemplateInterface}.}
     \label{tab:key-types}
   \end{table}
   
\begin{texexample}{xtemplate short example}{}
\DeclareObjectType{obj}{0}
\DeclareTemplateInterface{obj}{tmpt1}{0}
{
  section-name: tokenlist = section,
  section-numbering: tokenlist =Roman,
  section-color: tokenlist = blue,
}
\end{texexample}

The |\DeclareTemplateInterface| part of the code is just a macro, whose fourth argument is written in a funny way.
We could have just written it as:

\begin{teXXX}
\DeclareTemplateInterface{obj}{tmp1}{0}{section-numbering:tokenlist=arabic, ... }
\end{teXXX}

A confusing aspect of the |templates| package is how the code part is defined. Here for each
key declared in the \docAuxCommand{DeclareTemplateInterface} you will need to allocate it an appropriate
macro. This works like in normal \latex keys. 

\begin{texexample}{The template code}{}
\ExplSyntaxOn
\DeclareTemplateCode{obj}{tmpt1}{0}
{
  section-name         = \sectionname,
  section-numbering = \numberingtype, 
  section-color = \colorname,
  }
{
% the implementation part
\AssignTemplateKeys
  \sectionname\ ~ 
  {\cs:w\numberingtype\cs_end: {section}\scan_stop:}\\
}

\DeclareInstance {obj}{inst} {tmpt1}{section-numbering = roman}
\UseInstance{obj}{inst}
\DeclareInstance {obj}{inst2}{tmpt1}{section-name=SECTION,
                                                      section-numbering = arabic}
\UseInstance{obj}{inst2}
\ExplSyntaxOff
\end{texexample}

The implementation part is the part that starts with |\AssignTemplateKeys|. Here we can use the values stored in the key functions to do something useful. Again here, remember, we are using macros and |\AssignTemplateKeys| is a macro with five arguments. This is defined by the package as:

 \begin{verbatim}
   \@@_declare_template_code:nnnnn {#1} {#2} {#3} {#4} {#5}
\end{verbatim}

Looking back at our simple example, the formatting of the section number in |arabic| or |roman| did not make any particular checks for validity. This would have been better programmed as a |choice| key with all the choice words allowed hardcoded in the implementation part. 

\begin{teXXX}
 section-numbering  : choice { arabic, Roman, roman, words, Words, alph, Alph } = arabic
\end{teXXX}                                 

The |choice| key type implements multiple choice input. At the interface level only the list of valid choice is needed:

\begin{teXXX}
\DeclareTemplateInterface{ foo }{ bar }{ 0 }
    { key-name : choice { A, B, C } }
\end{teXXX}

Note that the choices are given in a comma delimited list (which must therefore be wrapped in braces). A default value can also be given:


\begin{teXXX}
\DeclareTemplateInterface{ foo }{ bar }{ 0 }
    { key-name : choice { A, B, C } = A }
\end{teXXX}

\begin{teXXX}
 section-numbering      =
      {
        roman =
          \cs_set_nopar:Npn \numberingtype:
            {
              ... code
            },
        roman  =
          \cs_set_nopar:Npn \numberingtype:
            {
              ... code 
            }
      },
\end{teXXX}

\begin{texexample}{The template code}{ex:unknownkey}
\ExplSyntaxOn
\DeclareTemplateInterface{obj}{section}{0}
{
  section-name: tokenlist = section,
  section-numbering: choice  {arabic, Roman, roman}=roman,
  section-color: tokenlist = blue,
}
\DeclareTemplateCode{obj}{section}{0}
{
  section-name         = \sectionname,
  section-numbering = 
     {
       roman     =   \cs_set_nopar:Npn  \numberingtypei: { \roman{section} \scan_stop: },
       Roman     =  \cs_set_nopar:Npn   \numberingtypei: { \Roman{section} \scan_stop: },
       arabic      =   \cs_set_nopar:Npn  \numberingtypei: { \arabic{section} \scan_stop: },
       unknown =   \cs_set_nopar:Npn  \numberingtypei: { ERROR~unknown~key }
     },  
  section-color = \colorname,
  }
{
% the implementation part

\AssignTemplateKeys
  \sectionname\ ~ 
  \numberingtypei: \par
}

\DeclareInstance {obj}{inst} {section}{section-numbering = roman}
\UseInstance{obj}{inst}
\DeclareInstance {obj}{inst2}{section}
    {
        section-name=SECTION,
        section-numbering = arabic
     }
     
\DeclareInstance {obj}{inst3}{section}
    {
        section-name=SECTION,
        section-numbering = Arabic
     }  
                                                          
\UseInstance{obj}{inst2}
\UseInstance{obj}{inst3}
\ExplSyntaxOff
\meaning\numberingtypei
\end{texexample} 

In Example~\ref{ex:unknownkey} we have introduced the |choice| type key. This also takes an option
\option{unknown}. If a value is given that has not been previously been defined, then it essentially acts as
an |else| branch to the code and executes the definition given, in our example just typesets an 
error message. The code in the example at this stage is very simplistic and it has not been abstracted properly. The example is simply here to demonstrate the various types of keys available. The |length| and the |skip| keys 
accept dimensions or skips and are simply coded. The |function| type of key can be very useful in many situations. 

In the next example we will add some skips before and after the section, as well introduce a boolean to choose betwen a block heading or an inline heading. 


\begin{texexample}{The template code}{ex:unknownkey}
\ExplSyntaxOn
\DeclareTemplateInterface{obj}{headings}{0}
{
  name: tokenlist = section,
  numbering: choice  {arabic, Roman, roman,none} = roman,
  color: tokenlist = blue,
  display: boolean = true,
  aboveskip: skip=10pt,
  belowskip: skip=10pt,
  }
 
\bool_new:N \l_display_bool

\DeclareTemplateCode{obj}{headings}{0}
{
  name         = \sectionname,
  numbering = 
     {
       roman     =  \cs_set_nopar:Npn \numberingtypei: {\roman{section}},
       Roman     = \cs_set_nopar:Npn \numberingtypei: {\Roman{section} },
       arabic      =  \cs_set_nopar:Npn \numberingtypei: {\arabic{section}},
       none       =  \cs_set_nopar:Npn \numberingtypei: {},
       unknown =  \cs_set_nopar:Npn\numberingtypei: {ERROR~unknown~key }
     },  
  color = \colorname,
  display = \l_display_bool,
  aboveskip = \l_tmpa_skip,
  belowskip = \l_tmpb_skip,
 }
 {
% the implementation part
  \AssignTemplateKeys
  \par\skip_vertical:N  \l_tmpa_skip
  \sectionname\ ~ 
  \numberingtypei: \par
}

\DeclareInstance {obj}{part} {headings}{name = PART,
                                                           numbering = Roman}
\DeclareInstance {obj}{section}{headings}{name=SECTION,
                                                               numbering = arabic}
\DeclareInstance {obj}{chapter}{headings}{name=CHAPTER,
                                                                numbering = Roman, 
                                                                aboveskip=5pt}    
                                                                                                                   
\UseInstance{obj}{part}
\UseInstance{obj}{section}
\UseInstance{obj}{chapter}

\ExplSyntaxOff

\end{texexample}   

Although the |xtemplate| manual recommends that booleans should be preferred over 
|choice| keys, but from a user interface point of view |choice| keys are more powerful. One can define
key variations such as (true, false, on, off, none) and other similar values. 

Another few notes for readers coming from \latexe. The  |\skip_vertical:N| is the
\tex |\vskip|.  There are also some questions arising from the approach, which can affect the
coding. The format and the flexibility of the final settings offered for the user. From a programmer’s
perspective the view is different.  We could view the three basic elements of a heading at a more
elementary level, consisting of a number, a label and a title. Consider the |HTML| element |<span>|,
how can we make an equivalent in \latex3? There are many approaches one could think of, but this time
having covered the basics of how to program templates, we will start from the Designer Level. The Designer
wishes to define commands that are normally used inline and are used for different type of purposes. Such functions can typeset words that are emphasized, others that represent computer code and are typeset verbatim, acronyms and abbreviations. These also can automatically add themselves to an index etc.

Of course we don’t want to offer the user a command called |\span| where he needs to type |\span[emph]|. What we want to offer the user is a series of commands. However at the Design Level, these can be created by means of templates.

\begin{texexample}{A template for spans}{ex:span}
\ExplSyntaxOn
\DeclareObjectType{inlineobj}{1}
\DeclareTemplateInterface{inlineobj}{span}{1}
{
  font-face: tokenlist,
  font-shape: choice {italic, slanted, normal},
  font-weight: choice {bold, normal},
  font-color: tokenlist,
  quote: function 1,
}
\cs_set_nopar:Npn \quote_format:n#1 {\enquote{#1}}
\cs_set_nopar:Npn \quote_format_none:n#1 {#1}

\DeclareTemplateCode{inlineobj}{span}{1}
{
  font-face         =  \l_font_tl,
  font-shape = {
     italic     = \cs_set_nopar:Nn \afontshape: {\itshape},
     slanted = \cs_set_nopar:Nn \afontshape: {\itshape},
     normal = \cs_set_nopar:Nn \afontshape: {\upshape}
  },
  font-weight = {
     bold    = \cs_set_nopar:Nn \afontseries: {\bfseries},
     normal =\cs_set_nopar:Nn \afontseries: {\mdseries}
   },
  font-color = \l_tmpa_tl,  
  quote = \quote_format:n,
}
{
% the implementation part
  \AssignTemplateKeys
  \group_begin:
  \color\l_tmpa_tl
   \cs:w \l_font_tl \cs_end: 
   \afontshape:
   \afontseries: 
       \quote_format:n{\detokenize{#1}} 
   \group_end:
 }
 
\ExplSyntaxOff

\DeclareInstance {inlineobj}{docFunction}{span}
    {
        font-face=arial,
        font-shape=normal, 
        font-weight=bold,
        font-color=green!40!black
     }

\DeclareDocumentCommand\docFunction{ m }{
   \IfInstanceExistTF {inlineobj}{docFunction} 
     {\UseInstance{inlineobj}{docFunction}{#1}}
     {ERROR                                                   }
}

\DeclareInstance {inlineobj}{tn}{span}
    {
        font-face=ttfamily,
        font-shape=normal, 
        font-weight=normal,
        font-color=green!40!black
     }

\DeclareDocumentCommand\tn{ m }{
   \IfInstanceExistTF {inlineobj}{tn} 
     {\UseInstance{inlineobj}{tn}{#1}}
     {ERROR                                                   }
} 
   
The function \docFunction {get_string ( )} is used throughout to get a string in LuaTeX, where macros in text paragraphs are shown as \docFunction\mymacro in green.
\end{texexample}
\ExplSyntaxOn
\DeclareObjectType{inlineobj}{1}
\DeclareTemplateInterface{inlineobj}{span}{1}
{
  font-face: tokenlist,
  font-shape: choice {italic, slanted, normal},
  font-weight: choice {bold, normal},
  font-color: tokenlist,
  quote: function 1,
}
\cs_set_nopar:Npn \quote_format:n#1 {\enquote{#1}}

\DeclareTemplateCode{inlineobj}{span}{1}
{
  font-face         =  \l_font_tl,
  font-shape = {
     italic     = \cs_set_nopar:Nn \afontshape: {\itshape},
     slanted = \cs_set_nopar:Nn \afontshape: {\itshape},
     normal = \cs_set_nopar:Nn \afontshape: {\upshape}
  },
  font-weight = {
     bold    = \cs_set_nopar:Nn \afontseries: {\bfseries},
     normal =\cs_set_nopar:Nn \afontseries: {\mdseries}
   },
  font-color = \l_tmpa_tl,  
  quote = \quote_format:n,
}
{
% the implementation part
  \AssignTemplateKeys
  \group_begin:
  \color\l_tmpa_tl
   \cs:w \l_font_tl \cs_end: 
   \afontshape:
   \afontseries: 
       \quote_format:n{\detokenize{#1}} 
   \group_end:
 }
 
\ExplSyntaxOff

\DeclareInstance {inlineobj}{docFunction}{span}
    {
        font-face=arial,
        font-shape=normal, 
        font-weight=bold,
        font-color=green!40!black
     }

\DeclareDocumentCommand\docFunction{ m }{
   \IfInstanceExistTF {inlineobj}{docFunction} 
     {\UseInstance{inlineobj}{docFunction}{#1}}
     {ERROR}
}

\DeclareInstance {inlineobj}{tn}{span}
    {
        font-face=ttfamily,
        font-shape=normal, 
        font-weight=normal,
        font-color=green!40!black
     }

\DeclareDocumentCommand\tn{ m }{
   \IfInstanceExistTF {inlineobj}{tn} 
     {\UseInstance{inlineobj}{tn}{#1}}
     {ERROR}
} 
   
With the last example I have introduced also the conditional \docAuxCommand*{IfInstanceTF} that provides a test if the template exist. In our case typesets |ERROR| if the instance does not exist.

\begin{docCommand}{IfInstanceExistTF}{\marg{object type} \marg{instance} \marg{true code} \marg{false code}}
Tests if the named \meta{instance} of an \meta{object type} exists, and then inserts the appropriate code into the input stream. 
\end{docCommand}


\section{Summary}

\latex3’s \pkgname{xtemplate} offers a flexible and robust way to enable  declarative
setting of typographical parameters for a document. For the package writer it has one major advantage. It can be used to expose an API through which users communicate with the package's important commands. I would go as far as to say that packages should only expose an API and no settings should occur during loading. This can reduce both errors during package loading with different key values, as well as perhaps stop the race at the |AtBeginDocument|. 

If you want to study a longer non-trivial example you can have a look at the \pkgname{xfrac} package. In this package Will Robertson used |xtemplate| extensively. He also used some of the more esoteric commands of the package and is worth studying the code, before you start using |xtemplate| in your package.

All the functionality made available by the package can easily be provided by |pgfkeys| and the creation of some custom commands. This will remain as a competitor to the package until some of the limitations of |xparse| are addressed. The main limitation currently from my point of view is the addition of custom types  in a similar fashion to pgfkeys \emph{handlers}, although the \tn{code} and the \tn{function} types can be used in this respect.



