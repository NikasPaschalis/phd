\chapter{Expandafter}

One of the most often misunderstood \TeX\ commands is \cmd{\expandafter}
expandafter is an instruction that reverses the order of expansion. It is not a typesetting instruction, but an instruction that influences the expansion of macros. But what is \textit{expansion}? The term expansion means the replacement of the macro and its arguments, if there are any, by the \textit{replacement} text of the macro. If we have defined a macro

\begin{teX}
\def\test{ABC};
\end{teX}


\noindent then the replacement text of |\test| is |ABC| and the \textit{expansion} of |\test| is |ABC|.

As a control sequence |expandafter| can be followed by any number of tokens.

\begin{teX}
\expandafter\token_e\token_1\token_2\token_n etc
\end{teX}

\noindent then the following rules describe the execution of |expandafter|:

\begin{enumerate}
\item  $<token_e$, the token immediately following |\expandafter|, is saved without expansion.
\item $<token_1>$, which is the token after the saved $token_e$, is analyzed. The following cases can be distinguished:
\begin{enumerate}
\item If is a macro: The macro will be expanded. In other words, the macro and its arguments, if any, will be replaced by the replacement text. After this \tex will \textbf{not} look at the first token of this new replacement text to expand it again or to execute it.
\end{enumerate}



\begin{teX}
\def\xx [#1]{[#1]}
\def\yy{[ABC]}

\expandafter\xx\yy
\end{teX}

This results in 
\def\xx [#1]{[#1]}
\def\yy{[ABC]}

\texttt{> \expandafter\xx\yy}


\item token1 is primitive: Normally a primitive token can not be expanded so the |\expandafter| has no effect; but there are exceptions, which we will discuss after the example.

\begin{texexample}{Expansion}{}
\expandafter AB
\end{texexample}

Character A is saved. Then \tex\ tries to expand it, but \textit{not} print B, because B cannot be expanded. Finally A is put back in front of the B ; in other words, the two characters are printed in the given order, and we may well have omitted the |\expandafter|. So what's the point here? |\expandafter| reverses the order of expansion, not of execution.

\noindent But there are exceptions to the above:
\begin{enumerate}
\item \textbf{temporarily suspend an opening curly brace} token 1 is is an opening curly brace which leads to the opening curly brace temporarily suspended. This is listed as a separate case because it has some interesting, applications;

\begin{teX}
\newtoks\ta
\newtoks\tb
\ta = {\a\b\c}
\tb=\expandafter{\the\ta}
\tb={\the\ta}
\tb
\end{teX}

\begin{texexample}{Expansion}{}
\begingroup

\def\a{A}
\def\b{B}
\def\c{C}
\newtoks\ta
\newtoks\tb
\ta = {\a\b\c}
\tb=\expandafter{\the\ta}
\tb={\the\ta}

\texttt{> \the\tb}

\texttt{> \the\ta}

\endgroup
\end{texexample}

\item \meta{$token_1$} is another expandafter. The best way to understand this is to write a \tex mnmal example and watch it in action

\begin{teX}
\tracingmacros=2  \tracingcommands=2
\def\a{A}
\def\b{B}
\def\c{C}

\expandafter\expandafter\expandafter\a\expandafter\b\c

\bye
\end{teX}

Checking the log file with |\tracingmacros=2 \tracingcommands=2| we get

\begin{verbatim}
{vertical mode: \def}
{blank space  }
{\def}
{blank space  }
{\def}
{blank space  }
{\par}
{\expandafter}
{\expandafter}
{\expandafter}

\c ->C
{\expandafter}

\b ->B

\a ->A
{the letter A}
{horizontal mode: the letter A}
{\par}

\meaning\futurenonspacelet
\end{verbatim}


\end{enumerate}


\section{Defining Macros on the fly}

This is a very common requirement. 

\begin{texexample}{csname}{}
\def\newtest#1#2{
  \expandafter\def\csname#1\endcsname{#2}%
}
\newtest{letters}{test for letters}
\letters
\end{texexample}





\end{enumerate}









