\parskip1ex 
\newfontfamily\tibetan{TibMachUni.ttf}
\def\deva{{\protect\tibetan\symbol{"0F7C} xx ༃}}

\chapter{Indices}
\pagebreak

\starttemplate{kroll}
\thispagestyle{empty}
    \begin{leftcolumn}
       \begin{center} 
          \huge \noindent PREPARING\\
                   INDEXES
       \end{center}
     
      \medskip

       {\justifying \small\noindent And in such indexes, although small pricks\\
To their subsequent volumes, there is seen\\
The baby figure of the giant mass\\
Of things to come at large. \par
\hfill \textit{--Shakespeare}\par
\hfill\hfill{ \RaggedRight from \textit{Troilus and Cressida}}}
\medskip
       \putimage[width=1.0\linewidth]{./images/animalium01.jpg}\par
       \aheader{Handwritten cards compiled by  Sherborn for his publication \textit{Index Animalium}}
   \end{leftcolumn}
   \begin{rightcolumn}
       \putimage[width=\linewidth]{./images/animalium.jpg}
       \onelinecaption{{\resizebox{\linewidth}{5.5pt}{\bfseries Sherborn’s `Index  Animalium'. Credit Natural History Museum, London\footnote{This monumental publication became the basis for all zoological nomenclature work having gathered together all the relevant data in one place, just as an online database does today.}}}\par}
%  \centerline{\onelineheader{TYPESETTING AN INDEX}}
      \begin{multicols}{2}
        
\parindent1em      \lettrine{T}{he} first English Language index, appeared in Christopher Marlowe's \textit{Hero and Leander} in 1593. At that period, as often as not, by an ``index to a book'' was meant what we should now call a table of contents. Among the first indexes---in the modern sense---to a book in the English language was one in Plutarch's Parallel Lives, in Sir Thomas North's 1595 translation\footnote{Borko, Harold \& Bernier, Charles L. (1978). \textit{Indexing Concepts and Methods}, ISBN 0-12-118660-1.}.  

A section entitled ``An Alphabetical Table of the most material contents of the whole book'' may be found in Henry Scobell's Acts and Ordinances of Parliament of 1658. This section comes after ``An index of the general titles comprised in the ensuing Table''. Both of these indexes predate the index to Alexander Cruden's Concordance (1737) see \citep{farrow96}, which is erroneously held to be the earliest index found in an English book.

      \end{multicols}
   \end{rightcolumn}
\stoptemplate

\pagestyle{headings}


\index{Quantum Mechanics>History|(}
\setlength{\columnsep}{2em}
\begin{multicols}{2}

\section{Preparing an index}

In order to produce an index, we need to load the
package \pkg{makeidx}  and immediately issue the command \cmd{\makeindex}.\footnote{You will also need to at least run the file once using |MakeIndex| on |MikTex|. Check your distribution if you getting problems.}
At the place where
we want the index to be printed,we use \cmd{\printindex}.
\parindent1em

In \latex, we include a word
in the index by using the command \cmd{\index}\meta{arg}, so if the word Kroll should be included in
the index, we should use the command |\index{Kroll}|.

If the word is to be printed in bold, we use

% : = |
% = = @
% \catcode `|
\bgroup
 \catcode`|=11
\gdef\idxmain#1{%
   \def\idxmainentryi##1##2##3;{%
      \index{Kr0=\textbf{#1}|textbf}
    }
   \idxmainentryi#1;    
}  


\egroup

\idxmain{Kroll}

\index{Kroll=\textbf{Kroll}>Leon Kroll}


\emphasis{usepackage,makeidx,makeindex,index,printindex}
\begin{teXXX}
\documentclass{article}
\usepackage{makeidx}
\makeindex
\begin{document}
   To prepare an index, 
   just include the
   package\index{package}

\printindex
\end{document}
\end{teXXX}


There is a  special character |@| is used to denote that what appears on its left side must be
typeset as it appears on its right side.  the first occurrence of perl will also be used
by the sorting algorithm. is is very useful since what is used for sorting and what
will be printed may be different! For example,we may want to have the name 
\texttt{Donald Knuth}  under the letter K., we should write


\verb+\index{Knuth@{Donald Knuth}}+

Another thing we may want to change is the way that the page number is typeset.
If we want, for example, to have the page number in bold, we would write
\verb+\index{perl|textbf}+.  Notice that we wrote "textbf"  without the backslash. Of course,
the above can be combined with the  command


\verb+\index{perl@\textbf{perl}|textit}+


will print the word perl in the index (the entry will be typeset in boldface type) sorted
as "perl",’ and its page number will be italic. A common application of this is through
the command . If we want to send the reader to another index entry, say, to send
the reader from the \verb+$\omega$+ to the \verb+$\Omega$+ command, we can write

\verb+\index{omega@$\omega$|see{$\Omega$}}+


Here, we ask for the entry to be sorted according to the word omega and, in its place,
the program must use 

\begin{teX}
$\omega$|see{$\Omega$}
\end{teX}

If a word is used repeatedly in a range of pages and we want to have this range
in the index, we do not write the relative \texttt{index} command all of the time. Instead,
we write \verb+\index\{convex\|(\}+  at the place where we have the first occurrence and
\verb+\index\{convex|)\}+  at the place where we have the last occurrence. This will produce a
page range in the index for the word "convex".

\subsection{Subindices}
Subindices can be produced using an exclamation mark. If we want the word `Zeus'
to appear in the category of Greek which is in the category of Gods, we will write:

\verb+\index{Gods!Greek!Zeus}+

The actual symbol used will depend on the |.ist| file used when the file is compiled. For example in the |ltxdoc| class this is redefined as |>|.
\DeclareRobustCommand\textat{%
  \bgroup\makeatother \egroup
}


\robustify{\ttdefault}
\robustify{\ttfamily}
\robustify{\color}

 \makeatletter
\def\IDX#1#2{%
   \def\xx{\expandafter\@gobble\string#2}
   \index{\bgroup\small\texttt{\textbackslash #1}>\small\texttt{\textbackslash \xx}\egroup}
}
 
 \makeatother

\IDX{chapter}{\@introduction}
\IDX{chapter}{\intro@duction}
\IDX{abr}{\@some@other}

\section{Multiple Pages}

To perform multi-page indexing, add a |( and |) to the end of the \cmd{\index} command, as in 
\index{Indexing>multi-page}.

{\small
\verb+\index{Quantum Mechanics!History|(}+

\narrower\narrower
In 1901, Max Planck released his theory of radiation dependant 
on quantized energy. While this explained the ultraviolet catastrophe
 in the spectrum of blackbody radiation, this had far larger consequences 
as the beginnings of quantum mechanics.\ldots

\verb+\index{Quantum Mechanics!History|)}+
}

\end{multicols}

\index{Quantum Mechanics>History|)}


\section{Summary of commands}

\begin{tabular}{lll}
\toprule
Example	&Index Entry	&Comment\\
\midrule
\textbackslash index\{hello\}	          &hello, 1	&Plain entry\\
\textbackslash index\{hello!Peter\}	      &Peter, 3	&Subentry under 'hello'\\
\textbackslash index\{Sam@\textbackslash textsl\{Sam\}\}	&Sam, 2	&Formatted entry\\
\textbackslash index\{Lin@\textbackslash textbf\{Lin\}\}	&\textbf{Lin}, 7	&Same as above\\
\textbackslash index\{Jennytextbf\}	     &Jenny, 3	&Formatted page number\\
\textbackslash index\{Joe textit\}	&Joe, 5	          &Same as above\\
\textbackslash index\{ecole@\'ecole\}	&école, 4	&Handling of accents\\
\textbackslash index\{Peter see\{hello\}\}	&Peter, see hello	&Cross-references\\
\textbackslash index\{Jen see also\{Jenny\}\}	&Jen, see also Jenny	 &Same as above\\
\bottomrule
\end{tabular}

\section{Indexing Class Documentation}

\index{Indexing=\textbf{Indexing}}
\index{Indexing>general}
\index{Indexing>doc}

Indexing \latex2e classes is normally achieved using the |doc| and |docstrip| program, which they are not so clear with examples, if you need to deviate from the standard methods. The important thing here to remember is that you need to use different characters |=| |>| |*|.

\begin{tabular}{ll}
\toprule
normal    & doc \\
\midrule
\string @ & \texttt{=} \\
\string ! & \texttt{>}\\
\bottomrule
\end{tabular}

\begin{verbatim}
\index{Indexing=\textbf}
\index{Indexing>general}
\index{Indexing>doc}
\end{verbatim}

This manual, was build using a large |ltxdoc| class and these problems appeared while I was developing it. As normal with such problems, they were very time consuming to debug. There are still issues in some parts and one day, I am hoping to come back and correct them. One needs at this point to query the need to use the |doc| and |docstrip| method of documenting macros and if it shouldn't have a pre-processor written in a higher language to ease development. 

\section{Customization}\index{Indexing>customizing}

When creating an index with \pkgname{makeindex} one can create a \docfile{sample.ist} file that can be used together with the |makeidx| program to customize the way the index will look.

\begin{verbatim}
heading_prefix "{\\bfseries\\hfil "
heading_suffix "\\hfil}\\nopagebreak\n"
headings_flag 1
delim_0 "\\dotfill"
delim_1 "\\dotfill"
delim_2 "\\dotfill"
\end{verbatim}

This will write the first alphabet symbol in bold font, and uses dots as delimiters. This file is generally  used jointly with \texttt{makeindex} using

\verb|makeindex -s sample.ist filename.idx|

where |filename.idx| has been craeted by executing |latex| or one of the other engine commands such as |pdflatex| on |filename.tex|.


According to \citep{gabora}, you may use

\begin{verbatim}
sort_rule "\." "\b\."
sort_rule "\:" "\b\:"
sort_rule "\," "\b\,"
\end{verbatim}


\section{Writing custom indexing commands}

For complex documents it is easier to write a number of macros to assist with indexing and to also provide consistency. For example if you want to index the Devanagari alphabet we might need to get quite creative as to how to both index it as well as get the symbols in the index.
\DeclareRobustCommand\ta{{\tibetan ༃ }}

%\begin{texexample}{Writing Indexing Commands}{ex:zs}
%\gdef\ZZs#1{\incsyms%
%   \indexcommand[\string#1]{#1}%
%   #1}
%\DeclareRobustCommand\ta{{\tibetan ༃}\xparse}
%\ta
%\end{texexample}
%
%The command |\ZZs{\ta}| typesets the command in the document, as (\ZZs{\ta}) and also adds it to the index and typesets the symbol.
%
%
%\begin{phdverbatim}
%\def\ZZs#1{\incsyms%
%   \indexcommand[\string#1]{#1}
%   \string#1}
%\end{phdverbatim}























