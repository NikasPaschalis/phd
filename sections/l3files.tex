\chapter{Expl3 File Operations}
\label{ch:l3files}


 
\tex provides only some basic primitive control sequences for dealing with files. \tex is also limited to 16 input streams and 16 output streams making it considerably difficult to manipulate too many files. In most of the examples here we have used, streams allocated by the \latexe kernel for temporary operations and hence we can re-use them, but with extreme care. 

Checking for the existence of a file is simple and we can use the |\file_if_exist:nTF| function. 

\section{Creating streams}

Before you can use a file in \tex you need to allocate a stream. With \latex3 you can use \docAuxCommand*{ior_new:N} or \docAuxCommand*{iow_new:N} depending if is a stream for write or read. All I/O operations are global: streams are declared with global names and treated accordingly.   

As one can run out of handles very quickly, the \pkgname{phd} package loads the \pkgname{morewrites} and also patches the \pkgname{filecontents} package to enable us to use it. This was renamed to |phdfilecontents|. In Example~\ref{ex:createstreams} 

\begin{texexample}{File streams}{ex:createstreams}
\ExplSyntaxOn
\iow_new:N \scratch_filea
\iow_new:N \scratch_fileb
\iow_new:N \scratch_filec
\iow_new:N \scratch_filed
\ExplSyntaxOff
\end{texexample}

Another way used widely by \latexe and package authors is to use \latexe's |\@inputcheck| file handle. This is a file handle used by |filecontents| as well as internally by the \latexe kernel for  and for building the |\IfFileExists| control sequence and hence its name. 


\begin{texexample}{LaTeXe examples}{ex:inputcheck}
\makeatletter
\bgroup
\ttfamily \meaning\@inputcheck\\
\number\@inputcheck %
\egroup
\makeatother
\end{texexample}

Stream management goes back to \tex and Plain\tex which use an allocation meachanism to assign the the names to numbers and then . This mechanism was then moved onto \latex2.  \latex3 uses a slightly different mechanism but the basic logic is still the same. With \latex3 the allocations are kept in a property list. 

\section{File Operations}
\begin{texexample}{File operations}{ex:fileops}
\ExplSyntaxOn
% Check if a file exists 
  \file_if_exist:nTF { filetest.txt } { \PASS } { \FAIL }
% Check for another one
  \file_if_exist:nTF { filetest } { \PASS } { \FAIL }
  \g_file_current_name_tl
\ExplSyntaxOff
\end{texexample}

\subsection{Handling paths}

The |l3files| module provides a mechanism to add paths to the search path used to search for 
a file. This is pretty much similar to |\graphicspath|.

\begin{docCommand*}{file_path_include:n} {\marg{path}}
Adds \meta{path} to the list of those used to search when reading files. The assignment is local.
The \meta{path} is processed in the same way as a \meta{file name}, i.e., with x-type expansion
except active characters. Spaces are not allowed in the \meta{path}.
\end{docCommand*}

In Example~\ref{ex:corpora1} we add to the search path the directory, where our corpora data is residing. Then we check to see if the file |female.txt| exist. This is a large text file (extension |.txt|), containing common female names
in the US. We will use this file later on for some for examples.

\begin{texexample}{File operations}{ex:corpora1}
\ExplSyntaxOn
% Add path
\file_path_include:n {./corpora/}
\file_if_exist:nTF {female.txt} {\PASS}{\FAIL}
\ExplSyntaxOff
\end{texexample}
\index{path}\index{file operations>path}

\subsection{Loading files}

Loading a full file can of course be achieved with the |\input| command. Under the experimental section of the |\expl3| there is also a function that can load a file on condition that it exists. 

\begin{docCommand}[color command=blue] {file_if_exist_input:n} { \marg{file name} }
Searches for \meta{file name} using the current TEX search path and the additional paths
controlled by |\file_path_include:n|). If found, inserts the \meta{true code} then reads in
the file as additional \latex source as described for |\file_input:n|. Note that 
|\file_if_exist_input:n| does not raise an error if the file is not found, in contrast to |\file-input:n|.
\end{docCommand}

\begin{texexample}{Loading a file only if it exists}{}
\ExplSyntaxOn
\file_if_exist_input:n {filetest.txt}
\ExplSyntaxOff
\end{texexample}

Since the file is loaded within the expl3 block, it will cause all spaces to be removed from the output, we can overcome this by following the |expl3| design pattern of declaring a function with |xparse| and calling it outside the block.

\begin{texexample}{Loading a file only if it exists}{}
\ExplSyntaxOn
\DeclareDocumentCommand \CorporaInput{ m }
  {
    \file_if_exist_input:n { #1 }
  }
\ExplSyntaxOff
\CorporaInput {filetest.txt}  
\end{texexample}

As is usual with |expl3| functions the |nTF| signature form of the |\file_if_exist_input:| is also available. 

\begin{texexample}{Loading a file only if it exists}{}
\ExplSyntaxOn
\DeclareDocumentCommand \CorporaInput{ m }
  {
     \file_if_exist_input:nTF {#1}
  }
\ExplSyntaxOff
\CorporaInput {filetest.txt}{ \PASS } { \FAIL }
\CorporaInput { filetest.txt }{ \PASS } { \FAIL }
\end{texexample}

The above code fails if we leave spaces between the \{\verb*+ filetest.txt +\}, we can easily remove them using the  |>{ \TrimSpaces }| argument processor.

\begin{texexample}{Loading a file only if it exists}{ex:trimspaces}
\ExplSyntaxOn
\DeclareDocumentCommand \CorporaInput{  >{  \TrimSpaces } m }
  {
     \file_if_exist_input:nTF {#1}
  }
\ExplSyntaxOff
\CorporaInput {filetest.txt}{ \PASS } { \FAIL }
\CorporaInput { filetest.txt }{ \PASS } { \FAIL }
\end{texexample}


\section{Reading and writing to streams}

Typical file operations are reading, writing and appending. Common file management operations are creating, deleting, opening, closing, copying an renaming.

\begin{texexample}{File operations}{ex:fileops}
\edef\someheading{Another test}
\ExplSyntaxOn
\iow_open:Nn \tempstream { filetest.txt }
\iow_now:Nx \tempstream {\someheading}
\iow_close:N \tempstream
\let\getfile\file_input:n
%\file_input:n {filetest.txt}
\ExplSyntaxOff
\getfile {filetest.txt}
%\getfile{./corpora/female.txt}
\end{texexample}   

\section{Input-output streams}

Reading one line at a time from a file, uses the \tex primitive |\read|. One important item to watch is that there are different commands for read and write you need to use the |\io|\textcolor{thered}{\texttt{r}}, rather than |iow_|

Streams  are precious in \tex  as we only have 16 available , so when reading from a file, when we use \LaTeXe, we can use some of \LaTeX build-in streams, so we will be using |\@inputcheck|

\begin{texexample}{File operations}{ex:fileops}
\ExplSyntaxOn

\makeatletter
\global\let\ltx_scratch_stream \@inputcheck
\makeatother
\ior_open:Nn \ltx_scratch_stream {male-a.txt}
\ior_get:NN \ltx_scratch_stream \l_tmpa_tl
\tl_use:N \l_tmpa_tl\par
\ior_get:NN \ltx_scratch_stream \l_tmpa_tl
\tl_use:N \l_tmpa_tl
\ior_close:N \ltx_scratch_stream

\ExplSyntaxOff
\end{texexample}   


Reading line by line is not very useful. What we need is the ability to read all he lines
recursively until the end of the file. 

\begin{texexample}{File operations}{ex:fileops}
\ExplSyntaxOn
% add the file path to the name
\file_path_include:n {./corpora/}

% open the stream
\ior_open:Nn \ltx_scratch_stream {male-a.txt}

% define a macro so we can do recursion
\cs_set:Npn \read_loop {
  \if_eof:w \ltx_scratch_stream
    \ior_close:N \ltx_scratch_stream
    \let\next\relax
 \else
   \ior_get:NN \ltx_scratch_stream \tmpa
   \tl_use:N \tmpa,~
   \let\next\read_loop
 \fi      
 \next 
}

% read the file and typeset the words with a comma
\read_loop
\ExplSyntaxOff
\RaggedRight
\end{texexample}   

In the following example we do some more changes to the example. This time instead of typesetting the names,
we will add them to a clist. We will also read a different file that contains an alphabetical list of male names. Then we will check if the name Zacharias is in the list. 

\begin{texexample}{File operations}{ex:fileops}
\ExplSyntaxOn
% \clist
\clist_new:N \males
% open the stream
\file_path_include:n {./corpora/}
\ior_open:Nn \ltx_scratch_stream {male.txt}

% define a macro so we can do recursion
\cs_set:Npn \read_loop {
  \ior_if_eof:NTF \ltx_scratch_stream
    {
      \ior_close:N \ltx_scratch_stream
      \cs_set_eq:NN \next\relax
    }
    {  
      \ior_get:NN \ltx_scratch_stream \tmpa
      \clist_put_right:Nx \males {\tl_use:N \tmpa}
      \cs_set_eq:NN \next\read_loop
   } 
 \next 
}

% read the file and parse the words as a clist \males
\read_loop

% check if Zacharias or Mary are in the list
\clist_if_in:NnTF\males {Zacharias} {\PASS} {\FAIL}
\clist_if_in:NnTF\males {Mary} {\PASS} {\FAIL}
\ExplSyntaxOff
\end{texexample}   

\section{Appending to a file}

To append to a file, first we need to read the contents of the file into a |tl_var| then apend the material and close the file. We then open it again in write mode and write the contents of the |tl_var|. Let us try it out.


    
\section{Writing to the log or aux files}    

There are some constant input-output streams. There is a somewhat different programming philosophy here are these are normally via messages and not directly as shown in the example here.  These are handled in the chapter dealing with messages.

\begin{texexample}{Writing to log and terminal}{ex:log}
\ExplSyntaxOn
\iow_term:x {Something}
\ExplSyntaxOff
\end{texexample}
      
Armed with all these it maybe time to review again our database functions that we have created in the earlier chapter on clists.

                
          
            
              
                
                  
                      



