\chapter{Rules and Leaders}

Dot leaders are a row of dots that visually connect the chapter titles and section headings to their corresponding page numbers. 
You can use leaders to fill a space with copies of a pattern,
\eg, to put repeated dots between a title and a page number in a table
of contents. A leader is a single copy of the pattern. The specification of
leaders contains three pieces of information:

\begin{enumerate}
\item  what a single leader is
\item  how much space needs to be filled
\item  how the copies of the pattern should be arranged within the space
\end{enumerate}

\begin{macro}{\leaders}
\tex  provides three commands for specifying leaders:\cs{leaders},\cs{cleaders},
and\cs{xleaders} (p. 174). The argument of each command specifies the
leader. The command must be followed by glue; the size of the glue specifies
how much space is to be filled. The choice of command determines how
the leaders are arranged within the space.
\end{macro}

Rule leaders \textit{fill} the specified amount of space with a rule extending in the direction of the skip
specified. \index{Rules and Leaders!rule leaders}

The most common application for leaders is to fill the space with either a rule or with dots, such as shown in the example below.

\emphasis{leaders,hbox,hfill}
\begin{texexample}{}{}
\hbox{Exa\leaders\hrule\hskip20pt e}
\hbox to \linewidth{Section 1.2 \leaders\hbox{..}\hfill\space 15}
Section 1.3 \leaders\hbox{..}\hfill\space 15

\parfillskip=0pt plus1fil

\lipsum*[1]\leaders\hbox{..}\hfill\space 15
\end{texexample}

One thing to note is that the leaders must be in a box, such as an \cs{hbox}.

\hbox to \textwidth{g\leaders\hbox{+}\hfill 112}

because a horizontal rule has a default height of .4pt. On the other hand,\index{Rules and Leaders!default value}

\verb+\hbox{g\leaders\vrule\hskip10pt f}+

gives

\hbox{g\leaders\vrule\hskip10pt f}

because the height and depth of a vertical rule by default fill the surrounding box.
Spurious rule dimensions are ignored: in horizontal mode

\verb+\leaders\hrule width 10pt \hskip 20pt+

is equivalent to

\verb+\leaders\hrule \hskip 20pt+

If the width or height-plus-depth of either the skip or the box is negative, TEX uses ordinary glue
instead of leaders.

\section{Box leaders}

Box leaders fill the available spaces with copies of a given box, instead of with a rule.
For all of the following examples, assume that a box register has been allocated:

\newbox\centerdot \setbox\centerdot=\hbox{\hskip.7em*\hskip.7em}

Now the output of

\verb+\hbox to 8cm {here\leaders\copy\centerdot\hfil there}+

is

\hbox to 8cm {here\leaders\copy\centerdot\hfil there}

That is, copies of the box register fill up the available space.

Dot leaders, as in the above example, are often used for tables of contents. In such applications it
is desirable that dots on subsequent lines are vertically aligned. The\cs    {leaders} command does this
automatically:

\begin{teX}
\hbox to 8cm {here\leaders\copy\centerdot\hfil there}
\hbox to 8cm {over here\leaders\copy\centerdot\hfil over there}
\end{teX}

gives

\hbox to 8cm {here\leaders\copy\centerdot\hfil there}
\hbox to 8cm {over here\leaders\copy\centerdot\hfil over there}


The mechanism behind this is the following: TEX acts as if an infinite row of boxes starts (invisibly)
at the left edge of the surrounding box, and the row of copies actually placed is merely the part of
this row that is not obscured by the other contents of the box.

Stated differently, box leaders are a window on an infinite row of boxes, and the row starts at the
left edge of the surrounding box. Consider the following example:

\begin{texexample}{}{}
\hbox to 8cm {\leaders\copy\centerdot\hfil}
\hbox to 8cm {word\leaders\copy\centerdot\hfil}
\end{texexample}

which gives,

\hbox to 8cm {\leaders\copy\centerdot\hfil}
\hbox to 8cm {word\leaders\copy\centerdot\hfil}

The row of leaders boxes becomes visible as soon as it does not coincide with other material.
The above discussion only talked about leaders in horizontal mode. Leaders can equally well be
placed in vertical mode; for box leaders the ‘infinite row’ then starts at the top of the surrounding
box.


\begin{macro}{cleaders}
\begin{macro}{xleaders}
The \cs{cleaders} command is similar to 
\cs{leaders}, but it splits excess space before and after the leaders into two equal parts, centring the row of boxes in the available space.
The \cs{xleaders} command is also similar, but spreads the space between and after the leaders evenly between all the boxes.
\end{macro}
\end{macro}

The differences are best explained with an example.

\emphasis{leaders,cleaders,xleaders}
\begin{texexample}{}{}
\def\leaderpattern{\hbox{\kern0.5em-\kern0.5em-\kern0.5em-}}
Lorem \leaders\leaderpattern\hfill 13\par
Lorem \cleaders\leaderpattern\hfill 13\par
Lorem \xleaders\leaderpattern\hfill 13\par
\end{texexample}


\section{Vertical leaders}

If vertical glue commands such as \cs{vfill} is used it is possible to have
vertical leaders. In Example~\ref{vleaders} we use a centered dot \cs{cdot} to fill the space between two paragraphs with leaders. We define a command
\cs{vdotfill} to do this that contains the instructions.

\begin{texexample}{Vertical leaders}{vleaders}
\newcommand{\vdotfill}{%
  \par\leaders\hbox{$\cdot$}\vfill}
  \vbox to 3cm{
  \lorem
  \vdotfill
  \lorem
  }
\end{texexample}





\section{Leaders and shifted margins}

If margins have been shifted, leaders may look different depending on how the shift has been realized.
For an illustration of how\cs    {hangindent} and\cs{leftskip} influence the look of leaders, consider
the following examples, where

\setbox0=\hbox{R a t a t a  }
\verb+\setbox0=\hbox{R a t a t a  }+



\hbox{\kern1em\hbox{\leaders\copy0\hskip5cm}}

\hangindent=1em \hangafter=-1 \noindent
\leaders\copy0\hskip5cm\hbox{}\par


gives (note the shift with respect to the previous example)
\medskip

{\hbox{\kern1em\hbox{\leaders\copy0\hskip5cm}}
\hangindent=1em \hangafter=-1 \noindent
\leaders\copy0\hskip5cm\hbox{}\par}

In the first paragraph the\cs{leftskip} glue only obscures the first leader box; in the second paragraph
the hanging indentation actually shifts the orientation point for the row of leaders. Hanging
indentation is performed in TEX by a\cs{moveright} of the boxes containing the lines of the
paragraph.

   

Leaders are a powerful tool, they take a little bit of time to understand, but once you familiar with them you can achieve all sorts of layouts with them.


\section{Applications}

Most of the useage of leaders is in table of contents and old tables fashioned the old way. The package \pkg{arydshln} by Hiroshi Nakashima uses \cs{xleaders} to give \latex’s \pkg{array} and \pkg{tabular} environments the capability to draw horizontal/vertical dash-lines. You can refer to it for more examples.

In the LateX kernel they are mostly found them in the definition of mathematical symbols and from where we derived Example~\ref{cleaders}.

\begin{texexample}{cleaders example}{cleaders}
 \makeatletter
 \def\rightarrowfill{$\m@th\smash-\mkern-7mu%
  \cleaders\hbox{$\mkern-2mu\smash-\mkern-2mu$}\hfill
  \mkern-7mu\mathord\rightarrow$}
 \makeatother
From here to \rightarrowfill the end.
\end{texexample}

Note in the example the use of mathematical kerns (|\mkern|) and the use of 
|\smash|. Another interesting area was the definition of various commands in the
picture environment using solely leaders.


















