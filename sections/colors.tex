\chapter{Colors}

\newthought{The figure below, shows the wavelengths} in nm of the visible light. It has been drawn using the \docpkg{xcolor} package and the native \latex environment \cmd{picture}. The colors can be typest using the wavelength of light.

\smallskip

\begin{texexample}{}{}
  \hbox{\color{thered} A TesT}
\end{texexample}




\newcount\WL \unitlength.75pt

\begin{figure}
\hskip-3pt\scalebox{0.9}{
\noindent

\begin{picture}(460,60)(355,-10)
\sffamily \tiny \linethickness{1.25\unitlength} \WL=360
\multiput(360,0)(1,0){456}%
{{\color[wave]{\the\WL}\line(0,1){50}}\global\advance\WL1}
\linethickness{0.25\unitlength}\WL=360
\multiput(360,0)(20,0){23}%
{\picture(0,0)
\line(0,-1){5} \multiput(5,0)(5,0){3}{\line(0,-1){2.5}}
\put(0,-10){\makebox(0,0){\the\WL}}\global\advance\WL20
\endpicture}
\end{picture}}
\caption{The visible spectrum nm}
\end{figure}

The |xcolor| package provides numerous macros for typesetting colors, using a variety of methods and color schemes. For example we can use the command \cs{color} to print a text sample in color.
\newlength\pull

\def\colorSample#1{%
\leavevmode
\parindent0pt
   \def\colorRule{\color[wave]{#1}\rule{\textwidth}{0.4pt}} 
   \colorRule
%% set to the width of the box
   \settowidth\pull{\framebox{\Large #1 nm}}
%% pull by one em
   \addtolength\pull{1em}
   \hskip -\pull{\color[wave]{#1}{{\framebox{\Large #1 nm}}}}%
   %% add story
   \hskip1em\noindent\onepar\par
   \colorRule
}
\bgroup
\colorSample{385}
\colorSample{809}
\egroup


\section{Specifying colors by name}

The easier way to specify colors is to use the pre-build names available
with the package drivers.


\section{Color boxes}

Now and then users want to place text in colorboxes. The macro \cs{colorbox} can be used to provide background text to a box.

\begin{macro}{\colorbox}
The colorbox macro takes one command. As it is modelled on the same concept as an
\cs{fbox} it will add a bit of space around the containing text. If you want
to remove it you will have to set the \cs{fboxsep=0pt}.
\end{macro}


\begin{texexample}{Color boxes}{}
\fboxsep0pt
\colorbox{green}{\begin{minipage}{3cm}
  Some text
\end{minipage}}

\fboxrule.2pt\fboxsep-.2pt
\fbox{\begin{minipage}{3cm}
  Some text
\end{minipage}}
\end{texexample}









