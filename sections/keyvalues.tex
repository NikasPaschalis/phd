\chapter{Key Value Interfaces}

The key value system greatly simplifies the \tex interface for authors. As \citep{joseph2009} wrote this ease of use was not transferred into settting up key-value systems for authors of pre-packaged \tex code. This Chapter and the one that follows that focus specifically on the \pkg{pgfkeys} package provides an overview and describes some of the more difficult areas. 

\section{keyval}

The \pkg{keyval} written by David Carlisle is still widely used by package authors to provide the means for users to easily specify numerous optional arguments for macros. The main advantages of using keyval are that  (1) the number of optional arguments is no longer limited to 9 and that (2) the arguments are named, and hence there is less chance of confusion about the syntax of a macro.

\section{xkeyval}

Before you start experimenting with the xkeyval package, I suggest that you load the package \pkg{xkview}. This is part of the \ctan{xkeyval}  bundle and can help you to view key value parameters in various ways. The \pkg{xkeyval} was developed by Hendri Adriaens and Uwe Kern \citep{xkeyval}. This package is an extension of the well-known |keyval| package. The package provides more flexible commands and syntax enhancements as well as newer option processing mechanism.

The main change of the |xkeyval| package is that it provides a means to namespace the keys, which all have the form |\KV@family@keyname|, where the KV is a literal prefix to avoid collisions. They take one argument to handle user input.

The main commands of the package are the same as those of keyval ||



\begin{texexample}{xkeyval }{}
\makeatletter

\define@key{phd}{pi}{\setlength{\parindent}{#1}}

\setkeys{phd}{pi=50pt}
\makeatother

\lorem

\setkeys{phd}{pi=0pt}
\lorem
\end{texexample}

Defining a default key, i.e., a key that can be used as |indent| or |indent=30pt| will stretch your memory, as it has an optional parameter as its third argument. 

\begin{verbatim}
\define@key{family}{key}[none]{The input is: #1}
\end{verbatim}

\begin{texexample}{xkeyval }{}
\makeatletter

\define@key{phd}{pi}[30pt]{\setlength{\parindent}{#1}}

\setkeys{phd}{pi}

\lorem

or \setkeys{phd}{pi=0pt}

\lorem
\makeatother
\end{texexample}

\section{Ordinary Keys}
\makeatletter
\define@key{phd}{pi}[1em]{\setlength{\parindent}{#1}}
\makeatother


Ordinary keys are keys that have values such as \texttt{animal=elephant} and your macro can be called like \texttt{animals[animal=elephant]\{14\}}.

   



\section{Keys and values in package options}

First of all, the package supplies macros to declare class or package options, execute them and process
them. The macros are available under the usual
LATEX names, but all with the suffix X, namely

\begin{macro}{\DeclareOptionX}
\begin{macro}{\DeclareOptionX*}
\begin{macro}{\ExecuteOptionsX}
\begin{macro}{\ProcessOptionsX}
These commands allow the user to assign a value to
an option just like when using |\setkeys|. The first
macro is based on |\define@key| and the final two
are based on |\setkeys|. Supposing that a package
|mypack| is set up with these commands, a user could
for instance do
\end{macro}
\end{macro}
\end{macro}
\end{macro}

\begin{verbatim}
\usepackage[textcolor=red,font=times]{mypack}
\end{verbatim}

These |xkeyval|macros are fully compatible with the \latex option conventions. They will allow packages to copy global options specified in the |\documentclass| command, to pass options to other classes or packages and to update the list of unused global options that will be displayed by \latex in the log file. 









