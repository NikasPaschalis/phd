\chapter{Endnotes}

There a number of packages that can help you add endnotes to your document.
This can be added at the end of every document or at the end of every chapter.

The first package we will use is \ctan{pagenote}. This was developed by Peter Wilson, 
and now maintained by Will Robertson. The pagenote package provides notes similar to footnotes except that
they are typeset on a different page. These are often called end notes.
Unless the memoir class is being used, the package requires the |ifmtarg|
package to be installed as well.

The package manager provides a key,


\begin{key}{/chapter/endnotes pagenote=\meta{pagenote or endnotes (default pagenote)}  }
 Set the key using |\cxset{endnotes package=pagenote}|.
\end{key}

One need to be careful, of the other commands required, in the preamble when you load the package. For example the |pagenote| package, requires that a |\makepagenote| command is issued in the preamble.
The entries are written in an auxiliary file with the extension |.ent|. 

\emphasis{document,usepackage,pagenote,printnotes,
          makepagenote,usepackage}

\begin{teXXX}
\documentclass{book}
\usepackage{pagenote}
\makepagenote
\renewcommand*{\notedivision}{\section*{%
   \notesname\ to chapter~\thechapter}}
\renewcommand*{\pagenotesubhead}[2]{}
\begin{document}
\chapter{bar}
  \section{blubb}
   Some text.\pagenote{The first endnote.}
\printnotes*

\chapter{foo}
  \section{foobar}
   Some text.\pagenote{The second endnote.}
   Some text.\pagenote{}
\printnotes*
\end{document}
\end{teXXX}

The \cmd{printnotes} command will cause the |.ent| file to be closed for any new
\cmd{printnotes*} notes, and then |\input| in order to print the collected notes. After |\printnotes|
no more notes will be collected, so use it after all are done.

The \ctan{endnotes} package works in a similar manner. To place all endnotes at the end of the chapter Each time you use \cmd{\theendnotes}, all endnotes that were stored previously will be put there. So just write \cmd{\theendnotes} at the end of each chapter.



Footnotes are notes\pagenote{There is a long discussion about the pros and cons of end notes in the Wikipedia.} at the foot of the page while endnotes are collected under a separate heading at the end of a chapter in a book or a document. Unlike footnotes, endnotes have the advantage of not affecting the image of the main text, but may cause inconvenience to readers who have to move back and forth between the main text and the endnotes\pagenote{The polynomial \[ polynomial{1,2,3,4,5} \] is a case in point. You can typeset a polynomial by using the \texttt{polynomial} package or if you are using special polynomials you should have a look at the \texttt{cool} package}.

The U.S. Government Printing Office Style Manual devotes over two pages to the topic of footnotes.[2] NASA has guidance for footnote usage in its historical documents.[3] About rules \pagenote{\notei}

\url{http://tex.stackexchange.com/questions/210/how-can-i-get-two-sequences-of-footnotes-in-one-latex-document-one-as-footnote}

 "The user may want separate endnotes for each chapter, or a big block of them at the end of the whole document. As it stands, either will work; you just say \cmd{theendnotes} wherever you want the endnotes so far to be inserted. However, you must add |\setcounter{endnote}{0}| after that if you want subsequent endnotes to start numbering at 1 again." 

\section*{Footnote and end note font sizes}

This is normally specified by the publications Style Manual. It is normal to use smaller type approximately 3-4 points smaller than the main text\pagenote{\noteii}\pagenote{\noteiv}.


\section*{Footnote marks}

Again for the footnote marks the publication Style manual should be consulted\pagenote{\noteiii}.

When symbols or signs are used for footnote reference marks, their 
sequence should be (*) asterisk, (\dag) dagger, (\ddag) double dagger, and 
(\S) section mark. Should more symbols be needed, these may be 
doubled or tripled, but for simplicity and greater readability, it is 
preferable to extend the assortment by adding other single-character symbols.(US Government Style Manual). \latex provides eleven marks and these should adequately cover most cases. End notes are normally marked in sequential superior numbers.
\sidenote{See also Chapter 8}

\section*{Separation of footnotes}

When two or more footnotes occur together use a thin space to separate them.
\pagenote{From the U.S. Governement Style Manual p.305, \S 15.19. Two or more superior footnote references occurring together are separated by thin spaces.}
\tex and most of the packages will adequately allow for this.


\section{Some tips}

If your footnotes and endnotes tend to be rather long, they can interefere with the editing of the works. In such case you can define some macros and accumulate these some where either at the beginning or end of the Chapter. Remember you cannot use numbers as variable names. An example of such a macro is shown below:

One question that comes quite often is:

I find it common in my writing to end up a sentence with a footnote reference mark. Should the footnote mark come before the stop or after it?

... this is some text$^a$.

... this is some text.$^b$


but in the world of typesetting, logic is not respected. At least in the field of scientific publishing, where footnotes and references are common, publishers tend to have very strict guidelines about where they want the marks. In that case, you're not really free to choose yourself. For example, most of the physics and chemistry journals want references after punctuation:

Both are valid ways to place a footnote reference, but they mean slightly different things.

If you want the footnote reference to belong to the entire sentence, then the second method is correct. However, if you want the footnote to apply only to the word text, then the first is correct.\footnote{AAAAAA}

See also \pagenote{For a discussion see \href{http://english.stackexchange.com/questions/9632/footnote-marks-at-end-of-a-sentence}{footnotemarks}}

\emphasis{def,notei}
\begin{teXXX}

\def\notei{\footnotesize{p[116] of \textit{The Printer's Grammar}, printed by L. Waylard,1775. writes about metal rules. Metal rules, like quadrats, are cast to m's, in such founts as are commonly employed in figure-work; which are casr to most of the small bodies. Metal rules are used in Schemes of Accounts, to direct and conect each Article with its summary contents, where they stand oposite, and distant from each other: in which case all the different sizes of rules are used, to prevent one rule falling upon another, especially of the same force;}}
\end{teXXX}

\def\notei{\footnotesize{p[116] of \textit{The Printer's Grammar}, printed by L. Waylard,1775. writes about metal rules. Metal rules, like quadrats, are cast to m's, in such founts as are commonly employed in figure-work; which are casr to most of the small bodies. Metal rules are used in Schemes of Accounts, to direct and conect each Article with its summary contents, where they stand oposite, and distant from each other: in which case all the different sizes of rules are used, to prevent one rule falling upon another, especially of the same force;}}




\def\noteii{\footnotesize{U.S. Government Style Manual (2008), 30th Edition, recommends that unless the copy is otherwise marked: (1) Footnotes to 12-point text 
are set in 8 point; (2) footnotes to 11-point text are set in 8 point, 
except in Supreme Court reports, in which they are set in 9 point; 
(3) footnotes to 10- and 8-point text are set in 7 point.}}

\def\noteiii{\footnotesize{For reference marks use: (1) Roman superior figures, (2) italic superior letters, and (3) symbols. Superior figures (preferred), letters, and symbols are separated from the words to which they apply by thin 
spaces, unless immediately preceded by periods or commas.}}

\def\noteiv{\footnotesize{Manual of Typography, pg 44 Foot-notes are set smaller than the text. There is no absolute rule, but generally it is two or three sizes less. If the text were in pica the notes might be set in brevier, or
even nonpareil. The same holds good for small pica or
long primer. Foot-notes have references in the text, and
when being set should be "flagged": that is, a piece of
paper inserted in the type stating what it is, to guide the
maker up, so that he can place it in its proper position.}}


\footnoterule

\renewcommand*{\notedivision}{\section*{\notesname\ to chapter~\thechapter}}
\renewcommand*{\pagenotesubhead}[2]{}

\printnotes



\normalsize







