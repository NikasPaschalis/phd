\chapter{The LateX Kernel}

\begin{multicols}{2}
\latex is the batteries of \tex. It has given authors the ability to write documents easily and in a consistent way. It has also provided pacakage writers with endless nights of work trying to figure out how things work. In this chapter we will describe briefly the workings of \latex and the areas where one could use it for improvements. The best source for how the \latex kernel works is \latex itself and the publication that comes with it \docfile{sourc2e.pdf}

\Paragraph{Code Organization.} The \latex source code is distributed in a number of classes. These classes are saved in files |a|..|z| and files |A-O| The source files are documented in |source2e|, just |texdoc source2e| to read it. What I am describing here, is a step by step analysis of the classes, supplemented by additional materials, in order to understand the inner workings. 
\end{multicols}

{\RaggedRight
\begin{tabular}{lp{5cm}}
\toprule
Filename  &Description \\
\midrule
a ltxdirchk.dtx &        \\
b ltplain.dtx    &        \\
c ltvers.dtx     & Version information       \\
c ltplain.dtx    & definitions, mostly from plain\\
h ltpar.dtx      & Paragraphs See (\pageref{pars})\\
i  ltspace        &Spacing commands. See~pg~[\pageref{spc}]\\
k files.dtx       & File handling, listing of files\\
n ltlengths      & Length setting commands. See~pg~[\pageref{kernel:lengths}]\\
\bottomrule
\end{tabular}
}

%\chapter{Paragraphs \texttt{File h: ltpar.dtx}}

\index{LaTeX kernel classes>File h ltpar.dtx}

This section of the kernel declares the commands used to set |\par| and |\everypar|
when ever their function needs to be changed for a long time. It is a very small class
and the commands that are defined are used extensively by other sections of the kernel.

As the kernel authors note, There are two situations in which |\par| may be changed:

\begin{enumerate}
\item  Long-term changes, in which the new value is to remain in effect until the
current environment is left. The environments that change |\par| in this way
are the following:

 All list environments (itemize, quote, etc.)
 
 Environments that turn |\par| into a noop: tabbing, array and tabular.

\item Temporary changes, in which |\par| is restored to its previous value the next
time it is executed. The following are all such uses.
|\end| when preceded by |\@endparenv|, which is called by |\endtrivlist|

 The mechanism for avoiding page breaks and getting the spacing right
after section heads.
\end{enumerate}

\Paragraph{\textbackslash @setpar} This initializes a long-term change to |\par|. The default definition of |\@par| will ensure that if |\@restorepar| defines |\par|
to execute |\@par| it will redefine itself to the primitive |\@@par| after one iteration.

\startlineat{3}
\begin{teX}
3 \def\@setpar#1{\def\par{#1}\def\@par{#1}}
4 \def\@par{\let\par\@@par\par}
\end{teX}

\Paragraph{\textbackslash @restorepar} Restore from a short-term change to |\par|.

\begin{teXX}
6 \def\@restorepar{\def\par{\@par}}
\end{teXX}

 Paragraphs
%\label{pars}
%\input{kernel-i-ltspace.tex}\label{spc}

%%% LaTeX2e file `./manual/kernel-F-ltsec.tex'
%% generated by the `filecontents' environment
%% from source `photo-book-class-final-test' on 2011/12/18.
%%
\clearpage
\section{Sectioning Commands - File F ltsect.dtx}

This file defines the declarations such as \cmd{author} which are used by \cmd{maketitle}.
|\maketitle| itself is defined by each class, not in the \latex kernel.
The second part of the file defines the generic commands used for defining sectioning
commands such as |\chapter|. Again the actual document level commands
are defined in the class files, in terms of these commands


The \cmd{@startsection} as the manual says is the mother of all sectioning commands.

The |\@startsection{name}{level}{indent}{beforeskip}|
|{afterskip}{hstyle}*[altheading]{heading}| command is the mother of all
the user level sectioning commands. The part after the *, including the * is
optional

\paragraph{name:} \eg subsection
\paragraph{level}: a number, denoting depth of section - eg., chapter=1, section=2, etc.
\paragraph{indent:} Indentation of heading from left margin

   Sectioning
%%% LaTeX2e file `./manual/kernel-d-ltxdefns.tex'
%% generated by the `filecontents' environment
%% from source `photo-book-class-final-test' on 2011/12/18.
%%
\section{Definitions File d ltxdefns.dtx}
This is one of themost useful sections of \latex for macro writers as it exposes a lot of macros hat one can find useful in packages. They are found in file |d|.


\cmd{two"@"digits} Prefixes a number less than 10 with a `|0|'.
typing,
\begin{teX}
\two@digits{21},
\two@digits{8}
\end{teX}

will give you:

\makeatletter
\two@digits{21},
\two@digits{8}
\makeatother


\cmd{typeout} Display something on the terminal.

\begin{teX}
 \def\typeout#1{\begingroup\set@display@protect
 \immediate\write\@unused{#1}\endgroup}
\end{teX}

\cmd{newlinechar} A char to be used as new-line in output to files.



\subsection{Saved versions of primitives}

The \tex primitives are saved in \latex using |@@| as a prefix for example the \tex primitive |\par| is saved as |\@@par|.

\begin{trivlist}

\item @@par The \tex primitive |\par|.

\item @@hyph This is the \tex primitive |\-|.

\item @@italiccorr saves the italic correction.

\item @height Use height instead of |\height|

\item  @width

\item  @depth Saves 5 tokens at the cost of time for macro expansion

\item @minus The |minus| command

\item  @plus The |plus| command
\end{trivlist}


\section{Command definitions}

This is one of the most useful sections as most of the macro writing commands are collected here.

For example typing the \latex equivalent for |csname|:

\begin{teX}
   \@namedef{123}{\texttt{One Two Three}}
   \@nameuse{123}
\end{teX}

\makeatletter
\@namedef{123}{\texttt{One Two Three}}

Will give you \@nameuse{123}. Besides saving a lot of typing, by using \latex kernel commands you provide a consistent interface to people that may wish to modify your code. One disantavantage though is tha tyour program will not be able to be used by |plain| \tex users. Each on its own.
\makeatother



\begin{teX}
\@ifundefined{123}{Yes}{No}
\end{teX}

Check if first arg is undefined or |\relax| and execute second or third arg depending,

\begin{teX}
   \def\@ifundefined#1{%
      \expandafter\ifx\csname#1\endcsname\relax
      \expandafter\@firstoftwo
      \else
      \expandafter\@secondoftwo
   \fi}
\end{teX}

The \cmd{@undefined} is normally used to check a named definition (one that has been created through csname..endcsname. Consider the following code:

\begin{teX}
\def\csUndefined#1{
\@ifundefined{#1}
   {The command \textbackslash \@nameuse{#1} is undefined}
   {The command \textbackslash #1 has been defined and expands to  \@nameuse{#1}}
}
\end{teX}

\def\csUndefined#1{
\@ifundefined{#1}
   {The command \textbackslash \@nameuse{#1} is undefined}
   {The command \textbackslash #1 has been defined and expands to  \@nameuse{#1}}
}

\noindent Typing, |\csUndefined{123}| we get \csUndefined{123}

\noindent Typing, |\csUndefined{texttt}| we get \csUndefined{texttt}{test}

This pattern is used fairly often and consistently in \latex when defining new commands on the fly, checking first if the command is defined and then taking some action. In most instances the first part issues an error message.

\makeatletter
\p@=1pt
\the\p@

\makeatother

It takes a while to get used in programming with variable names that include the |@|, symbol. If you find it difficult at first use your own style. You do not need to use it, but beware that other users or package writer can inadvertatly use the name symbol. However, with careful use of grouping you can aviod most mistakes like this.


\section{Fragile and robust commands}

Fragile and robust commands are one of the thornier issues in \latex's commands.\index{fragile commands},\index{robust commands}

Whilst typesetting documents, \latex makes use of many of \tex's features, such as
arithmetic, defining macros, and setting variables. However, there are (at least) three different ocassions when these commands are not safe. These are called \emph{moving arguments} by \latex, and consist of:

\begin{enumerate}
\item writing information to a file, such as indexes or tables of contents.
\item writing information to the screen.
\item inside an |\edef|, |\message|, |\mark|, or other command which evaluates its argument fully.
\end{enumerate}

\latex uses \cmd{protect} to make fragile commands robust. It does this by preceding the command with |\protect|. This can have one of five possible values:

\begin{enumerate}
\item |\relax|, for normal typesetting. So |\protect\foo| will execute |\foo|


\item \cmd{string}, for writing to the screen. So |\protect\foo| will write |\foo|.

\item \cmd{noexpand}, for writing to a file. So |\protect\foo| will write |\foo| followed by a space.


\item \cmd{@unexpandable@protect}, for writing a moving argument to a file. So |\protect\foo| will write |\protect\foo| followed by a space. This value is also used inside |\edefs|, |\marks| and other commands which evaluate their arguments fully.

\item \cmd{@unexpandable@noexpand}, for performing a deferred write inside an |\edef|.
So |\protect\foo| will write |\foo| followed by a space. If you want
|\protect\foo| to be written, you should use |\@unexpandable@protect|.
(This was Removed as it was never used).

\end{enumerate}

\latex provides the command \cmd{DeclareRobustCommand} to manage robust definitions.

\begin{teX}
\DeclareRobustCommand{\seq}[2][n]{%
\ifmmode
#1_{1}\ldots#1_{#2}%
\else
\PackageWarning{fred}{You can't use \protect\seq\space in text}%
\fi
}




So how does \latex does it? For this we need to look under |ltdefns.dtx|

\begin{teX}
 \def\@unexpandable@protect{\noexpand\protect\noexpand}
\end{teX}


\cmd{DeclareRobustCommand} is a package writers command, which has the same syntax as \cmd{newcommand}, but which declares a protected command.

It does this by having |\DeclareRobustCommand\foo|
define |\foo| to be |\protect\foo<space>|,
and then use |\newcommand\foo<space>|.

Since the internal command is |\foo<space>|, when it is written to an auxiliary
file, it will appear as |\foo|.
We have to be a bit cleverer if we're defining a short command, such as |\_|,
in order to make sure that the auxiliary file does not include a space after the
command, since |\_| a and |\_a| aren't the same. In this case we define |\_ |to be:

\begin{teX}
|\x@protect\_\protect\_<space>|
\end{teX}

which expands to:

\begin{teX}
\ifx\protect\@typeset@protect\else
\@x@protect@\_
\fi
\protect\_<space>
\end{teX}

Then if |\protect| is |\@typeset@protect| (normally |\relax|) then we just perform
|\_<space>|, and otherwise |\@x@protect@| gobbles everything up and expands to |\protect\_|


During typesetting we let the |@typeset@protect| to |\relax|

\begin{teX}
\let\@typeset@protect\relax
\end{teX}

%\def\strip@prefix#1>{}

\makeatletter
\def\test{A Test}
\let\tesT\test


Before we review the rest of the code let us make a small detour around the dangerous bends to re-examine the \emph{meaning} of \cmd{meaning}. This command (which is a \tex primitive), will return the definition of a macro.

\begin{teX}
\meaning\maketitlepage
\end{teX}

will produce

{
\tt

\meaning\maketitlepage


}


We can use the \latex kernel command |\strip@prefix|to strip the prefix of the macro


\begin{teX}
\expandafter\strip@prefix\meaning\maketitlepage
\end{teX}



As you can see the |\long macro:->| has now been removed. We will use this to use our own definition of a Macro similar to \latex first. What we will do we will define

\begin{teX}
\def\NewMacro#1{}  %\def\DeclareRobustCommand{}
\end{teX}

Our \cmd{NewMacro} will simply define a command to be |\edef|. Once we understand some of the concepts we will then move on to discuss the way \latex defines a |DeclareRobustCommand|.

What we need to capture is first the new command name and then the contents within the brackets. For example we may need to define a command

\begin{teX}
\NewMacro\test{This is test}
\end{teX}

We both need to capture the |test| as well as the |{This is a Test}|. We do this so that we can then |\edef\test{This is a test}|. Let us see one way.


\makeatletter

\def\@atest{\textsc{$\alpha$, Gamma}}

\@atest

\edef\TesT{\expandafter\strip@prefix\meaning\@atest}%

\@atest

\TesT

and the meaning of TesT is  {\tt \meaning\TesT}

What we have managed so far is to get the definition of a macro and place it in another macro. This is \tex magic! Whatremains now is to get the arguments (including the name of the macro and place it in front of the |edef|. Then we can have a macro that can be used to define other macros!

\makeatother

For example, if |\seq| is defined as follows:


\DeclareRobustCommand{\seq}[2][n]{%
\ifmmode
#1_{1}\ldots#1_{#2}%
\else
\PackageWarning{fred}{You can't use \protect\seq\space in text}%
\fi
}

$\seq{n1,n2}$

\seq{n1,n2}


You must admit that following what \latex does, involves mental acrobatics. With time though things start to fall into place!
\clearpage
 Definitions
%%% LaTeX2e file `./manual/kernel-d-ltxdefns.tex'
%% generated by the `filecontents' environment
%% from source `photo-book-class-final-test' on 2011/12/18.
%%
\section{Definitions File d ltxdefns.dtx}
This is one of themost useful sections of \latex for macro writers as it exposes a lot of macros hat one can find useful in packages. They are found in file |d|.


\cmd{two"@"digits} Prefixes a number less than 10 with a `|0|'.
typing,
\begin{teX}
\two@digits{21},
\two@digits{8}
\end{teX}

will give you:

\makeatletter
\two@digits{21},
\two@digits{8}
\makeatother


\cmd{typeout} Display something on the terminal.

\begin{teX}
 \def\typeout#1{\begingroup\set@display@protect
 \immediate\write\@unused{#1}\endgroup}
\end{teX}

\cmd{newlinechar} A char to be used as new-line in output to files.



\subsection{Saved versions of primitives}

The \tex primitives are saved in \latex using |@@| as a prefix for example the \tex primitive |\par| is saved as |\@@par|.

\begin{trivlist}

\item @@par The \tex primitive |\par|.

\item @@hyph This is the \tex primitive |\-|.

\item @@italiccorr saves the italic correction.

\item @height Use height instead of |\height|

\item  @width

\item  @depth Saves 5 tokens at the cost of time for macro expansion

\item @minus The |minus| command

\item  @plus The |plus| command
\end{trivlist}


\section{Command definitions}

This is one of the most useful sections as most of the macro writing commands are collected here.

For example typing the \latex equivalent for |csname|:

\begin{teX}
   \@namedef{123}{\texttt{One Two Three}}
   \@nameuse{123}
\end{teX}

\makeatletter
\@namedef{123}{\texttt{One Two Three}}

Will give you \@nameuse{123}. Besides saving a lot of typing, by using \latex kernel commands you provide a consistent interface to people that may wish to modify your code. One disantavantage though is tha tyour program will not be able to be used by |plain| \tex users. Each on its own.
\makeatother



\begin{teX}
\@ifundefined{123}{Yes}{No}
\end{teX}

Check if first arg is undefined or |\relax| and execute second or third arg depending,

\begin{teX}
   \def\@ifundefined#1{%
      \expandafter\ifx\csname#1\endcsname\relax
      \expandafter\@firstoftwo
      \else
      \expandafter\@secondoftwo
   \fi}
\end{teX}

The \cmd{@undefined} is normally used to check a named definition (one that has been created through csname..endcsname. Consider the following code:

\begin{teX}
\def\csUndefined#1{
\@ifundefined{#1}
   {The command \textbackslash \@nameuse{#1} is undefined}
   {The command \textbackslash #1 has been defined and expands to  \@nameuse{#1}}
}
\end{teX}

\def\csUndefined#1{
\@ifundefined{#1}
   {The command \textbackslash \@nameuse{#1} is undefined}
   {The command \textbackslash #1 has been defined and expands to  \@nameuse{#1}}
}

\noindent Typing, |\csUndefined{123}| we get \csUndefined{123}

\noindent Typing, |\csUndefined{texttt}| we get \csUndefined{texttt}{test}

This pattern is used fairly often and consistently in \latex when defining new commands on the fly, checking first if the command is defined and then taking some action. In most instances the first part issues an error message.

\makeatletter
\p@=1pt
\the\p@

\makeatother

It takes a while to get used in programming with variable names that include the |@|, symbol. If you find it difficult at first use your own style. You do not need to use it, but beware that other users or package writer can inadvertatly use the name symbol. However, with careful use of grouping you can aviod most mistakes like this.


\section{Fragile and robust commands}

Fragile and robust commands are one of the thornier issues in \latex's commands.\index{fragile commands},\index{robust commands}

Whilst typesetting documents, \latex makes use of many of \tex's features, such as
arithmetic, defining macros, and setting variables. However, there are (at least) three different ocassions when these commands are not safe. These are called \emph{moving arguments} by \latex, and consist of:

\begin{enumerate}
\item writing information to a file, such as indexes or tables of contents.
\item writing information to the screen.
\item inside an |\edef|, |\message|, |\mark|, or other command which evaluates its argument fully.
\end{enumerate}

\latex uses \cmd{protect} to make fragile commands robust. It does this by preceding the command with |\protect|. This can have one of five possible values:

\begin{enumerate}
\item |\relax|, for normal typesetting. So |\protect\foo| will execute |\foo|


\item \cmd{string}, for writing to the screen. So |\protect\foo| will write |\foo|.

\item \cmd{noexpand}, for writing to a file. So |\protect\foo| will write |\foo| followed by a space.


\item \cmd{@unexpandable@protect}, for writing a moving argument to a file. So |\protect\foo| will write |\protect\foo| followed by a space. This value is also used inside |\edefs|, |\marks| and other commands which evaluate their arguments fully.

\item \cmd{@unexpandable@noexpand}, for performing a deferred write inside an |\edef|.
So |\protect\foo| will write |\foo| followed by a space. If you want
|\protect\foo| to be written, you should use |\@unexpandable@protect|.
(This was Removed as it was never used).

\end{enumerate}

\latex provides the command \cmd{DeclareRobustCommand} to manage robust definitions.

\begin{teX}
\DeclareRobustCommand{\seq}[2][n]{%
\ifmmode
#1_{1}\ldots#1_{#2}%
\else
\PackageWarning{fred}{You can't use \protect\seq\space in text}%
\fi
}




So how does \latex does it? For this we need to look under |ltdefns.dtx|

\begin{teX}
 \def\@unexpandable@protect{\noexpand\protect\noexpand}
\end{teX}


\cmd{DeclareRobustCommand} is a package writers command, which has the same syntax as \cmd{newcommand}, but which declares a protected command.

It does this by having |\DeclareRobustCommand\foo|
define |\foo| to be |\protect\foo<space>|,
and then use |\newcommand\foo<space>|.

Since the internal command is |\foo<space>|, when it is written to an auxiliary
file, it will appear as |\foo|.
We have to be a bit cleverer if we're defining a short command, such as |\_|,
in order to make sure that the auxiliary file does not include a space after the
command, since |\_| a and |\_a| aren't the same. In this case we define |\_ |to be:

\begin{teX}
|\x@protect\_\protect\_<space>|
\end{teX}

which expands to:

\begin{teX}
\ifx\protect\@typeset@protect\else
\@x@protect@\_
\fi
\protect\_<space>
\end{teX}

Then if |\protect| is |\@typeset@protect| (normally |\relax|) then we just perform
|\_<space>|, and otherwise |\@x@protect@| gobbles everything up and expands to |\protect\_|


During typesetting we let the |@typeset@protect| to |\relax|

\begin{teX}
\let\@typeset@protect\relax
\end{teX}

%\def\strip@prefix#1>{}

\makeatletter
\def\test{A Test}
\let\tesT\test


Before we review the rest of the code let us make a small detour around the dangerous bends to re-examine the \emph{meaning} of \cmd{meaning}. This command (which is a \tex primitive), will return the definition of a macro.

\begin{teX}
\meaning\maketitlepage
\end{teX}

will produce

{
\tt

\meaning\maketitlepage


}


We can use the \latex kernel command |\strip@prefix|to strip the prefix of the macro


\begin{teX}
\expandafter\strip@prefix\meaning\maketitlepage
\end{teX}



As you can see the |\long macro:->| has now been removed. We will use this to use our own definition of a Macro similar to \latex first. What we will do we will define

\begin{teX}
\def\NewMacro#1{}  %\def\DeclareRobustCommand{}
\end{teX}

Our \cmd{NewMacro} will simply define a command to be |\edef|. Once we understand some of the concepts we will then move on to discuss the way \latex defines a |DeclareRobustCommand|.

What we need to capture is first the new command name and then the contents within the brackets. For example we may need to define a command

\begin{teX}
\NewMacro\test{This is test}
\end{teX}

We both need to capture the |test| as well as the |{This is a Test}|. We do this so that we can then |\edef\test{This is a test}|. Let us see one way.


\makeatletter

\def\@atest{\textsc{$\alpha$, Gamma}}

\@atest

\edef\TesT{\expandafter\strip@prefix\meaning\@atest}%

\@atest

\TesT

and the meaning of TesT is  {\tt \meaning\TesT}

What we have managed so far is to get the definition of a macro and place it in another macro. This is \tex magic! Whatremains now is to get the arguments (including the name of the macro and place it in front of the |edef|. Then we can have a macro that can be used to define other macros!

\makeatother

For example, if |\seq| is defined as follows:


\DeclareRobustCommand{\seq}[2][n]{%
\ifmmode
#1_{1}\ldots#1_{#2}%
\else
\PackageWarning{fred}{You can't use \protect\seq\space in text}%
\fi
}

$\seq{n1,n2}$

\seq{n1,n2}


You must admit that following what \latex does, involves mental acrobatics. With time though things start to fall into place!
\clearpage

%\cxset{chapter color=teal}
\chapter{Kernel Lengths}
\label{kernel:lengths}
\index{LaTeX kernel classes!File n  ltlength.dtx}
\section{File n, lengths and the ltlength.dtx}

This class defines a number of user commands for manipulating lengths. the code is straightforward. The |\newlength| command allocates a new internal skip register using the |\newskip| command from the allocations class.

\let\bs\textbackslash
\index{\bs newlength}\index{\bs setlength}\index{\bs addtolength}\index{\bs settowidth}\index{\bs settoheight}
\index{\bs settodepth}
\medskip
\begin{tabular}{ll}
\verb+\newlength+  &  Declare \#1 to be a new length command.\\
\verb+\setlength+    &  Set the length command, \#1, to the value \#2.\\
|\addtolength| & Increase the value of the length command, \#1, by the value \#2.\\
|\settowidth|   & Set the length, \#1 to the width of a box containing \#2. \\
|\settoheight|  & Set the length, \#1 to the height of a box containing \#2.\\
|\settodepth|   & Set the length, \#1 to the depth of a box containing \#2.\\
|\@settodim|   & internal macro\\
|\@settopoint| & internal macro\\
\end{tabular}
\medskip

\begin{teX}
3 \def\newlength#1{\@ifdefinable#1{\newskip#1}}
4 \def\setlength#1#2{#1#2\relax}
5 \def\addtolength#1#2{\advance#1 #2\relax}
\end{teX}
\medskip

The |setto| commands use a temporary box to store the contents and then measure them using the internal macro |\@settodim|,

\medskip
\begin{teX}
6 \def\@settodim#1#2#3{\setbox\@tempboxa\hbox{{#3}}#2#1\@tempboxa
%  Clear the memory afterwards (which might be a lot).
7       \setbox\@tempboxa\box\voidb@x}
8 \def\settoheight{\@settodim\ht}
9 \def\settodepth {\@settodim\dp}
10 \def\settowidth {\@settodim\wd}
\end{teX}
\medskip

\DescribeMacro{\@settopoint}\marg{skip register}
The |\@settopoint| macro takes the contents of the skip register that is supplied as its argument
and removes the fractional part to make it a whole number of points. This can be
used in class files to avoid values like 45.455pt when calulating a dimension. The method of
rounding is interesting. Also it is interesting that this macro, is not used in the kernel at all, but is defined
here for use with the standard classes (it is used to round off dimensions for page calculations).

\medskip
\begin{texexample}{settopoint}{}
  \makeatletter
  \def\@settopoint#1{\divide#1\p@\multiply#1\p@}
  \newlength\@test
  \setlength\@test{19.5pt}
  \@settopoint{\@test}
  \the\@test
  \makeatother
\end{texexample}




\ref{kernel:lengths}


\makeatletter

\section{File a ltdirchk.dtx}
This file implements the semi-automatic determination of various system dependent
parts of the initialisation. The actual definitions may be placed in a file
|texsys.cfg|. Thus for operating systems for which the tests here do not result in
acceptable settings, a `hand written' texsys.cfg may be produced.
Current directoty |\@currdir| 

|\input@path| For most common operating sytsems is let to undefined. % undefined on window

The routines define a useful macro to parse file name paths:

\filename@parse{./test/some other paths/path/tex.jpg}

\filename@area

\filename@base,

\filename@ext

The |\@TeXversion| is only defined for very old versions of |TeX|, on a reasonable moern distribution, it should be let to undefined.



\section{File k ltfiles.dtx}


\begin{tabular}{lp{5cm}}
|\document| &\\
|\nofiles| &\\
|\includeonly| &\\
|\include| &\\
|\input| &\\
|ifFileExists| &\\
|\InputIfFileExists| & If the file exists on a system, execute then input the name, otherwise execute \textit{else}\\
\end{tabular}

%\tt
%\meaning\@empty
%
%\meaning\frenchspacing
%
%\meaning\nonfrenchspacing

\begin{multicols}{2}
\Paragraph{Listing files}  A list of files so far. The initial value of |@gobble| eats the comma before the first file name. Here we start encountering \LaTeX's iteration macros:

The |\@filelist| is a comma delimited list that will hold the value of all files. It is let to |\@gobble| in order to eat the first comma. This is a nifty trick. \indexat{@filelist}\indexat{@addtofilelist}.

\begin{Code}
204 \let\@filelist\@gobble
\end{Code}

The next macro adds a file name to the list. If you are not familiar with lists this is an interestin way of understanding, how a comma delimited list is build.

\begin{Code}
205 %\def\@addtofilelist#1{\xdef\@filelist{\@filelist,#1}}
\end{Code}

Note that the |\@filelist| gets deactivated if |\listfiles| does not appear in the preamble. The |begin{document}|
contains code equivalent to:

\begin{Code}
\AtBeginDocument{%
\ifx\@listfiles\@undefined
\let\@filelist\relax
\let\@addtofilelist\@gobble
\fi}
229 \@onlypreamble\listfiles
230 \let\@dofilelist\relax
\end{Code} 

\indexat{onlypreamble}

\end{multicols}

\section{\texttt{File c ltvers.dtx}}

\begin{multicols}{2}
\Paragraph{Version name and version date.}
This is a small class, that its sole purpose is to provide version information. It is also supposed to check
if the format is too old. If it is older than 65 months it emits an error. 

\begin{Code}
2 \def\fmtname{LaTeX2e}
3 \edef\fmtversion{2011/06/27}
\end{Code}

After the format version is hardcoded, a macro using one of LaTeX's scatch names is defined. The parameters
of this macro are delimited in the same way as the date in the format version definition, thus a comparison can be made between the two periods. If it is longer than a preset period it emits an error.
\begin{Code}
4 \iffalse
5 \def\reserved@a#1/#2/#3\@nil{%
6 \count@\year
7 \advance\count@-#1\relax
8 \multiply\count@ by 12\relax
9 \advance\count@\month
10 \advance\count@-#2\relax}
11 \expandafter\reserved@a\fmtversion\@nil
12 \ifnum\count@>65
13 \typeout{^^J%
14 !!!!!!!!!!!!!!!!!!!!!!!!!!!!!!!!!!!!!!!!!!!!!!!^^J%
15 ! You are attempting to make a LaTeX format from a source file^^J%
16 ! That is more than five years old.^^J%
17 !^^J%
18 ! If you enter <return> to scroll past this message then the format^^J%
19 ! will be built, but please consider obtaining newer source files^^J%
20 ! before continuing to build LaTeX.^^J%
21 !!!!!!!!!!!!!!!!!!!!!!!!!!!!!!!!!!!!!!!!!!!!!!!!^^J%
22 }
23 \errhelp{To avoid this error message, obtain new LaTeX sources.}
24 \errmessage{LaTeX source files more than 5 years old!}
25 \fi
26 \let\reserved@a\relax
27 \fi
\end{Code}

It does not appear that this macro is currently activated, but I can be wrong. One cannot help but notice the conservatism in saving memory by the use of nameless scratch macros. Good practice dictates, that after their use they are let to |\relax|.

\fmtname

\fmtversion
\end{multicols}

\section{File d ltplain.dtx}

The routines covered by this class go deep into the heart of the kernel. Besides storing some of the plain commands in new macros, this section of the kernel defines its own defining commands like |\newcommand|, |\newenvironment| and similar other macros. We start by summarizing the author and internal commands available in this section.




\section{File  Bibliographies ltbibl.dtx} % Check on the file name

A bibliography is created by the \cmd{thebibliography} environment, which generates
a title such as "References", and a list of entries. The \texttt{BIBTEX} program will create
a file containing such an environment, which will be read in by the \cmd{bibliography}
command. With BIBTEX, the following commands will be used


\cmd{bibliography} This commands reads in all the filenames of the bibliography. |\bibliography{file1,file2,file3,file4}|
It will then write a \cmd{bibdata} entry on the |.aux| file and tries to read in |mainfile.bbl|.


\begin{teX}
\def\bibliography#1{%
 \if@filesw
 \immediate\write\@auxout{\string\bibdata{#1}}%
 \fi
 \@input@{\jobname.bbl}
}
\end{teX}

The \cmd{if@filesw} is defined in File k (ltfiles.dtx) as follows:
\begin{teX}
\newif\if@filesw \@fileswtrue
\end{teX}

The |bibliograhy| command simply writes the |bibdata|. Notice that |\string| is used to make sure  there is no expansion. \cmd{@input@} is a version of \cmd{@input} that does add the file to \cmd{@filelist}. This is also defined in |ltfiles.dtx|. 

\begin{teX}
\def\@input@#1{\InputIfFileExists{#1}{}{\typeout{No file #1.}}}
\end{teX}


\noindent It simply checks if the file exists and outputs to 

\startLineAt{32}
\begin{teX}
 \def\bibliographystyle#1{%
 \ifx\@begindocumenthook\@undefined\else
   \expandafter\AtBeginDocument
 \fi
 {\if@filesw
  \immediate\write\@auxout{\string\bibstyle{#1}}%
 \fi}}
\end{teX}



\makeatletter

|\if@filesw|~~~   \if@filesw

|jobname|          \jobname



The bibliography environment is a list environment. Instead of using |\item|, it uses \cmd{bibitem}. The rest of the commands are \cmd{cite} and \cmd{nocite}.

The \cmd{nocite}, puts information on the |.aux| file that causes |\bibtex| to include a citation list in the bibliography, but nothing in the document.\cmd{nocite\{*\}} is special it tells |\bibtex| to put the whole of a collection of citation with comment.
references into the bibiography.

Most of the Bibliography formatting comes later in the actual classes.


\cmd{if@tempswa} General boolean switch used by LATEX kernel commands.
defines as |\newif\if@tempswa|

The \cmd{@cite} hook determines the \textit{relative formatting} of the two logical parts of a citation

\begin{teX}
\def\@cite#1#2{[{#1\if@tempswa , #2\fi}]}
\end{teX}




\makeatother

Not the easiest of read and the whole think should have a rethink for better modularity.





\section{parboxes and other boxed things!}
\label{parbox}\index{parbox}\index{mode!paragraph}
A \cmd{parbox} is a box whose contents are created in paragraph mode. The |\parbox| has two mandatory arguments:

\begin{teX}
\parbox[position][height][inner-pos]{width}{text}
\end{teX}

\noindent For example,

\begin{teX}
\parbox{7cm}{\onepar}
\end{teX}

\noindent\parbox{7cm}{\onepar}

\bigskip

\begin{description}
\item[width]  specifies the width of the parbox, and
\item[text]   the text that goes inside the parbox.
\end{description}


\latex will position a parbox so its centre lines up with the centre of the text line. The optional position argument allows you to line up either the top or bottom line in the parbox (default is top).

If the height argument is not given, the box will have the natural height of the text.

The [inner-pos] argument controls the placement of the text inside the box. If it is not specified, position is used.

\begin{enumerate}
\item[t]  text is placed at the top of the box.
\item[c]  text is centred in the box.
\item[b]  text is placed at the bottom of the box.
\item[s]  stretch vertically. The text must contain vertically stretchable space for this to work.
\end{enumerate}

A parbox command is used for a parbox containing a small piece of text, with nothing fancy inside. In particular, you shouldn't use any of the paragraph-making environments inside a parbox argument. For larger pieces of text, including ones containing a paragraph-making environment, you should use a \docenv{minipage} environment 

\begin{teX}
\def\parbox{%
  \@ifnextchar[%]
   \@iparbox
   {\@iiiparbox c\relax[s]}}
\end{teX}

The definition of the user command |\parbox| is quite simple and follows the typical pattern found with \latex commands. The macro, uses the |@ifnextcharacter|, to check for a left square bracket ([), to check if an optional argument has been supplied or not. If one was supplied |@iiiparbox| is called else
|\@iiparbox| is called. This is typical of many \latex commands, where optional arguments are supplied. Four macros are used to provide a user command. The command, in this case |\parbox| and |@iparbox|, |@iiparbox| and |@iiiparbox|. The real work (in the case of |\parbox| is undertaken by the internal command |@iiiparbox|. It takes some time to get used to these patterns and my suggestion is for you to try a few macros for practice, plus read this chapter a couple of times trying some of the examples.

The \indexat{iparbox} just handles the case again of optional arguments,

\begin{teX}
\def\@iparbox[#1]{%
   \@ifnextchar[%]
   {\@iiparbox{#1}}%
   {\@iiiparbox{#1}\relax[s]}}
\end{teX}


  

\begin{teX}
\def\@iiparbox#1[#2]{%
   \@ifnextchar[%]
   {\@iiiparbox{#1}{#2}}%
   {\@iiiparbox{#1}{#2}[#1]}}

% The internal part of parbox
% \@iiiparbox
% \@parboxto
% The internal version of \parbox.
\end{teX}

\begin{teX}
   \let\@parboxto\@empty
\end{teX}

\hspace{-1cm}\texttt{\textbackslash @iiiparbox}

The internal version of \texttt{\textbackslash parbox}. The parameter text is as follows:

\begin{verbatim}
#1 position t, b (default is top)
#2 height
#3 inner position t, c, b, c
#4 width
#5 text
\end{verbatim}

\begin{teX}
  \long\def\@iiiparbox#1#2[#3]#4#5{%
    \leavevmode
    \@pboxswfalse
    \setlength\@tempdima{#4}%
    \@begin@tempboxa\vbox{\hsize\@tempdima\@parboxrestore#5\@@par}%
     %check height 
     \ifx\relax#2\else %empty
        \setlength\@tempdimb{#2}%set to height
        \edef\@parboxto{to\the\@tempdimb}%
        \fi
      \if#1b\vbox
        \else\if #1t\vtop
          \else\ifmmode\vcenter
            \else\@pboxswtrue $\vcenter
      \fi\fi\fi
      %set parbox (originally defined as empty) 
      \@parboxto{\let\hss\vss\let\unhbox\unvbox
        \csname bm@#3\endcsname}%
      \if@pboxsw \m@th$\fi
    \@end@tempboxa}
\end{teX}

\noindent The \cmd{bm@} takes various values and is used for spacing. 
|\bm@l|, |\bm@r|, |\bm@s|, |\bm@t|, |\bm@b|


\textbf{Set up spacing}

\begin{teXX}
21 \def\bm@c{\hss\unhbox\@tempboxa\hss}
22 \def\bm@l{\unhbox\@tempboxa\hss}\let\bm@t\bm@l
23 \def\bm@r{\hss\unhbox\@tempboxa}\let\bm@b\bm@r
24 \def\bm@s{\unhbox\@tempboxa}
\end{teXX}


\noindent If you are wondering what |\@pboxswfalse| and |\@pboxswtrue| you can checkit out by using \cmd{meaning}. There is also the package \docpkg{reflex} which you can use.


\makeatletter

\parindent0pt

\def\reflect{\@star@or@long\accommand}
\fboxrule=0.0pt

\long\def\accommand#1{\framebox[4cm][l]{%
     {\tt\string#1 \hfill:}} %
     \parbox[t]{7cm}{\tt\expandafter\strip@prefix\meaning#1}} 


%% Must change to minipage
\def\ccommand#1{\parbox[t]{4cm}{\string#1}\parbox[t]{7cm}{#1}}


\reflect{\bm@c}

\reflect{\@pboxswtrue}

\reflect{\@pboxswtrue}

\reflect{\leavevmode}


\meaning\@@par

\meaning\hss

\meaning\ifmmode

\meaning\vbox

\meaning\voidb@x

\def\texprim{\texttt{\protect\TeX\ primitive}}


\def\CMhss{Glue that is infinitely stretchable as well as infinitely strechable \texprim}

\def\CMi{The \BS i command is valid in math mode and text mode. It generates an i without dot (Unicode U+131, \i). The \BS I command expands to I, but is converted to \BS i by \BS MakeLowercase. \texprim}


\ccommand{\CMi}

\reflect{\CMhss}

\reflect{\CMi}

\CMi

\if{\meaning\par}{\meaning\par}{True} \else {false}

\ifx{\par\meaning}{\meaning\par}{True}\else{false}


\makeatother


To complete the code the \indexat{@arrayparboxrestore} is defined. This  restores various paragraph parameters.
The rational for allowing two normally global 
flags to be set locally here was
stated originally by Donald Arsenau and extended by Chris Rowley. It is because these 
flags are only set globally to true by section commands, and these should
never appear within boxes or, indeed, in any group; and they are only ever set globally to false when they are definitely true.
File B: ltboxes.dtx Date: 2006/05/18 Version v1.1g 237
If anyone is unhappy with this argument then both 
flags should be treated as in |\set@nobreak|; otherwise this command will be redundant.

\begin{teXX}
176 \def\@arrayparboxrestore{%
177 \let\if@nobreak\iffalse
178 \let\if@noskipsec\iffalse
179 \let\par\@@par
180 \let\-\@dischyph
\end{teXX}

Redefined accents to allow changes in font encoding
\begin{teXX}
181 \let\'\@acci\let\`\@accii\let\=\@acciii
182 \parindent\z@ \parskip\z@skip
183 \everypar{}%
184 \linewidth\hsize
185 \@totalleftmargin\z@
186 \leftskip\z@skip \rightskip\z@skip \@rightskip\z@skip
187 \parfillskip\@flushglue \lineskip\normallineskip
188 \baselineskip\normalbaselineskip
189 \sloppy}
\end{teXX}

Finally the \cmd{parboxrestore} restores various paragraph parameters, and also |\\|.

\begin{teXX}
190 \def\@parboxrestore{\@arrayparboxrestore\let\\\@normalcr}
\end{teXX}


\makeatletter
{\obeylines
\meaning\if@nobreak
\meaning\if@noskipsec
\meaning\@dischyph
\meaning\discretionary
}
\makeatother

The code between \indexat{\@begin@tempboxa} are helper macros for supporting \cmd{height}, \cmd{width} etc. It grabs \#1 into \indexat{@tempboxa} and measures it. It also allows for macros involved with the coloring of the box.


This was a long write and possibly a long read. As \latex was written a long time ago. One could improve the user interface of the command by
defining the |\parbox| with the |keyval| package. Will it be more intuitive to describe

\begin{teXX}

\parboX[width=7cm, 
        height=8cm,
        textposition=t, 
        outerposition=b]{teX}
\end{teXX}

%\makeatletter
%\def\ac{test}
%\def\Source{this,is, a, short, \string\ac }
% The string \emph{\Source} contains the following tokens:\\
% \whiledo{\not\equal{\Source}{}}
% {
%     \GetTokens{TokenOne}{TokenTwo}{\Source}
%     \def\tempa{\TokenOne}
%     \texttt{\meaning\ac}\\
%     \let\Source\TokenTwo
% }


\section*{Strip the backslash}

\tex's possibilities are almost infinite. Here is another example of
an argument that is thrown away\cite{amy1990}, that just strips the backslash:

\def\stripbackslash#1#2*{\def\one{#2}}

which only uses the second argument, throwing away the
first argument, in this case stripping away a backslash
from a control sequence supplied by the user. \cmd{stripbackslash}
can then be used in another macro which
needs a control sequence without its backslash to work
correctly, for instance:

\def\newdef#1{\expandafter
\stripbackslash\string#1* \one}



\newdef\testmacro

produces

\newdef\testmacro


Instead of simply printing the control sequence without
the backslash, |\newdef| can be rewritten to test to see
if a given macro has already been defined. In this example,
|\newdef| tests to see if the expansion of the control
sequence |\csname\one\endcsname|, (where |\one|,
was defined by |\stripbackslash| to be the control sequence
supplied by the user minus its backslash) is equal
to |\relax|. This takes advantage of the TEX convention
that a previously undefined control sequence invoked in
a |\csname...\endcsname| environment will be understood
to be equal to |\relax|, whereas an already defined
control sequence will not:


\def\newdef#1{%
\expandafter\stripbackslash\string#1*
%% \stripbackslash defines \one
\expandafter
\ifx\csname\one\endcsname\relax
%% \one is expanded to be the
%% control sequence the user supplied
%% minus the backslash.
%% If csname construction equals
%% \relax, do nothing
\else %% Else, give error message:
{\tt Sorry, \string#1 has already been
defined. Please supply a new name.}
\fi}

In the test below, notice that we do not get an error message
for |\cactus| which hasn’t been previously defined,
but we do get a message for \tex, which is defined:

\newdef\TeX
\newdef\cactus

\def\newdef#1{%
\expandafter\stripbackslash\string#1*
%% \stripbackslash defines \one
\expandafter
\ifx\csname\one\endcsname\relax
%% \one is expanded to be the
%% control sequence the user supplied
%% minus the backslash.
%% If csname construction equals
%% \relax, do nothing
\parbox{3.8cm}{\texttt{\textbackslash \one \hfill : }}\parbox{5cm}{~~\texttt{Not defined}}
\else %% Else, give error message:
{\reflect{#1}}
\fi}

\newdef\cactus

\newdef\TeX

\newdef\org@TeX

\newdef\newdimen

\newdef\et@xglob

\newdef\@preamblecmds

\newdef\ij

\newdef\textbaht

\makeatother



\chapter{\LaTeX counters}
\clearpage

\thispagestyle{plain}
{\centering\includegraphics[width=\textwidth]{./graphics/moneylenders}\par}

{\centering \onelineheader{THE LATEX COUNTERS}\par}

\begin{multicols}{2}
The |File:m ltcounts.dtx| provides the command sequences defined by \latex to use with counters. It is a fairly short file with approximately 
60 lines of code. It is a good file to study in order to polish your skills in programming \tex. 

The heart of the counter commands are the \indexat{definecounter}

\index{LaTeX counters}
\index{LaTeX counters!\textbackslash setcounter}\index{LaTeX counters! \textbackslash addtocounter}\index{LaTeX counters! \textbackslash value}
The class starts with some definitions for |\setcounter|, |\addtocounter|, |\newcounter|, |\value|:


\begin{Code}
2 \def\setcounter#1#2{%
3   \@ifundefined{c@#1}%
4      {\@nocounterr{#1}}%
5      {\global\csname c@#1\endcsname#2\relax}}
\end{Code}


Testing for something here

Notice in line [5] that the name of the counter is prefixed with |c@|. This is automatically done for newcounter commands which we explain later on.


\index{LaTeX counters!\textbackslash newcounter}
The |\newcounter{newctr}[oldctr]| macro  Defines |newctr| to be a counter, which is
reset when counter |oldctr|  is stepped. If |newctr| is  already defined produces
|c@newctr already| defined  error.

\begin{Code}
10 \def\newcounter#1{%
11 \expandafter\@ifdefinable \csname c@#1\endcsname
12 {\@definecounter{#1}}%
13 \@ifnextchar[{\@newctr{#1}}{}}
\end{Code}

The code checks to see, if an optional value is provided, using |\@ifnextchar| and branches to either
|\@definecounter| or to |\@newctr|. The two macros follow:



\begin{Code}
25 \def\@definecounter#1{\expandafter\newcount\csname c@#1\endcsname
26 \setcounter{#1}\z@
27 \global\expandafter\let\csname cl@#1\endcsname\@empty
28 \@addtoreset{#1}{@ckpt}%
29 \global\expandafter\let\csname p@#1\endcsname\@empty
30 \expandafter
31 \gdef\csname the#1\expandafter\endcsname\expandafter
32 {\expandafter\@arabic\csname c@#1\endcsname}}
\end{Code}

\Paragraph{\textbackslash @definecounter}\par
The command does a lot of work. Firstly, it defines a new counter using the \tex primitive \cmd{newcount} in line [25]. It then sets the counter using \cmd{setcounter} to zero (|\z@|). Lastly line [31], defines the counter as |thefoo|. This is an internal kernel command that provides the routines and definition of the counters to the rest of the macros.

\begin{Code}
15 \def\@newctr#1[#2]{%
16 \@ifundefined{c@#2}{\@nocounterr{#2}}{\@addtoreset{#1}{#2}}}
\end{Code}


Next follow a number of commands, for representing the values of counters in different forms.

\Paragraph{\textbackslash arabic} Representation of counter as arabic numerals. Changed 29 Apr 86 to make it
print the obvious thing if COUNTER is not positive.

\begin{Code}
34 \def\arabic#1{\expandafter\@arabic\csname c@#1\endcsname}
\end{Code}

\Paragraph{roman, Roman, alph, Alph.}The rest of the number definitions, follow in the same manner.  All of the commands, have internal macro representations. Of interest is the way |\Roman| is defined. The explanation follows after this block.

\begin{Code}
\roman Representation of hcounteri as lower-case Roman numerals.
35 \def\roman#1{\expandafter\@roman\csname c@#1\endcsname}
\Roman Representation of hcounteri as upper-case Roman numerals.
36 \def\Roman#1{\expandafter\@Roman\csname c@#1\endcsname}
\alph Representation of hcounteri as a lower-case letter: 1 = a, 2 = b, etc.
37 \def\alph#1{\expandafter\@alph\csname c@#1\endcsname}
\Alph Representation of hcounteri as an upper-case letter: 1 = A, 2 = B, etc.
38 \def\Alph#1{\expandafter\@Alph\csname c@#1\endcsname}
\end{Code}

The internal representation, is straightforward:

\begin{Code}
\@arabic \@arabic\FOOcounter Representation of \FOOcounter as arabic numerals.
40 \def\@arabic#1{\number #1} %% changed 29 Apr 86
\@roman \@roman\FOOcounter Representation of \FOOcounter as lower-case Roman numerals.
41 \def\@roman#1{\romannumeral #1}
\@Roman \@Roman\FOOcounter Representation of \FOOcounter as upper-case Roman numerals.
42 \def\@Roman#1{\expandafter\@slowromancap\romannumeral #1@}
\end{Code}



\begin{Code}
\@slowromancap Fully expandable macro to change a roman number to uppercase.
43 \def\@slowromancap#1{\ifx @#1% then terminate
44 \else
45 \if i#1I\else\if v#1V\else\if x#1X\else\if l#1L\else\if
46 c#1C\else\if d#1D\else \if m#1M\else#1\fi\fi\fi\fi\fi\fi\fi
47 \expandafter\@slowromancap
48 \fi
49 }
\end{Code}



\makeatletter
\begin{teX}
\@slowromancap iiiv@ 
\end{teX}



\def\@Slowroman#1{\ifx @#1% then terminate
 \else
   \@fnsymbol#1 \texttt{\textbackslash @fnsymbol\{#1\}\\ } \expandafter\@Slowroman
 \fi
 }


\noindent \@Slowroman 123456789 @




\section{Footnote symbols}

\LaTeXe provides the \indexat{fnsymbol} command that holds the symbols for old fashioned footnote symbols. These can be used both in text or math mode. The definition is shown below.

\reflect{\@fnsymbol}
\bigskip


The symbols are as follows:

\noindent \@Slowroman 123456789 @


As it happens the \indexat{fnsymbol} is one of those commands that have been revised in \texttt{fixltx2e} \sidenote{see \url{http://www.tex.ac.uk/tex-archive/macros/latex/unpacked/fixltx2e.sty}} and hence the definition shown is that of the \texttt{fixltx2e}.

\begin{teXX}
\ProvidesPackage{fixltx2e}

\def\@fnsymbol#1{%
   \ifcase#1\or \TextOrMath\textasteriskcentered *\or
   \TextOrMath \textdagger \dagger\or
   \TextOrMath \textdaggerdbl \ddagger \or
   \TextOrMath \textsection  \mathsection\or
   \TextOrMath \textparagraph \mathparagraph\or
   \TextOrMath \textbardbl \|\or
   \TextOrMath {\textasteriskcentered\textasteriskcentered}{**}\or
   \TextOrMath {\textdagger\textdagger}{\dagger\dagger}\or
   \TextOrMath {\textdaggerdbl\textdaggerdbl}{\ddagger\ddagger}\else
   \@ctrerr \fi
}

\end{teXX}

The old definition is shown below:

\begin{teXX}
58 \def\@fnsymbol#1{\ensuremath{\ifcase#1\or *\or \dagger\or \ddagger\or
59 \mathsection\or \mathparagraph\or \|\or **\or \dagger\dagger
60 \or \ddagger\ddagger \else\@ctrerr\fi}}
\end{teXX}

All nice and wonderfully, imaginative stuff that brings the discussion of the |ltcounts.dtx| to an end. This is one of the smaller classes, but important as counters are the heart of \latex. 



\indexat{stpelt}\index{LaTeX counters!\textbackslash stpelt}

This command takes one argument, a counter name and sets it to zero.

\begin{Code}
\@stpelt
23 \def\@stpelt#1{\global\csname c@#1\endcsname \z@}
\end{Code}


This can be used creatively in packages as for example the \docpkg{chappg}\sidenote{numbers are from the chappg documentation} \sidenote{see \url{http://www.tug.org/texlive/Contents/live/texmf-dist/doc/latex/chappg/chappg.pdf}}

\begin{Code}
The next magic makes the page counter be reset to one rather than zero
93 \renewcommand\@stpelt[1]{%
94 \global\csname c@#1\endcsname
95 \expandafter\ifx \csname c@#1\endcsname \c@page
96 \@ne
97 \else
98 \z@
99 \fi
100 }
\end{Code}



\texttt{\textbackslash cl@@ckpt}
\indexat{cl@@ckpt} is the  reset list of a dummy counter \indexat{ckpt}
used for taking checkpoints for the \cmd{include}\index{\textbackslash include} system.
\end{multicols}

\makeatletter
\def\@elt{,   }


%% print what is available
\topline

{\footnotesize 
\cl@@ckpt 
}

\bottomline


\begin{teXX}
\cl@@ckpt
24 \def\cl@@ckpt{\@elt{page}}
\end{teXX}

\sidenote{See also \url{http://www.tug.org/TUGboat/Articles/tb18-4/tb57work.pdf}}

The word \textit{elt}\index{elt} is short for element.

%This function returns the element of sequence indexed by index. Legitimate values of index are integers ranging from 0 up to one less than the length of sequence. If sequence is a list, then out-of-range values of index return nil; otherwise, they trigger an args-out-of-range error.
%
%\begin{verbatim} 	
%(elt [1 2 3 4] 2)
%=> 3
%\end{verbatim}
%\printindex 
%\end{document}
%
%\def\test#1{\def\res{#1}\ifx\foo\res True\\ \else Error \\ \fi}
%\edef\foo{\@car 123\@nil} \test{1}
%\edef\foo{\@car {1}23\@nil} \test{1}
%\edef\foo{\@car {123}{456}{7}\@nil} \test{123}
%\edef\foo{\@carcube1234567\@nil}\test{123}
%\edef\foo{\@cdr 123\@nil} \test{23}
%\edef\foo{\@cdr {134}{x}\@nil}   \test{x}
%\edef\foo{\@cdr {134}{{x}}\@nil} \test{{x}}
%
%\let\foo\@nnil \test{\@empty}
%
%
%
%\toks@={abc\foo}\addto@hook\toks@{x\bar}
%\expandafter\def\expandafter\foo\expandafter{\the\toks@} \test{abc\foo x\bar}
%\g@addto@macro\foo{y\gee} \test{abc\foo x\bar y\gee}
%\def\xx{456}
%\def\foo{123} \@cons\foo{\xx78}\test{123\@elt45678}
%
%
%
%http://www.tug.org/TUGboat/Articles/tb15-4/tb45braa.pdf
%
%
%

\makeatother



%{ltmiscen.dtx}{flushleft}
%/{ltmiscen.dtx}{flushright}
%{ltmiscen.dtx}{center}

\section*{Introspection}
\index{Introspection}
In everyday life, introspection is the act of self-examination. Introspection refers to the examination of one's own thoughts, feelings, motivations, and actions. The great philosopher Socrates spent much of his life in self-examination, encouraging his fellow Athenians to do the same. He even claimed that, for him, "the unexamined life is not worth living." We can use some \tex trickery and \latex acrobatics to print the listing of a macro. We will call this macro \cmd{reflect}.\index{\textbackslash meaning}

\begin{teX}
\def\reflect{\@star@or@long\accommand}
\def\accommand#1{\string#1:%
  \expandafter\strip@prefix\meaning#1} 
\end{teX}

\noindent The definition appears deceptively simple (by having removed some cosmetic additions to format the output). Let us see as an example the output of |\reflect{\frogking}|.
\bigskip


\hfil\hfill Results of |\reflect{frogking}| 

\smallskip


\hrule
\medskip
\makeatletter
\def\showcommand{\@star@or@long\accommand}
\fboxrule=0.0pt
\def\accommand#1{\framebox[3cm][l]{\bf\color{red} 
    \string#1:~~} %
     \parbox[t]{7cm}{\expandafter\strip@prefix\meaning#1}} 
\showcommand*{\frogking}
\makeatother
\smallskip



\bigskip


The code captures both the name of the macro (strips the |macro:->| part and displays the definition). I did include some formatting commands to get it to display better.

If you have difficulty in understanding the code the best advice I can give you is to stop driving and take public transport. This applies to hackers as well as writers, but it may not be applied fully to mathematicians and physicists. If a typical commute is 30 minutes by car or 1 hour by public transportation, you may feel like you are losing an hour a day leaving your car at home, but you are actually gaining an hour of time if you bring your laptop, assuming you want to spend at least 2 hours a day working on things on your laptop outside work. Other upsides: cheaper, less stress, safer, better for the environment and you will get closer to understanding \tex.


\section*{Lisp relics in \protect\LaTeX\ }
\index{Lisp}
\newthought{Introduced in the Lisp programming language}, |car| and |cdr| are primitive operations upon linked lists composed of |cons cells| (or "non-atomic S-expressions"). A |cons cell| is composed of two pointers; the |car| operation extracts the first pointer, and the |cdr| operation extracts the second.
\index{S-expressions}
Thus, the expression |(car (cons x y))| evaluates to |x|, and |(cdr (cons x y))| evaluates to |y|.

When |cons| cells are used to implement singly-linked lists (rather than trees and other more complicated structures), the |car| operation returns the first element of the list, while cdr returns the rest of the list. For this reason, the operations are sometimes given the names first and rest or head and tail.

What does this have to do with \latex? Obviously the \latex authors knew their computer science well, as you can find many instances where these concepts are used. They are easily definable using \tex and the definitions can be found in the |ltdefn.dtx| file.

\begin{teX}
\def\@car#1#2\@nil{#1}
\def\@cdr#1#2\@nil{#2}
\@carcube \@carcube T1 ... Tn\@nil = T1 T2 T3 , n > 3
\def\@carcube#1#2#3#4\@nil{#1#2#3}
\end{teX}





\begin{teX}
%\LaTeXe The LATEX2" logo as proposed by A-W designers.
\makeatletter
 \def\LaTeXe{%
   \mbox{\m@th%
    \if b\expandafter\@car\f@series\@nil\boldmath\fi
     \LaTeX\kern.15em2$_{\textstyle\varepsilon}$}}
\makeatother
\end{teX}

\scalebox{5}{\LaTeXe}

To understand the command we first need to check the definition of |\m@th|, which defines the mathsurround to be equal to |\z@|. |\def\m@th{\mathsurround\z@}| \index{\textbackslash mathsurround}. So the first thing we did was to make sure that the space around the maths is set to zero. The strange setting is the |@car|, which picks up the first letter of |\boldmath|?

\makeatletter
\def\Latex{%
   \mbox{\m@th%
    %\if b\expandafter\@car\f@series\@nil\boldmath\fi
    \boldmath
     \LaTeX\kern.15em2$_{\textstyle\varepsilon}$}}
 

\makeatother

\scalebox{5}{\Latex}


Lisp was originally implemented on the IBM 704 computer, in the late 1950s. The 704 hardware had special support for splitting a 36-bit machine word into four parts, an "address part" and "decrement part" of 15 bits each and a "prefix part" and "tag part" of three bits each.
Precursors to Lisp included functions:

\begin{description}
\item{|car|} (short for "Contents of the Address part of Register number"),\index{car}
\item{|cdr|} ("Contents of the Decrement part of Register number"),\index{cdr}
\item{|cpr|} ("Contents of the Prefix part of Register number"), and\index{cpr}
\item{|ctr|} ("Contents of the Tag part of Register number"),\index{ctr}
\end{description}
each of which took a machine address as an argument, loaded the corresponding word from memory, and extracted the appropriate bits.












