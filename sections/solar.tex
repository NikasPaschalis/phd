

\newcommand{\be}{\begin{equation}}
\newcommand{\ee}{\end{equation}}
\newcommand{\vv}[1]{\mathbf #1}						% 3-vector
\newcommand{\la}[1]{\label{#1}}
\newcommand{\Eq}[1]{Eq.~(\ref{#1})}
\newcommand{\Eqs}[2]{Eqs.~(\ref{#1},\ref{#2})}
\newcommand{\bea}{\begin{eqnarray}}
\newcommand{\eea}{\end{eqnarray}}
\newcommand{\Sec}[1]{Sec.~\ref{#1}}
\newcommand{\Secs}[2]{Secs.~\ref{#1}, \ref{#2}}

\newcommand{\Fig}[1]{Fig.~\ref{#1}}
\def\nn{\nonumber \\ }

\chapter{Solar Position}
\label{ch:solar}

In Chapter i18n, we primarily dealt with the necessary translation strings and we briefly touched on 
calendric calculations. In this Chapter we will describe equations that describe the position of the sun, as seen from observers on earth and develop some interesting typographical sidelines. The main object of the Chapter is to outline the usage of LuaTeX in the preparation of complex documents.

\section{Spherical coordinates for the sun}

\begin{figure} [tbh]
\begin{center}
	\includegraphics[width=0.37 \textwidth]{./images/solar/spherical.pdf}
\end{center}
\caption{\small In spherical coordinates, a three-dimensional vector $\vv r$ is expressed in terms of a radial distance $r$, a polar angle $\theta$, and an azimuthal angle $\phi$.  The vector {\boldmath $\rho$} is the projection of $\vv r$ onto the $x$-$y$ plane.  In terms of the rectangular coordinates, $x = r \sin \theta \cos \phi$, $y = r \sin \theta \sin \phi$, and $z = r \cos \theta$.\la{fig:spherical}}
\end{figure}
Any point in three-dimensional space can be defined by the spherical co-ordinates $(r, \theta, \phi)$, where $r$ is the distance, $\theta$ the polar angle, and $\phi$ is the azimuthal angle. 

In terms of the rectangular coordinates $(x, y, z)$, we have
\be
\vv r = \left( \begin{array}{c} x \\ y \\ z \end{array} \right) = \left( \begin{array}{c} r \sin \theta \cos \phi  \\ r \sin \theta \sin \phi \\ r \cos \theta \end{array} \right)~,
\la{eq:spherical}
\ee

The {\it celestial sphere} is an imaginary spherical surface, sharing a center with the Earth's globe, and with a very large, indefinite radius.  The positions of the stars, planets, and other heavenly bodies are characterized by their radial projection onto this surface.  The largeness of the radius of the celestial sphere, compared to the radius of the Earth, allows us, when convenient, to picture it as centered at the position of an observer standing on the Earth's surface, rather than at the center of the Earth.

For simplicity, we take the radius $r$ of the celestial sphere to be equal to 1 (in undetermined units).  We shall use a subscript $\odot$ (the astronomical symbol for the Sun) to indicate that a vector or a coordinate thereof refers to the position of the Sun.

\subsection{Ecliptic frame}
\la{sec:ecliptic}

From the Earth, the Sun appears to move, against the background of the distant stars, along a great circle on the celestial sphere called the {\it ecliptic}.\footnote{The ecliptic is sometimes defined as the plane of the Earth's orbit around the Sun.  The circle that we call the ``ecliptic'' is the intersection of that plane with the celestial sphere.}  We will therefore start by working in an ``ecliptic frame,'' in which the position of the distant stars is fixed,\footnote{For this reason the distant stars, which form the constellations, are also referred to as the ``fixed stars.''} and in which the polar angle of the Sun is always $\theta_\odot = \pi / 2$, whereas the azimuthal angle $\phi_\odot$ varies over the course of the year, as shown in \Fig{fig:ecliptic}.  If the Earth's orbit were perfectly circular, then $\phi_\odot$ would increase at a constant rate, completing a full revolution in a year.  In \Sec{sec:center} we will see how to account for the fact that the Earth's orbit is slightly elliptical, but for now we will simply express the azimuthal angle of the Sun as a function of the time $t$.  We therefore express the position of the Sun, in the ecliptic frame of reference, as:
\be
\vv r_\odot (t) = \left( \begin{array}{c} \cos \phi_\odot (t) \\ \sin \phi_\odot (t) \\ 0 \end{array} \right)~.
\la{eq:ecliptic}
\ee

\begin{figure} [tbh]
\begin{center}
	\includegraphics[width=0.47 \textwidth]{./images/solar/ecliptic.pdf}
\end{center}
\caption{\small The Sun moves along the ecliptic during the course of the year.  In the ecliptic frame of reference, the Sun's polar angle is fixed, $\theta_\odot = \pi / 2$, while the azimuthal angle $\phi_\odot$ increases with time at an approximately constant rate of $2 \pi$ per year.\la{fig:ecliptic}}
\end{figure}

\subsection{Equatorial frame}
\la{sec:equatorial}

The axis of rotation of the Earth is tilted with respect to the ecliptic frame by an angle of obliquity
\be
\varepsilon = 23.44^\circ = 0.4091 ~.
\la{eq:obliquity}
\ee
It is therefore convenient to change coordinates to an ``equatorial frame,'' by rotating about the $x$-axis by an angle $\varepsilon$, as shown in \Fig{fig:equatorial}(a), so that the new $z'$-axis coincides with the Earth's axis of rotation.  The motion of the Sun in this equatorial frame is illustrated in \Fig{fig:equatorial}(b), in which the celestial north pole is labelled $P$ and the celestial south pole $\overline P$.  The ecliptic intersects the celestial equator at two points, $e$ and $\bar e$, known as the {\it equinoxes}.  At $e$ the Sun crosses the equator from south to north, and this is therefore known as the northward equinox (or ``first equinox'', since it occurs first in the calendar year).  Conversely, $\bar e$ is known as the southward, or second equinox.  The points of maximum displacement between the position of the Sun and the celestial equator are known as {\it solstices}, and are marked in \Fig{fig:equatorial}(b) by $s$ and $\bar s$.

The position of Sun in the equatorial coordinate frame is given by:
\bea
\vv r'_\odot =
\left( \begin{array}{c} x'_\odot \\ y'_\odot \\ z'_\odot \end{array} \right) &=&
\left( \begin{array}{c} \sin \theta'_\odot \cos \phi'_\odot \\ \sin \theta'_\odot \sin \phi'_\odot \\ \cos \theta'_\odot \end{array} \right) \nn
&=&
\left( \begin{array}{r r r}
1 & 0 & 0 \\
0 & \cos \varepsilon & - \sin \varepsilon \\
0 & \sin \varepsilon & \cos \varepsilon
\end{array} \right)
\left( \begin{array}{c} \cos \phi_\odot  \\ \sin \phi_\odot \\ 0 \end{array} \right) = 
\left( \begin{array}{c} \cos \phi_\odot  \\ \cos \varepsilon \sin \phi_\odot \\ \sin \varepsilon \sin \phi_\odot \end{array} \right) ~.
\la{eq:equatorial}
\eea
The polar angle for the Sun in this equatorial reference frame is therefore
\be
\theta'_\odot = \arccos z'_\odot = \arccos \left( \sin \varepsilon \sin \phi_\odot \right)~.
\la{eq:polar-sun}
\ee

In the astronomical literature, the value of $\phi_\odot$, measured with respect to the first equinox $e$, is called the Sun's ``ecliptic longitude.''  The corresponding $\phi'_\odot$ in the equatorial frame is called the Sun's ``right ascension,'' while $\pi / 2 - \theta'_\odot$ is its ``declination.''\footnote{It is common to approximate the Sun's declination as \hbox{$\pi / 2 - \theta'_\odot \simeq \varepsilon \sin \phi_\odot$}; see, e.g., \cite{Khavrus}.  This is accurate to within $0.26^\circ$ for the value of $\varepsilon$ in \Eq{eq:obliquity}, but rather obscures the geometry involved, as Sproul comments in \cite{Sproul}.}  For a full discussion of the celestial sphere and of the coordinate systems that astronomers use to characterize points on it, see \cite{celestial-sphere}.

\begin{figure} [htb]
\begin{center}
	\includegraphics[width=0.3 \textwidth]{./images/solar/obliquity.pdf} \hskip 2.5 cm
	\includegraphics[width=0.42 \textwidth]{./images/solar/equinoxes.pdf}
\end{center}
\caption{\small (a): We may transform from the ecliptic to the equatorial reference frame by rotating along the $x$-axis by an angle equal to the obliquity $\varepsilon$, given in \Eq{eq:obliquity}, so that the new axis $z'$-axis is also the axis of the Earth's rotation.  (b):  The motion of the Sun along the ecliptic, as seen in the new equatorial reference frame.  Point $P$ marks the celestial north pole and $\overline P$ the celestial south pole.  The northward and southward equinoxes are marked by $e$ and $\bar e$, respectively.  The northern and southern solstices are indicated by $s$ and $\bar s$, respectively.\la{fig:equatorial}}
\end{figure}

\subsection{Terrestrial frame}
\la{sec:terrestrial}

Seen from the Earth, objects in the sky rotate azimuthally in the equatorial frame (i.e., about the $z'$-axis), with constant angular velocity
\be
\omega = \frac{2\pi}{23.9345 ~\hbox{hours}}~,
\la{eq:omega}
\ee
where $23.9345$ hours is the duration of the ``sidereal day,'' equal to the amount of time that it takes the Earth to complete one rotation about its axis (and therefore also for a distant star to return to the same position in the sky).  This is slightly less than the ``mean solar day'' of 24 hours, because of the Sun's motion along the ecliptic during the course of one sidereal day.

To characterize the position of the Sun, as seen from a point on the surface of the Earth, we must also adjust for the geographic latitude $L$.  We can achieve this by rotating about the $x$-axis by an angle equal to the co-latitude $\pi / 2 - L$.  The transformation from the equatorial frame to the terrestrial frame therefore gives: 
\bea
\vv r''_\odot =
\left( \begin{array}{c} x''_\odot \\ y''_\odot \\ z''_\odot \end{array} \right) &=&
\left( \begin{array}{c} \sin \theta''_\odot \cos \phi''_\odot \\ \sin \theta''_\odot \sin \phi''_\odot \\ \cos \theta''_\odot \end{array} \right) \nn
&=& \left( \begin{array}{r r r}
1 & 0 & 0 \\
0 & \sin L & - \cos L \\
0 & \cos L & \sin L
\end{array} \right)
\left( \begin{array}{r r r}
\cos \left[ \omega (t - t_0) \right] & \sin \left[ \omega (t - t_0) \right] & 0 \\
-\sin \left[ \omega (t - t_0) \right] & \cos \left[ \omega (t - t_0) \right] & 0 \\
0 & 0 & 1
\end{array} \right)
\left( \begin{array}{c} \cos \phi_\odot  \\ \cos \varepsilon \sin \phi_\odot \\ \sin \varepsilon \sin \phi_\odot \end{array} \right)~,
\la{eq:terrestrial}
\eea
where $t - t_0$ is the interval during which the Earth has rotated, measured with respect to a reference time $t_0$.\footnote{Note that the signs of the off-diagonal $\pm \sin[\omega(t-t_0)]$ entries in the corresponding rotation matrix in \Eq{eq:terrestrial} reflect the fact that the Earth's rotation displaces the Sun in an azimuthal direction {\it opposite} to that of the Sun's yearly motion along the ecliptic.  This is the reason why the mean solar day of 24 hours is {\it longer} than the sidereal day of 23.9345 hours: the extra 4 minutes of rotation are need to compensate for the change in $\phi_\odot$ in \Eq{eq:ecliptic}.}  We will discuss how to choose the value of $t_0$ (which will depend on the geographic longitude $\ell$) in \Sec{sec:longitude}.

By \Eq{eq:terrestrial}, the altitude (or ``elevation'') of the Sun above the horizon, as a function of the latitude and the time $t$, is
\bea
\alpha_\odot (L, t) = \frac{\pi}{2} - \theta''_\odot (L, t) &=& \arcsin [z''_\odot (L, t)] \nn
&=& \arcsin \left( - \cos L \cdot \cos [\phi_\odot (t)] \cdot \sin \left[ \omega (t - t_0) \right]
+ \cos L \cdot \cos \varepsilon \cdot \sin [\phi_\odot (t)] \cdot \cos \left[ \omega (t - t_0) \right] \right. \nn
&&  \hskip 1.5 cm \left. +  \sin L \cdot \sin \varepsilon \cdot \sin [\phi_\odot (t)] \right)~.
\la{eq:altitude}
\eea
When $\alpha_\odot = 0$, the Sun is either rising or setting.  When $\alpha_\odot = \pi / 2$, the Sun is directly overhead, at the ``zenith'' (this can occur only at tropical latitudes $-\varepsilon \leq L \leq \varepsilon$).  

Meanwhile, the azimuthal angle $\phi''_\odot$ can be computed from \Eq{eq:terrestrial}, using the relation
\be
\tan \phi''_\odot = \frac{y''_\odot}{x''_\odot}~.
\la{eq:terr-azimuth}
\ee
In \Sec{sec:longitude} we will work out the relation between this $\phi''_\odot$ and the cardinal directions (North, East, South, and West).

\section{Astronomical adjustments}
\la{sec:astronomical}

For some purposes, it may be acceptable to approximate the angle $\phi_\odot (t)$ as increasing linearly and completing a full revolution in one year (as do the authors of \cite{Khavrus}). A more precise expression can be obtained from Kepler's first and second laws of planetary motion, which state that the Earth moves along an ellipse, with the Sun at a focus, while the line segment from the Sun to the Earth sweeps out equal areas in equal times.

\subsection{Equation of the center}
\la{sec:center}

The angle subtended by the line from the Sun to the Earth, with respect to the major axis of the elliptical orbit, is known to astronomers as the ``true anomaly'' and is usually represented by the letter $v$.  Finding $v$ as a function of time has no exact analytic solution,\footnote{Newton offered a rigorous proof that no analytic solution could exist, using concepts now associated with topology, long before topology was invented.  This fascinating proof (the first impossibility proof since the ancient Greeks) is discussed in \cite{Arnold,Chandra}.} but an expansion can be obtained, which converges rapidly for small orbital eccentricity $e$, known as the ``equation of the center:''
\be
v = M + 2 e \sin M + \frac{5}{4} e^2 \sin 2M + \frac{1}{12} e^3 \left( 13 \sin 3M - 3 \sin M \right) + {\cal O}(e^4)~.
\la{eq:center}
\ee
The ``mean anomaly'' in \Eq{eq:center} can be expressed as
\be
M = M_0 + M_1 t~,
\la{eq:mean}
\ee
with constant $M_{0,1}$; it would be equal to the angle $v$ for a perfectly circular orbit ($e=0$) of equal area to the true elliptical orbit; see \cite{Danby}.  The values of $v$ and $M$ in \Eq{eq:center} are measured with respect to the {\it perihelion}, which is the point of closest approach between the Earth and the Sun, as shown in \Fig{fig:orbit}.  The value of $2 \pi / M_1$ is slightly greater than one calendar year because of the slow precessions of the equinoxes and the perihelion, which we will discuss in \Sec{sec:precession}. 

\begin{figure} [htb]
\begin{center}
	\includegraphics[width=0.6 \textwidth]{./images/solar/orbit.pdf}
\end{center}
\caption{\small Diagram of the Earth's orbit around the Sun.  The eccentricity is exaggerated for clarity.  The ``true anomaly'' $v$ is measured from perihelion, which is the point of closest approach between the Sun and the Earth.  The point of greatest separation between Sun and Earth is called the ``aphelion.''  We measure the ecliptic azimuthal angle of the Sun, $\phi_\odot$, from the Earth's position at the time of the first equinox.  Therefore $\phi_\odot = v - v_0$, where $v_0$ is the angular displacement between the perihelion and the equinox.\la{fig:orbit}}
\end{figure}

For our purposes it will be convenient to measure the angle $\phi_\odot$ from the first equinox of the year.  Therefore we let
\be
\phi_\odot = v - v_0 ~,
\la{eq:v0}
\ee
where $v_0$ is the angular displacement between the perihelion and the first equinox, as shown in \Fig{fig:orbit}.  Using the current astronomical data for the parameters $M_0$, $M_1$, $e$, and $v_0$ \cite{orbit-parameters}, we can write the equation of the center for the Earth as:
\be
M(t) = -0.0410 + 0.017202 \, t
\la{eq:mean-numbers}
\ee
and
\be
\phi_\odot (t) = - 1.3411 + M(t) + 0.0334 \sin [M(t)] + 0.0003 \sin [2M(t)]~,
\la{eq:center-numbers} 
\ee
where $t = 0$ corresponds to 1 January 2013, 0:00, Universal Coordinated Time (UTC), and $t$ is measured in mean solar days of 24 hours.

We may see from \Eq{eq:center-numbers} that the correction to $\phi_\odot$ introduced by the eccentricity of the Earth's orbit is small: less than $2^\circ$ at any given time of the year.  For a planet like Mercury, whose orbit is more eccentric and whose rotation is slower than the Earth's, the motion of the Sun in the sky is qualitatively different, as discussed in \citep{Mercury}.

\subsection{Precession of equinoxes and perihelion}
\la{sec:precession}

In the second century BCE, the Greek astronomer Hipparchos of Nicaea found that the positions of the equinoxes moved along the ecliptic (i.e., with respect to the distant stars) by about $1^\circ$ per century (the modern estimate is $1.38^\circ$ per century).  Newton correctly explained this as due to the tidal forces that the Moon and the Sun exert on the Earth, which is not perfectly spherical.  If the Earth did not spin, those tidal forces would pull the Earth's equatorial bulge onto the orbital plane of the corresponding perturbing body (i.e., of either the Moon or the Sun).  The Earth's spinning turns the action of that tidal torque into a {\it precession}, so that the axis of the Earth's rotation describes a cone, and the position of the celestial north pole therefore moves slowly along a circle, with respect to the constellations.\footnote{This slow change of the positions of the poles, equinoxes, and solstices, relative to the distant stars, implies that the signs of the Zodiac are not fixed with respect to the solar calendar.  For example, the ``Tropic of Cancer'' was so named because the position of the Sun at the time of the northern solstice used to lie within the constellation of Cancer, but today the northern solstice actually lies in Taurus.  The first equinox, which used to lie in Aries when the ancient Babylonians developed the calendar, has since shifted to Pisces and will move into Aquarius around the year 2,600.  This last circumstance has been the source of much mystical twaddle about the ``dawning of the Age of Aquarius.''}

The period of the precession of the Earth's axis is about 26,000 years.  Since the recurrence of the seasons depends on the periodicity of the equinoxes, rather than on the actual time it takes the Earth to go once around the Sun, the modern calendar is based on the ``mean tropical year,'' which is shorter than the sidereal year by about $20$ minutes (i.e., 1/26,000 of a sidereal year).

The position of the perihelion with respect to the distant stars also varies, but more slowly, with a period of about 112,000 years, which is equivalent to a displacement of about $0.32^\circ$ per century.  This precession results from perturbations to the motion of the Earth around the Sun caused by the gravitational pull of the Moon and the other planets, and to a lesser extent also by relativistic corrections to Newtonian gravity.\footnote{One of the most convincing early demonstrations of the validity of Einstein's theory of general relativity was that it explained the anomalous precession of the perihelion for the orbit of Mercury, which astronomers had until then failed to account for by the gravitational influence of the known planets; see \cite{dInverno} for a detailed discussion.}

The respective precessions of the equinox and the perihelion proceed in opposite directions along the ecliptic, causing the value of $v_0$ in \Eq{eq:v0} to {\it decrease} by about $1.7^\circ$ per century.\footnote{The quantity $2\pi - v_0$ is known to astronomers the ``longitude of perihelion.''}  Though for our purposes such precision is hardly justified, if we wished to take into account those precessions, we could make $v_0$ in \Eq{eq:v0} a time-dependent parameter.

\section{Duration of daylight}
\la{sec:daylight}

The computation only up to \Eq{eq:polar-sun} suffices to obtain a good estimate of the number of hours of daylight for a given day of the year, if we do not care for the precise time of sunrise and sunset.  Here the main approximation is that that the azimuthal angle of the Sun in the ecliptic frame, $\phi_\odot$, will be taken to be fixed during a given calendar date $d$.  For definiteness, let us say that $\phi_\odot$ is computed at noon for the date and location of interest, the corresponding time being translated to Universal Coordinated Time (UTC), for use in Eqs.~(\ref{eq:mean-numbers}) and (\ref{eq:center-numbers}).

\begin{figure} [t]
\begin{center}
	\includegraphics[width=0.5 \textwidth]{./images/solar/daylight-sphere.pdf} \hskip 1.5 cm
	\includegraphics[width=0.325 \textwidth]{./images/solar/daylight-circle.pdf}
\end{center}
\caption{\small (a): Cross-section of the celestial sphere along the Earth's axis of rotation $P \overline P$, centered at the position $O$ of an observer at geographic latitude $L$.  The point $m$ corresponds to the maximum altitude of the Sun, and $\overline m$ to the minimum altitude.  (b): Cross-section of the celestial sphere, centered at point $a$ and perpendicular to the Earth's axis of rotation.  The point $c$ corresponds to sunrise and $\bar c$ to sunset.  The arrows show the direction in which the celestial sphere rotates with respect to the observer at $O$.\la{fig:daylight}}
\end{figure}

Figure \ref{fig:daylight}(a) shows a cross-section of the celestial sphere, parallel to the Earth's axis of rotation $P \overline P$.  As the sphere rotates about the observer at point $O$, the celestial pole $P$ maintains a fixed altitude, equal to the observer's geographic latitude $L$.\footnote{If we take $L$ to be positive for points on the northern hemisphere of the Earth, then $P$ is the north celestial pole, and $\overline P$ is the south celestial pole.  The opposite convention would be more convenient for observers in the southern hemisphere.}  Point $m$ marks the maximum altitude of the Sun, while point $\overline m$ marks its minimum altitude.

The path of the Sun in the sky corresponds to the circle $am$, shown in \Fig{fig:daylight}(b) (again, as long as we neglect the change in $\phi_\odot$, and therefore also in $\theta'_\odot$, during the course of one day).  This circle is a cross-section of the celestial sphere, perpendicular to the axis $P \overline P$ and parallel to the line $m \overline m$.

In terms of the angle $\delta$ in \Fig{fig:daylight}(b),\footnote{Astronomers call $\delta$ the Sun's ``local hour angle'' at the times of rising and setting.  See, e.g., \cite{Meeus-rising}.} the number of hours of daylight is simply
\be
H = 24 \left( 1 - \frac{\delta}{\pi} \right)~,
\ee
since the Sun moves uniformly along the circle $am$, with a period of 24 hours.\footnote{By making the period of rotation of the Sun about the celestial poles in \Fig{fig:daylight} equal to the mean solar day of 24 hours, rather than the sidereal day of 23.9345 hours, we are taking into account the average change in $\phi_\odot$ during the course of one day.}  Examining Figs.~\ref{fig:daylight}(a) and (b), we see that
\be
\delta = \arccos \frac{ab}{am} = \arccos \left( \tan L \cot \theta'_\odot \right)~.
\la{eq:delta}
\ee
Therefore we can express the number of hours of daylight as a function of geographic latitude and day of the year in the form:
\bea
H(L,d) &=& 24 \left[ 1 - \frac{\arccos \left( \tan L \cot \theta'_\odot (d) \right)}{\pi} \right] \nn
&=& 24 \left[ 1 - \frac{1}{\pi} \arccos \left( \tan L \frac{\sin \varepsilon \sin [\phi_\odot (d)]}
{\sqrt{1 - \sin^2 \varepsilon \sin^2 [\phi_\odot (d)]}} \right) \right]
\la{eq:hours}
\eea
(which agrees with the expression obtained in \cite{Khavrus}).

\begin{figure} [t]
\begin{center}
	\includegraphics[width=0.5 \textwidth]{./images/solar/daylight-hours.pdf}
\end{center}
\caption{\small Number of hours of continuous daylight $H$, as a function of the day of the year $d$ (starting on 1 January), computed using \Eq{eq:hours}, for: the latitude of Cartagena de Indias, Colombia, $10^\circ 24'$ N (red curve); the latitude of Boston, Massachusetts, USA, $42^\circ 21'$ N (blue curve); the latitude of Stockholm, Sweden, $59^\circ 20'$ N (green curve); and the Arctic Circle, $66^\circ 34'$ N (dashed black curve).\la{fig:daylight-hours}}
\end{figure}

Figure \ref{fig:daylight-hours} shows plots of $H$ as a function of the day of the year $d$, at the latitudes of Cartagena de Indias (Colombia), Boston (USA), Stockholm (Sweden), and the Arctic Circle, all in the northern hemisphere.  Note that, for the northern hemisphere, the midyear solstice (which occurs around 21 June, or $d=171$) is always the longest day, whereas it is the shortest day everywhere in the southern hemisphere.  Conversely, the year-end solstice (around 21 December, or $d = 354$) is always the longest day in the southern hemisphere and the shortest in the northern hemisphere.
 
\section{Calculations}

With the theoretical part now behind us, we will revisit and extend our module for the calendar
to include the solar position calculations. We first load the module and then assign a local variable \textbf{doy}, day of year to get the day of the year in a leap year. We will also get the same for a normal year to compare and test our function.

\begin{texexample}{Loading the Lua module}{}
\begin{luacode}
local c = require("i18n.calendar") 
local doy = c.calcDayOfYear(11, 13, true)
tex.print("Day of year for 13 Nov in a leap year", doy, "\\par")
doy = c.calcDayOfYear(11, 13, false)
tex.print("Day of year for 13 Nov in a normal year", doy)
\end{luacode}
\end{texexample}
As expected in a leap year the day number was higher by one day.  This function will be the starting point for most of the calculations that follow. Similarly to the day number the year needs to be expressed fractionally before we can use it in other equations. This is given in radians.

\begin{texexample}{Calculate fractional year}{}
\begin{luacode}
local c = require("i18n.calendar") 
local doy = c.calcDayOfYear(12, 13, false)
local fy = c.fractionalYear(doy, 12)
tex.print("Day of year for 13 Dec in a normal year", doy, "\\par")
tex.print("Fractional year at doy = ",doy, fy.. "rad", math.deg(fy))
\end{luacode}
\end{texexample}

The next two calculations calculate the Equation of Time and the declineation angle.

\begin{texexample}{Calculate equation of time and declneation}{}
\begin{luacode}
local c = require("i18n.calendar") 
local doy = c.calcDayOfYear(12, 13, false)
local fy = c.fractionalYear(doy, 12)
local decl = c.declineation(fy)
local ET   = c.equation_of_time(fy)

tex.print("Day of year for 13 Dec in a normal year", doy, "\\par")
tex.print("Fractional year at doy = ", doy, fy.. " rad", math.deg(fy))
tex.print("Declineation", decl, math.deg(decl))
tex.print("Equation of time", ET)
\end{luacode}
\end{texexample}

\begin{texexample}{Calculate equation of time and declneation}{}
\begin{luacode}
local c = require("i18n.calendar") 
local ET, doy   

for i = 1, 30 do 
    doy = c. calcDayOfYear(i, 0, false)
  ET= c.equation_of_time(doy)
  tex.print(ET .. ', ')
end 
\end{luacode}
\end{texexample}

Now that we have generated all these points, we might make a small diversion and plot them. We will use pgfplots to typeset the chart for us. To plot the equation of time against the day of the year, we first generate the data within a 
\cmd{\luadirect}. 
\bigskip

\begin{scriptexample}{test}{}
\begin{verbatim}
\edef\equationoftime{\luadirect{
  local c = require("i18n.calendar") 
  local ET   
  for fy = 1, 365, 1  do 
    ET= c.equation_of_time(fy)
    tex.sprint('(',fy ..', '.. ET .. ') ')
  end 
}}
\end{verbatim}
\end{scriptexample}
\bigskip

The next step is to write the pgfplots code. We will keep it simple and without a lot of formatting at first.
\edef\equationoftime{\luadirect{
local c = require("i18n.calendar") 
local ET   
for fy = 1, 365, 1  do 
  ET= c.equation_of_time(fy)
  tex.sprint('(',fy ..', '.. ET .. ') ')
end 
}}

\begin{scriptexample}{}{}
\begin{verbatim}
\begin{tikzpicture}
\begin{axis}[height=5cm,
                  width = 0.6\textwidth,
                  xlabel = day number,
                  ylabel = equation of time (min),
                  ]
\addplot[smooth] coordinates {
\equationoftime
};
\end{axis}
\end{tikzpicture}
\end{verbatim}
\end{scriptexample}

Combining the code and inserting it in our document we obtain a nice plot of the equation of time, possibly correct to 12 decimal places.

\begin{figure}[htb]
\begin{scriptexample}{}{}
\bgroup
\centering
\begin{tikzpicture}
\begin{axis}[height=5cm,
                  width = 0.6\textwidth,
                  xmin = 1, xmax=365,
                  ymax = 15,
                  xlabel = day number,
                  enlarge y limits = true,
                  enlarge x limits = true,
                  ylabel = equation of time (min),
                  grid = major
                  ]
\addplot[smooth] coordinates {
\equationoftime
};
\end{axis}
\end{tikzpicture}

\egroup
\caption{Equation of time, generated via LuaTeX mark-up.}
\end{scriptexample}
\end{figure}

The graph of course still needs a lot of polishing. The axes can have better numbering, we may want to add a title and other enhancements. The one modification which we will do is to make the chart a bit wider and get the x-axis numbering a bit better.  

\begin{figure}[htb]
\begin{scriptexample}{}{}
\bgroup
\centering
\begin{tikzpicture}
\begin{axis}[height=5cm,
                  width = \textwidth,
                  xmin = 30, xmax=365,
                  ymax = 15,
                  xtick = {30,60,...,360},
                  xlabel = day number,
                  enlarge y limits = true,
                  enlarge x limits = true,
                  ylabel = equation of time (min),
                  grid = major,
                  grid style ={color = black!50},
                  ]
\addplot[smooth] coordinates {
\equationoftime
};
\end{axis}
\end{tikzpicture}

\egroup
\caption{Equation of time, generated via LuaTeX mark-up.}
\end{scriptexample}
\end{figure}

The advantage of having the data the program and the documentation in one document, has major advantages. The generated plots guaranteed that our methods and functions have been tested properly. Of course once we have such a powerful tool in our disposal we can generate from the same routines tabular figures. 



The problem we need to solve is that we need to print the values in rows, one row for each value of the day number for all the months in a year. We also need to cater for all of TeX's mark-up without breaking up anything. Although at first this looks like a difficult task, now that we have sorted the rads from degrees, we need to go back to our module and handle the table from there. It is always easier to generate TeX commands from Lua rather than working from TeX and exporting and importing to Lua.

But how do we organize the code? 


\begin{scriptexample}{}{}
\bgroup
\scriptsize
\begin{luacode}
local c = require("i18n.calendar") 
c.equation_of_time_table ()
\end{luacode}
\captionof{table}{Equation of Time (ET) values for a normal year. }
\egroup
\end{scriptexample}

Modifying the routines to cater for a leap year is fairly simple, we can introduce a variable leap and modify the \luacmd{equation\_of\_time\_table}. 


\bgroup
\footnotesize
\begin{luacode}
local c = require("i18n.calendar") 
c.debug = true
c.equation_of_time_table (true, 12)
\end{luacode}
\captionof{table}{Equation of Time (ET) values for a leap year. }
\egroup


The coding of the tables (for the normal year and the leap year) although somewhat unexciting, is important to test that the routines are performing as required, as they provide an easy way to examine the output. As you can observe the day number for both the years are correct. 

It is good practice, that when developing code, you include testing routines and if appropriate also logging routines. The leap table was produced using |c.debug = true|  to view the day number, which was programmed as a subscript to the numerical values of the equation of time.

\subsection{declination}

The declineation angle also takes the day of the year parameter and returns the angle. For this we have already
added a function

\begin{scriptexample}{}{}
\begin{verbatim}
\begin{luacode}
local c = require("i18n.calendar")
c.printdeclination()
\end{luacode}
\end{verbatim}
\end{scriptexample}

\subsection{Calculating the Solar Azimuth and Altitude}

Now that we have most of the programming behind us, we will calculate the solar azimuth and altitude angles. These require effectively the calling of most of the functions we have coded so far. Once these are calculated then the sunset, zenith and sunrise can be calculated. These will be required later on to build a model for the solar gains and losses of buildings. 

\subsection{Assembling the Model}

The model specification required that from a given location, defined by its latitude and longitude at a particular month, day and time of the year, we should be able to determine the location of the sun on the celestial sphere.

We define an object |location = {latitude, longitude, name}| 











