\makeatletter
\cxset{title font-family=\sffamily,
       title font-size=\Huge,
       title font-weight=\bfseries}

\@specialtrue

\cxset{custom=steward}
\cxset{steward,
  chapter toc=true,
  numbering=arabic,
  custom = stewart,
  offsety=0cm,
  image={./chapters/hine03.jpg},
  texti={When Lamport designed the original \LaTeX\ sectioning commands, limitations of computer power forced him to restrict the abstraction of complicated chapter layouts. With current tools available improvements are much easier to program.},
  textii={In this chapter we discuss a method that allows the production of fancy chapter headings and formatting, based on a set of key values. Central  to this process is the separation of content from presentation.
We also discuss the basic formatting tools that are available and how one can modify them to mould new book designs.},
}


\@specialtrue

\cxset{chapter opening=left}
\chapter{Special Designs}
\section{Introduction}


The strength of the package lies in having defined mechanisms to enable easier abstraction of special designs.
We will first outline a simple mechanism for such definitions.

To define any special chapter you need to either redefine a command or create a new one. Let us look at
an example, which simply uses the tikZ package to draw a chapter header at the top of the page. Every time the \cs{chapter} command is called, this command will be indirectly activated at the appropriate point.
What is available for you to use is all the chapter settings information. You can also add additional keys.


The strength of the system lies in defining an adequate number of variables to abstract the design. We also need to decide which are the important parameters. Let us for demonstration purposes just add two new keys.
One for the band color and another for the rounded colour.

Note that any design based on tikZ's  \texttt{remember picture, overlay} requires possibly two and sometimes three runs in order to stabilize.



\section{Naming conventions}

When you set the key custom it redefines the command \cs{customdesign@cx} to hold the name of your special macro.  So the only place where you need to add a definition is one macro. You can name your style anything you want, however I recommend that variants are named in two or more words, the second one simulating a theme. For example you can name your theme \option{stephan} and a sub-theme as  \option{stephan blue}.


\section{Themes and styles}

Once you have a design abstracted and its major components defined as keys, you can think of it as a template. A template then can be extended to different \textit{themes}. For example if we name our template as \textit{stefan}, we can have themes as \textit{stefan blue}, \textit{stefan green} or other similar and appropriate names. This is closer to what is currently used in CMS systems on the web.




\cxset{stefan/.style={%
        title font-color=\color{white},
        band height=5cm,
        fill= teal,
        custom=tikzspecials}}

\cxset{stefan}

\chapter{Special Designs}

\lipsum[1-3]








\chapter{Introduction to TikZ Style Chapters}

The \lstinline{tikZ} package brings a lot of capabilities to the design of fancy style headings, including shading effects and the like. I expect this type of design to grow in the future. Since tikZ is part of the PGF family it is easy to integrate with the package.

\section{Integrating the code}

Code integration, especially with a document that might have different chapter headings presents a challenge. However, if we do touch the chapter command it might make things easier. We provide a key called special that instead of calling the \string\@make... calls a special
routine to handle the tikz commands (as one would expect that all the code will then be here).


First we define a special key.


\begin{teX}
\cxset{custom/.code=\gdef\customdesign@cx{#1}\@specialtrue,
       fill/.store in=\fill@cx}
\cxset{custom=tikzspecial,
       title font-size=\Large,
       title font-color=\color{white}}
\end{teX}

We have assumed that the only value we want to pass is the Chapter title, as the rest can be handled quite easily, by means of key values.

\section{Key management}

When you develop a generic template all the standards keys are available to you. For example the chapter opening commands. However, if you positioning using fixed parameters, the anywhere key cannot work properly, by adding a yshift into the definition of the special and adjusting you can achieve it.

\clearpage



\cxset{stefan,
       chapter opening=anywhere,
       fill= purple,
       band height=2cm,
       section color= purple}

\renewsection

\chapter{A test}

\section{Conclusions}

In this chapter we have seen how to design and code special templates for special openings. In most cases you will use TikZ to produce them, so familiarity with the graphics program is essential. In general I advice that before you embark on a special design to select the method you will used based on the following:

\begin{enumerate}
\item If the template requires positioning of pictures and text at exact positions, you can use the 
        picture environment and the built-in commands provided in this package.
\item If it requires any special graphics, coloured blocks and the like, use the TikZ package or pstricks.
\item If you only manipulating textual information you don't need a special use the key value interface provided by the package (see for example the verso style).
\end{enumerate}


\clearpage

\newgeometry{top=1cm,bottom=1cm,left=0cm, right=0cm} 


\pagestyle{empty}

\@specialfalse
\cxset{
 custom= genetics,
 name={CHAPTER CONCEPT},
 numbering=none,
 number font-size=,
 number font-family=,
 number font-weight=,
 number color=white,
 numbering=arabic,
 chapter opening=right,
 chapter color={black},
 chapter font-family=\sffamily,
 chapter font-size=\Large,
 chapter font-weight=\bfseries,
 title font-family=\sffamily,
 title font-color=teal,
 rule off,
 image={./images/genetics-dogs.png},
 image caption={Labrador retriever\\
         puppies expressing\\
         brown (chocolate),\\
         golden (yellow),\\
         and black\\
         coat colors,\\
         traits controlled\\
         by two gene pairs.},
 textiii={\begin{itemize}
\large
\item This is some text describing the main chapter concepts. Many
      modern textbooks have chapter openings, with complicated
      layouts, such as this one.
\item Each layout is probably unique, but some designs
      might be possible to be abstracted.
\item Keys are defined for the special textboxes in a anti-clockwise pattern.
      The image caption is set using image caption, the chapter concepts list
      is defined using the key textiii.
\item The chapter number is available via the normal chapter number keys.
\item This layout has been developed using normal \tex macros and does not
      utilize absolute positioning via tikz or similar. This way the layout
      is more flexible. It can certainly be improved using LaTeX3 techniques.
  \end{itemize}}
}


\newpage
\@specialtrue
\chapter[Special Design]{Extensions\\ of Mendelian\\ Genetics}


\cxset{
 custom=genetics,
 name={CHAPTER CONCEPT},
 numbering=none,
 number font-size=,
 number font-family=,
 number font-weight=,
 number color=white,
 numbering=arabic,
 chapter opening=right,
 chapter color={black},
 chapter font-family=\sffamily,
 chapter font-size=\Large,
 chapter font-weight=\bfseries,
 title font-family=\sffamily,
 title font-color=teal,
 rule off,
 image= {./images/chromosome.png},
 image caption={Labrador retriever\\
         puppies expressing\\
         brown (chocolate),\\
         golden (yellow),\\
         and black\\
         coat colors,\\
         traits controlled\\
         by two gene pairs.},
 textiii={\begin{itemize}
\large
\item While alleles are transmitted from parent to   offspring
according to Mendelian principles, they often do not
display the clear-cut dominant/recessive relationship
observed by Mendel.
\item In many cases, in a departure from Mendelian genetics,
two or more genes are known to influence the phenotype
of a single characteristic.
\item Still another exception to Mendelian inheritance occurs
when genes are located on the X chromosome, because one
of the sexes receives only one copy of that chromosome,
eliminating the possibility of heterozygosity.
\item Phenotypes are often the combined result of genetics and
the environment within which genes are expressed.
\item The result of the various exceptions to Mendelian principles
is the occurrence of phenotypic ratios that differ from those
produced by standard monohybrid, dihybrid, and trihybrid
crosses.
  \end{itemize}
}}

\newpage

\chapter[Jia Lu's paintings]{The Human\\Chromosome}



\newpage

\cxset{custom = genetics,
 image = {./images/swords.jpg},
 image caption={Liu's paintings\\
         reflect traditional\\
         aesthetics of,\\
         her teacher\\
         Fang Zeng.\\
         Her work\\
         reflects strength\\
         and wisdom.},
 textiii={\begin{itemize}
\large
\item While alleles are transmitted from parent to   offspring
according to Mendelian principles, they often do not
display the clear-cut dominant/recessive relationship
observed by Mendel.
\item In many cases, in a departure from Mendelian genetics,
two or more genes are known to influence the phenotype
of a single characteristic.
\item \lorem
  \end{itemize}
}}

\chapter{Jia Lu's Sensual Paintings}

\restoregeometry

\section{Jia Lu}

Jia Lu is an oil painter working in America, known for blending Asian and European imagery in her paintings, predominantly of women. Jia Lu's works include Chinese ink paintings, oil paintings, watercolors, drawings, sculpture and prints. Her early work strongly reflected the traditional aesthetics of her teacher Fan Zeng, but by the time she exhibited in Canada, she was critiquing new social developments, consumerism and power relations in China through a series of mixed-media self-portraits. Her mature work in oils demonstrates an interest in Buddhism and a purely feminine aesthetic and can be seen as a response to the masculine, sensual approach to the female nude. However, she has also painted male figures. In numerous interviews she has emphasized the importance of beauty in her work, which she describes as "strength and wisdom".

\makeatletter
\@specialfalse

\cxset{style13}










