\section{xparse}

The LaTeX3 Team provided the package \pkg{xparse} which provides document level 
authors withs some powerful commands that extend those such as \cs{newcommand}
of LaTeX2e, they have been stable and the interface is not expected to change. My
understanding is that the package and the code is very stable and is targetting
package authors.\footnote{http://tex.stackexchange.com/questions/98152/always-use-newdocumentcommand-instead-of-newcommand}


\DescribeMacro{\DeclareDocumentCommand}

This family of commands are used to create a document-level \textit{function}. The argument
specification for the function is given by \textit{arg spec}, and expanding to be replaced by the
\textit{code}. Unlike \latex's definition commands, all xparse commands take two arguments.
The first one is the \textit{argument specifier}, and the second is the \textit{code.} It is best to 
start with an example to illustrate its use.

\emphasis{s,m,o,m,v}
\begin{texexample}{DeclareDocumentCommand}{l3:1}
\DeclareDocumentCommand \foo { m o m } {% 
    arg 1 = #1 , arg 2 =  #2 arg3 = #3 }

\foo{mandatory}[optional]{mandatory}

\DeclareDocumentCommand \foobar { m m m } { 
    arg 1 = #1 , arg 2 =  #2 arg3 = #3 }

\foobar{A}{B}{C}

\foobar ABC

\DeclareDocumentCommand \chapter { s o m }
{
\IfBooleanTF {#1}
  { \typesetnormalchapter {#2} {#3} }
  { \typesetstarchapter {#3} }
}
\DeclareDocumentCommand\test{ s m }{
\IfBooleanTF{#1}
  {\MakeTextUppercase{#2}}
  {\MakeTextLowercase{#2}}
}
\test{This is some text}
\test*{This is SOME test}

\DeclareDocumentCommand\test{ s +m +v }{
\IfBooleanTF{#1}
  {\MakeTextUppercase{#3}}
  {\MakeTextLowercase{#2}}
 \par \leavevmode
  #3
}
\RaggedRight

\test*{This is SOME test\par}{\text
                                      \othercommands
                                      \or \a \full \macro}
\end{texexample}

\DeclareDocumentCommand\test{ s +m +v }{
\IfBooleanTF{#1}
  {\MakeTextUppercase{#3}}
  {\MakeTextLowercase{#2}}
 \par \leavevmode
  #3
}


\begin{flushleft}
\test*{This is SOME test\par}{|\text
                                      \othercommands
                                      \or \a \full \macro|}\par
\end{flushleft}
As you can observe from the example, the \cs{DeclareDocumentCommand} requires
that you provide an argument specifier, that is we need to tell it what type of
input to expect. By adding this abstraction when creating a new command or
\textit{function} as the LaTeX3 Team like to call it, it eliminates a lot of the
code that the package writer has to write. 



\begin{argumentlist}{m}
\item [m] Mandatory
\item [o] Optional argument in  []. Returns NoValue if not present.
\item [O] 
\item [s] Starred version
\item [v] Verbatim
\item [l]
\item [u]
\item [d]
\item [D] 
\item [t]  An optional \meta{token}, which will result in a value \cs{BooleanTrue} if htokeni is 
            present and \cs{BooleanFalse} otherwise. Given as thtokeni.
\item [g] An optional argument given inside a pair of TEX group tokens (in standard LATEX,
              { . . . }), which returns \cs{NoValue} if not present.
\item [G] G As for g but returns \marg{default} if no value is given: G\marg{defaulti}.
\end{argumentlist}











