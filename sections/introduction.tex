\cxset{style13}
\chapter{Introduction}

Browse any books in a library or bookshop and the striking thing is that their design is very individualistic. They might have similarities but their main features vary. In many respects they resemble people's faces where minor differences have striking effects.

This package arose out of a question at stackexchange. How to redefine chapter heads. Having seen the popularity of pgf I realized that LaTeX users prefer this method of styling rather the traditional LaTeX method.

The user interface can be extended to basically all major packages. The principle is to keep to a minimum changes that can affect the LaTeX core commands. If there are any additions a key setting is provided to be able to revert back to normal LaTeX.

The workflow can be simplified. In addition I want to believe that the interface can provide a useful addition to the open source community and that other people will contribute style libraries, which will be simpler to write. It is also possible
to device an easy and uncomplicated web interface to handle
such a great number of variables.

\section{Why this package}

Most people when they get started with \LaTeX\ will either use one of the standard classes such as the \file{book.cls} or one of the generic classes notably koma-script or memoir. Most students will be forced to use on of the many thesis classes available.

\section{The key value concept}

The key-value concept that originated with \LaTeX\ has been extended many times, the last and most serious implementation of it by Tantau in the PGF package. What essentially Tantau allows us to do is to use a scripting language to script TeX code. The TikZ and pgfplots packages are two major packaged that use keys effectively. Their popularity is gowing and what \athena\ does is to offer a user interface that in many respects.


\section{On typography}

This package hopefully will assist in improving the typography of books set with \LaTeX\ and this package. Any typographical comments on the various styles are my own and not necessarily correct. Like fashion and art typography has opinions rather than absolute truths. In many styles the design is slightly adapted to blend a bit better with this manual. Also I did not select fonts as per the samples but this is left on you the user to decide.

I selected samples based on their programming demands and not on  typography. For example Chapter\ref{style:46}, was selected to demonstrate the use of TikZ to produce leaders and decorations and not for its looks, which is not terribly attractive.

\section{This manual}

This manual is rather long as it is aimed at people that may want to add extensions to the package. If all you need is to change the font of the standard LaTeX class, you maybe better off with another package. 















