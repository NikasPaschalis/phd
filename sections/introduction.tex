\cxset{style13}
\chapter{Introduction}

 \LaTeX2e, does not provide a standard library, but comes equipped with
 a package mechanism that allows code extensions to be loaded as required.
 This has created a strong vibrant community, hundreds of packages and a 
 headache to both new and seasoned users. What packages are available, when
 to use them and in which order is a common theme for many questions on
 lists and |TX.SE|.

 It is quite common during the writing of a thesis or book
 for the author to keep on adding macros and packages
 at the preamble of the document. In most cases this can
 be satisfactory but in many others it leads to
 incompatibilities and errors. This package aims at
 minimizing one's preamble, by prefetching a number of
 commonly used packages. It also aims at loading them
 in the right order and providing patches for conflicts.
 
 I am hoping that using this package, will lead to less
 frustrations with the intricacies of \LaTeX2e\ packages.

 Although I expect most users to just load the package, using
 the standard command:

 \begin{teX}
  \usepackage{phd}
 \end{teX}

 it will be better for the longer projects to just fork the
 package and adapt it to your needs. In this respect, I have
 uploaded the package to |github|.

 My goal in selecting the packages and adding a number of 
 commands for the authors was to be able to typeset a 
 document for most common use cases, without the need of
 additional packages. The packages I selected are biased
 towards academic publications, although they can find use
 in almost any fields. The package provides a mechanism via
 PGF keys to provide a settings file. 
 
 Most of the documentation can be found in the implementation part.

Browse any books in a library or bookshop and the striking thing is that their design is very individualistic. They might have similarities but their main features vary. In many respects they resemble people's faces where minor differences have striking effects.

This package arose out of a question at stackexchange. How to redefine chapter heads. Having seen the popularity of pgf I realized that LaTeX users prefer this method of styling rather the traditional LaTeX method.

The user interface can be extended to basically all major packages. The principle is to keep to a minimum changes that can affect the LaTeX core commands. If there are any additions a key setting is provided to be able to revert back to normal LaTeX.

The workflow can be simplified. In addition I want to believe that the interface can provide a useful addition to the open source community and that other people will contribute style libraries, which will be simpler to write. It is also possible
to device an easy and uncomplicated web interface to handle
such a great number of variables.

\section{Why this package}

Most people when they get started with \LaTeX\ will either use one of the standard classes such as the \file{book.cls} or one of the generic classes notably koma-script or memoir. Most students will be forced to use on of the many thesis classes available.

\section{The key value concept}

The key-value concept that originated with \LaTeX\ has been extended many times, the last and most serious implementation of it by Tantau in the PGF package. What essentially Tantau allows us to do is to use a scripting language to script TeX code. The TikZ and pgfplots packages are two major packaged that use keys effectively. Their popularity is growing and what this package does is to offer a user interface that in many respects resembles css.

\begin{longtable}{lp{3cm}p{6cm}}
\toprule
Key           &Remarks\\ 
\midrule
chapter color && the color of the chapter head\\
chapter before && code that goes before the \textit{Chapter}\\
chapter number &&style of the number roman, Roman, arabic\\
 chapter name = CHAPTER &&\\
 chapter toc = false && Simulates |chapter*| only typesets the chapter name and the chapter title.\\
 chapter color= thegray &&\\
 chapter opening = right &&\\
 chapter numbering = arabic &&\\
 chapter font-family= sffamily &&\\
 chapter font-weight= bold &&\\
 chapter font-size= LARGE &&\\
 chapter before= &|\thinrule|\ldots& Items before the chapter name\\
 chapter after = &&items after the chapter name\\
\bottomrule
\end{longtable}

Many of the key values do not need to be written as commands, i.e., you can write |large| rather than |\large|. I thought this would bring the concept nearer to CSS and also to make it


\section{On typography}

This package hopefully will assist in improving the typography of books set with \LaTeX\ and this package. Any typographical comments on the various styles are my own and not necessarily correct. Like fashion and art typography has opinions rather than absolute truths. In many styles the design is slightly adapted to blend a bit better with this manual. Also I did not select fonts as per the samples but this is left on you the user to decide.

I selected samples based on their programming demands and not on  typography. For example Chapter\ref{style:46}, was selected to demonstrate the use of TikZ to produce leaders and decorations and not for its looks, which is not terribly attractive.

\section{This manual}

This manual is rather long as it is aimed at people that may want to add extensions to the package. If all you need is to change the font of the standard LaTeX class, you maybe better off with another package. 

 \section{Version control with Git and Github}
 If you are involved with code or a publication that will have frequent changes, you should consider
 some type of version control system. My own recommendation is to use |git| and an online repository such
 as |github|. The latter is currently very fashionable and makes sharing code easier. Note that the |github|
 offers both public as well as private repositories. The general recommendation is that for unpublished work
 such as a thesis or code under development, it is preferable to go for a private repository. 
 
 \subsection*{What is the difference between Git and GitHub?}
git is a version control, system think of it as a series of snapshots (commits) of your code. You see 
a path of this snapshots, in which order they where created. You can make branches to experiment and come back to snapshots you took.
GitHub, is a web-page on which you can publish your git repositories and collaborate with other people.

 \subsection*{Is git saving every repository locally (in the user's machine) and in GitHub?}

 No, it's only local. You can decide to push (publish) some branches on GitHub.

 \subsection*{Can you use Git without GitHub? If yes, what would be the benefit for using GitHub?}

 Yes, it runs local. You could back it up with dropbox. See also \href{http://dotmonster.co/backup-and-sync-folders-with-dropbox-and-symbolic-links/}{how to sync your git folders to dropbox}.

\subsection*{How does Git compare to a backup system such as Time Machine?}

It's a different thing, git lets you track changes and your development process. If you use git with GitHub, it becomes effectively a backup. However usually you would not push all the time to GitHub, at which point you do not have a full backup if things go wrong. I use git in a folder that is synchronized with dropbox.

 To synchronize the two, you create a symlink in your Git folder, i.e., in your repository. This can even be done in windows although it is normally called a \textit{short-cut}. Write click in the folder and create a short-cut to the drop-box folder. For example if you create a folder \texttt{pic} to hold your pictures, this folder will be automatically uploaded at \texttt{dropbox}. You don't want your pictures tracked at Github, as they do not really need version control. Do not forget to exclude them by typing in the \texttt{.gitignore} file the directive  \verb|*.lnk|.

\subsection*{Is this a manual process, in other words if you don't commit you wont have a new version of the changes made?}

Yes, commiting and pushing are both manual.

\subsection*{If are not collaborating and you are already using a backup system why would you use Git?}

 You program worked. You developed more, your program does not work. git diff, shows you the difference between the current code and the last working commit.

 Or you just go back to the last working. You want to try a change, but are not sure it really will work. 
 You create a branch, test you code change. If it works fine, you merge it to the main branch. If it does not you throw the branch away and go back to the main branch.
 You did some debugging. Before you commit you always look at the changes from the last commit. You see your debug print statement that you forgot to delete.

Make sure you check gitimmersion.com.















