\makeatletter\@specialfalse\makeatother
\cxset{style13}
\chapter{Introduction}
\addtocimage{-12pt}{-20pt}{../images/tocblock-fish.jpg}


\epigraph{``Begin at the beginning,'' the king said
"and then go on till you come to the end, then stop."}{
---Lewis Carroll, Alice in Wonderland}



This package and its documentation attempts to eliminate some common 
problems encountered when using \LaTeX2e. The first one is the loading of 
recommended packages for a large and perhaps complicated document and 
the second is the re-designing of styles for a document.

 \LaTeX2e, does not provide a standard library, but comes equipped with
 a package mechanism that allows code extensions to be loaded as required.
 This has created a strong vibrant community, hundreds of packages and a 
 headache to both new and seasoned users. What packages are available, when
 to use them and in which order is a common theme for many questions on
 lists and |TX.SE|.

 It is quite common during the writing of a thesis or book
 for the author to keep on adding macros and packages
 at the preamble of the document. In most cases this can
 be satisfactory but in many others it leads to
 incompatibilities and errors. This package aims at
 minimizing one's preamble, by prefetching a number of
 commonly used packages. It also aims at loading them
 in the right order and providing patches for conflicts.
 
 I am hoping that using this package, will lead to less
 frustrations with the intricacies of \LaTeX2e\ packages.

 Although I expect most users to just load the package, using
 the standard command:

 \begin{commands}[]{}
 | \usepackage{phd}|
 \end{commands}

 it will be better for the longer projects to just fork the
 package and adapt it to your needs. In this respect, I have
 uploaded the package to |github|.\footnote{\url{https://github.com/yannisl/phd}}

 My goal in selecting the packages and adding a number of 
 commands for the authors was to be able to typeset a 
 document for most common use cases, without the need of
 additional packages. The packages I selected are biased
 towards academic publications, although they can find use
 in almost any fields. The package provides a mechanism via
 PGF keys to provide a settings file. 
 
 Most of the documentation can be found in the implementation part.

Browse any books in a library or bookshop and the striking thing is that their design is very individualistic. They might have similarities but their main features vary. In many respects they resemble people's faces where minor differences have striking effects.

This package arose out of a question at stackexchange. How to redefine chapter heads. Having seen the popularity of the |pgf| package \cite{pkg-pgf} I realized that \latex users prefer this method of styling rather the traditional \latex method.

The user interface can be extended to basically all major packages. The principle is to keep to a minimum changes that can affect the LaTeX core commands. If there are any additions a key setting is provided to be able to revert back to normal LaTeX.

The workflow can be simplified. In addition I want to believe that the interface can provide a useful addition to the open source community and that other people will contribute style libraries, which will be simpler to write. It is also possible
to device an easy and uncomplicated web interface to handle
such a great number of variables.

\section{Why this package}

Most people when they get started with \LaTeX\ will either use one of the standard classes such as the \docfile{book.cls} or one of the generic classes notably koma-script or memoir. Most students will be forced to use on of the many thesis classes available.

\section{The key value concept}

The key-value concept that originated with \LaTeX\ has been extended many times, the last and most serious implementation of it by Tantau in the PGF package. What essentially Tantau allows us to do is to use a scripting language to script TeX code. The TikZ and pgfplots packages are two major packaged that use keys effectively. Their popularity is growing and what this package does is to offer a user interface that in many respects resembles css.

\begin{longtable}{lp{3cm}p{6cm}}
\toprule
Key           &Remarks\\ 
\midrule
chapter color && the color of the chapter head\\
chapter before && code that goes before the \textit{Chapter}\\
chapter number &&style of the number roman, Roman, arabic\\
 chapter name = CHAPTER &&\\
 chapter toc = false && Simulates |chapter*| only typesets the chapter name and the chapter title.\\
 chapter color= thegray &&\\
 chapter opening = right &&\\
 chapter numbering = arabic &&\\
 chapter font-family= sffamily &&\\
 chapter font-weight= bold &&\\
 chapter font-size= LARGE &&\\
 chapter before= &|\thinrule|\ldots& Items before the chapter name\\
 chapter after = &&items after the chapter name\\
\bottomrule
\end{longtable}

Many of the key values do not need to be written as commands, i.e., you can write |large| rather than |\large|. I thought this would bring the concept nearer to CSS and also to make it


\section{On typography}

This package hopefully will assist in improving the typography of books set with \LaTeX\. Any typographical comments on the various styles are my own and not necessarily correct. Like fashion and art typography has opinions rather than absolute truths. In many styles the design is slightly adapted to blend a bit better with this manual. Also I did not select fonts as per the samples but this is left on you the user to decide.

I selected samples based on their programming demands and not on  typography. For example Chapter\ref{style:46}, was selected to demonstrate the use of TikZ to produce leaders and decorations and not for its looks, which is not terribly attractive.

\section{Packages and Fonts}

This manual has been typeset with numerous fonts in order to enable the typsetting of almost all the scripts provided by the Unicode standard. In order to process it from the |.dtx| file, these fonts must be available in your system, otherwise \XeLaTeX\ will have a problem finding the fonts and it will take an awful long time to process. This is especially true for the scripts section, where virtually all the Unicode defined scripts are discussed. You will need a fast computer and a fast hard disk to process the document within a reasonable time. When using \pkgname{fontspec} always define your fonts with the \cmd{\newfontfamily} this will speed up processing by an order of magnitude. Compiling from the command prompt will speed up compilation. Average speed 2-3 pages per second.

Many of \tex's parameters are stretched to the limit with a complicated document such as this manual. You will require a full distribution otherwise expect some errors. Important packages is \pkgname{morefloats} and \pkgname{morewrites}. The package will also expect that you have e-tex installed. Ubuntu users are normally one year behind in updates, so you might wish to update manually. It will take upwards of 5 minutes to compile fully on an old laptop and a couple of minutes on a state of the art computer.

The |dtx| should be processed best with its own make file provided for Windows only |phd.bat|. The make file will process the documentation using \lualatex. You can also process the document with \xelatex but is prone to produce errors. Using \latexe the sections on scripts etc will not be printed and a much shorter version of the manual is provided. 

\section{Scripts and Languages}

The package and the documentation offer a full repertoire of font selection keys for different scripts and languages. It hasn't been possible, however hard I tried to compile this section of the documentation with \xelatex, as it kept giving errors of too many files open. This was also not possible even with the \pkgname{morewrites} package loaded. With \lualatex the document compiled with no major problems other than the font rendering being of a lower quality to that of XeLaTeX om windows, other than disabling incompatible packages and a number of commands that were redefined. 

Some good news for multi-script typesetting is the Noto fonts from Google. These fonts named Noto from "No Tofu" meaning you do not see any little square blocks for undefined glyphs, are fast to load. Disantvantage you need to switch between font commands fairly often.

\section{This manual}

This manual is rather long as it is aimed at people that may want to add extensions to the package. If all you need is to change the font of the standard LaTeX class, you maybe better off with another package. 

 \section{Version control with Git and Github}
 If you are involved with code or a publication that will have frequent changes, you should consider
 some type of version control system. My own recommendation is to use |git| and an online repository such
 as |github|. The latter is currently very fashionable and makes sharing code easier. Note that the |github|
 offers both public as well as private repositories. The general recommendation is that for unpublished work
 such as a thesis or code under development, it is preferable to go for a private repository. 
 

 \section{Ordering of Packages}
 
One package that normally leads to errors is the 
\pkgname{hyperref}. The package which is an outstanding example of software engineering and supported single handledy by Heiko Oberdiek redefines a a lot of internal commands of the kernel. As a lot of other packages do the same it has to be loaded at the end of the preable with the exception of some packages! 
 
 This manual is typeset according to the conventions of the
 \LaTeX \textsc{docstrip} utility which enables the automatic
 extraction of the \LaTeX{} macro source files~\cite{GOOSSENS94}.

 
 \href{http://tex.stackexchange.com/questions/96350/problem-with-algorithmic-and-hyperref}{problem with algorithmic and hyperref}

 \begin{verbatim}
\usepackage{float}  % load float package first!

\usepackage{hyperref} % let hyperref patch the float package stuff
.
 \usepackage{algorithm} % let algorithm use the patched version of the float package
 \end{verbatim}
 \section{Conventions}
 \subsection{Defining Colors}
 All color definitions are of the form |the<color>|. So the setting for |theblue| is called
 |theblue|. This provides easy to remember commands.
 
 \section{Document Structure}
 We do not load too many packages for document structure, a these are expected to be
 treated at class level.
 \begin{table}[ht]
 \centering
 \caption{Packages used for structure.}
 \begin{tabular}{ll}
   \toprule
   Package  & Default\\
   \midrule
   |multicol| & called by author  \\
   \bottomrule
 \end{tabular}
 \end{table}

\section{Known problems}

Perhaps the biggest issue with the package is the speed of
compilation with \XeLaTeX\ or \LuaTeX. This is to be expected, as both engines spend a lot of resources in font management. On demand loading of packages is something I have in the back of my mind. This should be done via document styles i.e., if a book is for the humanities, perhaps only a rudimentary amount of maths packages should be loaded.

















