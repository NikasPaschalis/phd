\newlength{\egwidth}\setlength{\egwidth}{0.48\textwidth}

\newenvironment{ega}%
{\begin{list}{}{\setlength{\leftmargin}{0.02\textwidth}%
\setlength{\rightmargin}{\leftmargin}}\item[]\footnotesize}%
{\end{list}}

\newenvironment{egbox}%
{\begin{minipage}[t]{\egwidth}}%
{\end{minipage}}

\newcommand{\egstart}{\begin{ega}\begin{egbox}}
\newcommand{\egmid}{\end{egbox}\hfill\begin{egbox}}
\newcommand{\egend}{\end{egbox}\end{ega}}

% one or two other commands

\newcommand{\fn}[1]{\hbox{\tt #1}}
\newcommand{\llo}[1]{(see line #1)}
\newcommand{\lls}[1]{(see lines #1)}


\egstart
\begin{verbatim}
Here is some advice to remember:
\begin{quotation}
Environments for making
...other things as well.

Many problems
...environments.
\end{quotation}
\end{verbatim}
\egmid%
Here is some advice to remember:
\begin{quotation}
Environments for making quotations
can be used for other things as well.

Many problems can be solved by
novel applications of existing
environments.
\end{quotation}
\egend

The \cs{tabbing} environment overcomes this problem. Within it you set
tabstops and tab to them much like you do on a typewriter.  Tabstops are
set with the |\=| command, and the |\>| command moves to the
next stop.  The
|\\| command is used to separate each line.  A line that ends |\kill|
produces no output, and can be used to set tabstops:


\begin{teX}
\begin{tabbing}
 Income \=Expenditure \= \kill
 Income \>Expenditure \>Result\\
 20s 0d  \>19s 11d \>Happiness\\
 20s 0d  \>20s 1d  \>Misery \\
\end{tabbing}
\end{teX}

\smallskip

\begin{tabbing}
Income \=Expenditure \=    \kill
Income \>Expenditure \>Result \\
20s 0d \>19s 11d \>Happiness   \\
20s 0d \>20s 1d  \>Misery    \\
\end{tabbing}


Unlike a typewriter's tab key, the |\>| command always moves to the next
tabstop in sequence, even if this means moving to the left.  This can cause
text to be overwritten if the gap between two tabstops is too small.


\begin{comment}
\section{Environment}

\begin{teX}
\def\beginstory{
  \vskip 0.5in                 % Skip down 1/2 before story
   \begingroup                  % Start of formatting properties
   \leftskip 1in\rightskip 1in  % Wider margins for narrower text
   \itshape                     % Italic font
   \noindent{.\dotfill{}.\par}  % Make dotted line
	% Text after close of \beginstory will be story formatted
}

\def\endstory{
  \par\noindent{.\dotfill{}.\par}  % Make dotted line
  \endgroup                        % End of formatting properties
  \vskip 0.5in                    % Skip final 1/2 inch
}

\beginstory 

  Just a short story 

\endstory

\end{teX}



Here's a story about the formative era of personal computing. I
originally wrote it in 1999, but the point it makes is still valid.
Hope you like it.



\subsection*{The \protect\latex way}
LaTeX implements macros |\begin| and |\end|. These are a generic pair whose argument determines the environment that's being begun or ended.

LaTeX makes it much easier to code environments. Here's a generic environment definition:

|\newenvironment{environment_name}{stuff to do before text}{stuff to do after text}|

That's it -- the \cs{newenvironment} macro takes three arguments:
The name of the environment being created
The stuff to do before the text being assigned that environment
The stuff to do after the text being assigned that environment

The resemblance to \tex paired macros is obvious, but \latex  environments make it generic across all environments, and place the beginning and ending code in one place. Not only that, but because the environment has one name instead of two different names, it's very easy for a front end like LyX to assign environments to highlighted stretches of text.

The |\newenvironment|  macro works only when the environment name is undefined. If there's already an environment with that name, use|\renewenvironment|  instead. If you don't know, there are ways to test.

The following is a LaTeX version of the |\beginstory| \ldots | \endstory| example, with the two macros folded into the definition of one environment called story.


Lamport, cleverly defined macros that automatically, create the necessary \tex \cs{begingroup} and \cs{endgroup} commands. You can find the code in the |source2e| file and which is shown below:\footnote{You can find the full code in \texttt{File y, for ltmiscen.dtx}}

\begin{teX}
\def\begin#1{%
  \@ifundefined{#1}%
  {\def\reserved@a{\@latex@error{Environment #1 undefined}\@eha}}%
  {\def\reserved@a{\def\@currenvir{#1}%
  \edef\@currenvline{\on@line}%
  \csname #1\endcsname}}%
  \@ignorefalse
  \begingroup\@endpefalse\reserved@a}


\def\end#1{%
  \csname end#1\endcsname\@checkend{#1}%
  \expandafter\endgroup\if@endpe\@doendpe\fi
  \if@ignore\@ignorefalse\ignorespaces\fi}
\end{teX}

In \latex environments are defined as 
|\begin{foo}| and |\end{foo}| which are are used to delimit environment |foo|.
|\begin{foo}| starts a group and calls |\foo| if it is defined, otherwise it does
nothing.

|\end{foo}| checks to see that it matches the corresponding |\begin| and if so,
it calls |\endfoo| and does an |\endgroup|. Otherwise, |\end{foo}| does nothing.
If |\end{foo}| needs to ignore blanks after it, then |\endfoo| should globally set
the |@ignore| switch true with |\@ignoretrue| (this will automatically be global).

NOTE: |\@@end| is defined to be the |\end| command of TEX82.
|\enddocument| is the user's command for ending the manuscript file.
|\stop| is a panic button to end \tex in the middle.


\section*{Checking the environment}

This is interesting in that we can use \cs{@currenvir} to check if a command is within a particular environment. The following code will be used to typeout the environment.

\begin{teX}
\begin{enumerate}
   \item Check environment with |@currenvir|
   \makeatletter
   \item The current environment is \@currenvir
   \makeatother
\end{enumerate} 
\end{teX}

\begin{enumerate}
\item Check environment with |@currenvir|
\makeatletter
 \item The current environment is \@currenvir
\makeatother
\end{enumerate}

The \cs{@checkend} \index{Latex kernel=@checkend} uses the \cs{@currenvir}\index{Latex kernel=@currenvir} to see if there is a matching
begin environment and if it cannot find it produces an error.

\begin{teX}
\def\@checkend#1{%
   \def\reserved@a{#1}
   \ifx\reserved@a\@currenvir 
   \else
     \@badend{#1}
   \fi
}
\end{teX}

It is a pity that there is no real guide for explaining the \latex macros, other than just reading through them. Lamport and later his associates managed to produce code that offers the user a friendly API. Besides the scenes of this API, it also offers the package writers hundreds of useful commands.
\end{comment}
