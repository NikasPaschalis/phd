\subsection{Coptic calendar}

The Coptic calendar or Alexandrian calendar, which is a descendant of the Ancient Egyptian calendar, is still used by the Coptic Orthodox Church and still used in Egypt.   This calendar is based on the ancient Egyptian calendar. To avoid the calendar creep of the latter, a reform of the ancient Egyptian calendar was introduced at the time of Ptolemy III (Decree of Canopus, in 238 BC) which consisted of the intercalation of a sixth epagomenal day every fourth year. However, this reform was opposed by the Egyptian priests, and the idea was not adopted until 25 BC, when the Roman Emperor Augustus formally reformed the calendar of Egypt, keeping it forever synchronized with the newly introduced Julian calendar. To distinguish it from the Ancient Egyptian calendar, which remained in use by some astronomers until medieval times, this reformed calendar is known as the Coptic calendar. Its years and months coincide with those of the Ethiopian calendar but have different numbers and names.

\subsubsection{Coptic Year}

The Coptic Year retained the ancient Egyptian civil year and its divisions of three seasosn, four months each.  The three seasons are commemorated by special prayers in the Coptic Divine Liturgy. This calendar is still in use all over Egypt by farmers to keep track of the various agricultural seasons. The Coptic calendar has 13 months, 12 of 30 days each and an intercalary month at the end of the year of 5 or 6 days depending whether the year is a leap year or not. The year starts on 11 September in the Gregorian Calendar or on the 12th in the year before (Gregorian) Leap Years. The Coptic Leap Year follows the same rules as the Gregorian so that the extra month always has 6 days in the year before a Gregorian Leap Year. The Coptic Year is divided as follows:


\subsubsection{Coptic Months}  The Coptic calendar has thirteen months. 

\begin{table}[p]
\begin{scriptexample}[\arial]{Script}

\captionof{table}{Coptic calendar months.}
{\small
\begin{tabular}{l >{\pan}p{1.3cm} >{\pan\raggedright }p{1.3cm} >{\raggedright}p{1.3cm} l l >{\RaggedRight}p{3cm}}
\toprule
      &Bohairic & Sahidic & Coptic &Arabic &Date &Origin\\
\midrule      
1	& Ⲑⲱⲟⲩⲧ
      &{\pan Ⲑⲟⲟⲩⲧ} 	
      &Thout	  
      &Tout	
      &11 Sept – 10 Oct	
      &Akhet (Inundation)	Thoth, god of Wisdom \& Science\\
      
 2	&Ⲡⲁⲟⲡⲓ	&Ⲡⲁⲱⲡⲉ	&Paopi	&Baba	&11 Oct – 10 Nov	 &Akhet (Inundation)	Hapi, god of the Nile (Vegetation)\\
  
 
3	&Ⲁⲑⲱⲣ	&Ϩⲁⲑⲱⲣ	&Hathor	&Hatour	&10 Nov – 9 Dec	&Akhet (Inundation)	Hathor, goddess of beauty and love (the land is lush and green)\\  

4	&Ⲭⲟⲓⲁⲕ	&Ⲕⲟⲓⲁⲕ	&Koiak	&Kiahk	&10 Dec – 8 Jan	&Akhet (Inundation)	Ka Ha Ka = Good of Good, the sacred Apis Bull\\   
     
5	&Ⲧⲱⲃⲓ	&Ⲧⲱⲃⲉ	&Tobi	&Touba	&9 Jan – 7 Feb 	&Proyet, Peret, or Poret (Growth)	Amso Khem, a form of Amun-Ra (growth of nature and rain)\\
     
6	&Ⲙⲉϣⲓⲣ	&Ⲙϣⲓⲣ 	&Meshir	&Amshir	&8 Feb – 9 Mar &Proyet, Peret, or Poret (Growth)	Mechir, genius of wind (month of storms and wind) \\   

 7	&Ⲡⲁⲣⲉⲙϩⲁⲧ	&Ⲡⲁⲣⲙ̀ϩⲟⲧⲡ	&Paremhat	&Baramhat	&10 Mar – 8 Apr	 &Proyet, Peret, or Poret (Growth)	Mont, god of war (high temperatures; month of the sun)\\

8	&Ⲫⲁⲣⲙⲟⲩⲑⲓ	&Ⲡⲁⲣⲙⲟⲩⲧⲉ	&Parmouti	&Baramouda	&9 Apr – 8 May	&Proyet, Peret, or Poret (Growth)	Renno, severe wind and death (vegetation ends; earth is dry)\\ 

9	&Ⲡⲁϣⲟⲛⲥ	&Ⲡⲁϣⲟⲛⲥ	&Pashons	&Bashans	&9 May – 7 Jun	&Shomu or Shemu (Harvest)	Khenti, a form of Horus, god of metals\\

10	&Ⲡⲁⲱⲛⲓ	&Ⲡⲁⲱⲛⲉ	&Paoni	&Ba'ouna	&8 Jun – 7 Jul   &	Shomu or Shemu (Harvest)	p3-n-In = valley festival\\


11	&Ⲉⲡⲓⲡ	 &Ⲉⲡⲓⲡ	  &Epip	&Abib	&8 Jul – 6 Aug	&Shomu or Shemu (Harvest)	Apida, the serpent that Horus, son of Osiris, killed\\

12	&Ⲙⲉⲥⲱⲣⲓ	&Ⲙⲉⲥⲱⲣⲏ	&Mesori	&Mesra	 &7 Aug – 5 Sept	&Shomu or Shemu (Harvest)	Mesori, birth of the sun\\

13	&Ⲡⲓⲕⲟⲩϫⲓ  ⲛ̀ⲁ̀ⲃⲟⲧ	&Ⲕⲟⲩϫⲓ  ⲛ̀ⲁ̀ⲃⲟⲧ	 &Pi Kogi Enavot	&Nasie	&6 Sep – 10 Sep &	Shomu or Shemu (Harvest)	The Little Month  \\                

\bottomrule    
\end{tabular}

}
\vspace*{3cm}
\end{scriptexample}

\end{table}









                  
                  
                  
                  