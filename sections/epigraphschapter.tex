%%%%%%%%%%% EPIGRAPHS %%%%%%%%%%%%%%%%%%%%%%%%%%%%%%%%%%
\cxset{custom = stewart}
\cxset{chapter toc=true,
       numbering=arabic,
       image=hine02,
       name=Chapter}
\cxset{blank page text=\epigraph{The great tragedy of science is the slaying of a beautiful theory
by an ugly fact.}{Thomas Huxley}}

\chapter{Epigraphs}\index{epigraphs}
\epigraph{Example is the school of mankind, and they
will learn at no other.}{Letters on a Regicide Peace}



\section{Introduction}
Epigraphs or quotations before or after chapters are quite common in books. Peter Wilson's epigraph package, 
does a good job and we have adapted it where necessary to allow for a key value interface. The command:

\cs{epigraph}\marg{text}\marg{source}. By default the epigraph is placed at the right
hand side of the textblock, and the \marg{source} is typeset at the bottom right of the \marg{text}. 
Numerous settings allow for manipulating the width of the epigraph, the location and other 
variables. If the package is available we use it otherwise we use other internal commands.



\section{Key-value interface}
The key value interface provided by the package is shown below. It mostly follows the 
naming conventions of the epigraph package to make the transition easier for experienced users.
\medskip

\keyval{epigraph align}{\marg{left, center, right}}{A font-size command such as \cs{footnotesize}, 
\cs{small} and other similar commands.}

\keyval{epigraph rule width}{\marg{dim}}{A font-size command such as \cs{footnotesize}, \cs{small} 
and other similar commands.}

\keyval{epigraph font-size}{\marg{dim}}{A font-size command such as \cs{footnotesize}, \cs{small} and 
other similar commands.}

\keyval{epigraph beforeskip}{\marg{dim}}{Space before the epigraph.}
\keyval{epigraph afterskip}{\marg{dim}}{Space after the epigraph.}
\keyval{epigraph source align}{\marg{left, center, right}}{Align the source text to the right, left or center.}
\keyval{epigraph source font-size}{\marg{dim}}{Align the source text to the right, left or center.}
\keyval{epigraph source font-shape}{\marg{dim}}{Align the source text to the right, left or center.}
\keyval{epigraph source font-family}{\marg{dim}}{Align the source text to the right, left or center.}
\keyval{epigraph source font-weight}{\marg{bold,normal}}{Align the source text to the right, left or center.}


\section{Example usage}
To set the style and an example usage is shown in .

\begin{example}{epigraph example}{}
\cxset{epigraph width=0.5\linewidth,
       epigraph font-size=\small,
       epigraph rule width=0.4pt,
       epigraph align=right,
       epigraph source align=right,
       epigraph text align=right}

%\chapter{Epigraphs}\index{epigraphs}
\epigraph{Example is the school of mankind, and they
will learn at no other.}{Letters on a Regicide Peace}
\end{example}

Another example for a somewhat longer quote:

\begin{example}{epigraph example}{}
\cxset{epigraph width=0.5\linewidth,
          epigraph font-size=\small,
          epigraph rule width=0.4pt,
          epigraph align=left,
          epigraph source align=right,
          epigraph text align=left}
%\chapter{Epigraphs}\index{epigraphs}
\epigraph{Everything written with vitality expresses that vitality; there are no dull subjects, 
only dull minds.}{Raymond Chandler\\\textit{Letters on a Regicide Peace}}
\end{example}

More usage examples can be found in relevant style examples (See Chapter~\ref{ch:41}) for a rather 
nice example with non-traditional alignment.

\section{Epigraphs on empty pages}

When a chapter open on an odd page sometimes the  previous page is left empty. Some book designers 
add the words ``this page left intentionally blank'' and other might add a quote. To add such a quote use:

\begin{tcolorbox}
\begin{lstlisting}
\cxset{blank page text=\epigraph{The great tragedy of science is the slaying of a beautiful theory
by an ugly fact.}{Thomas Huxley}}
\end{lstlisting}
\end{tcolorbox}
