\chapter{The LaTeX3 l3token package}
\label{ch:l3token}

The \tex concept of tokens is central to its operation. In earlier chapters we discussed extensively the use of category codes and other important aspects of \tex’s tokens. Rememeber a \tex token is either a single character or a control sequence such as a the control sequence |\test|.

\begin{texexample}{makeatletter}{}
\ExplSyntaxOn
\group_begin:
\char_set_catcode_letter:N @
\char_set_catcode_letter:N 1
\def\@store1a{AAAA}
\@store1a\\
\token_to_meaning:N @\\
\token_to_meaning:N 1\\
\char_set_catcode_other:N @
\char_set_catcode_other:N 1
\token_to_meaning:N @\\
\token_to_meaning:N 1\\
\group_end:
\ExplSyntaxOff
\end{texexample}

There are sixteen different commands to set the catcode to any of the predefined groups used by \tex. If you cannot remember the catcode number for a tab character, try and remember its command!

\begin{verbatim}
\char_set_catcode_escape:N 
\char_set_catcode_group_begin:N
\char_set_catcode_group_end:N
\char_set_catcode_math_toggle:N
\char_set_catcode_alignment:N
\char_set_catcode_end_line:N
\char_set_catcode_parameter:N
\char_set_catcode_math_superscript:N
\char_set_catcode_math_subscript:N
\char_set_catcode_ignore:N
\char_set_catcode_space:N
\char_set_catcode_letter:N
\char_set_catcode_other:N
\char_set_catcode_active:N
\char_set_catcode_comment:N
\char_set_catcode_invalid:N
\end{verbatim}

\section{Token predicate functions}

\begin{docCommand}{token_if_macro:NTF} { \meta{token} \marg{true code} \marg{false code}}
tests if the \meta{token} is a \tex macro.
\end{docCommand}

\begin{texexample}{Test if is a macro}{}
\ExplSyntaxOn
\csname sometest\endcsname
\expandafter\def\csname sometesti\endcsname{}
\token_if_macro:NTF \par { \PASS } { \FAIL }
\token_if_macro:NTF \minipage { \PASS } {\FAIL }
\token_if_macro:NTF \sometest { \PASS } {\FAIL }
\token_if_macro:NTF \sometesti { \PASS } {\FAIL }
\par
\meaning\sometest
\ExplSyntaxOff
\end{texexample}

 If it is a primitive we can find out, using yet another boolean construction \docAuxCommand*{token_if_primitive:NTF}  We can also check its meaning. It is interesting to note that \docAuxCommand*{par} is not a macro. Interestingly we can view what \tex does when we say |\csname somecs\endcsname|. It justs sets it equal to |\relax|. 
 
 Again this is important in parsing and in automating the generation of commands. For example  in the |phd| package, we allow for a key value to be entered either as a control sequence for example, |\Large| or simply as a |large|. A test could be provided before further processing such type of input.

\begin{texexample}{Test if is a macro}{ex:ifprimitive1}
\ExplSyntaxOn
\token_to_meaning:N \par\\
\token_to_meaning:N \toks
\token_if_primitive:NTF \par { \PASS } { \FAIL }\\
\ExplSyntaxOff
\end{texexample}

Example~\ref{ex:ifprimitive1} can be used to test if a primitive has been redefined (this can be important for your code and to restore its meaning if necessary or issue an error message.  Another test which is available is to check if a token is a macro. 

%% Dangerous??
\begin{texexample}{Test if is a macro}{ex:active}
\ExplSyntaxOn
\group_begin:
\token_if_cs:NTF \char_set_catcode_active:N  { \PASS } { \FAIL }
\group_end:
\ExplSyntaxOff
\end{texexample}

The next set of available commands are helper functions equivalent to the output of |\ifcat| 

\begin{docCommand} {token_if_group_begin:NTF} {\meta{token} \marg{true code} \marg{false code}}
Tests if \meta{token} has the category code of a begin group token (\{) when normal TEX
category codes are in force). Note that an explicit begin group token cannot be tested in
this way, as it is not a valid N-type argument. To test it you have to use |\c_group_begin_token|. This is mostly
used in conjuction with |futurelet| type constructions and or parsing.
\end{docCommand}


\begin{texexample} {Test if group begin} {ex:ifgroubbegin}
\ExplSyntaxOn
 \token_if_group_begin:NTF \c_group_begin_token { \PASS }{ \FAIL }
 \token_if_group_end:NTF    \c_group_end_token { \PASS }{ \FAIL }\par
 \the\catcode`{
\ExplSyntaxOff
\end{texexample}

Behind the scenes |expl3| uses the |\ifcat| primitive to test the token against the catcode values. Constructions for all categories are available and summarized in the test below rather than described.
\begin{texexample} {Test if group begin} {ex:ifgroubbegin}
\ExplSyntaxOn
 \token_if_group_begin:NTF \c_group_begin_token { \PASS }{ \FAIL }
 \token_if_group_end:NTF    \c_group_end_token { \PASS }{ \FAIL }\par
 \token_if_alignment:NTF     \c_alignment_token { \PASS }{ \FAIL }\par
 \token_if_parameter:NTF    \c_parameter_token { \PASS }{ \FAIL }\par
\ExplSyntaxOff
\end{texexample}


\section{LaTeX3 Futurelet type functions}

In Chapter Futurelet, we spend considerable effort to understand how \tex’s futurelet macro works. There is often a need to look ahead at the next token in the input stream while leaving
it in place. This is handled using the “peek” functions. The generic \docAuxCommand*{peek_after:Nw} is
provided along with a family of predefined tests for common cases. As peeking ahead does
not skip spaces the predefined tests include both a space-respecting and space-skipping
version.

\begin{texexample}{Peek ahead ignoring spaces} {}
\ExplSyntaxOn
\peek_catcode_remove_ignore_spaces:NTF =  
    { 
      \PASS  
      \token_if_letter:NTF
          {l_peek_token ~= ~\token_to_meaning:N \l_peek_token \\  } 
          {   }
    } 
    { \FAIL }  
 = abcde \\
\ExplSyntaxOff
\end{texexample}

Most applications would require to recursively pick up tokens from the input stream and only terminated once a special token is found. This is the most powerful method to parse input strings and create really powerful functions. 

You will understand better if we hide the code in a function.

\begin{texexample}{Peek ahead ignoring spaces} {ex}
\ExplSyntaxOn
\cs_new:Npn \checkletter #1 {
\peek_catcode_remove_ignore_spaces:NTF #1  
    { 
      \PASS  
      \token_if_letter:NTF
          {l_peek_token ~= ~\token_to_meaning:N \l_peek_token \\  } 
          {   }
    } 
    { \FAIL } }

\checkletter {=} =abcde \par
\checkletter {A} Abcde \par
\ExplSyntaxOff
\end{texexample}

\begin{texexample}{Peek ahead ignoring spaces} {}
\ExplSyntaxOn
\cs_set:Npn \check_letter_and_removeall #1 {
\peek_catcode_remove_ignore_spaces:NTF #1  
    { 
      \PASS  
      \removeallaux:w  
    } 
   { \FAIL } 
 }

\cs_set:Npn \removeallaux:w #1; { removed~#1~ }

\check_letter_and_removeall {W}  W 12pt; \par
\ExplSyntaxOff
\end{texexample}


\begin{texexample}{ex:recursivefl}  { }                            
\lorem 
\ExplSyntaxOn
\def\recurse {
   \peek_catcode_remove_ignore_spaces:NTF ; 
   {TRUE} 
   {
     \recurse
   } abcdouiop;
}
\ExplSyntaxOff
\end{texexample}















