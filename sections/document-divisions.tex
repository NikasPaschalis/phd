\chapter{Document Divisions}

The publishing world has different names for different type of documents \emph{books, journals, articles, reports}. The |phd| package simplifies the production and styling of these documents and their divisions. 

\def\test{}
\cxset{document levels/.code =\test}
\cxset{document levels={a1,a2,a3},}

\begin{key}{/chapter/document type = book}
\end{key}
\begin{key}{/chapter/document levels = \meta{book, part, chapter, section, subsection, subsubsection, paragraph, subparagraph}}
\end{key}

Unlike the standard classes or classes such as the |memoir| \citep{memoir} and |koma|, the |phd| class comes in a single form, but is capable of typesetting most document types. If a type is not available in the standard library it can be easily created by forking one of the existing ones. 

\latex introduced the concept of \emph{document division levels}
\begin{table}[h]
 \centering
 \caption{Document division levels}\label{tab:levels}
 \begin{tabular}{lr} \hline
   Division & Level \\ \hline
   book          & -2 \\
   part          & -1 \\
   chapter       &  0 \\
   section       &  1 \\
   subsection    &  2 \\
   subsubsection &  3 \\
   paragraph     &  4 \\
   subparagraph  &  5 \\ 
 \hline
 \end{tabular}
\end{table}

\begin{key}{/chapter/toc levels = \meta{integer}} The key tells
the typesetting engine how many levels to include in the Table of Contents. An equivalent \latexe command is \cmd{\setcounter}\meta{tocdepth}.

\end{key}



\DescribeMacro{\secdef}
    The macro \cs{secdef} can be used when a sectioning command is
    defined without using \cs{@startsection}. It has two arguments:

    \cs{secdef}\meta{unstarcmds}\meta{starcmds}

    \begin{description}
    \item[\meta{unstarcmds}] Used for the normal form of a
          sectioning command.
    \item[\meta{starcmds}] Used for the $*$-form of a
          sectioning command.
    \end{description}

    You can use \cs{secdef} as follows:
 \begin{verbatim}
       \def\chapter { ... \secdef \CMDA \CMDB }
       \def\CMDA    [#1]#2{ ... }  % Command to define
                                   % \chapter[...]{...}
       \def\CMDB    #1{ ... }      % Command to define
                                   % \chapter*{...}
 \end{verbatim}


Perhaps the most dominant part in defining the stylistic aspects of a document is the styling of the document subdivisions such as the sections and chapter heads.



