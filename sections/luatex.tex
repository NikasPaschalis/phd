\chapter{LuaTeX}

\begin{texexample}{Checking for Lua}{}
\ifluatex
  LuaTEX is running
\else
  Without LuaTEX
\fi

\ifxetex
 XeTeX is running
\else
 Without XeTeX
\fi
\end{texexample}

\section{Getting started}
For an introduction to the most important gotchas of \string\directlua, see lualatex-doc.pdf.
Before presenting the tools in this package, let me insist that the best way to manage a nontrivial piece of Lua code is probably to use an external file and source it from Lua, as explained in the cited document.

\section{Basic print statements}

To print you can use the Lua command \lstinline{print}.

\directlua{tex.print(2+2)}

Lua can handle subtraction, negative numbers, numbers with decimal points, multiplication (using *),
division (using /), exponentiation (using \texttt{\^} ), and combinations of these. Here are some examples:
\emphasis{directlua, tex,print}
\begin{tcblisting}{}
\directlua{tex.print(2+2^3)}
\end{tcblisting}

\subsection{Scientific notation}

You can also write numbers using the \textit{scientific notation}, where the aprt before the upper multiplied by 10 to the power after $\textup{e}$

\begin{tcblisting}{}
\directlua{tex.print(1.2193263111264E17 / 987654321)}
\end{tcblisting}

\subsection{Hexadecimal numbers}
Lua also understands hexadecimal (base 16) numbers, using the letters
a-f (or A-F) to represent 10 through 15. Hexadecimal numbers should start with \texttt{0x} or \texttt{0X}.
\begin{tcblisting}{}
\directlua{tex.print(0XA)}
\end{tcblisting}


\section{Lua variables}

\begin{tcblisting}{}
\directlua{
   A=2^0.5
   B=42 
   tex.print(A+B)
}
\end{tcblisting}

Lua is case sensitive, which means NUM, Num and num are different variables.

\begin{tcblisting}{}
\directlua{
   NUM=1
   Num=2 
   num=3
   tex.print(NUM+Num+num)
}
\end{tcblisting}


\subsection{Multiple Assignment}

You can assign multiple values to multiple variables at the same time. You can also print multiple values
at the same time. The comma is used for both. Here’s how you use the comma for multiple assignment,
and for printing multiple values at the same time:

\begin{tcblisting}{}
\directlua{
  Product, Exponent = 10*10, 999
  A,B=1,2
  tex.print(Product, Exponent)
  tex.print(A,B)
}
\directlua{
    A,B=1,2
    tex.sprint(A,B)
}
\end{tcblisting}



\section{Strings}
The double quote characters (") mark the beginning and end of the string. Marking the beginning and
end is all they do; they are not actually part of the string, which is why print doesn't print them, as in
this example:

\begin{tcblisting}{}
\directlua{
 tex.sprint("This is a string")
}
\end{tcblisting}

Like numbers, strings are values, which means they can be assigned to variables. Here's an example

\begin{tcblisting}{}
\directlua{
  Name, Phone = "Jane X. Doe", "248-555-5898"
  tex.sprint(Name, Phone)
}
\end{tcblisting}

\subsection{Quoting Strings with Square Brackets}

\begin{tcblisting}{}
\directlua{
 tex.sprint([[
    There are some 
    funky  characters in this
    string. 
]])
}
\end{tcblisting}

Now what do you do if you want to print to square brackets together?

\begin{tcblisting}{}
\ttfamily \directlua{
 tex.sprint([=[
    [[There]] are some 
    funky  characters in this
    string. 
]=])
}
\end{tcblisting}

\subsection{Escaping Characters}

\begin{tcblisting}{}
\ttfamily \directlua{
 tex.sprint([=[
    [[There]] are some 
    funky \string\\ characters in this \string\\
    string. 
]=])
}
\end{tcblisting}


\begin{tcblisting}{}
\ttfamily \directlua{
 tex.sprint("2"+21)
}
\end{tcblisting}

\subsection{Concatenation operator}

\begin{tcblisting}{}
\ttfamily \directlua{
 tex.sprint("99"..21) -- this is a comment
}
\end{tcblisting}

\section{Comments}
\begin{tcblisting}{}
\ttfamily \directlua{
 tex.sprint("99"..21) -- this is a comment
}
\end{tcblisting}

\section{Conditionals}
\begin{tcblisting}{}
\directlua{
 a=3
 b=12^2
 if a<b then 
   tex.sprint("$a<b^2$ "..a..", "..b) 
 else
 end
}
\end{tcblisting}

\subsection{while loop}
\begin{tcblisting}{}
\directlua{
counter = 1
 while counter <= 100 do
 tex.print(counter)
 counter = counter + 1
 end
}
\end{tcblisting}

\let\exec\directlua
\subsection{for loop}
\begin{tcblisting}{}
\exec{
for N = 1, 10 do
 tex.print(N)
 end
}
\end{tcblisting}

\subsection{repeat until}
\let\exec\directlua
\begin{tcblisting}{}
\exec{
  N,Z=1,21
  F=1
  repeat
    tex.print(N+Z)
    N=N+1
    F=F+10
  until F>=99
  tex.print(Z)
}
\end{tcblisting}

\subsection{The break and do Statements}
\begin{tcblisting}{}
\exec{
for N = 1, 10 do
 if N > 5 then
 break
 end
 tex.print(N)
 end
}
\end{tcblisting}

\chapter{Functions}

\begin{tcblisting}{}
\exec{
function Average(Num1, Num2)
 return (Num1 + Num2) / 2
 end
 tex.print(Average(10,12))
}
\end{tcblisting}

Here is one that prints the Ackerman function. This is an interesting function as the value grows very quickly and can cause a stack overflow. It can also slow compilation quite a bit for higher values. ack(4,1) will cause a stack overflow. Whereas ack(3,11) compiles giving a value of 16381 ack(3,12) causes a stack overflow.  

\[
\begin{matrix}
   a\uparrow\uparrow b & = {\ ^{b}a}  = & \underbrace{a^{a^{{}^{.\,^{.\,^{.\,^a}}}}}} & 
   = & \underbrace{a\uparrow (a\uparrow(\dots\uparrow a))} 
\\  
    & & b\mbox{ copies of }a
    & & b\mbox{ copies of }a
  \end{matrix}
\]

\[
\begin{bmatrix}
   a\times b & = & \underbrace{a+a+\dots+a} \\
   & & b\mbox{ copies of }a
\end{bmatrix} 
\]


\begin{tcblisting}{}
\exec{
function ack(M,N)
    if M == 0 then return N + 1 end
    if N == 0 then return ack(M-1,1) end
    return ack(M-1,ack(M, N-1))
end
tex.sprint(ack(3,8))
}
\end{tcblisting}


\subsection{Returning multiple values}
This return an error if Arg1 is a number
\begin{tcblisting}{}
\exec{
function ReturnArgs(Arg1, Arg2, Arg3)
  return Arg1, Arg2, Arg3
end
   tex.sprint(ReturnArgs("aaaa", 2, 3))
function Fact(N)
  local Ret = 1
  for I = 1, N do
    Ret = Ret * I
  end
  return Ret
  end
tex.print(Fact(10))
}
\end{tcblisting}


\begin{tcblisting}{}
\exec{
function Fact(N)
  local Ret = 1
  for I = 1, N do
    Ret = Ret * I
  end
  return Ret
  end
tex.print(Fact(10))
}
\end{tcblisting}

\subsection{Anonymous Functions}
\begin{tcblisting}{}
\exec{
(function(A, B)
   tex.print(A + B^2)
 end)(2, 3)
}
\end{tcblisting}

\chapter{Working with Tables}
This chapter explores a new data type called a table. It's a data structure, which means that it lets
you combine other values. Because of its flexibility, it is Lua’s only data structure. (It is possible to
create other, special-purpose data structures in C.)
\begin{tcblisting}{}
\begin{filecontents*}{test.lua}
hash = {}
hash[ "key-1" ] = "val1"
hash[ "key-2" ] = 1
hash[ "key-3" ] = {}
tex.sprint(hash["key-2"])
\end{filecontents*}
\directlua{dofile("test.lua")}
\end{tcblisting}


\section{Using dofile}

Although using \lstinline{directlua} has been a useful function so far, this is limiting in many respects.
\begin{filecontents*}{test.lua}
 function test(N)
  return N
end
tex.print(test(5))
\end{filecontents*}

\begin{texexample}{Working with lua file}{ex:dofile}
\begin{filecontents*}{test.lua}
 function test(N)
   return N
 end
 tex.print(test(5))
\end{filecontents*}
\directlua{dofile("test.lua")}
\end{texexample}

As we want to keep ourselves within a document, rather than a separate file, we
use the \verb+filecontents+ package to save a lua file on disc and then use it with
\lstinline{dofile}.

\begin{tcblisting}{}
\begin{filecontents*}{test.lua}
hash = {}
hash[ "key-1" ] = "val1"
hash[ "key-2" ] = 1
hash[ "key-3" ] = {}
\end{filecontents*}
\directlua{dofile("test.lua")}
\end{tcblisting}


\begin{tcblisting}{}
\begin{filecontents*}{test.lua}
Squares = {} -- A table constructor can be empty.
 for I = 1, 5 do
   Squares[I] = I ^ 2
 end
 for I = 1, 5 do
   tex.sprint(I .. " squared is " .. Squares[I].."\\\\")
 end
\end{filecontents*}
\directlua{dofile("test.lua")}
\end{tcblisting}


\begin{tcblisting}{}
\begin{filecontents*}{test01.lua}
 NameToInstr = {John = "rhythm guitar",
 Paul = "bass guitar",
 George = "lead guitar",
 Ringo = "drumkit"}
 for Name, Instr in ipairs(NameToInstr) do
   tex.sprint(Name)
 end
\end{filecontents*}
\directlua{dofile("test01.lua")}
\end{tcblisting}

\clearpage
Using tables that contain functions is a handy way to organize functions, and Lua keeps many of its
built-in functions in tables, indexed by strings. For example, the table found in the global variable table
contains functions useful for working with tables. If you assign another value to table, or to one of the
other global variables used to store built-in functions, the functions won’t be available anymore unless
you put them somewhere else beforehand. If you do this accidentally, just restart the interpreter

\subsection{table.sort}
\begin{tcblisting}{}
\begin{filecontents*}{test.lua}
Names = {"Scarlatti", "Telemann", "Corelli", "Purcell",
   "Vivaldi", "Handel", "Bach","Yiannis"}
table.sort(Names)
 for I, Name in ipairs(Names) do
   tex.sprint(I.." ".. Name.."\\par")
 end
\end{filecontents*}
\directlua{dofile("test.lua")}
\end{tcblisting}

\subsection{table.concat}
\subsection{table.sort}
\begin{tcblisting}{}
\begin{filecontents*}{test.lua}
require "lualibs"
-- require "luasql.mysql"
-- commas (and spaces):
 function CommaSeparate(Arr)
   return table.concat(Arr, ", ")
 end
 tex.print(CommaSeparate({"a", "bc", "d"}))
 FileStr = os.tmpname()
    tex.sprint("\\string"..os.tmpname().." \\% temporary file name")
 Hnd = io.open(FileStr, "w")
 if Hnd then
   tex.sprint("Opened temporary file ", FileStr, " for writing\n")
   Hnd:write("Line 1\nLine 2\nLine 3\n")
Hnd:close()
for Str in io.lines(FileStr) do
  io.write(Str, "\n")
end
  os.remove(FileStr)
else
   io.write("Error opening ", FileStr, " for writing\n")
end
\end{filecontents*}

\directlua{dofile("test.lua")}
\end{tcblisting}

\chapter{Packages}
The luacode package provides commands to make entering of code from a teX
file easier. It provides a better escape mechanism

\emphasis{luaexec,luacode}
\begin{tcblisting}{}
\def\foo{156789.0001}
\(
 \luaexec{
 texio.write_nl("Special chars: _ ^ & $ { } ~ working.\string\n"
 .. "Backslashes still need a bit of care.\string\n"
 .. "Single sharps are easier now: \#")
  % a tex comment: we also get a % below
 tex.sprint("\\pi \\neq ", tostring(math.pi):gsub('\%.', '+'))
 % we can use TeX macros
tex.sprint("-", math.sqrt(\foo))
 }
\)
\end{tcblisting}
\section{Using \texttt{luacode}}
The luacode environment is similar to luaexec
You can pass a macro, which must be fully expandable in order
to work properly.
\begin{tcblisting}{}
\edef\foo{156789.0001}
\[
 \begin{luacode}
    tex.sprint("a=", math.sqrt(\foo),"+ z^3")
 \end{luacode}
\]
\end{tcblisting}

\section{Creating new environments}
\emphasis{luacode,endluacode,tex}
If you need to create new environment you will have to use the \lstinline!\luacode...\endluacode! form  rather than the begin...end.
\medskip

\begin{tcblisting}{}
\begin{luacode}
  require "lfs"
  local temp=lfs.currentdir()
 -- tex.sprint(-2, temp)

  function attrdir (path)
    for file in lfs.dir(path) do
        if file ~= "." and file ~= ".." then
            local f = path..'/'..file
            tex.sprint (-1, f.."\\par     ")
            local attr = lfs.attributes (f)
            assert (type(attr) == "table")
            if attr.mode == "directory" then
                attrdir (f)
            else
               -- for name, value in pairs(attr) do
                    --tex.sprint (-2,name, value)
                     -- tex.sprint (-2,name)
               -- end
            end
        end
    end
  end
  local z="C:/test"
  attrdir (z)
\end{luacode}
\end{tcblisting}

\chapter{Finding where everything is}

One of the most frustrating things when you starting with a new language, is to find out where
everything is and if the various paths are correct. The snippet below does just that. It tests for a file using \lstinline!kpse.find_file!
\bigskip

\emphasis{kpse,find}
\begin{tcblisting}{}
\begin{filecontents}{atest.tex}
\end{filecontents}
\begin{luacode}
  -- kpse.set_program_name("lualatex")
  -- uncomment if having problems
  local file = "atest.tex"
  local path = kpse.find_file(file, 'tex')
    if path==nil then
     tex.tprint({"\\color{red}"},{"File not found!"})
   else
     tex.tprint({path})
   end
\end{luacode}
\end{tcblisting}

In the example, we created a small file on the fly, using \lstinline!filecontents! package. Then we test using \lstinline!kpse.find_file!. It returns true and we print the full path.

\subsection{the os library}
Getting the operating system type and name
\emphasis{kpse,find,require,os,name,lualibs,type}
\begin{tcblisting}{}
\begin{filecontents}{atest.tex}
\end{filecontents}
\begin{luacode}
  -- get the os name
  require("lualibs-os")
  local G,Y=os.name,os.type
  tex.tprint({G.."\\par"},{Y.."\\par"})
\end{luacode}
\end{tcblisting}

Gettting the runtime of a program

\begin{tcblisting}{}
\begin{filecontents}{atest.tex}
\end{filecontents}
\begin{luacode}
  -- get the os name
  require("lualibs-os")
  tex.sprint (">> ",os.runtime()," ms")
  tex.tprint({-2,os.uuid()})
\end{luacode}
\end{tcblisting}

\subsection{uuid}
\emphasis{os,uuid}
Getting a uuid or a slight variant of it we use \lstinline!os.uuid()!

\section{Reading from a file}


\usepackage{luatextra}


\begin{filecontents}{testdata.dat}
  A  B
  1.0 20
  1.1 21
  1.2 22
\end{filecontents}

\begin{texexample}{Reading from files}{ex:readfile}
\begin{luacode}
  function readtxt()
    file = io.open("testdata.dat", "r")
    for line in file:lines() do
      print(line)
      tex.print(line)
    end
  end
\end{luacode}
\directlua{readtxt()}
\end{texexample}














