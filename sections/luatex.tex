\chapter{LuaTeX}

\section{What is LuaTeX}

Although \tex and \latex can be used to program complex commands, and the proof is this book, they are not a general programming language. Many tasks that can be easily be done
in other programming languages, they are extrememely complicated or very slow when done with \tex. Due tothis limitation many auxiliary programs have been developed to assist in common tasks, such as |bibtex| or |biber| that are used to build bibliographies and |makeindex| or |xindex| to generate indexes. In both cases, sorting a list alphabetically is a relatively easy task, but very complicated using \tex.


Another limitation is \tex's math capabilities. One can just marvel at the efforts of \tex macro authors in achieving what they have achieved. A cursory look at the code in the \pkgname{fp} can give you an idea of the difficulties involved.

To address the need to do more complex functions within \TeX, an extension of \TeX{} called Lua\TeX{} was begun a few years ago.  
(The leaders of the project and main developers are Taco Hoekwater, Hartmut Henkel and Hans Hagen.) The idea was to enhance \TeX{} with a previously existing general purpose programming language. After a careful evaluation of possible candidates, the language chosen was Lua (see \href{http://www.lua.org/}{lua.org}), a powerful, fast, lightweight, embeddable scripting language that has,  of course, a free license suitable to be used with \TeX.

Moreover, Lua is easy to learn and use, and anyone with basic programming skills can use it without difficulty. (Many examples of Lua code can be found in this book, and also in \url{http://rosettacode.org/wiki/Category:Lua},
and \url{http://lua-users.org/}).

Lua\TeX{} is not \TeX{}, but an extension of \TeX{}, in the same way that pdf\TeX{} or \XeTeX{} are also extensions.
In fact, Lua\TeX{} includes pdf\TeX{} (it is an extension of pdf\TeX{}, and offers backward compatibility), 
and also has many of the features of \XeTeX.

Lua\TeX{} is still in a beta stage, but the current versions are usable (the first public beta was launched in 2007,
and when this paper was written on \today, the release used was version \the\luatexversion). 

\section{Is LuaTeX for you}

The choice of TeX engine should in a way be immaterial to you, if you are just thinking about typesetting normal books.

\section{Embedding Lua in a TeX document}

\begin{macro}{\directlua}
Although it is recommended to put Lua code in Lua files, from time to time one may want or need to go Lua in the middle of a document. To this end, LuaTeX has two commands: \cmd{\directlua} and \cmd{\latelua}. They work the same, except \cmd{\latelua} is processed when the page where it appears is shipped out, whereas \cmd{\directlua} is processed at once; the distinction is immaterial here, and what is said of \cmd{\directlua} also applies to \cmd{\latelua}.
\end{macro}

\section{From Lua to TeX}

Inside Lua code, one can pass strings to be processed by TeX with the functions |tex.print()|, |tex.sprint()| and |tex.tprint()|. All such calls are processed at the end of a \cmd{\directlua} call, even though they might happen in the middle of the code. This behavior is worth noting because it might be surprising in some cases, although it is generally harmless.

The function can also can also take an optional number as its first argument; it is interpreted as referring to a catcode table (as defined by \cmd{\initcatcodetable} and \cmd{\savecatcodetable}), and each line is processed by TeX with that catcode regime. For instance (note that with such a minimal catcode table, braces don't even have their usual values):

\begin{texexample}{catcodes}{}
\directlua {tex.enableprimitives('',tex.extraprimitives()) }
\bgroup
\initcatcodetable1
\catcode`\_=0
\savecatcodetable1
\egroup
\directlua{tex.print(1, "_TeX")}
\luadirect{tex.print(1, "_TeX")}
\end{texexample}

The string will be read with |_| as an escape character, and thus interpreted as the command commonly known as |\TeX|. The catcode regime holds only for the strings passed to |tex.print()| and the rest of the document isn't affected.

If the optional number is -1, or points to an invalid (i.e. undefined) catcode table, then the strings are processed with the current catcodes, as if there was no optional argument. If it is -2, then the strings are read as if the result of \cmd{\detokenize}: all characters have catcode 12 (i.e. `other', characters that have no function beside representing themselves), except space, which has catcode 10 (as usual).

\begin{scriptexample}[]{}
Note the |\directlua {tex.enableprimitives('',tex.extraprimitives()) }|, without this directive the extra primitives are not loaded, and the example will produce errors. this is rather hidden in the documention in version \the\luatexversion running under \formatname
\end{scriptexample}

\section{Checking for LuaTeX}

The first thing you will need to program, when developing a \luatex\ package, is to check that the engine is in fact using \luatex. This can be done using the \cs{ifluatex} from the package \pkgname{ifluatex}. 

\begin{texexample}{Checking for Lua}{}
\ifxetex
  Using XeTeX
\else
  \ifluatex
   Using LuaTeX
  \else
   Using LaTeX
  \fi
\fi
\end{texexample}


%---------------
\begin{figure}
\begin{luacode*}
-- Fourier series
function partial_sum(n,x)
    partial = 0;
    for k = 1, n, 1 do 
        partial = partial + math.sin(k*x)/k 
    end;
    return partial
end
-- Code to write PGFplots data as coordinates
function print_partial_sum(n,xMin,xMax,npoints,option)
    local delta = (xMax-xMin)/(npoints-1)
    local x = xMin
    if option~=[[]] then
        tex.sprint("\\addplot["..option.."] coordinates{")
    else
        tex.sprint("\\addplot coordinates{")
    end
    for i=1, npoints do
        y = partial_sum(n,x)
        tex.sprint("("..x..","..y..")")
        x = x+delta
    end
    tex.sprint("}") -- We can write "};" and then it is not necessary to put ";" when used
end
\end{luacode*}
\newcommand\addLUADEDplot[5][]{\directlua{print_partial_sum(#2,#3,#4,#5,[[#1]])}}
\centering
\pgfplotsset{width=15cm, height=7cm}  
\begin{tikzpicture}\small
\begin{axis}[xmin=-0.2, xmax=31.6, ymin=-1.85, ymax=1.85, 
  xtick={0,5,10,15,20,25,30},
  ytick={-1.5,-1.0,-0.5,0.5,1.0,1.5},
  minor x tick num=4,
  minor y tick num=4,
  axis lines=middle,
  axis line style={-}
  ] 
%%\addplot[color=red] {pi/2-x/2};
% SYNTAX: Partial sum 30, from x = 0 to 10*pi, and sampled in 1000 points
\addLUADEDplot[color=blue,smooth]{30}{0}{10*math.pi}{1000};
\end{axis} 
\end{tikzpicture}
\caption{The partial sum $\sum_{k=1}^{30} \frac{\sin(kx)}{k}$ of the Fourier series  of $f(x)=(\pi-x)/2$ illustrating the Gibbs phenomenon.}\label{fig:Gibbs}
\end{figure}

\begin{figure}[htp]
\begin{scriptexample}[]{}
\begin{luacode*}
-- Fourier series
function partial_sum(n,x)
    partial = 0;
    for k = 1, n, 1 do 
        partial = partial + math.sin(k*x)/k 
    end;
    return partial
end
-- Code to write PGFplots data as coordinates
function print_partial_sum(n,xMin,xMax,npoints,option)
    local delta = (xMax-xMin)/(npoints-1)
    local x = xMin
    if option~=[[]] then
        tex.sprint("\\addplot["..option.."] coordinates{")
    else
        tex.sprint("\\addplot coordinates{")
    end
    for i=1, npoints do
        y = partial_sum(n,x)
        tex.sprint("("..x..","..y..")")
        x = x+delta
    end
    tex.sprint("}") -- We can write "};" and then it is not necessary to put ";" when used
end
\end{luacode*}
\newcommand\addLUADEDplot[5][]{\directlua{print_partial_sum(#2,#3,#4,#5,[[#1]])}}
\centering
\pgfplotsset{width=15cm, height=7cm}  
\begin{tikzpicture}\small
\begin{axis}[xmin=-0.2, xmax=31.6, ymin=-1.85, ymax=1.85, 
  xtick={0,5,10,15,20,25,30},
  ytick={-1.5,-1.0,-0.5,0.5,1.0,1.5},
  minor x tick num=4,
  minor y tick num=4,
  axis lines=middle,
  axis line style={-}
  ] 
%%\addplot[color=red] {pi/2-x/2};
% SYNTAX: Partial sum 30, from x = 0 to 10*pi, and sampled in 1000 points
\addLUADEDplot[color=blue,smooth]{30}{0}{10*math.pi}{1000};
\end{axis} 
\end{tikzpicture}
\begin{verbatim}
\begin{luacode*}
-- Fourier series
function partial_sum(n,x)
    partial = 0;
    for k = 1, n, 1 do 
        partial = partial + math.sin(k*x)/k 
    end;
    return partial
end
-- Code to write PGFplots data as coordinates
function print_partial_sum(n,xMin,xMax,npoints,option)
    local delta = (xMax-xMin)/(npoints-1)
    local x = xMin
    if option~=[[]] then
        tex.sprint("\\addplot["..option.."] coordinates{")
    else
        tex.sprint("\\addplot coordinates{")
    end
    for i=1, npoints do
        y = partial_sum(n,x)
        tex.sprint("("..x..","..y..")")
        x = x+delta
    end
    tex.sprint("}") -- We can write "};" and then it is not necessary to put ";" when used
end
\end{luacode*}
\end{verbatim}
\end{scriptexample}
\end{figure}

\section{Getting started}

For an introduction to the most important gotchas of \cmd{\directlua}, see the guide \texttt{lualatex-doc.pdf}, which is available at ctan or possibly in your distribution.

Before presenting the tools in this package, let me insist that the best way to manage a nontrivial piece of Lua code is preferable to use an external file and source it from Lua, as explained in the cited document.

\LuaTeX communicates with TeX with only a few commands. We will explore these commands first.

\section{Basic print statements}

\CMDI{\directlua\{tex.print\}}
To print you can use the Lua command \lstinline{tex.print}.

\directlua{tex.print(2+2)}

Lua can handle subtraction, negative numbers, numbers with decimal points, multiplication (using *),
division (using /), exponentiation (using \texttt{\^} ), and combinations of these. Here are some examples:
\emphasis{directlua, tex,print}
\begin{tcblisting}{}
\directlua{tex.print(2+2^3)}
\end{tcblisting}

\subsection{Scientific notation}

You can also write numbers using the \textit{scientific notation}, where the aprt before the upper multiplied by 10 to the power after $\textup{e}$

\begin{tcblisting}{}
\directlua{tex.print(1.2193263111264E17 / 987654321)}
\end{tcblisting}

\subsection{Hexadecimal numbers}
Lua also understands hexadecimal (base 16) numbers, using the letters
a-f (or A-F) to represent 10 through 15. Hexadecimal numbers should start with \texttt{0x} or \texttt{0X}.
\begin{tcblisting}{}
\directlua{tex.print(0XA)}
\end{tcblisting}


\section{Lua variables}

\begin{tcblisting}{}
\directlua{
   A=2^0.5
   B=42 
   tex.print(A+B)
}
\end{tcblisting}

Lua is case sensitive, which means NUM, Num and num are different variables.

\begin{tcblisting}{}
\directlua{
   NUM=1
   Num=2 
   num=3
   tex.print(NUM+Num+num)
}
\end{tcblisting}


\subsection{Multiple Assignment}

You can assign multiple values to multiple variables at the same time. You can also print multiple values
at the same time. The comma is used for both. Here’s how you use the comma for multiple assignment,
and for printing multiple values at the same time:

\begin{tcblisting}{}
\directlua{
  Product, Exponent = 10*10, 999
  A,B=1,2
  tex.print(Product, Exponent)
  tex.print(A,B)
}
\directlua{
    A,B=1,2
    tex.sprint(A,B)
}
\end{tcblisting}



\section{Strings}
The double quote characters (") mark the beginning and end of the string. Marking the beginning and
end is all they do; they are not actually part of the string, which is why print doesn't print them, as in
this example:

\begin{tcblisting}{}
\directlua{
 tex.sprint("This is a string")
}
\end{tcblisting}

Like numbers, strings are values, which means they can be assigned to variables. Here's an example

\begin{tcblisting}{}
\directlua{
  Name, Phone = "Jane X. Doe", "248-555-5898"
  tex.sprint(Name, Phone)
}
\end{tcblisting}

\subsection{Quoting Strings with Square Brackets}

\begin{tcblisting}{}
\directlua{
 tex.sprint([[
    There are some 
    funky  characters in this
    string. 
]])
}
\end{tcblisting}

Now what do you do if you want to print to square brackets together?

\begin{tcblisting}{}
\ttfamily \directlua{
 tex.sprint([=[
    [[There]] are some 
    funky  characters in this
    string. 
]=])
}
\end{tcblisting}

\subsection{Escaping Characters}

\begin{tcblisting}{}
\ttfamily \directlua{
 tex.sprint([=[
    [[There]] are some 
    funky \string\\ characters in this \string\\
    string. 
]=])
}
\end{tcblisting}


\begin{tcblisting}{}
\ttfamily \directlua{
 tex.sprint("2"+21)
}
\end{tcblisting}

\subsection{Concatenation operator}

\begin{tcblisting}{}
\ttfamily \directlua{
 tex.sprint("99"..21) -- this is a comment
}
\end{tcblisting}

\section{Comments}
\begin{tcblisting}{}
\ttfamily \directlua{
 tex.sprint("99"..21) -- this is a comment
}
\end{tcblisting}

\section{Conditionals}
\begin{tcblisting}{}
\directlua{
 a=3
 b=12^2
 if a<b then 
   tex.sprint("$a<b^2$ "..a..", "..b) 
 else
 end
}
\end{tcblisting}

\subsection{while loop}
\begin{tcblisting}{}
\directlua{
counter = 1
 while counter <= 100 do
 tex.print(counter)
 counter = counter + 1
 end
}
\end{tcblisting}

\let\exec\directlua
\subsection{for loop}
\begin{tcblisting}{}
\exec{
for N = 1, 10 do
 tex.print(N)
 end
}
\end{tcblisting}

\subsection{repeat until}
\let\exec\directlua
\begin{tcblisting}{}
\exec{
  N,Z=1,21
  F=1
  repeat
    tex.print(N+Z)
    N=N+1
    F=F+10
  until F>=99
  tex.print(Z)
}
\end{tcblisting}

\subsection{The break and do Statements}
\begin{tcblisting}{}
\exec{
for N = 1, 10 do
 if N > 5 then
 break
 end
 tex.print(N)
 end
}
\end{tcblisting}

\chapter{Functions}

\cxset{epigraph width=0.6\textwidth,
       epigraph text align=left}
\epigraph{“A computer is like a violin. You can imagine a novice trying first a phonograph and then a violin. The latter, he says, sounds terrible. That is the argument we have heard from our humanists and most of our computer scientists. Computer programs are good, they say, for particular purposes, but they aren’t flexible. Neither is a violin, or a typewriter, until you learn how to use it.” }{Marvin Minsky}

\section{What are functions}

The main mechanism for abstraction of statements and expressions in Lua are \textit{functions}. Function carry out specific tasks. In other languages sometimes they are called a \textit{procedure} or a \textit{subroutine}. Functions are first class citizens in Lua. They can be used both as statements or as expressions.

\begin{texexample}{Functions used in expressions}{ex:expressions}
\begin{luacode}
a = math.sin(2) + math.cos(10)
tex.print(a)
\end{luacode}
\end{texexample}

The arguments are enclosed in parenthese denoting a call. If the function does not have any arguments, we must still call it with an empty list () to denote the call.

\section{Multiple Results}

An unconventional, feature of Lua is that functions can return multiple results. Many predefined functions in Lua return multiple values. 

Example \ref{ex:functions} has two parameters and returns the average of these numbers.

\begin{texexample}{Functions}{ex:functions}
\exec{
function Average(Num1, Num2)
    return (Num1 + Num2) / 2
end
tex.print(Average(10,12))
}
\end{texexample}

Here is one that prints the Ackerman function. This is an interesting function as the value grows very quickly and can cause a stack overflow. It can also slow compilation quite a bit for higher values. ack(4,1) will cause a stack overflow. Whereas ack(3,11) compiles giving a value of 16381 ack(3,12) causes a stack overflow.  

\[
\begin{matrix}
   a\uparrow\uparrow b & = {\ ^{b}a}  = & \underbrace{a^{a^{{}^{.\,^{.\,^{.\,^a}}}}}} & 
   = & \underbrace{a\uparrow (a\uparrow(\dots\uparrow a))} 
\\  
    & & b\mbox{ copies of }a
    & & b\mbox{ copies of }a
  \end{matrix}
\]

\[
\begin{bmatrix}
   a\times b & = & \underbrace{a+a+\dots+a} \\
   & & b\mbox{ copies of }a
\end{bmatrix} 
\]


\begin{tcblisting}{}
\exec{
function ack(M,N)
    if M == 0 then return N + 1 end
    if N == 0 then return ack(M-1,1) end
    return ack(M-1,ack(M, N-1))
end
tex.sprint(ack(3,8))
}
\end{tcblisting}


\subsection{Returning multiple values}

This return an error if Arg1 is a number
\begin{tcblisting}{}
\exec{
function ReturnArgs(Arg1, Arg2, Arg3)
  return Arg1, Arg2, Arg3
end
   tex.sprint(ReturnArgs("aaaa", 2, 3))
function Fact(N)
  local Ret = 1
  for I = 1, N do
    Ret = Ret * I
  end
  return Ret
  end
tex.print(Fact(10))
}
\end{tcblisting}


\begin{tcblisting}{}
\exec{
function Fact(N)
  local Ret = 1
  for I = 1, N do
    Ret = Ret * I
  end
  return Ret
  end
tex.print(Fact(10))
}
\end{tcblisting}

\subsection{Anonymous Functions}
\begin{tcblisting}{}
\exec{
(function(A, B)
   tex.print(A + B^2)
 end)(2, 3)
}
\end{tcblisting}

\chapter{Working with Lua Tables}

This section explores a new data type called a table. It's a data structure, which means that it lets
you combine other values. Because of its flexibility, it is Lua’s only data structure. (It is possible to
create other, special-purpose data structures in C.)
\begin{tcblisting}{}
\begin{filecontents*}{test.lua}
hash = {}
hash[ "key-1" ] = "val1"
hash[ "key-2" ] = 1
hash[ "key-3" ] = {}
tex.sprint(hash["key-2"])
\end{filecontents*}
\directlua{dofile("test.lua")}
\end{tcblisting}


\section{Using dofile}

Although using \lstinline{directlua} has been a useful function so far, this is limiting in many respects.
\begin{filecontents*}{test.lua}
 function test(N)
  return N
end
tex.print(test(5))
\end{filecontents*}

\begin{texexample}{Working with lua file}{ex:dofile}
\begin{filecontents*}{test.lua}
 function test(N)
   return N
 end
 tex.print(test(5))
\end{filecontents*}
\directlua{dofile("test.lua")}
\end{texexample}

As we want to keep ourselves within a document, rather than a separate file, we
use the \verb+filecontents+ package to save a lua file on disc and then use it with
\lstinline{dofile}.

\begin{tcblisting}{}
\begin{filecontents*}{test.lua}
hash = {}
hash[ "key-1" ] = "val1"
hash[ "key-2" ] = 1
hash[ "key-3" ] = {}
\end{filecontents*}
\directlua{dofile("test.lua")}
\end{tcblisting}


\begin{tcblisting}{}
\begin{filecontents*}{test.lua}
Squares = {} -- A table constructor can be empty.
 for I = 1, 5 do
   Squares[I] = I ^ 2
 end
 for I = 1, 5 do
   tex.sprint(I .. " squared is " .. Squares[I].."\\\\")
 end
\end{filecontents*}
\directlua{dofile("test.lua")}
\end{tcblisting}


\begin{tcblisting}{}
\begin{filecontents*}{test01.lua}
 NameToInstr = {John = "rhythm guitar",
 Paul = "bass guitar",
 George = "lead guitar",
 Ringo = "drumkit"}
 for Name, Instr in ipairs(NameToInstr) do
   tex.sprint(Name)
 end
\end{filecontents*}
\directlua{dofile("test01.lua")}
\end{tcblisting}

\clearpage
Using tables that contain functions is a handy way to organize functions, and Lua keeps many of its
built-in functions in tables, indexed by strings. For example, the table found in the global variable table
contains functions useful for working with tables. If you assign another value to table, or to one of the
other global variables used to store built-in functions, the functions won’t be available anymore unless
you put them somewhere else beforehand. If you do this accidentally, just restart the interpreter

\subsection{table.sort}
\begin{tcblisting}{}
\begin{filecontents*}{test.lua}
Names = {"Scarlatti", "Telemann", "Corelli", "Purcell",
   "Vivaldi", "Handel", "Bach","Yiannis"}
table.sort(Names)
 for I, Name in ipairs(Names) do
   tex.sprint(I.." ".. Name.."\\par")
 end
\end{filecontents*}
\directlua{dofile("test.lua")}
\end{tcblisting}

\subsection{table.concat}
\subsection{table.sort}
\begin{tcblisting}{}
\begin{filecontents*}{test.lua}
require "lualibs"
-- require "luasql.mysql"
-- commas (and spaces):
 function CommaSeparate(Arr)
   return table.concat(Arr, ", ")
 end
 tex.print(CommaSeparate({"a", "bc", "d"}))
 FileStr = os.tmpname()
    tex.sprint("\\string"..os.tmpname().." \\% temporary file name")
 Hnd = io.open(FileStr, "w")
 if Hnd then
   tex.sprint("Opened temporary file ", FileStr, " for writing\n")
   Hnd:write("Line 1\nLine 2\nLine 3\n")
Hnd:close()
for Str in io.lines(FileStr) do
  io.write(Str, "\n")
end
  os.remove(FileStr)
else
   io.write("Error opening ", FileStr, " for writing\n")
end
\end{filecontents*}

\directlua{dofile("test.lua")}
\end{tcblisting}

\chapter{Packages}
The luacode package provides commands to make entering of code from a teX
file easier. It provides a better escape mechanism

\emphasis{luaexec,luacode}
\begin{tcblisting}{}
\def\foo{156789.0001}
\(
 \luaexec{
 texio.write_nl("Special chars: _ ^ & $ { } ~ working.\string\n"
 .. "Backslashes still need a bit of care.\string\n"
 .. "Single sharps are easier now: \#")
  % a tex comment: we also get a % below
 tex.sprint("\\pi \\neq ", tostring(math.pi):gsub('\%.', '+'))
 % we can use TeX macros
tex.sprint("-", math.sqrt(\foo))
 }
\)
\end{tcblisting}
\section{Using \texttt{luacode}}
The luacode environment is similar to luaexec
You can pass a macro, which must be fully expandable in order
to work properly.
\begin{tcblisting}{}
\edef\foo{156789.0001}
\[
 \begin{luacode}
    tex.sprint("a=", math.sqrt(\foo),"+ z^3")
 \end{luacode}
\]
\end{tcblisting}

\section{Creating new environments}
\emphasis{luacode,endluacode,tex}
If you need to create new environment you will have to use the \lstinline!\luacode...\endluacode! form  rather than the begin...end.


\endinput

\chapter{Finding where everything is}

One of the most frustrating things when you starting with a new language, is to find out where
everything is and if the various paths are correct. The snippet below does just that. It tests for a file using \lstinline!kpse.find_file!
\bigskip

\emphasis{kpse,find}
\begin{tcblisting}{}
\begin{filecontents}{atest.tex}
\end{filecontents}
\begin{luacode}
  -- kpse.set_program_name("lualatex")
  -- uncomment if having problems
  local file = "phd.dtx"
  local path = kpse.find_file(file, 'dtx')
    if path==nil then
     tex.tprint({"\\color{red}"},{"File not found!"})
   else
     tex.tprint({path})
   end
\end{luacode}
\end{tcblisting}

In the example, we created a small file on the fly, using \lstinline!filecontents! package. Then we test using \lstinline!kpse.find_file!. It returns true and we print the full path.

\subsection{the os library}
Getting the operating system type and name
\emphasis{kpse,find,require,os,name,lualibs,type}
\begin{tcblisting}{}
\begin{filecontents}{atest.tex}
\end{filecontents}
\begin{luacode}
  -- get the os name
  require("lualibs-os")
  local G,Y=os.name,os.type
  tex.tprint({G.."\\par"},{Y.."\\par"})
\end{luacode}
\end{tcblisting}

Gettting the runtime of a program

\begin{tcblisting}{}
\begin{filecontents}{atest.tex}
\end{filecontents}
\begin{luacode}
  -- get the os name
  require("lualibs-os")
  tex.sprint (">> ",os.runtime()," ms")
  tex.tprint({-2,os.uuid()})
\end{luacode}
\end{tcblisting}

\subsection{uuid}
\emphasis{os,uuid}
Getting a uuid or a slight variant of it we use \lstinline!os.uuid()!

\section{Reading from a file}

|\usepackage{luatextra}|


\begin{filecontents}{testdata.dat}
  A  B
  1.0 20
  1.1 21
  1.2 22
\end{filecontents}

\begin{texexample}{Reading from files}{ex:readfile}
\begin{luacode}
  function readtxt()
    file = io.open("testdata.dat", "r")
    for line in file:lines() do
      print(line)
      tex.print(line)
    end
  end
\end{luacode}
\directlua{readtxt()}
\end{texexample}














