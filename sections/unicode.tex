\chapter{Unicode}

Unicode is an encoding of \textit{characters}, and it is the first encoding that took the trouble to define what a
\textit{character} is. The distinction between a character and a \textit{glyph} has come to be of interest een to philosophers with the Japanese philosopher Shigeki Moro to say that Unicode’s approach is Aristotelian essentialist. 

In this book we adopt the practical definition given by Spyropoulos in his book Unicode \& Encodings. 

\begin{itemize}

\item A glyph is the image of a symbol used in a writing system (in an alphabet, a syllabary, a set of ideographs, etc.) or in a notational system (like music, mathematics, cartography etc.)

\item A \textit{character} is the simple description, primarily linguistic or logical, of an equivalence class of glyphs.
\end{itemize}

\section{Unicode's Principles.}

Unicode subscribes to ten principles.

