\MakePercentComment
\makeatletter
\cxset{toc image=\@empty,
       chapter toc=true}

\@specialtrue


\renewcommand\stewart[2][]{%
\fancypagestyle{fancy}{%
\lhead{}\rhead{}
\chead{}
\cfoot{}
\lfoot{}
\rfoot{\thepage}
\def\footrule#1{{\color{blue}%
  \hrule width\paperwidth}\vskip3pt
}

\renewcommand{\headrulewidth}{0pt}
\renewcommand{\footrulewidth}{0.4pt}}

\clearpage

\begin{tikzpicture}[remember picture,overlay]
% Main shading block
\node [xshift=5cm,yshift=-\paperheight] at (current page.north west)
[text width=0.98\textwidth,text height=\paperheight, fill=thecream!30,rounded corners,above right]
{};
\node [xshift=6.5cm,yshift=-1.5cm-\soffsety] at (current page.north west)
[text width=0.9\textwidth,below right]{\sffamily \bfseries \huge #2};

\node [xshift=3cm,yshift=-1.5cm] at (current page.north west)
[text width=3cm,align=center,minimum height=2.5cm, fill=blue,below right]
{\[\text{\HHUGE\bfseries\sffamily\color{white}\thechapter}\]
\par\vspace*{3pt}
};

\node [xshift=-0.2cm,yshift=-21.5cm] at (current page.north west)
[text width=3cm,above right]%
{\includegraphics[width=1.0\paperwidth]{\image@cx}};
% second box left
\node [xshift=3cm,yshift=-19.5cm] at (current page.north west)
[text width=9cm,minimum height=2.5cm,inner sep=0.5em, fill=blue,below right]
{\color{white}
  \bfseries\sffamily \texti@cx
};
% Last block
\node [xshift=6.5cm,yshift=-26cm] at (current page.north west)
[text width=12cm,above right]
{\textii@cx
};
\end{tikzpicture}
\par
\clearpage
}





\cxset{steward,
  numbering=arabic,
  custom=stewart,
  offsety=0cm,
  image= {./images/hine02.jpg},
  texti={When Lamport designed the original \LaTeX\ sectioning commands he did not provide a fully comprehensive interface for modifying their design. With current tools available improvements are much easier to program and this chapter provides the details.},
  textii={\precis{In this chapter we discuss a method that allows the production of fancy chapter headings and formatting, based on a set of key values. Central  to this process is the separation of content from presentation.
We also discuss the basic formatting tools that are available and how one can modify them to mould new book designs.}
 }
}


\chapter{Designing Chapter Headings}
\addtocimage{-12pt}{-20pt}{./images/tocblock-man-01.jpg}

\section{Introduction}

A crowded first page is as unsightly as a crowded title page, so wrote De Vinne in \emph{Modern Methods of Book Composition} in 1904.  Not much has changed since. A new chapter must make a good impression and must give an immediate signal that a different topic is going to be discussed. Traditionally chapter openings in LaTeX are an unimpressive and dry event. Our aim is to brighten it up a bit, while keeping true separation of content from presentation, but avoiding the pit traps of over ornamenting the design. A book is to be read and we should provide minimal ornamentation. 



\section{Package Usage}

To use the package include it just like any other package:


\begin{teX}
\documentclass{book}
\usepackage{phd}
\cxset{style13}
\begin{document}
\chapter{Introduction}
\end{document}
\end{teX}

The command \cs{cxset} sets the default style for the example to the style defined as \marg{style13}. The package currently offers  100 templates and numerous keys to manipulate them further. Styles are similar to \enquote{themes} used in web programming; they are a collection of keys that resemble in many ways \texttt{css}. Styles can have any names and I am sure as package usage increases and evolve,they will get better names. 


\section{Chapter opening page}

The standard LaTeX classes offer only two options to either open a chapter on an odd page or at any page. This package offers five alternatives:

\begin{key}{/phd/ chapter opening = \marg{any, left, right, anywhere, ifafter}}
For documents that are primarily to be read on the web, use |any| for normal books, use |right|. Some templates that we provide use |any| and the examples use |anywhere| to enable us to display the heading at any position on the page.
\end{key}

\begin{marglist}
\item [any] Opens a chapter at any page, either \textit{verso} or \textit{recto}.
\item [left] Opens a chapter on an even page
\item [right] Opens a chapter on a right page.
\item [anywhere] Opens a chapter at the point where the \cs{chapter} is typed.
\item [none] Alias for \marg{anywhere}.
\item [ifafter] Opens a chapter at the next page if the page has material that does not exceed a certain portion of \cs{textheight}.
\end{marglist}

To change a setting you just modify the value of the key \option{chapter opening} to one of the values described earlier. We use this key to print the many examples that follow (see the example~\ref{ex:anywhere}). 



\begin{tcolorbox}
\begin{lstlisting}
\cxset{chapter opening=anywhere}
\end{lstlisting}
\end{tcolorbox}


\@specialfalse


\cxset{chapter toc=false}
\begin{texexample}{title=Inline Chapter Example}{ex:anywhere}
\cxset{chapter opening=anywhere, chapter before=\bigskip, chapter font-family=\sffamily,title font-family=\sffamily}

\lipsum[2]
\chapter{Chapter Example}
\lipsum[3]
\chapter{Another Chapter Example}
\lipsum[4]
\end{texexample}

\addtocounter{chapter}{-1}

Examples for other types of chapter openings follow in the rest of the documentation.

\subsection{Blank pages before chapters}

In the standard LaTeX book class when the \texttt{openany} option is not given or in the report class when the openright is given, chapters start at odd-numbered pages. This can cause a blank page to be printed. Some book designers prefer this page to be completely empty, without any headers or footers. This cannot be done with \lstinline{\thispagestyle} as this command will have to be issued on the \textit{previous} page. However by a suitable redefinition of the
\lstinline{\clearpage} this can be done automatically.
\medskip

\begin{tcolorbox}
\begin{lstlisting}
\makeatletter
\def\cleardoublepage{\clearpage\if@twoside\ifodd\c@page\else
  \hbox{}
  \vspace*{\fill}
  \begin{center}
    This page left intentionally blank.
  \end{center}
  \vspace{\fill}
  \thispagestyle{empty}
  \newpage
  \if@twocolumn\hbox{}\newpage\fi\fi\fi}
\makeatother
\end{lstlisting}
\end{tcolorbox}
\medskip

This is achieved easily by setting the following options:
\bigskip

\begin{tcolorbox}
\lstinline{chapter blank page=empty}\par
\lstinline{chapter blank page text=Some text.}\par
\lstinline{chapter blank page=plain}\par
\end{tcolorbox}
\medskip



The last one refers to a \lstinline!\thispagestyle{plain}!.



\section{Keys for chapter head formatting}

A chapter heading can be considered of being constructed of several parts, the \textit{chapter number}, the chapter name typically \textit{chapter} and the \textit{title}. Predefined keys handle all the elements of formatting. Additional keys are defined to handle other elements such as inclusion of images or producing complicated examples with graphics constructed with \texttt{TikZ} and other similar packages.

\medskip

\begin{key}{/phd/  numbering font-size = \oarg{sizing commands}}
\end{key}

\begin{key}{/phd/ chapter numbering font-family = \oarg{sizing commands}}
\end{key}

\subsection{Keys for numbering}

Chapter numbering follows that of the standard \LaTeX\ classes and is extended to cover some additional cases such as fully spelled out numbers. This of course is only good for languages that use the arabic numeralsn. For other languages numerals in different formats can be added with simple keys and without the need of \pkgname{polyglossia} or \pkgname{babel}. 

Note that the package uses Heiko Oberdiek's package \pkgname{alphalph} to allow for alphabetic numbering that extends beyond the normal 26 letters of the alphabet. Examples for numbering can be seen in \ref{ex:romannumbering}

\medskip

\begin{key}{/phd/ number numbering = \oarg{alph,Alph,roman,Roman,none,WORDS,words,none}}

Style of numbering.
\end{key}

\medskip

\begin{marglist}
\item [arabic] Despite that the Arabs call what the West calls Arabic numbers Indian numbers, we provide the value arabic to have normal numbers printed.
\item [alph] Lowercase alphabetic numbering.
\item [Alph] Uppercase alphabetic numbering.
\item [roman] Lowercase roman numbering.
\item [Roman] Uppercase roman numbering.
\item [words] The number is in lowercase words.
\item [WORDS] The number is in uppercase literal numerals.
\item [Words] Prints the number in words and capitalizes the first letter, for example the number 21 will be printed as `Twenty One'\footnote{Currently limited to the first hundred numbers}.
\item [none] This is equivalent to using the star version of the command. It does not print any number and does not increment the chapter counter.\footnote{I am ambivalent about this, perhaps it will be better to increment it, as it can give a more general approach.}
\end{marglist}

\cxset{chapter opening=anywhere, numbering=Roman}
\index{chapter design>numbering>roman}

%\begin{texexample}{Setting up keys for numbering}{ex:romannumberingx}
%\bgroup
%\cxset{numbering=Roman}
%
%\chapter{Roman numbering}
%
%\egroup
%\end{texexample}


\index{chapter design>numbering>words}

%\begin{texexample}{Literal Numbering}{ex:literal}
%\cxset{numbering=WORDS} 
%\chapter{Literal numbering}
%\cxset{numbering=words,chapter name=chapter}
%\chapter{Literal numbering} TODO
%\end{texexample}


\cxset{numbering=arabic}
\subsection{Setting font information}
\subsection{Letter spacing}

Chapter letter spacing can be achieved using the soul package in a combination with the key |spaceout|.


The following examples illustrate the usage.

\begin{texexample}{Letter Spacing}{ex:letterspacing}
\cxset{numbering=Roman,
         chapter spaceout=none,
         title spaceout=soul,
         title font-size=Large,
         title font-family=rmfamily,
         title font-shape=scshape}
\chapter{Letter Spacing}
\end{texexample}





\section{Styling the chapter title}

Similarly to the number and chapter styling keys exist for styling the chapter title. We summarize the available standard keys below:

\index{chapter design!labels!letter spacing}
\begin{texexample}{Styling the Title}{ex:title} 
\cxset{chapter spaceout=soul,numbering=arabic, title spaceout=soul}
\chapter{Chapter title}
\end{texexample}


\begin{key}{/phd/  title font-family = \marg{family}}Selects a predefined font family
\end{key}

\begin{key}{/phd/ chapter title font-weight = \marg{\cs{bfseries},\cs{normalseries}}} Font weight.
\end{key}

\begin{key}{/phd/ chapter title font-size = \marg{\cs{large},\cs{Large},\cs{huge},\cs{Huge},\cs{HUGE},\cs{HHuge}}}Font sizing commands or their names. Both \cmd{\HUGE} and HUGE are allowed to be used as values for the key.
\end{key}

\begin{key}{/phd/ chapter title font-color = \marg{color}}
The color of the chapter title letters. This takes any predefined color name. 
\end{key}


\begin{key}{/phd/ chapter title spaceout = \marg{soul,none}}
 This key will space out the title. 
\end{key}

\begin{texexample}{}{}
\cxset{chapter name=,
          numbering=none,
          chapter title align=center,
          title font-size=Large,
          title font-color=black,
          chapter title width=0.6\textwidth,
          title spaceout=soul}
\chapter{The Prehistoric Period in South-East Asia: 2300 BC--AD 400}          
\end{texexample}

\begin{key}{/phd/ chapter title before = \marg{soul,none}}
Sets the font color of the title.
\end{key}

\begin{key}{/phd/ chapter title after = \marg{soul,none}}
Hook to add items after the title block.
\end{key}
 
\begin{key}{/phd/ chapter title before skip = \marg{soul,none}}
Before title string skip.
\end{key}
  \begin{key}{/phd/ chapter title after skip = \marg{soul,none}}
After title string skip.
\end{key}

\lorem 
%
%\begin{texexample}{letter spacing the chapter title block}{ex:title3}
%
%\cxset{chapter spaceout=none,
%         numbering=arabic}
%         
%\chapter{Chapter Title Styling}
%\end{texexample}
%
%\end{document}



\cxset{chapter opening=right}
\section{Table of Contents}\index{table of contents!key settings}

Traditionally a chapter will be added to the Table of Contents if the \cs{chapter} command is issued. The starred version will not produce a number and will not add a contents line. Since we have adopted an approach where we use a key value interface we can dispense with the starred version of the command, by setting the \option{chapter toc} option to false. For example if we want to define a command for a ``Foreward'' or ``Epiloque'' without wishing them to be added to the table of contents we can use the following setting.\index{Foreward>definitions}\index{Epilogue>definitions}



\begin{texexample}{changing the chapter label name}{}
\cxset{chapter toc=false, name=, numbering=none,}
\chapter{Foreward}
\lorem
\end{texexample}

Note that the key \option{numbering=none} still has to be set.


Please note that when \textbf{numbering=none} the chapter number is not available anymore and yo may have to reset it if required again. Although this might be seen as rather cumbersome than simply using \cs{chapter*} the advantage is consistency in the user interface and the use of appropriate semantic definitions for all sectioning commands thus achieving a bit more separation of context from style.


\cxset{chapter toc=true}

\section{Defining styles}

Named styles can be defined using the standard \textsc{PGF} conventions. To define a style for the forward above we can use:

\begin{texexample}{}{}
\cxset{foreward/.style={chapter toc=false,numbering=none,
          name=,
          title font-size= Large,
          title font-family= sffamily,
          numbering=none}}
\cxset{foreward}
\chapter{Foreward.}
\lorem
\end{texexample}



\cxset{numbering=arabic}
\section{Creating semantic names for commands and environments}

To keep our search for semantic commands and true separation of contents it is prudent to define some macros for typesetting the  `foreward' section.

\bgroup
\begin{texexample}{defining a \textit{Foreward} macro.}{}
\begin{lstlisting}
\cxset{foreward/.style={chapter toc=false,
          name=,
          title font-size = Large,
          title font-family = sffamily,
          numbering=none}}
\newcommand\forewardname{foreward}
\expandafter\newenvironment\expandafter{\forewardname}{%
\cxset{foreward}\chapter{Foreward}}%
{}
\begin{foreward}
\lorem
\end{foreward}
\end{lstlisting}
\end{texexample}
\egroup

Notice the use of a new command \cmd{\forewardname} to allow for internationlization using Babel or other methods. One is tempted to let the English name, but a better approach perhaps is to define both.

\makeatletter
\MakePercentIgnore
