\newfontfamily\emoji{Symbola}
\makeatletter
\@specialfalse
\makeatother

%\cxset{steward,
%  numbering=arabic,
%  custom=stewart,
%  offsety=0cm,
%  image={asia.jpg},
%  texti={An introduction to the use of font related commands. The chapter also gives a historical background to font selection using \tex and \latex. },
%  textii={Typesetting Middle Eastern scripts, Hebrew, Samaritan, Arabic, Thaana, Syriac
%  with \XeLaTeX. Selection and definition of related commands. Right to left writing systems. Image A fallah woman and her child. 1878 Elizabeth Jerichau Baumann.
% },
%}



\chapter{Those Other Languages}

\epigraph{New York\\
          1. Act making it a misdemeanor to make a speech or talk in public manner, in any language other
than English upon any subject relating to the form of a character of the government or the administration or enforcement of the laws of this state or the United States. }{\itshape Introduced in the Assembly by Mr Hamill, Feb.23, and referred to the Codes Committe (A.878.)}

\parindent1em



\section{The world's scripts and languages}


On May 23, 1918, Iowa Gov. William Harding banned the use of any foreign language in public: in schools, on the streets, in trains, even over the telephone.   \cite{frese} published a detailed history of this even in American history during the Great War years and its effects, which can even today be seen in Iowa. Such events of course during stressful times in a history of a country are not unique to America and similar politics can be observed throughout all human history. Today most Americans' answer to the calling of such a law would probably be the Unicode Character \unicodenumber{U+1F4A9}\footnote{\protect\emoji\protect\char"1F4A9}. Getting the character to print as a footnote in a document takes some doing. As many of the world's languages are facing extinction and the inclusion of a section in the |phd| package to deal with different scripts and appropriate fonts has been done in this spirit.

 
Probably there are more users of \latexe whose mother tongue is not English than those who speak the language. \tex out of the box does not offer facilities for using non-latin based scripts easily; this presents numerous problems. The biggest problem---which has been solved to a large extent---was the entering of text without having to mark all the special
characters such as umlauts (\"o) with commands. The second issue and which has been addressed by packages such as Babel, is redefining the strings such as ``Chapter" to another language. In software this is called internationalization and a governing standard is |i18n|. None of the current packages take such an approach and none of them as yet offer a satisfactory solution for |LuaLaTeX|. 



Another issue with writing systems and scripts is that of appropriate fonts. Most writing systems that have ever existed are now extinct. Only minute vestiges of one of the most ancient---Egyptian hieroglyphs---live on, unrecognized, in the Latin alphabet in which English, among hundreds of other languages, is conveyed today. The latin \textit{m}, for example, ultimately derives from the Egyptian's cononantal n-sign, depicting waves. There may never be a font that includes all the unicode characters (code2000) came close. Good fonts with well over ten thousand characters, keyed to the Unicode system, are now readily available. 

Bringhurst in the Elements of Typographic Style \citeyearpar{Bringhurst2005} critisized the allotment of only 256 characters in the extended ASCII specification and other software and considered this practice by software developers as `typographically sectarian and culturally stunted’. 


Many of the scripts are widely different to the Latin script. Hanunó'o is written vertically from bottom to top, whereas Tibetan and sometimes Chinese from top to bottom. Many Middle Eastern scripts such as Hebrew and Arabic are written from right to left. Some of the scripts have other peculiarities as they can have different forms when they are at the middle of a word or at the end. Ancient scripts such as hieroglyphics could be written from top to botton or from right to left or left to right or boustrostrephon. The glyphs of the latter could also face either left or right and the writing direction can be determined based on the direction the figures are ``seeing''. 

In this chapter we will be using the word ``script" instead of a ``writing system". Many people associate the word ``script" with a small program which is normally used on the command line. Here ``script" means a collection of letters and other characters, meant for writing human languages in a systematic way.  We say that languages such as English, Dutch and Icelandic and Vietnam use the Latin script, although they have different repertoires of characters.


\section{TeX's support for different languages}

TeX's primitives such as \cmd{\language}=\meta{number} can be used to store hyphenation patterns and exceptions for up to 256 different languages. This primitive is then used by TeX to apply an appropriate set of hyphenation rules for each paragraph or part of a paragraph in a document\footnote{\url{http://www.tug.org/utilities/plain/cseq.html language-rp}}. When TeX begins a ne paragraph it sets the \emph{current language} to \cmd{\language}. Just before it adds each new character to the paragraph in unrestricted horizontal mode, it compares the current language to \cmd{\language}. If they are different, TeX : a) changes the current language to \cmd{\language}; b) inserts a whatsit\index{whatsit>language} containing the new language and the values of |\lefthyphenmin| and |\righthyphenmin|; and c) inserts the character. The |\setlanguage| command should be used to change languages in restricted horizontal mode (i.e., inside an |\hbox|). If \meta{number} is less than 0 or greater than 255, 0 is used [455]. Plain TeX has a |\newlanguage| command which may be used to allocate numbers for languages [347]. Changes made to |\language| are local to the group containing the change 

\section{LaTeX}

As far as hyphenation patterns are concerned \latexe follows very closely to the methods employed by \tex and Plain Tex. In the source2e the File |lthyphen.dtx| describes the approach to loading the default file |hyphen.ltx| . If a file hyphen.cfg is found \latexe will load the appropriate hyphenaion patterns. Traditionally language management was achieved via Johan 
Braams package Babel which we describe in the next section. Numerous packages to assist in using different languages with \latex can be found at \url{http://www.ctan.org/tex-archive/language/}. Some are discussed in the Chapters that follow.


\section{The Babel Package} 

The \pkgname{Babel} by \citet{babel} was the first package to systematically offer foreign language
support for \tex. It has been updated for use with |XeTeX| and |LuaTeX| and provides an environment
in which documents can be typeset in a language
other than US English, or in more than one language
or script. However, no attempt has been done to
take full advantage of the features provided by the
latter, which would require a completely new core
(as for example polyglossia or as part of \latex3).

The package has a number of predefined language files with the extension |ldf|. Each \emph{language definition file} contains commands appropriate for setting strings and hyphenation patterns in the particular language, as well as
many ancillary macros to typeset dates and numbers in the typographical convention of the language. 


\Describe\selectlanguage{\marg{language}}{}
When a user wants to switch from one language to another he can
do so using the macro |\selectlanguage|. This macro takes the
language, defined previously by a language definition file, as
its argument. It calls several macros that should be defined in
the language definition files to activate the special definitions
for the language chosen. For ``historical reasons'', a macro name is
converted to a language name without the leading |\|; in other words,
the two following declarations are equivalent:
\begin{verbatim}
\selectlanguage{german}
\selectlanguage{\german}
\end{verbatim}

\Describe\foreignlanguage{\marg{language}\marg{text}}
The command |\foreignlanguage| takes two arguments; the second
argument is a phrase to be typeset according to the rules of the
language named in its first argument. This command (1) only
switches the extra definitions and the hyphenation rules for the
language, \emph{not} the names and dates, (2) does not send
information about the language to auxiliary files (i.e., the
surrounding language is still in force), and (3) it works even if
the language has not been set as package option (but in such a
case it only sets the hyphenation patterns and a warning is shown).

\Describe{otherlanguage*}%
{\marg{language}{otherlanguage*}}

Same as |\foreignlanguage| but as environment. Spaces after the
environment are \textit{not} ignored.



\section{The Polyglossia package}

The \pkgname{polyglossia} package has a lot of potential and has solved many issues
but its integration with large parts of the traditional |pdfLaTeX| world
is still under development and will probably take a while before one could
declare it easy to use and bug free \cite{polyglossia}. For example anything with the |bidi| package has issues with loading orders for a number of packages and least of which is with
the Ams packages. So if you are going to mix a number of languages in a \XeTeX\ document
you need to take extra care.

 Polyglossia is a package for facilitating multilingual typesetting with
 \XeLaTeX\ and (at an early stage) \LuaLaTeX.  Basically, it
 can be used as a replacement of \pkg{babel} for performing the following
 tasks automatically:
 
 \begin{enumerate}
 \item Loading the appropriate hyphenation patterns.
 \item Setting the script and language tags of the current font (if possible and
       available), via the package \pkg{fontspec}.
 \item Switching to a font assigned by the user to a particular script or language.
 \item Adjusting some typographical conventions according to the current language
       (such as afterindent, frenchindent, spaces before or after punctuation marks,
       etc.).
 \item Redefining all document strings (like chapter, “figure”, “bibliography”).
 \item Adapting the formatting of dates (for non-Gregorian calendars via external
       packages bundled with polyglossia: currently the Hebrew, Islamic and Farsi
       calendars are supported).
 \item For languages that have their own numbering system, modifying the formatting
       of numbers appropriately (this also includes redefining the alphabetic sequence
       for non-Latin alphabets).\footnote{ %
         For the Arabic script this is now done by the bundled package \pkg{arabicnumbers}.}
 \item Ensuring proper directionality if the document contains languages
       that are written from right to left (via the package \pkg{bidi},
       available separately).
 \end{enumerate}
 
 Several features of \pkg{babel} that do not make sense in the \XeTeX\ world (like font
 encodings, shorthands, etc.) are not supported.
 Generally speaking, \pkg{polyglossia} aims to remain as compatible as possible
 with the fundamental features of \pkg{babel} while being cleaner, light-weight,
 and modern. The package \pkg{antomega} has been very beneficial in our attempt to
 reach this objective.


\section{Loading language definition files}

The recommended way of \pkg{polyglossia} to load language definition files
is given in the manual as:
 
\Describe{\setdefaultlanguage}{\oarg{options}\marg{lang}}
 (or equivalently \cmd\setmainlanguage).
 Secondary languages can be loaded with

\Describe{\setotherlanguage}{\oarg{options}\marg{lang}}
 These commands have the advantage of being explicit and of allowing you to set
 language-specific options.\footnote{ %
 More on language-specific options below.}
 It is also possible to load a series of secondary languages at once using

\Describe\setotherlanguages{\marg{lang1,lang2,lang3,\ldots}}

 Language-specific options can be set or changed at any time by means of
\Describe\setkeys{\marg{lang}\marg{opt1=value1,opt2=value2,\ldots}}

\subsection{Bidirectional languages}





\begin{comment}
\begin{Arabic}
ّ هو إذ الغاية؛ شريف الفوائد، جم المذهب، عزيز فنّ التاريخ فنّ أنّ اعلم
والملوك سيرهم، في والأنبياء أخلاقهم، في الأمم من الماضين أحوال على يوقفنا
ّ أحوال في يرومه لمن ذلك في الإقتداء فائدة تتم حتّى وسياستهم؛ دولهم في
والدنيا. الدين
\end{Arabic}
\end{comment}

The Greek language is represented both in modern Greek as well as its ancient variants.

\begin{verbatim}
\begin{greek}
\textbf{Η ελληνική γλώσσα} είναι μία από τις ινδοευρωπαϊκές γλώσσες, για την
οποία έχουμε γραπτά κείμενα από τον 15ο αιώνα π.Χ. μέχρι σήμερα. Αποτελεί το
μοναδικό μέλος ενός κλάδου της ινδοευρωπαϊκής οικογένειας γλωσσών. Ανήκει
επίσης στον βαλκανικό γλωσσικό δεσμό.\\	
\end{greek}
\end{verbatim}

\topline

\textbf{Η ελληνική γλώσσα} είναι μία από τις ινδοευρωπαϊκές γλώσσες, για την
οποία έχουμε γραπτά κείμενα από τον 15ο αιώνα π.Χ. μέχρι σήμερα. Αποτελεί το
μοναδικό μέλος ενός κλάδου της ινδοευρωπαϊκής οικογένειας γλωσσών. Ανήκει
επίσης στον βαλκανικό γλωσσικό δεσμό.\\	

\bottomline

\begin{verbatim}
\begin{russian}
\textbf{Русский язык} — один из восточнославянских языков, один из 
крупнейших языков мира, в том числе самый распространённый из славянских
языков и самый распространённый язык Европы, как географически, так и по
числу носителей языка как родного (хотя значительная, и географически бо́
льшая, часть русского языкового ареала находится в Азии).	\\

\end{russian}
\end{verbatim}



\textbf{Русский язык} — один из восточнославянских языков, один из крупнейших языков мира, в том числе самый распространённый из славянских языков и самый распространённый язык Европы, как географически, так и по числу носителей языка как родного (хотя значительная, и географически бо́льшая, часть русского языкового ареала находится в Азии).	\\





\section{The Translator package}

The \pkgname{translator} package was developed by Till Tantau \cite{translator}. It provides a flexible
mechanism for translating individual words into different languages.
For example, it can be used to translate a word like ``figure'' into,
say, the German word ``Abbildung''. Such a translation mechanism is
useful when the author of some package would like to localize the
package such that texts are correctly translated into the language
preferred by the user. The translator package is \emph{not} intended
to be used to automatically translate more than a few words. 

You may wonder whether the translator package is really necessary
since there is the (very nice) |babel| package available for
\LaTeX. This package already provides translations for words like
``figure''. Unfortunately, the architecture of the babel package was
designed in such a way that there is no way of adding translations of
new words to the (very short) list of translations directly build into
babel.

The translator package was specifically designed to allow an easy
extension of the vocabulary. It is both possible to add new words that
should be translated and translations of these words.

\subsection{Using the Translator Package}

  The \pkg{Translator} needs to be used with \pkgname{Babel} and I am not too sure yet 
  if it is ready  to be used with Polyglossia.

Once the package has loaded a language or a set of languages the optional argument to the
\cmd{\translate} can be used to translate a string. 

\begin{texexample}{Translating strings}{ex:translator}
  \translate[to=german]{rightpagename}
  \translate[to=dutch]{rightpagename}
\end{texexample}

Before you can provide the translations you need to provide your own dictionaries, where you require them. These need to be installed at a place where \tex can find them.

\CMDI{\ProvidesDictionary}


The dictionary has to be saved in a specific format that relates to the \cmd{\ProvidesDictionary} command. The second argument of the command must be appended to the file name; for the example the file is saved as\footnote{This  example is from the translator package bundle and is under the folder \texttt{base}}:

|translator-basic-dictionary-German|

The concepts take a bit of time to sink in, but once you have everything set up, it is quite easy and straight forward to incorporate it, into your package. 

\begin{teXXX}
\ProvidesDictionary{translator-basic-dictionary}{German}

\providetranslation{Abstract}{Zusammenfassung}
\providetranslation{Addresses}{Adressen}
\providetranslation{addresses}{Adressen}
\providetranslation{Address}{Adresse}
\providetranslation{address}{Adresse}
\providetranslation{and}{und}
\providetranslation{Appendix}{Anhang}
\providetranslation{Authors}{Autoren}
\providetranslation{authors}{Autoren}
\providetranslation{Author}{Autor}
\providetranslation{author}{Autor}
\end{teXXX} 

This is in contrast to Babel and Polyglossia that define
commands for each string to be translated such as,

\begin{teXXX}
\def\captionsdutch{%
    \def\prefacename{Voorwoord}%
    \def\refname{Referenties}%
    \def\abstractname{Samenvatting}%
    \def\bibname{Bibliografie}%
    \def\chaptername{Hoofdstuk}%
    \def\appendixname{Bijlage}%
    \def\contentsname{Inhoudsopgave}%
    \def\listfigurename{Lijst van figuren}%
    \def\listtablename{Lijst van tabellen}%
    \def\indexname{Index}%
    \def\figurename{Figuur}%
    \def\tablename{Tabel}%
    \def\partname{Deel}%
    \def\enclname{Bijlage(n)}%
    \def\ccname{cc}%
    \def\headtoname{Aan}%
    \def\pagename{Pagina}%
    \def\seename{zie}%
    \def\alsoname{zie ook}%
    \def\proofname{Bewijs}%
    \def\glossaryname{Verklarende woordenlijst}%
    \def\today{\number\day~\ifcase\month%
      \or januari\or februari\or maart\or april\or mei\or juni\or
      juli\or augustus\or september\or oktober\or november\or
      december\fi
      \space \number\year}}
\end{teXXX}

\begin{macro}{\usedictionary}\marg{kind}
  This command tells the |translator| package, that at the beginning of
  the document it should load \textit{all} dictionaries of kind \meta{kind} for
  the languages used in the document. Note that the dictionaries are
  not loaded immediately, but only at the beginning of the document.

  If no dictionary of the given \emph{kind} exists for one of the
  language, nothing bad happens.

  Invocations of this command accumulate, that is, you can call it
  multiple times for different dictionaries.
\end{macro}

\begin{command}{\uselanguage\marg{list of languages}}
  This command tells the |translator| package that it should load the
  dictionaries for all languages in the \meta{list of languages}. The
  dictionaries are loaded at the beginning of the document.
\end{command}

\section{The phd package Language facilities}

The \pkgname{phd} package provides facilities for language handling, but albeitly still at an experimental stage. Sectioning command strings can easily be set in one's language by just typing the key in the appropriate language.

\begin{texexample}{Example of changing language in headings}{ex:lheadings}
\cxset{chapter name=Kapital,
          chapter opening=anywhere}
\chapter{Das German}          
\end{texexample}


\section{Fonts for All the World Scripts}

Many commercial as well as open source fonts exist that can be used to typeset text the world's scripts and languages. The aim of this section of the documentation is to present an overview of the most common scripts represented in the Unicode~7.0 standard. All the examples require the use of the \XeTeX\ engine. In addition you need to have a copy of the font on your own system. If you do not have them, the font loading mechanism of \XeTeX\ will take some time to search all the directories and slows compilation tremendously. 

\subsection{Pan-Unicode Fonts}

Thousands of fonts exist on the market, but fewer than a dozen fonts—sometimes described as ``pan-Unicode" fonts—attempt to support the majority of Unicode's character repertoire. Instead, Unicode-based fonts typically focus on supporting only basic ASCII and particular scripts or sets of characters or symbols. Several reasons justify this approach: applications and documents rarely need to render characters from more than one or two writing systems; fonts tend to demand resources in computing environments; and operating systems and applications show increasing intelligence in regard to obtaining glyph information from separate font files as needed, i.e. font substitution. Furthermore, designing a consistent set of rendering instructions for tens of thousands of glyphs constitutes a monumental task; such a venture passes the point of diminishing returns for most typefaces.

The \texttt{NotoSerif} fonts from Google\footnote{\protect\url{http://www.google.com/get/noto/}} have good support for 96 language fonts and the list is gorwing. Since these are widely available most of the scripts that follow use these fonts. Follow the instructions at the website to install them. It is just a matter of dragging them into the fonts folder for most operating systems.

Another freeware pan-Unicode font is Titus
This is an extended version of this font is TITUS Cyberbit Unicode, includes 36,161 characters in v4.0.

On Windows systems |Arial Unicode MS| contains glyphs for all code points within the Unicode Standard version 2.1.  

The code2000.ttf provides 63546 glyphs and is the nearest font to a universal font to handle Unicode. Development stopped in 2008. As a comparison Linux Libertine O, provides 2674 glyphs. 

CJK fonts naturally will have the most glyphs, \idxfont{MingLiU} 34046 glyphs and is a very good font for CJK typesetting.

The \href{http://ftp.gnu.org/gnu/freefont/}{FreeFont Project} currently supports most of the useful set of free outline (i.e. OpenType) fonts covering as much as possible of the Unicode character set. The set consists of three typefaces: one monospaced and two proportional (one with uniform and one with modulated stroke). There are tw

The idea of having lots of different writing systems into a single font at all? How good does such a font need to be?
There are two extreme views.  The first one is that glyphs in a font shold comprise a unified design entity. This in practice makes sense only within a single language script. Different script systems, such a Latin, Arabic and Devanagari, have different typesetting traditions and conventions.  A good discussion of the prons and cons can be found at the gnu website \footnote{\protect\url{https://www.gnu.org/software/freefont/articles/Why_Unicode_fonts.html}}. For TeX it is a better proposition in order to avoid switching of fonts that can distract the writer. At least one requires fonts that are inclusive of one's usage. 

\section{The \texttt{ucharclasses} package}

For multilingual texts font switching can become cumbersome. The use of a pan-Unicode font as the default can help. However, if the languages are distinct enough to use different Unicode blocks, which are not covered by the \pkgname{polyglossia} package Mike Kamermans' package \pkgname{ucharclasses} can be used. This package does not work with LuaTeX. 

\begin{verbatim}
% and the font switching magic
\usepackage[CJK, Latin, Thai, 
           Sinhala, Malayalam, 
           DominoTiles, 
           MahjongTiles]{ucharclasses}
\usepackage{fontspec}

% default transition uses the widest coverage font I know of
\setDefaultTransitions{\fontspec{Code2000.ttf}}{}

% overrides on the default rules for specific informal groups
\setTransitionsForLatin{\fontspec{Palatino Linotype}}{}
\setTransitionsForCJK{\fontspec{code2000.ttf}}{}%HAN NOM A
\setTransitionsForJapanese{\fontspec{code2000.ttf}}{}%Ume Mincho

% overrides on the default rules for specific unicode blocks
\setTransitionTo{CJKUnifiedIdeographsExtensionB}{\fontspec{SimSun-ExtB}}
\setTransitionTo{Thai}{\fontspec{IrisUPC}}
\setTransitionTo{Sinhala}{\fontspec{Iskoola Pota}}
\setTransitionTo{Malayalam}{\fontspec{Arial Unicode MS}}

\end{verbatim}

{
\newfontfamily\mahjong{FreeSerif.ttf}
domino tiles, 🁇 🀼 🁐 🁋 🁚 🁝, and mahjong tiles: 🀑 🀑 🀑 🀒 🀒 🀒 🀕 🀕 🀕 🀗 🀗 🀗 🀅 🀅 (using FreeSerif)

}

The interaction between Polyglossia and Fontspec can result in infinite looping and memory leaks. I do not recommend that you use these commands as yet. The use of the charclasses will also slow down compilation possibly by a factor of 10.



\section{PhD Settings}

\def\test{}
\cxset{language/.code=\test}
\cxset{language=greek}
\cxset{languages/.code=\test}
\cxset{languages={english,greek,spanish,chinese}}
\cxset{greek font/.code=\test}
\cxset{greek font=code2000.ttf}

\begin{key}{/chapter/language=\meta{language name}}  
The key language sets the main language for the document. This language will be used for the sectioning commands and common string translations.

If the language is English Polyglossia or Babel are not loaded automatically. If the language is other than English we load either Babel or Polyglossia depending on the engine used.
\end{key}


\begin{key}{/chapter/languages = \meta{language1, language2, language3}}  
The key |languages|, determines all the other scripts available for typesetting. For each language default font commands are create automatically. The aim is to be able to run a fully multilingual system with the minimum of upfront settings. These we leave to customize in the style template files.
\end{key}

\begin{key}{/chapter/greek font = \meta{options}\meta{font file}}  
The package comes with numerous language and appropriate default fonts
for each operating system. 
\end{key}










