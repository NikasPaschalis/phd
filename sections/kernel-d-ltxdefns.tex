%% LaTeX2e file `./manual/kernel-d-ltxdefns.tex'
%% generated by the `filecontents' environment
%% from source `photo-book-class-final-test' on 2011/12/18.
%%
\section{Definitions File d ltxdefns.dtx}
This is one of themost useful sections of \latex for macro writers as it exposes a lot of macros hat one can find useful in packages. They are found in file |d|.


\cmd{two"@"digits} Prefixes a number less than 10 with a `|0|'.
typing,
\begin{teX}
\two@digits{21},
\two@digits{8}
\end{teX}

will give you:

\makeatletter
\two@digits{21},
\two@digits{8}
\makeatother


\cmd{typeout} Display something on the terminal.

\begin{teX}
 \def\typeout#1{\begingroup\set@display@protect
 \immediate\write\@unused{#1}\endgroup}
\end{teX}

\cmd{newlinechar} A char to be used as new-line in output to files.



\subsection{Saved versions of primitives}

The \tex primitives are saved in \latex using |@@| as a prefix for example the \tex primitive |\par| is saved as |\@@par|.

\begin{trivlist}

\item @@par The \tex primitive |\par|.

\item @@hyph This is the \tex primitive |\-|.

\item @@italiccorr saves the italic correction.

\item @height Use height instead of |\height|

\item  @width

\item  @depth Saves 5 tokens at the cost of time for macro expansion

\item @minus The |minus| command

\item  @plus The |plus| command
\end{trivlist}


\section{Command definitions}

This is one of the most useful sections as most of the macro writing commands are collected here.

For example typing the \latex equivalent for |csname|:

\begin{teX}
   \@namedef{123}{\texttt{One Two Three}}
   \@nameuse{123}
\end{teX}

\makeatletter
\@namedef{123}{\texttt{One Two Three}}

Will give you \@nameuse{123}. Besides saving a lot of typing, by using \latex kernel commands you provide a consistent interface to people that may wish to modify your code. One disantavantage though is tha tyour program will not be able to be used by |plain| \tex users. Each on its own.
\makeatother



\begin{teX}
\@ifundefined{123}{Yes}{No}
\end{teX}

Check if first arg is undefined or |\relax| and execute second or third arg depending,

\begin{teX}
   \def\@ifundefined#1{%
      \expandafter\ifx\csname#1\endcsname\relax
      \expandafter\@firstoftwo
      \else
      \expandafter\@secondoftwo
   \fi}
\end{teX}

The \cmd{@undefined} is normally used to check a named definition (one that has been created through csname..endcsname. Consider the following code:

\begin{teX}
\def\csUndefined#1{
\@ifundefined{#1}
   {The command \textbackslash \@nameuse{#1} is undefined}
   {The command \textbackslash #1 has been defined and expands to  \@nameuse{#1}}
}
\end{teX}

\def\csUndefined#1{
\@ifundefined{#1}
   {The command \textbackslash \@nameuse{#1} is undefined}
   {The command \textbackslash #1 has been defined and expands to  \@nameuse{#1}}
}

\noindent Typing, |\csUndefined{123}| we get \csUndefined{123}

\noindent Typing, |\csUndefined{texttt}| we get \csUndefined{texttt}{test}

This pattern is used fairly often and consistently in \latex when defining new commands on the fly, checking first if the command is defined and then taking some action. In most instances the first part issues an error message.

\makeatletter
\p@=1pt
\the\p@

\makeatother

It takes a while to get used in programming with variable names that include the |@|, symbol. If you find it difficult at first use your own style. You do not need to use it, but beware that other users or package writer can inadvertatly use the name symbol. However, with careful use of grouping you can aviod most mistakes like this.


\section{Fragile and robust commands}

Fragile and robust commands are one of the thornier issues in \latex's commands.\index{fragile commands},\index{robust commands}

Whilst typesetting documents, \latex makes use of many of \tex's features, such as
arithmetic, defining macros, and setting variables. However, there are (at least) three different ocassions when these commands are not safe. These are called \emph{moving arguments} by \latex, and consist of:

\begin{enumerate}
\item writing information to a file, such as indexes or tables of contents.
\item writing information to the screen.
\item inside an |\edef|, |\message|, |\mark|, or other command which evaluates its argument fully.
\end{enumerate}

\latex uses \cmd{protect} to make fragile commands robust. It does this by preceding the command with |\protect|. This can have one of five possible values:

\begin{enumerate}
\item |\relax|, for normal typesetting. So |\protect\foo| will execute |\foo|


\item \cmd{string}, for writing to the screen. So |\protect\foo| will write |\foo|.

\item \cmd{noexpand}, for writing to a file. So |\protect\foo| will write |\foo| followed by a space.


\item \cmd{@unexpandable@protect}, for writing a moving argument to a file. So |\protect\foo| will write |\protect\foo| followed by a space. This value is also used inside |\edefs|, |\marks| and other commands which evaluate their arguments fully.

\item \cmd{@unexpandable@noexpand}, for performing a deferred write inside an |\edef|.
So |\protect\foo| will write |\foo| followed by a space. If you want
|\protect\foo| to be written, you should use |\@unexpandable@protect|.
(This was Removed as it was never used).

\end{enumerate}

\latex provides the command \cmd{DeclareRobustCommand} to manage robust definitions.

\begin{teX}
\DeclareRobustCommand{\seq}[2][n]{%
\ifmmode
#1_{1}\ldots#1_{#2}%
\else
\PackageWarning{fred}{You can't use \protect\seq\space in text}%
\fi
}




So how does \latex does it? For this we need to look under |ltdefns.dtx|

\begin{teX}
 \def\@unexpandable@protect{\noexpand\protect\noexpand}
\end{teX}


\cmd{DeclareRobustCommand} is a package writers command, which has the same syntax as \cmd{newcommand}, but which declares a protected command.

It does this by having |\DeclareRobustCommand\foo|
define |\foo| to be |\protect\foo<space>|,
and then use |\newcommand\foo<space>|.

Since the internal command is |\foo<space>|, when it is written to an auxiliary
file, it will appear as |\foo|.
We have to be a bit cleverer if we're defining a short command, such as |\_|,
in order to make sure that the auxiliary file does not include a space after the
command, since |\_| a and |\_a| aren't the same. In this case we define |\_ |to be:

\begin{teX}
|\x@protect\_\protect\_<space>|
\end{teX}

which expands to:

\begin{teX}
\ifx\protect\@typeset@protect\else
\@x@protect@\_
\fi
\protect\_<space>
\end{teX}

Then if |\protect| is |\@typeset@protect| (normally |\relax|) then we just perform
|\_<space>|, and otherwise |\@x@protect@| gobbles everything up and expands to |\protect\_|


During typesetting we let the |@typeset@protect| to |\relax|

\begin{teX}
\let\@typeset@protect\relax
\end{teX}

%\def\strip@prefix#1>{}

\makeatletter
\def\test{A Test}
\let\tesT\test


Before we review the rest of the code let us make a small detour around the dangerous bends to re-examine the \emph{meaning} of \cmd{meaning}. This command (which is a \tex primitive), will return the definition of a macro.

\begin{teX}
\meaning\maketitlepage
\end{teX}

will produce

{
\tt

\meaning\maketitlepage


}


We can use the \latex kernel command |\strip@prefix|to strip the prefix of the macro


\begin{teX}
\expandafter\strip@prefix\meaning\maketitlepage
\end{teX}



As you can see the |\long macro:->| has now been removed. We will use this to use our own definition of a Macro similar to \latex first. What we will do we will define

\begin{teX}
\def\NewMacro#1{}  %\def\DeclareRobustCommand{}
\end{teX}

Our \cmd{NewMacro} will simply define a command to be |\edef|. Once we understand some of the concepts we will then move on to discuss the way \latex defines a |DeclareRobustCommand|.

What we need to capture is first the new command name and then the contents within the brackets. For example we may need to define a command

\begin{teX}
\NewMacro\test{This is test}
\end{teX}

We both need to capture the |test| as well as the |{This is a Test}|. We do this so that we can then |\edef\test{This is a test}|. Let us see one way.


\makeatletter

\def\@atest{\textsc{$\alpha$, Gamma}}

\@atest

\edef\TesT{\expandafter\strip@prefix\meaning\@atest}%

\@atest

\TesT

and the meaning of TesT is  {\tt \meaning\TesT}

What we have managed so far is to get the definition of a macro and place it in another macro. This is \tex magic! Whatremains now is to get the arguments (including the name of the macro and place it in front of the |edef|. Then we can have a macro that can be used to define other macros!

\makeatother

For example, if |\seq| is defined as follows:


\DeclareRobustCommand{\seq}[2][n]{%
\ifmmode
#1_{1}\ldots#1_{#2}%
\else
\PackageWarning{fred}{You can't use \protect\seq\space in text}%
\fi
}

$\seq{n1,n2}$

\seq{n1,n2}


You must admit that following what \latex does, involves mental acrobatics. With time though things start to fall into place!
\clearpage
