
 \chapter{File Handling}

 The following user commands are defined in this part (ltfiles.dtx):

  \begin{macro}{\document} (ie |\begin{document}|)
      Reads in the .AUX files and |\catcode|'s |@| to 12.
  \end{macro}

  \begin{macro}{\nofiles}
       Suppresses all file output by setting |\@filesw| false.
  \end{macro}

  \begin{macro}{\includeonly}\marg{NAME1, ... ,NAMEn}\\
       Causes only parts NAME1, ... ,NAMEn to be read by
         their |\include| commands.  Works by setting \@partsw true
         and setting |\@partlist| to NAME1, ... ,NAMEn.
  \end{macro}

  \begin{macro}{\include}\marg{NAME}\\
        Does an |\input| NAME unless |\@partsw| is true and
         NAME is not in |\@partlist|.  If |\@filesw| is true, then
         it directs .AUX output to NAME.AUX, including a
         checkpoint at the end.
  \end{macro}

 \begin{macro}{\input}\marg{NAME}\\
        The same as TeX's |\input|, except it allows optional
         braces around the file name. In \LaTeXe, it also avoids
         the primitive `missing file' error, if the file can not be
         found.
 \end{macro}

 \begin{macro}{\IfFileExists}\marg{NAME}\marg{then}\marg{else}\\
   If the file exists on the system, execute \emph{then} otherwise
   execute \emph{else}.
 \end{macro}

 \begin{macro}{\InputIfFileExists}\marg{NAME}\marg{then}\marg{else}\\
   If the file exists on the system, execute \emph{then} and input
   \emph{NAME}  otherwise execute \emph{else}.
 \end{macro}


\section{Variables, switches and internal commands}

\begin{tabular}{lp{6cm}}

    \cs{@mainaux}  & Output file number for main .AUX file.\\

    \cs{@partaux}  & Output file number for current part's .AUX file.\\

    \cs{@auxout}   & Either \cs{@mainout} or \cs{@partout}, depending on
                   which .AUX file output goes to.\\

    \cs{@input}\marg{foo} & If file foo exists, then \cs{input}'s it,
                   otherwise types a warning message.\\

    \cs{@filesw}      & Switch -- set false if no .AUX, .TOC, .IDX etc
                     files are to be written\\

    \cs{@partsw}      & Set true by a \cs{includeonly} command.\\

    \cs{@partlist}   & Set to the argument of the \cs{includeonly} command.\\

    \cs{cp@FOO}      & The checkpoint for \cs{include}'d file FOO.TEX, written
                   by \cs{@writeckpt} at the end of file FOO.AUX\\
\end{tabular}


\begin{comment}
% \includeonly{FILELIST} ==
%  BEGIN
%   \@partsw   := T
%   \@partlist := FILELIST
%  END
%
% \include{FILE} ==
%  BEGIN
%   \clearpage
%   if \@filesw = T
%     then  \immediate\write\@mainaux{\string\@input{FILE.AUX}}
%   fi
%   if  \@partsw = T
%     then \@tempswa := F
%          \reserved@b == FILE
%          for \reserved@a := \@partlist
%              do if eval(\reserved@a) = eval(\reserved@b)
%                   then \@tempswa := T          fi
%              od
%   fi
%
%   if \@tempswa = T
%      then \@auxout := \@partaux
%           if \@filesw = T
%             then  \immediate\openout\@partaux{FILE.AUX}
%                   \immediate\write\@partaux{\relax}
%           fi
%           \@input{FILE.TEX}
%           \clearpage
%           \@writeckpt{FILE}
%           if @filesw then \closeout \@partaux fi
%           \@auxout := \@mainaux
%      else \cp@FILE
%   fi
%  END
%
% \@writeckpt{FILE} ==
%  BEGIN
%    if \@filesw = T
%        \immediate\write on file \@partaux:
%                  \@setckpt{FILE}{                  %% }
%        for \reserved@a := \cl@@ckpt
%           do  \immediate\write on file \@partaux:
%                   \global\string\setcounter
%                       {eval(\reserved@a)}{eval(\c@eval(\reserved@a))}
%           od                                     %% {
%        \immediate\write on file \@partaux:  }
%    fi
%  END
%
% \@setckpt{FILE}{LIST} ==
%  BEGIN
%    G \cp@FILE := LIST
%  END
%
%  INITIALIZATION
%    \@tempswa := T
%
% \end{oldcomments}
%
%
% \task{???}{Do we use @unused or mainaux?}
%  \begin{macro}{\@inputcheck}
%  \begin{macro}{\@unused}
%    Allocate read stream for testing and output stream.
% \changes{v1.0l}{1994/11/07}
%      {move here from ltdefns, remove duplicate \cs{@mainaux}}
%    \begin{macrocode}
\newread\@inputcheck
\newwrite\@unused
%    \end{macrocode}
%  \end{macro}
%  \end{macro}
%
\end{comment}

 \begin{macro}{\@mainaux}
 \begin{macro}{\@partaux}
  These two macros allocate write streams for the main auxiliary file
  and the part file.
 \end{macro}
 \end{macro}
    \begin{teX}
\newwrite\@mainaux
\newwrite\@partaux
    \end{teX}


Next the switches are defined as well as a newcount for |\@clubpenalty|.
The latter still puzzles me as to why here.

 \begin{macro}{\if@filesw}
 \begin{macro}{\if@partsw}
 \end{macro}
 \end{macro}
    \begin{teX}
\newif\if@filesw \@fileswtrue
\newif\if@partsw \@partswfalse
    \end{teX}

 \begin{macro}{\@clubpenalty}
    This stores the current normal (non-infinite) value of
    \cs{clubpenalty}; it should therefore be reset whenever the
    normal value is changed (as in the bibliography in the standard
    styles).
  \end{macro}
    \begin{teX}
\newcount\@clubpenalty
\@clubpenalty \clubpenalty
    \end{teX}


  \begin{macro}{\document}
    Cancel the |\begingroup| from |\begin|
  \end{macro}
  \begin{teX}
    \def\document{\endgroup
   \end{teX}

    If some options on |\documentclass| haven't been used by any
    package we will now give a warning since this is most certainly a
    misspelling.

  \begin{teX}
  \ifx\@unusedoptionlist\@empty\else
    \@latex@warning@no@line{Unused global option(s):^^J%
            \@spaces[\@unusedoptionlist]}%
  \fi
  \@colht\textheight
  \@colroom\textheight \vsize\textheight
  \columnwidth\textwidth
  \@clubpenalty\clubpenalty
  \if@twocolumn
    \advance\columnwidth -\columnsep
    \divide\columnwidth\tw@ \hsize\columnwidth \@firstcolumntrue
  \fi
  \hsize\columnwidth \linewidth\hsize
  \begingroup\@floatplacement\@dblfloatplacement
    \makeatletter\let\@writefile\@gobbletwo
    \global \let \@multiplelabels \relax
    \@input{\jobname.aux}%
  \endgroup
  \if@filesw
    \immediate\openout\@mainaux\jobname.aux
    \immediate\write\@mainaux{\relax}%
  \fi
  \end{teX}


Next 
 FMi added |\process@table| to support NFSS;
 This will also work with old lfonts if no other style defines
 |\process@table|.  The following line forces the initialization of
 the math fonts.
 \begin{teX}
  \process@table
  \let\glb@currsize\@empty  %% Force math initialization.
  \normalsize
  \everypar{}%
 \end{teX}
%

 So that punctuation in headings is not disturbed by verbatim
 or other local changes to the space factor codes, save the document
 default here. This will be locally reset by the output routine.
 For special cases a class may want to define |\normalsfcodes|
 directly, in case that definition will be used.
 (This is an old bug, problem existed in \LaTeX2.0x and plain \TeX.)
 \changes{v1.1k}{1997/04/14}
            {Set the document space factor defaults. latex/2404}
    \begin{teX}
  \ifx\normalsfcodes\@empty
    \ifnum\sfcode`\.=\@m
      \let\normalsfcodes\frenchspacing
    \else
      \let\normalsfcodes\nonfrenchspacing
    \fi
  \fi
 \end{teX}

 Way back in 1991 (08/26) FMi \& RmS set the |\@noskipsec| switch
 to true in the preamble and to false here.
 This was done to trap lists and related text in the preamble but it
 does not catch everything; hence Change 1.1g was introduced.
    \begin{teX}
  \@noskipsecfalse
    \end{teX}


 changes{v1.1a}{1995/10/24}
           {Removed refundefined switch}
  \begin{teX}
  \let \@refundefined \relax
  \end{teX}

    Just before disabling the preamble commands we execute the begin
    document hook which contains any code contributed by
    |\AtBeginDocument|. Also disable the gathering of the file list,
    if no |\listfiles| has been issued. |\AtBeginDocument| is redefined
    at this point so that and such commands that get into the hook do
    not chase their tail\ldots

  \begin{teX}
  \let\AtBeginDocument\@firstofone
  \@begindocumenthook
  \end{teX}

    Most of the following assignments will be done globally in case
    the user adds something like |\begin{multicols}| to the document
    hook, i.e. starts are group in |\begin{document}|.
    Since a value of exactly 0pt for \cs{topskip} causes
    \cs{twocolumn[]} to misbehave, we add this check, hoping
    that it will not cause any problems elsewhere.
  \begin{teX}
  \ifdim\topskip<1sp\global\topskip 1sp\relax\fi
  \global\@maxdepth\maxdepth
  \global\let\@begindocumenthook\@undefined
  \ifx\@listfiles\@undefined
    \global\let\@filelist\relax
    \global\let\@addtofilelist\@gobble
  \fi
  \end{teX}

    At the very end we disable all preamble commands. This has to
    happen after the begin document hooks was executed so that this
    hook can still use such commands.

  \begin{teX}
  \gdef\do##1{\global\let ##1\@notprerr}%
  \@preamblecmds
  \end{teX}

   Added disabling of \cs{@nodocument}
  \begin{teX}
  \global\let \@nodocument \relax
  \end{teX}
   The next line is a pure safety measure in case a do list is ever
   expanded at the wrong place. In addition it will save a few
   tokens to get rid of the above definition.
  \begin{teX}
  \global\let\do\noexpand
  \end{teX}

    Use of |\AtBeginDocument| hook might mean that we are already in
    horizontal mode, so ignore the space after |\begin{document}|.
  \begin{teX}
  \ignorespaces}
  \end{teX}

  \begin{teX}
\@onlypreamble\document
  \end{teX}

 \begin{macro}{\normalsfcodes}
 The setting of |\@empty| is just a flag. This command may be defined
 in a class or package file. If it is still |\@empty| at 
 |\begin{document}| it will be defined to be |\frenchspacing| or
 |\nonfrenchspacing|, depending on which of those appears to be in
 effect at that point.
 \end{macro}
    \begin{teX}
\let\normalsfcodes\@empty
    \end{teX}


 \begin{macro}{\nofiles}
 Set |\@fileswfalse| which suppresses the places where \LaTeX\ makes
 |\immediate| writes. The |\makeindex| and |\makeglossary| are
 disabled. |\protected@write| is redefined not to write to the file
 specified, but rather to write a blank line to the log file. This
 ensures that a \meta{whatsit} node is still created, and so spacing
 is not affected by the |\nofiles| command; to ensure this more
 generally, the |\if@nobreak| test is needed.
 \end{macro}
 \begin{teX}
\def\nofiles{%
  \@fileswfalse
  \typeout{No auxiliary output files.^^J}%
  \long\def\protected@write##1##2##3%
    {\write\m@ne{}\if@nobreak\ifvmode\nobreak\fi\fi}%
  \let\makeindex\relax
  \let\makeglossary\relax}
\@onlypreamble\nofiles
 \end{teX}

%
% \begin{macro}{\protected@write}
%    This takes three arguments: an output stream, some initialization
%    code, and some text to write.  It then writes this, with
%    appropriate handling of |\protect| and |\thepage|.
% \changes{v1.0k}{1994/11/04}{Macro added  ASAJ.}
% \changes{v1.0t}{1995/05/25}
%         {(CAR) added \cs{long}}
%    \begin{macrocode}
\long\def \protected@write#1#2#3{%
      \begingroup
       \let\thepage\relax
       #2%
       \let\protect\@unexpandable@protect
       \edef\reserved@a{\write#1{#3}}%
       \reserved@a
      \endgroup
      \if@nobreak\ifvmode\nobreak\fi\fi
}
%    \end{macrocode}
% \end{macro}
%
    \begin{teX}
\let\@auxout=\@mainaux
    \end{teX}


 \begin{macro}{\includeonly}
 \end{macro}
 \begin{teX}
\def\includeonly#1{%
  \@partswtrue
  \edef\@partlist{\zap@space#1 \@empty}}
  \@onlypreamble\includeonly
  \end{teX}

%
% \begin{macro}{\include}
% \changes{v0.9p}{1994/01/18}
%         {Use \cs{@input@} so include files are listed.}
% In the definition of |\include|, |\def\reserved@b| changed to
% |\edef\reserved@b| to be consistent with the |\edef| in
% |\includeonly|.
% (Suggested by Rainer Sch\"opf \& Frank Mittelbach.
% Change made 20 Jul 88.)
%
% Changed definition of |\include| to allow space at end of file name
% --- otherwise, typing |\include{foo }| would cause \LaTeX\ to
% overwrite |foo.tex|.  Change made 24 May 89, suggested by Rainer
% Sch\"opf  and Frank Mittelbach
%
% Made |\include| check for being used inside an |\include|'d file, as
% this will not work and cause surprising results.
%    \begin{macrocode}
\def\include#1{\relax
  \ifnum\@auxout=\@partaux
    \@latex@error{\string\include\space cannot be nested}\@eha
  \else \@include#1 \fi}
%    \end{macrocode}
% \end{macro}
%
 \begin{macro}{\@include} The definition of |\@include| is fairly
 straightforward less one puzzle, why would the |\clearpage| be
 issued and thus have a behaviour which is totally different from
 |\input|. My suspicion it has to do with early bugs.
 
 \end{macro}
    \begin{teX}
\def\@include#1{%
  \clearpage
  \if@filesw
    \immediate\write\@mainaux{\string\@input{#1.aux}}%
  \fi
  \@tempswatrue
  \if@partsw
    \@tempswafalse
    \edef\reserved@b{#1}%
    \@for\reserved@a:=\@partlist\do
      {\ifx\reserved@a\reserved@b\@tempswatrue\fi}%
  \fi
  \if@tempswa
    \let\@auxout\@partaux
    \if@filesw
      \immediate\openout\@partaux #1.aux
      \immediate\write\@partaux{\relax}%
    \fi
    \@input@{#1.tex}%
    \clearpage
    \@writeckpt{#1}%
    \if@filesw
      \immediate\closeout\@partaux
    \fi
  \else
     \end{teX}
 If the file is not included, reset |\deadcycles|, so that a long
 list of non-included files does not generate an `Output loop'
 error.

    \begin{teX}
    \deadcycles\z@
    \@nameuse{cp@#1}%
  \fi
  \let\@auxout\@mainaux}
    \end{teX}



 \begin{macro}{\@writeckpt}
 \end{macro}
    \begin{teX}
\def\@writeckpt#1{%
  \if@filesw
    \immediate\write\@partaux{\string\@setckpt{#1}\@charlb}%
    {\let\@elt\@wckptelt \cl@@ckpt}%
    \immediate\write\@partaux{\@charrb}%
  \fi}
    \end{teX}


 \begin{macro}{\@wckptelt}
 \end{macro}
    \begin{teX}
\def\@wckptelt#1{%
  \immediate\write\@partaux{%
    \string\setcounter{#1}{\the\@nameuse{c@#1}}}}
    \end{teX}


 \begin{macro}{\@setckpt}
 \end{macro}
    \begin{teX}
\def\@setckpt#1{\global\@namedef{cp@#1}}
    \end{teX}


 \begin{macro}{\@charlb}
 \begin{macro}{\@charrb}
 \end{macro}
 \end{macro}
 The following defines |\@charlb| and |\@charrb| to be |{| and |}|,
 respectively with |\catcode| 11.
    \begin{teX}
{\catcode`[=1 \catcode`]=2
\catcode`{=11 \catcode`}=11
\gdef\@charlb[{]
\gdef\@charrb[}]
]% }brace matching
    \end{teX}


 \subsection{Safe Input Macros}

 \begin{macro}{\IfFileExists}
 \end{macro}

 \begin{teX}
\long\def \IfFileExists#1#2#3{%
  \openin\@inputcheck#1 %
  \ifeof\@inputcheck
    \ifx\input@path\@undefined
      \def\reserved@a{#3}%
    \else
      \def\reserved@a{\@iffileonpath{#1}{#2}{#3}}%
    \fi
  \else
    \closein\@inputcheck
    \edef\@filef@und{#1 }%
    \def\reserved@a{#2}%
  \fi
  \reserved@a}
 \end{teX}

\end{document}
%
% \begin{macro}{\@iffileonpath}
% If the file is not found by |\openin|, and |\input@path| is defined,
% look in all the directories specified in |\input@path|.
% \changes{v0.9p}{1994/01/18}{Macro added}
% \changes{v1.0f}{1994/05/02}
%     {\cs{@break@loop} renamed to \cs{@break@tfor}}
% \changes{v1.0t}{1995/05/25}
%         {(CAR) added \cs{long}}
% \changes{v1.1d}{1996/01/10}
%     {Change argument handling to not require doubled hash. latex/2024}
%    \begin{macrocode}
\long\def\@iffileonpath#1{%
  \let\reserved@a\@secondoftwo
  \expandafter\@tfor\expandafter\reserved@b\expandafter
             :\expandafter=\input@path\do{%
    \openin\@inputcheck\reserved@b#1 %
    \ifeof\@inputcheck\else
      \edef\@filef@und{\reserved@b#1 }%
      \let\reserved@a\@firstoftwo%
      \closein\@inputcheck
      \@break@tfor
    \fi}%
  \reserved@a}
%    \end{macrocode}
%  \end{macro}
%  \end{macro}
%
% \begin{macro}{\InputIfFileExists}
% \changes{v0.9b}
%         {1993/12/04}{Macro added}
% \changes{v0.9p}
%         {1994/01/18}{New Definition}
% \changes{v0.3b}{1994/03/13}
%         {Use new cmd \cs{@addtofilelist}}
% Now define |\InputIfFileExists| to input |#1| if it seems to exist.
% Immediately prior to the input, |#2| is executed.
% If the file |#1| does not exist, execute `|#3|'.
% \changes{v1.0t}{1995/05/25}
%         {(CAR) added \cs{long}}
%    \begin{macrocode}
\long\def \InputIfFileExists#1#2{%
  \IfFileExists{#1}%
    {#2\@addtofilelist{#1}\@@input \@filef@und}}
%    \end{macrocode}
%  \end{macro}
%
  \begin{macro}{\input}
    Input a file: if the argument is given in braces use safe input
    macros, otherwise use \TeX's primitive |\input| command (which is
    called |\@@input| in \LaTeX).
  \end{macro}

    \begin{teX}
\def\input{\@ifnextchar\bgroup\@iinput\@@input}
    \end{teX}

%
% \begin{macro}{\@iinput}
%    Define |\@iinput| (i.e., |\input|) in terms of
%    |\InputIfIfileExists|.
% \changes{v0.9b}{1993/12/04}{Macro reimplemented}
%    \begin{macrocode}
\def\@iinput#1{%
  \InputIfFileExists{#1}{}%
  {\filename@parse{#1}%
   \edef\reserved@a{\noexpand\@missingfileerror
     {\filename@area\filename@base}%
     {\ifx\filename@ext\relax tex\else\filename@ext\fi}}%
   \reserved@a}}
%    \end{macrocode}
%  \end{macro}
%
% \begin{macro}{\@input}
% \changes{v0.9b}{1993/12/04}{Macro reimplemented}
% \changes{v0.9p}{1994/01/18}%
%         {do not use a different definition for \cs{input@path}}
%    Define |\@input| in terms of |\IfIfileExists|.
%    So this is a `safe input' command, but the files input are not
%    listed by |\listfiles|.
%
%    We don't want |.aux|, |.toc| files etc be listed by |\listfiles|.
%    However, something like |.bbl| probably should be listed and thus
%    should be implemented not by |\@input|.
%    \begin{macrocode}
\def\@input#1{%
  \IfFileExists{#1}{\@@input\@filef@und}{\typeout{No file #1.}}}
%    \end{macrocode}
%  \end{macro}
%
% \begin{macro}{\@input@}
% \changes{v0.9p}{1994/01/18}{Macro added}
% \task{???}{share code with `@input ?}
% Version of |\@input| that does add the file to |\@filelist|.
%    \begin{macrocode}
\def\@input@#1{\InputIfFileExists{#1}{}{\typeout{No file #1.}}}
%    \end{macrocode}
%  \end{macro}
%
% \begin{macro}{\@missingfileerror}
% This `error' command avoids \TeX's primitive missing file loop.
%
% Missing file error. Prompt for a new filename, offering a default
% extension.
% \changes{LaTeX2e}{1993/11/21}
%         {Stop infinite looping on \cs{@er@ext}}
% \changes{LaTeX2e}{1993/11/28}
%         {Use filename parser from dircheck}
% \changes{LaTeX2e}{1994/03/15}
%         {Quit on x or X just like a real error}
% \changes{LaTeX2e}{1994/05/26}
%         {Modify message format}
% \changes{v1.0j}{1994/11/03}
%         {Move here from ltclass}
% \changes{v1.0w}{1995/10/06}
%         {Autoload error}
%    \begin{macrocode}
%<autoload>\def\@missingfileerror{\@autoerr\@missingfileerror}
%</2ekernel|autoload>
%<*2ekernel|autoerr>
\gdef\@missingfileerror#1#2{%
     \typeout{^^J! LaTeX Error: File `#1.#2' not found.^^J^^J%
      Type X to quit or <RETURN> to proceed,^^J%
      or enter new name. (Default extension: #2)^^J}%
     \message{Enter file name: }%
      {\endlinechar\m@ne
       \global\read\m@ne to\@gtempa}%
    \ifx\@gtempa\@empty
    \else
      \def\reserved@a{x}\ifx\reserved@a\@gtempa\batchmode\@@end\fi
      \def\reserved@a{X}\ifx\reserved@a\@gtempa\batchmode\@@end\fi
      \filename@parse\@gtempa
      \edef\filename@ext{%
        \ifx\filename@ext\relax#2\else\filename@ext\fi}%
     \edef\reserved@a{%
       \noexpand\InputIfFileExists
         {\filename@area\filename@base.\filename@ext}%
         {}%
         {\noexpand\@missingfileerror
            {\filename@area\filename@base}{\filename@ext}}}%
      \reserved@a
    \fi}
%</2ekernel|autoerr>
%<*2ekernel|autoload>
%    \end{macrocode}
% \end{macro}
%
% \begin{macro}{\@obsoletefile}
%    For compatibility with \LaTeX~2.09 document styles, we distribute
%    files called |article.sty|, |book.sty|, |report.sty|,
%    |slides.sty| and |letter.sty|.  These use the command
%    |\@obsoletefile|, which produces a warning message.
%    \changes{v0.9m}{1993/12/20}{Added this command, removed
%                               @oldfilewarning}
% \changes{v1.0f}{1994/05/02}{Make \cs{@onlypreamble}}
%    \begin{macrocode}
\def\@obsoletefile#1#2{%
   \@latex@warning@no@line{inputting `#1' instead of obsolete `#2'}}
\@onlypreamble\@obsoletefile
%    \end{macrocode}
%
% \subsection{Listing files}
%
% \begin{macro}{\@filelist}
% A list of files input so far. The initial value of |\@gobble| eats
% the comma before the first file name.
%    \begin{macrocode}
\let\@filelist\@gobble
%    \end{macrocode}
% \end{macro}
%
% \begin{macro}{\@addtofilelist}
% Add to the  list of files input so far.
% This `real' definition is only used for `cfg' files during initex.
% An initial definition of |\@gobble| has already been set.
%    \begin{macrocode}
%\def\@addtofilelist#1{\xdef\@filelist{\@filelist,#1}}
%    \end{macrocode}
% \end{macro}
%
% \begin{macro}{\listfiles}
% A preamble command to cause |\end{document}| to list files input
% from the main file.
%    \begin{macrocode}
\def\listfiles{%
  \let\listfiles\relax
  \def\@listfiles##1##2##3##4##5##6##7##8##9\@@{%
     \def\reserved@d{\\}%
     \@tfor\reserved@c:=##1##2##3##4##5##6##7##8\do{%
       \ifx\reserved@c\reserved@d
         \edef\filename@area{ \filename@area}%
       \fi}}%
%    \end{macrocode}
%
% \changes{v1.0o}{1994/11/30}
%         {Use \cs{@dofilelist}}
%    \begin{macrocode}
  \def\@dofilelist{%
     \typeout{^^J *File List*}%
     \@for\@currname:=\@filelist\do{%
       \filename@parse\@currname
       \edef\reserved@a{%
          \filename@base.%
          \ifx\filename@ext\relax tex\else\filename@ext\fi}%
       \expandafter\let\expandafter\reserved@b
                              \csname ver@\reserved@a\endcsname
       \expandafter\expandafter\expandafter\@listfiles\expandafter
             \filename@area\filename@base\\\\\\\\\\\\\\\\\\\@@
       \typeout{%
         \filename@area\reserved@a
         \ifx\reserved@b\relax\else\@spaces\reserved@b\fi}}%
     \typeout{ ***********^^J}}}
%    \end{macrocode}
%
% \changes{LaTeX2e}{1994/03/13}
%         {Reset \cs{@addtofilelist} at begin document}
% \changes{LaTeX2e}{1994/03/16}
%         {Move this code directly into \cs{document}}
% The |\@filelist| will be de-activated if |\listfiles| does not
% appear in the preamble. |\begin{document}| contains code equivalent
% to the following:
%\begin{verbatim}
% \AtBeginDocument{%
%   \ifx\@listfiles\@undefined
%     \let\@filelist\relax
%     \let\@addtofilelist\@gobble
%   \fi}
%\end{verbatim}
%    \begin{macrocode}
\@onlypreamble\listfiles
%    \end{macrocode}
% \end{macro}
%
%  \begin{macro}{\@dofilelist}
% \changes{v1.0o}{1994/11/30}
%         {Macro added}
%    \begin{macrocode}
\let\@dofilelist\relax
%    \end{macrocode}
%  \end{macro}

