\chapter{Mathlists}
\parindent1em

It is not an easy undertaking to try and descibe a long algorithm in words. The best source
of the algorithm is \tex itself, which is available at ctan. However, I have attempted
as best as I can to describe the algorithm and its main constituents below in a summary format,
drawing heavily from the tex source, \citep{texbytopic} and \citep{breiten}. Another good source
is the LuaTeX reference manual (which puts things in a more modern perspective).\footnote{texdoc luatexref-t to get it.}


Similar to the way horizontal and vertical lists are build \tex also constructs another type of list, called a mathlist\footnote{mlist in the code.}. An entire formula is parsed into such a 
syntax tree before any of the actual
typesetting is done, because the current style of type is usually not
known until the formula has been fully scanned. For example, when the
formula |$a+b \over c+d$| is being read, there is no way to tell
that |a+b| will be in script size until |\over| has appeared.

During the scanning process, each element of the mlist being built is
classified as a relation, a binary operator, an open parenthesis, etc.,
or as a construct like |\sqrt| that must be built up. This classification
appears in the mlist data structure.

After a formula has been fully scanned, the mlist is converted to an hlist
so that it can be incorporated into the surrounding text. This conversion is
controlled by a recursive procedure that decides all of the appropriate
styles by a ``top-down'' process starting at the outermost level and working
in towards the subformulas. The formula is ultimately pasted together using
combinations of horizontal and vertical boxes, with glue and penalty nodes
inserted as necessary

A math list is essentially a syntax tree which is a collection of nodes and  math nodes. The
math nodes are called `noad's and the nodes that are specifically associated with it. Most of these
nodes contain sub-nodes so that the list of possible fields is actually quite small.

\begin{enumerate}
\item Atoms. $\mathbf{a+b+c}$
\item A generalized fraction resulting from basic \tex primitives such as \cmd{\above}, \cmd{\over}, etc.
\item A boundary from |\left| or |\right|.
\item A four way choice resulting from |\mathchoice|
\item A style change
\item Other elements such as horizontal material 
\end{enumerate}

Atoms consist of three fields the \emph{nucleus}, \emph{subscript} and \emph{superscript}. The nucleus of an atom contains a type. There are thirteen different atom types. Each portion of a formula is clasified as
Ord,  Op, Bin, Rel, Ope,
Clo, Pun, or Inn, for purposes of spacing and line breaking. An
|ord_noad|, |op_noad|, |bin_noad|, |rel_noad|, |open_noad|, |close_noad|,
|punct_noad|, or |inner_noad| is used to represent portions of the various
types. For example, an `=' sign in a formula leads to the creation of a
|rel_noad| whose |nucleus| field is a representation of an equals sign
(usually |fam=0|, |character=@'75|).  A formula preceded by |\mathrel|
also results in a |rel_noad|.

\begin{longtable}{lp{8cm}}
  Ordinary & ordinary atom, as for instance, x,y,z,1,2,3 and lowercase Greek charaters.
                                The command \docAuxCommand{mathord} forces this class\\
  Large  operator  & large operator atom as for instance, $\sum$, $\prod$ etc. The command
                               \docAuxCommand{mathop} forces this class. Characters that are large
                               operators are centered vertically, and behave differently in display styles from
                               in the other styles.\\
  Binary  & binary operator atom e.g. $+ - /$. The command \docAuxCommand{mathbin} forces this
                               class.\\
  Rel  & realtional operator, such as $lt$.\\
  Open & an opening atom like, \{. \\
  Close & a closing atom such as \}. \\
  Punct &a punctuation atom such as a comma.\\
  Inner & inner atom such as one representing a fraction $x^3/2$.\\
  Over  & an overline atom such as $\bar{x}$\\
  Under &an underline atom, such as x\\
  Acc     & an accented atom such $\hat{x}$\\
  Rad    & a  radical atom such as $\sqrt{x^3}$ \\
  Vcent & a |vbox| to be centered that is generated by \cmd{\vcenter}\\
\end{longtable}



\section{Math characters and mathcodes}

Every character used for math lists has an asssociated \emph{mathcode}.  Every character has
the following associated with it. In order for the algorithm to work it needs to know certain parameters.
In a more modern language these would have been programmed as properties of an object type
atom.

\begin{description}
\item [class] A class with a number range from 0 \ldots 7.  There are seven classes predefined in \tex.
\item [family] A family (a number in the range 0\ldots15.
\item [character code] A character code (a number on the range 0\ldots127 in the selected family.


For efficiency reasons \tex encoded these three numbers into one hexadecimal number,
four digits long.

\begin{enumerate}
\item The first digit represents the class
\item The second digit represents the family
\item The third or fourth digits represent the character code.
\end{enumerate}

This hexademical number is termed the \emph{mathcode}. For example a mathcode of class x, family y and character code zz is the four digit long hexadecimal number xyzz.
\ExplSyntaxOn
\int_to_hex:n {\the\mathcode`,}
\ExplSyntaxOff

\item [subformula] A subformula is assigned a class only.

\end{description}


\section{Making Up your own symbols}

In some cases you would like to define your own symbol

\begin{figure}[htbp]
\includegraphics[width=\textwidth]{mathlistalgo}
\caption{The scheme of the algorithm of the math list processing; the numbers at the left of the boxes refer to the steps of the algorithm, as described in Appendix G. (after Jackowski (2006)}
\end{figure}

\begin{figure}[htbp]
\includegraphics[width=\textwidth]{mathlimits}
\end{figure}








